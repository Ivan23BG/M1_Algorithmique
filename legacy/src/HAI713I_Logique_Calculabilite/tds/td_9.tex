% ----- Consignes exo 1 ----- %
\begin{td-exo}[Théorie de l'incrément (TD logique 3)]\,\\ % 1
    Dans le modèle \(\left( \bb N, S(x)\right)\) où \(S(x)\) est interprété comme ajoutant 1 au nombre \(x\).

    \begin{enumerate}
        \item Montrez qu'il existe une formule \(F_p(x,y)\) qui exprime la propriété \(x = y+p\).

        \item Montrez qu'il existe une formule \(F_p(x,y)\) de taille \(O(\log(p))\) qui exprime la propriété \(x = y+p\).
    \end{enumerate}
\end{td-exo}

% ----- Solutions exo 1 ----- %
\iftoggle{showsolutions}{
	\begin{td-sol}[]\, % 1
		\begin{enumerate}
            \item On pose 
            \begin{equation*}
                F_p(x,y) \equiv x = y + p
            \end{equation*}
            où
            \begin{equation*}
                x = \underbrace{S(S(\cdots(S(y))))}_{p \text{ fois}}
            \end{equation*}
            Alors, on a
            \begin{equation*}
                |F_p(x,y)| = O(p)
            \end{equation*}

            \item On a deux cas:
            \begin{itemize}
                \item Si \(p = 2k\), on pose 
                \begin{equation*}
                    x = y + p \iff  \exists z (F_k(x,y) \land F_k(z,y))
                \end{equation*}
                \item Si \(p = 2k+1\), in pose 
                \begin{equation*}
                    x = y + p \iff  \exists z (F_k(x,y) \land F_k(S(z),y))
                \end{equation*}
            \end{itemize}
            On peut alors montrer par induction que 
            \begin{equation*}
                |F_p(x,y)| = O(\log(p))
            \end{equation*}
        \end{enumerate}
	\end{td-sol}
}{}

% ----- Consignes exo 2 ----- %
\begin{td-exo}[Théories complètes]\, % 2
    \begin{enumerate}
        \item Montrez qu'une théorie \(T\) est complète si et seulement si tous ses modèles sont élémentairement équivalents.
        
        \item Soit \(T\) un ensemble de formules quelconque.
        On note \(\theta (T)\) les conséquences de \(T\), donc 
        \begin{equation*}
            \theta(T) = \{\phi, T \vdash \phi\}.
        \end{equation*}
        Montrez que \(T\) et \(\theta(T)\) ont les mêmes théorèmes.
    \end{enumerate}
\end{td-exo}

% ----- Solutions exo 2 ----- %
\iftoggle{showsolutions}{
	\begin{td-sol}[]\, % 2
		\begin{enumerate}
            \item Exercice solution
        \end{enumerate}
	\end{td-sol}
}{}

% ----- Consignes exo 3 ----- %
\begin{td-exo}[Optional title xx]\,\\ % 3
Exercise xx content
\end{td-exo}

% ----- Solutions exo 3 ----- %
\iftoggle{showsolutions}{
	\begin{td-sol}[]\,\\ % 3
		Exercice solution
	\end{td-sol}
}{}