% Pour l'exercice 1
% Le plus simple quand on a deux groupes d'éléments en association est de penser à un graphe biparti. Ici le graphe bipartie contiendrait alors les associations dans un groupe et les personnes dans l'autre.

% Pour l'exercice 4.4
% On veut un exemple de graphe cubique sans couplage parfait, il faut mettre 3 graphes reguliers de degré 3 reliés a un unique sommet central, alors c'est impossible de trouver un couplage parfait dans ce graphe
% Pour la version k, on prend 1 vers 2k, 2k vers 2k et couplage dans ce dernier

% Exercice 5
% On veut couvrir toutes les aretes de K_n avec un nombre minimum de couplages.

% On appelle ce qu'on étudie ici l'indice chromatique
% La solution est n-1 si n est pair, n sinon
