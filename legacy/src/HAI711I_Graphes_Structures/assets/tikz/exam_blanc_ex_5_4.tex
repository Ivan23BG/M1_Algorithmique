\ffigbox[\FBwidth]{%
\caption{\centering Complémentaire du graphe \(G \cup P\)}\label{Fig:exam_blanc_ex_5_4}
}{
    \fbox{
        \begin{tikzpicture}[scale=1, main node/.style={circle, draw, fill=blue!20, inner sep=1pt, font=\scriptsize, minimum size=6mm}]
            % les sommets de G
            \node[main node] (g1) at (-1, 2) {\(g1\)};
            \node[] (dots) at (-1, 0) {\(\vdots\)};
            \node[main node] (gn) at (-1, -2) {\(g_n\)};

            % les sommets de P
            \node[main node] (p1) at (4, 3) {\(p1\)};
            \node[main node] (p2) at (5, 1) {\(p2\)};
            \node[main node] (p3) at (3, -1) {\(p3\)};
            \node[main node] (p4) at (4, -3) {\(p4\)};

            % on entoure G
            \node[inner sep=12pt, fit=(g1) (gn), name=GFIT] {};

            \draw[green!60, thick, fill=green!20, opacity=0.4]
                (GFIT.center) ellipse [x radius=1.3cm, y radius=2.5cm];
            \node at (GFIT.south) [yshift=-10pt] {\(G\)};

            % les aretes
            % dans P
            \draw (p1) -- (p3);
            \draw (p1) -- (p4);
            \draw (p2) -- (p4);
            
            % entre G et P
            \draw (p2) -- (g1);
            \draw[dashed] (p2) -- (dots);
            \draw (p2) -- (gn);

            \draw (p3) -- (g1);
            \draw[dashed] (p3) -- (dots);
            \draw (p3) -- (gn);
        \end{tikzpicture}
    }
}