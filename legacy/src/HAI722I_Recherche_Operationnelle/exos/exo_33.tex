% ----- Consignes exo 33 ----- %
\begin{td-exo}[Détermination du dual] % 33
	Déterminer le dual des programmes suivants:
	\begin{enumerate}
		\item \, \begin{equation*}
			(PL_0) = 
			\begin{cases}
				\min z(x_1, x_2, x_3) = 5x_1 + 2x_2 + x_3\\
				2x_1 + 3x_2 + x_3 \geq 20\\
				6x_1 + 8x_2 + 5x_3 \geq 30\\
				7x_1 + x_2 + 3x_3 \geq 40\\
				x_1 + 2x_2 + 4x_3 \geq 50\\
				x_1, x_2, x_3 \geq 0.
			\end{cases}
		\end{equation*}

		\item \, \begin{equation*}
			(PL_1) = 
			\begin{cases}
				\max z(x_1, x_2) = 2x_1 + x_2\\
				x_1 + 5x_2 \leq 10\\
				x_1 + 3x_2 \leq 6\\
				2x_1 + 2x_2 \leq 8\\
				x_1, x_2 \geq 0.
			\end{cases}
		\end{equation*}
	\end{enumerate}
\end{td-exo}

% ----- Solutions exo 33 ----- %
\iftoggle{showsolutions}{
	\begin{td-sol}[]\ %
		On a
		\begin{enumerate}
			\item Le dual du programme \((PL_0)\) est
			\begin{equation*}
				\begin{cases}
					\max 20 y_1 + 30 y_2 + 40 y_3 + 50 y_4\\
					2y_1 + 6y_2 + 7y_3 + y_4 \leq 5\\
					3y_1 + 8y_2 + y_3 + 2y_4 \leq 2\\
					y_1 + 5y_2 + 3y_3 + 4y_4 \leq 1\\
					y_1, y_2, y_3, y_4 \geq 0.
				\end{cases}
			\end{equation*}

			\item Le dual du programme \((PL_1)\) est
			\begin{equation*}
				\begin{cases}
					\min 10y_1 + 6y_2 + 8y_3\\
					y_1 + y_2 + 2y_3 \geq 2\\
					5y_1 + 3y_2 + 2y_3 \geq 1\\
					y_1, y_2, y_3 \geq 0.
				\end{cases}
			\end{equation*}
		\end{enumerate}

		Pour ce qui est de leur résolution, on a:
		\begin{enumerate}
			\item \, % TODO add answer

			\item On a

			\begin{center}
				\begin{tabular}{|ccc|ccccc|} % chktex 44
					\hline  % chktex 44
					\, & \, &\(c\)&\(2\)&\(1\)&\(0\)&\(0\)&\(0\)\\
					\hline % chktex 44
					\multicolumn{1}{|c|}{\(c^J\)}& \multicolumn{2}{c|}{variables de base}&\(y_1\)&\(y_2\)&\(y_3\)&\(y_4\)&\(y_5\)\\
					\hline % chktex 44
					\multicolumn{1}{|c|}{\(0\)}& \multicolumn{1}{c|}{\(x_1^{1}=x_3\)} &\(10\)&\(1\)&\(5\)&\(1\)&\(0\)&\(0\)\\
					\hline % chktex 44
					\multicolumn{1}{|c|}{\(0\)}& \multicolumn{1}{c|}{\(x_2^{1}=x_4\)} &\(6\)&\(1\)&\(3\)&\(0\)&\(1\)&\(0\)\\
					\hline % chktex 44
					\multicolumn{1}{|c|}{\(0\)}& \multicolumn{1}{c|}{\(x_3^{1}=x_5\)} &\(8\)&\(2\)&\(2\)&\(0\)&\(0\)&\(1\)\\
					\hline % chktex 44
					\multicolumn{1}{|c|}{} &\(z(x)\)& \multicolumn{1}{|c|}{\(0\)} &\(-2\)&\(-1\)&\(0\)&\(0\)&\(0\)\\
					\hline % chktex 44
				\end{tabular}
			\end{center}
			\becomes{}
			\begin{center}
				\begin{tabular}{|ccc|ccccc|} % chktex 44
					\hline  % chktex 44
					\, & \, &\(c\)&\(2\)&\(1\)&\(0\)&\(0\)&\(0\)\\
					\hline % chktex 44
					\multicolumn{1}{|c|}{\(c^J\)}& \multicolumn{2}{c|}{variables de base}&\(y_1\)&\(y_2\)&\(y_3\)&\(y_4\)&\(y_5\)\\
					\hline % chktex 44
					\multicolumn{1}{|c|}{\(0\)}& \multicolumn{1}{c|}{\(x_1^{2}=x_3\)} &\(\vphantom{\dfrac16}6\)&\(0\)&\(4\)&\(1\)&\(0\)&\(-\frac12\)\\
					\hline % chktex 44
					\multicolumn{1}{|c|}{\(0\)}& \multicolumn{1}{c|}{\(x_2^{2}=x_4\)} &\(\vphantom{\dfrac16}2\)&\(0\)&\(2\)&\(0\)&\(1\)&\(-\frac12\)\\
					\hline % chktex 44
					\multicolumn{1}{|c|}{\(2\)}& \multicolumn{1}{c|}{\(x_3^{2}=x_1\)} &\(\vphantom{\dfrac16}4\)&\(1\)&\(1\)&\(0\)&\(0\)&\(\frac12\)\\
					\hline % chktex 44
					\multicolumn{1}{|c|}{} &\(z(x)\)& \multicolumn{1}{|c|}{\(8\)} &\(0\)&\(1\)&\(0\)&\(0\)&\(1\)\\
					\hline % chktex 44
				\end{tabular}
			\end{center}

			Comme \(x_3,x_4\) sont en bases, leurs variables duales associées valent 0.

			Pour \(x_5\), on prend la valeur associée donc 1. Cela donne:
			\begin{equation*}
				(x_3,x_4,x_5) = (0, 0, 1) \implies (y_1,y_2,y_3) = (0, 0, 1).
			\end{equation*}
		\end{enumerate}
	\end{td-sol}
}{}
