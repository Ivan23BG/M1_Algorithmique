
% ----- Consignes exo 29 ----- %
\begin{td-exo}[La méthode du simplexe n'est pas une méthode polynomiale]\,\\ % 29
	Dans cet exercice nous allons montrer que la méthode du simplexe n'est pas une méthode polynomiale.
	Pour cela, nous allons considérer le programme linéaire suivant:
	\begin{equation*}
		(PL_n)=
		\begin{cases}
			\max z = 2^{n-1} x_1 + 2^{n-2} x_2 + \cdots + 2x_{n-1} + x_n \\
			x_1 \leq 5\\
			4x_1 + x_2 \leq 25\\
			8x_1 + 4x_2 + x_3 \leq 125\\
			\ldots\\
			2^n x_1 + 2^{n-1} x_2 + \cdots + 4x_{n-1} + x_n \leq 5^n\\
			x_i \geq 0, \quad i=1,\ldots,n.
		\end{cases}
	\end{equation*}\,
	\begin{enumerate}
		\item Donner la forme standard de \((PL_n)\).
		\item Donner la solution optimale.
		\item Montrer que \(\forall i, x_i\) et \(y_i\) (où \(y_i\) est la variable d'écart associée à la \(i\)-ième
		variable) ne peuvent être en même temps des variables hors base.
		\item Nous allons montrer que \(2^n\) tableaux sont nécessaires pour résoudre ce problème.
		\begin{itemize}
			\item Vérifez-le pour \(n=1\).
			\item Montrez-le pour \(n=2\).
			\begin{itemize}
				\item Résoudre graphiquement.
				\item Résoudre par la méthode des tableaux. Combien de tableaux sont nécessaires
				pour résoudre \((PL_2)\).
				\item Quel est le chemin des visites des points extrêmes. Montrez que le 
				dernier tableau peut se mettre sous la forme (voir le tableau 2).
				\begin{equation*}
					2cx + x_n \leq 5^n
				\end{equation*}
				où
				\begin{equation*}
					l\in \{1,\ldots,n-1\}, \quad c = (2^{n-1}, 2^{n-2}, \ldots, 2),
				\end{equation*}
				et \(A\) est la matrice associé au \(n-1\) premières contraintes.
			\end{itemize}
			\item Supposons par hypothèse de récurrence que la résolution du programme linéaire 
			\(PL_n\) nécessite \(2^n\) tableaux. Nous allons montrer que la résolution du
			programme linéaire \(PL_{n+1}\) nécessite \(2^{n+1}\) tableaux.
			\begin{itemize}
				\item 
			\end{itemize}
		\end{itemize}
	\end{enumerate}
\end{td-exo}

% ----- Solutions exo 29 ----- %
\iftoggle{showsolutions}{
	\begin{td-sol}[]\ %
		\begin{enumerate}
			\item On réécrit les équations avec les \(n\) variables d'écart \(y_i\):
			\begin{equation*}
				\begin{cases}
					\max z = 2^{n-1} x_1 + 2^{n-2} x_2 + \cdots + 2x_{n-1} + x_n + 0\cdot y_1 + \cdots + 0\cdot y_n \\
					x_1 + y_1 = 5\\
					4x_1 + x_2 + y_2 = 25\\
					8x_1 + 4x_2 + x_3 + y_3 = 125\\
					\vdots\\
					2^n x_1 + 2^{n-1} x_2 + \cdots + 4x_{n-1} + x_n + y_n = 5^n\\
					x_i, y_i \geq 0, \quad i=1,\ldots,n.
				\end{cases}
			\end{equation*}
			\item Pour \(n=1\), la solution optimale est \(x_1 = 5\).

			Pour \(n=2\), la solution optimale est \((0, 25)\). On peut
			vérifier avec le tableau du simplexe qui suit:
			\begin{center}
				\begin{tabular}{|ccc|cccc|} % chktex 44
					\hline  % chktex 44
					& \ &\(c\)&\(2\)&\(1\)&\(0\)&\(0\)\\
					\hline % chktex 44
					\multicolumn{1}{|c|}{\(c^J\)}& \multicolumn{2}{c|}{variables de base}&\(x_1\)&\(x_2\)&\(y_1\)&\(y_2\)\\
					\hline % chktex 44
					\multicolumn{1}{|c|}{\(0\)}& \multicolumn{1}{c|}{\(x_1^{1}=y_1\)} &\(5\)&\(1\)&\(0\)&\(1\)&\(0\)\\
					\hline % chktex 44
					\multicolumn{1}{|c|}{\(0\)}& \multicolumn{1}{c|}{\(x_2^{1}=y_2\)} &\(25\)&\(4\)&\(1\)&\(0\)&\(1\)\\
					\hline % chktex 44
					\multicolumn{1}{|c|}{} &\(z(x)\)& \multicolumn{1}{|c|}{\(0\)} &\(-2\)&\(-1\)&\(0\)&\(0\)\\
					\hline % chktex 44
				\end{tabular}
			\end{center}
			ensuite
			\begin{center}
				\begin{tabular}{|ccc|cccc|} % chktex 44
					\hline  % chktex 44
					& \ &\(c\)&\(2\)&\(1\)&\(0\)&\(0\)\\
					\hline % chktex 44
					\multicolumn{1}{|c|}{\(c^J\)}& \multicolumn{2}{c|}{variables de base}&\(x_1\)&\(x_2\)&\(y_1\)&\(y_2\)\\
					\hline % chktex 44
					\multicolumn{1}{|c|}{\(2\)}& \multicolumn{1}{c|}{\(x_1^{2}=x_1\)} &\(5\)&\(1\)&\(0\)&\(1\)&\(0\)\\
					\hline % chktex 44
					\multicolumn{1}{|c|}{\(0\)}& \multicolumn{1}{c|}{\(x_2^{2}=y_2\)} &\(5\)&\(0\)&\(1\)&\(-4\)&\(1\)\\
					\hline % chktex 44
					\multicolumn{1}{|c|}{} &\(z(x)\)& \multicolumn{1}{|c|}{\(10\)} &\(0\)&\(-1\)&\(2\)&\(0\)\\
					\hline % chktex 44
				\end{tabular}
			\end{center}
			ensuite
			\begin{center}
				\begin{tabular}{|ccc|cccc|} % chktex 44
					\hline  % chktex 44
					& \ &\(c\)&\(2\)&\(1\)&\(0\)&\(0\)\\
					\hline % chktex 44
					\multicolumn{1}{|c|}{\(c^J\)}& \multicolumn{2}{c|}{variables de base}&\(x_1\)&\(x_2\)&\(y_1\)&\(y_2\)\\
					\hline % chktex 44
					\multicolumn{1}{|c|}{\(2\)}& \multicolumn{1}{c|}{\(x_1^{3}=x_1\)} &\(5\)&\(1\)&\(0\)&\(1\)&\(0\)\\
					\hline % chktex 44
					\multicolumn{1}{|c|}{\(1\)}& \multicolumn{1}{c|}{\(x_2^{3}=x_2\)} &\(5\)&\(0\)&\(1\)&\(-4\)&\(1\)\\
					\hline % chktex 44
					\multicolumn{1}{|c|}{} &\(z(x)\)& \multicolumn{1}{|c|}{\(10\)} &\(0\)&\(0\)&\(-2\)&\(1\)\\
					\hline % chktex 44
				\end{tabular}
			\end{center}
			et enfin
			\begin{center}
				\begin{tabular}{|ccc|cccc|} % chktex 44
					\hline  % chktex 44
					& \ &\(c\)&\(2\)&\(1\)&\(0\)&\(0\)\\
					\hline % chktex 44
					\multicolumn{1}{|c|}{\(c^J\)}& \multicolumn{2}{c|}{variables de base}&\(x_1\)&\(x_2\)&\(y_1\)&\(y_2\)\\
					\hline % chktex 44
					\multicolumn{1}{|c|}{\(0\)}& \multicolumn{1}{c|}{\(x_1^{4}=y_1\)} &\(5\)&\(1\)&\(0\)&\(1\)&\(0\)\\
					\hline % chktex 44
					\multicolumn{1}{|c|}{\(1\)}& \multicolumn{1}{c|}{\(x_2^{4}=x_2\)} &\(25\)&\(4\)&\(1\)&\(0\)&\(1\)\\
					\hline % chktex 44
					\multicolumn{1}{|c|}{} &\(z(x)\)& \multicolumn{1}{|c|}{\(25\)} &\(2\)&\(0\)&\(0\)&\(1\)\\
					\hline % chktex 44
				\end{tabular}
			\end{center}
			ce qui confirme que la solution optimale est \((0, 25)\).
			\item On ecrit les deux lignes du tableau et on 
			regarde ce qui se passe en \(x_i\) et \(y_i\):
			\begin{equation*}
				\begin{aligned}
					&2^{i-1}x_1 + 2^{i-2}x_2 + \cdots + 4x_{i-2} + x_{i-1} + y_{i-1} = 5^{i-1}\\
					&2^i x_1 + 2^{i-1}x_2 + \cdots + 4x_{i-1} + x_i + y_i = 5^i
				\end{aligned}
			\end{equation*}
			Si \(y_i = 0\), alors dans l'équation du haut on a
			\begin{equation*}
				4x_{i-1} = 5^{i-1} - 2^{i-1}x_1 - 2^{i-2}x_2 - \cdots - x_{i-2} \geq 0
			\end{equation*}
			mais dans celle du bas on a
			\begin{equation*}
				4x_{i-1} = 4\cdot 5^{i-1} - 2^i x_1 - 2^{i-1}x_2 - \cdots - x_i > 0
			\end{equation*}
			
			Si on fait cela aussi avec \(x_i = 0\), on trouve que \(x_i = y_i = 0\)
			est impossible.

			\item On a
			\begin{enumerate}
				\item \,%
			\end{enumerate}
		\end{enumerate}
	\end{td-sol}
}{}