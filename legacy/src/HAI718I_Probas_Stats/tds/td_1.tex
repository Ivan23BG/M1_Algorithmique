\subsection*{Combinatoire et premiers calculs de probabilités}\label{subsec:ss_1}
\addcontentsline{toc}{subsection}{\nameref{subsec:ss_1}}


% ----- Consignes exo 1 ----- %
\begin{td-exo}[]\, % 1
    \begin{enumerate}
        \item En considérant les 26 lettres de l'alphabet, combien peut-on former de mots de 2 lettres?
        Combien peut-on former de mots de deux lettres constitués d'une consonne suivie d'une voyelle?
        Combien peut-on former de mots de deux lettres constitués d'une consonne et d'une voyelle?
        \item Combien d'équipes différentes de 3 personnes peut-on former à partir d'un groupe de 5 personnes?
        \item Avec 17 chevaux au départ, combien y a-t-il de tiercés possibles? Dans le désordre?
    \end{enumerate}
\end{td-exo}

% ----- Solutions exo 1 ----- %
\iftoggle{showsolutions}{
	\begin{td-sol}[]\, % 1
		\begin{enumerate}
            \item On peut former \(26 \times 26 = 676\) mots de 2 lettres. On peut former \(20 \times 6 = 120\) 
            mots de deux lettres constitués d'une consonne suivie d'une voyelle. On peut former \(20 \times 6 + 6 \times 20 = 240\) 
            mots de deux lettres constitués d'une consonne et d'une voyelle.
            \item On peut en former \(\binom{5}{3} = 10\) équipes différentes de 3 personnes à partir d'un groupe de 5 personnes.
            \item Avec 17 chevaux au départ, il y a \(17 \times 16 \times 15 = 4080\) tiercés possibles. Dans le désordre,
            il y en a \(4080 / 6 = 680\).
        \end{enumerate}
	\end{td-sol}
}{}


% ----- Consignes exo 2 ----- %
\begin{td-exo}[]\,\\ % 2
    Une urne contient \(n\) boules blanches (\(n\geq 5\)) et 10 boules noires. 
    On tire au hasard et simultanément 10 boules de l'urne.
    \begin{enumerate}
        \item Quelle est la probabilité \(p_n\) pour que l'on ait tiré 
        exactement 3 boules noires?
        \item Etudier le sens de variation de la suite \(p_n\) et calculer \(\lim_{n \to +\infty} p_n\).
    \end{enumerate}
\end{td-exo}


% ----- Solutions exo 2 ----- %
\iftoggle{showsolutions}{
	\begin{td-sol}[]\,\\ % 2
		\begin{enumerate}
            \item Exercice solution
        \end{enumerate}
	\end{td-sol}
}{}

\subsection*{Loi binomiale}\label{subsec:ss_2}
\addcontentsline{toc}{subsection}{\nameref{subsec:ss_2}}


% ----- Consignes exo xx ----- %
\begin{td-exo}[]\,\\ % xx
    On jette une fois un dé non truqué.
    \begin{enumerate}
        \item Quelle est la probabilité d'obtenir la face 1?
        Quelle est la loi de cet événement?

        On jette 18 fois le dé en question, quelles sont les probabilités 
        des événements suivants:
        \item Obtenir la face 1 exactement 3 fois.
        \item Obtenir la face 1 au moins 3 fois.
        \item Obtenir la face 1 au plus 16 fois.
    \end{enumerate}
\end{td-exo}

% ----- Solutions exo xx ----- %
\iftoggle{showsolutions}{
	\begin{td-sol}[]\,\\ % xx
		Exercice solution
	\end{td-sol}
}{}

\subsection*{La loi normale (gaussienne)}\label{subsec:ss_3}
\addcontentsline{toc}{subsection}{\nameref{subsec:ss_3}}

On suppose dans cette section que \(U \sim \mathcal{N}(0,1)\) (loi normale centrée réduite).

% ----- Consignes exo xx ----- %
\begin{td-exo}[]\,\\ % xx
    Calculer les probabilités suivantes:
    \begin{enumerate}
        \item \(P(U < 1.5)\)
        \item \(P(U > 2.5)\)
        \item \(P(U < -1.5)\)
        \item \(P(-1.5 < U < 2.5)\)
    \end{enumerate}
\end{td-exo}

% ----- Solutions exo xx ----- %
\iftoggle{showsolutions}{
	\begin{td-sol}[]\,\\ % xx
		Exercice solution
	\end{td-sol}
}{}


% ----- Consignes exo xx ----- %
\begin{td-exo}[]\,\\ % xx
    Trouver la valeur de \(u\) telle que:
    \begin{enumerate}
        \item \(P(U < u) = 0.95\)
        \item \(P(U < u) = 0.1\)
        \item \(P(U > u) = 0.99\)
        \item \(P(-u < U < u) = 0.95\)
    \end{enumerate}
\end{td-exo}

% ----- Solutions exo xx ----- %
\iftoggle{showsolutions}{
	\begin{td-sol}[]\,\\ % xx
		Exercice solution
	\end{td-sol}
}{}

On suppose dans la suite que \(X \sim \mathcal{N}(\mu = 2, \sigma^2 = 5^2)\) (loi normale de moyenne 2 et d'écart-type 5).
On a alors \( \frac{X - \mu}{\sigma} = \frac{X - 2}{5} \sim \mathcal{N}(0,1) \).


% ----- Consignes exo xx ----- %
\begin{td-exo}[]\,\\ % xx
    Calculer les probabilités suivantes:
    \begin{enumerate}
        \item \(P(X < 10)\)
        \item \(P(0 < X < 10)\)
    \end{enumerate}
\end{td-exo}

% ----- Solutions exo xx ----- %
\iftoggle{showsolutions}{
	\begin{td-sol}[]\,\\ % xx
		Exercice solution
	\end{td-sol}
}{}

% ----- Consignes exo xx ----- %
\begin{td-exo}[]\,\\ % xx
    Trouver la valeur de \(x\) telle que:
    \begin{enumerate}
        \item \(P(X < x) = 0.95\)
        \item \(P(X < x) = 0.05\)
        \item \(P(2 - x < X < 2 + x) = 0.95\)
    \end{enumerate}
\end{td-exo}

% ----- Solutions exo xx ----- %
\iftoggle{showsolutions}{
	\begin{td-sol}[]\,\\ % xx
		Exercice solution
	\end{td-sol}
}{}

\subsection*{La loi du Chi-deux \(\chi^2\) (ou loi de Pearson)}\label{subsec:ss_4}
\addcontentsline{toc}{subsection}{\nameref{subsec:ss_4}}

Pour \(U_1, \ldots, U_p\) des variables \(\mathcal{N}(0,1)\) indépendantes,
on appelle loi du Chi-deux à \(p\) degrés de liberté \((\chi^2_p)\) la loi de la variable aléatoire
\(\sum_{i=1}^p U_i^2\).

% ----- Consignes exo xx ----- %
\begin{td-exo}[]\, % xx
    \begin{enumerate}
        \item Que vaut la somme de deux variables de loi \(\chi^2\) indépendantes de degrés de liberté respectifs \(p\) et \(q\)?

        \item Soit \(X \sim \chi^2_{15}\) et \(Y \sim \chi^2_{10}\). Calculer:
        \begin{enumerate}
            \item \(P(X < 6.26)\) 
            \item \(P(Y > 3.25)\)
            \item \(P(X + Y > 11.52)\)
        \end{enumerate}

        \item Soit \(X \sim \chi^2_{15}\). Trouver la valeur de \(x\) telle que:
        \begin{enumerate}
            \item \(P(X < x) = 0.01\)
            \item \(P(X < x) = 0.05\)
            \item \(P(X < x) = 0.99\)
        \end{enumerate}
    \end{enumerate}
\end{td-exo}

% ----- Solutions exo xx ----- %
\iftoggle{showsolutions}{
	\begin{td-sol}[]\,\\ % xx
		Exercice solution
	\end{td-sol}
}{}

\subsection*{Théorème de la limite centrale}\label{subsec:ss_5}
\addcontentsline{toc}{subsection}{\nameref{subsec:ss_5}}

On rappelle quelques propriétés de la loi normale.



% % ----- Consignes exo xx ----- %
% \begin{td-exo}[]\,\\ % xx
%   
% \end{td-exo}

% % ----- Solutions exo xx ----- %
% \iftoggle{showsolutions}{
% 	\begin{td-sol}[]\,\\ % xx
% 		Exercice solution
% 	\end{td-sol}
% }{}

% % ----- Consignes exo xx ----- %
% \begin{td-exo}[]\,\\ % xx
%   
% \end{td-exo}

% % ----- Solutions exo xx ----- %
% \iftoggle{showsolutions}{
% 	\begin{td-sol}[]\,\\ % xx
% 		Exercice solution
% 	\end{td-sol}
% }{}