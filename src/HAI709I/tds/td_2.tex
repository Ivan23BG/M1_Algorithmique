% ----- Consignes exo 1 ----- %
\begin{td-exo}[] % 1
	Dire si les fonctions suivantes sont:
	\begin{itemize}
		\item bornées par un polynôme:
		\begin{gather*}
			f_1(n) = n, \quad f_2(n) = n^3 - 4n^2 + 2n + 9, \quad f_3(n) = n \log_2(n), \\
			f_4(n) = {1.1}^n, \quad f_5(n) = 2^{\frac12n}, \quad f_6(n) = 2^{\sqrt{n}}, \quad
			f_7(n) = n^{\log_2(n)}
		\end{gather*}
		\item négligeables:
		\begin{gather*}
			f_1(n) = \frac{1}{n^2+n+1}, \quad f_2(n) = \frac{3}{\log_2(n)}, \quad f_3(n) = 2^{-n}, \\
			f_4(n) = 2^{-\frac12n}, \quad f_5(n) = \frac{n^2+n+1}{e^n}, \quad f_6(n) = 2^{-\sqrt{n}}, \quad
			f_7(n) = n^{-\log_2(n)}
		\end{gather*}
	\end{itemize}
\end{td-exo}

% ----- Solutions exo 1 ----- %
\iftoggle{showsolutions}{
	\begin{td-sol}[]\ %
		
	\end{td-sol}
}{}


% ----- Consignes exo 2 ----- %
\begin{td-exo}[] % 2
	Considérons un chiffrement de Vigenère \(\pi\) où l’espace de messages est constitué de toutes les
	chaînes de 3 caractères et \(\gen\) fonctionne de la manière suivante: d’abord, une période \(t\) est
	tirée uniformément au hasard dans \({2, 3}\), puis une clé \(k\) est tirée uniformément au hasard dans
	\({0, . . . , 25}^t\).
	\begin{itemize}
		\item Définissez un adversaire \(\mathcal{A}\) tel que \(\bb P(\text{PrivK}^{eav}_{\mathcal{A}, \Pi} = 1) > \frac12\).
		\item Définissez un adversaire \(\mathcal{A}\) tel que \(\bb P(\text{PrivK}^{eav}_{\mathcal{A}, \Pi} = 1) \geq \frac34\).
	\end{itemize}
\end{td-exo}

% ----- Solutions exo 2 ----- %
\iftoggle{showsolutions}{
	\begin{td-sol}[]\ %
		
	\end{td-sol}
}{}


% ----- Consignes exo 3 ----- %
\begin{td-exo}[] % 3
	Soit \(G\) un PRG avec un facteur d’expansion \(l(n) > 2n\). Dans chacun des cas suivants, prouver que
	\(G'\) est aussi un PRG, si tel est le cas. Sinon, trouver un contre-exemple.
	\begin{itemize}
		\item Définissez \(G' \overset{\text{def}}= G(s_1,\ldots,s_{\lceil \frac n4\rceil})\),
		où \(s=s_1,\ldots s_n\).
		\item Définissez \(G' \overset{\text{def}}= G(s_1,\ldots,s_{\lceil \frac n2\rceil})\),
		où \(s=s_1,\ldots s_n\).
		\item Définissez \(G' \overset{\text{def}}= G(s\mid\mid0^{\n s})\).
	\end{itemize}
\end{td-exo}

% ----- Solutions exo 3 ----- %
\iftoggle{showsolutions}{
	
}{}


% ----- Consignes exo 4 ----- %
\begin{td-exo}[] % 4
	TO FILL
\end{td-exo}


% ----- Solutions exo 4 ----- %
\iftoggle{showsolutions}{
	\begin{td-sol}[]\ %
		TO FILL
	\end{td-sol}
}{}


% ----- Consignes exo 5 ----- %
\begin{td-exo}[] % 5
	TO FILL
\end{td-exo}

% ----- Solutions exo 5 ----- %
\iftoggle{showsolutions}{
	\begin{td-sol}[]\ %
		TO FILL
	\end{td-sol}
}{}
