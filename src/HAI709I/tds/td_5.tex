% ----- Consignes exo 1 ----- %
\begin{td-exo}[] % 1
	Soit \(k\in {\{0,1\}}^n\) la clé d'un MAC.\@
    Prouvez que, si la longueur de l'étiquette
    \begin{equation*}
        t(n) = \mathsf{O}(\log n),
    \end{equation*}
    alors le MAC n'est pas sûr (c'est-à-dire, qu'un MAC sûr
    doit toujours avoir un \(t(n)\) super-logarithmique).
\end{td-exo}

% ----- Solutions exo 1 ----- %
\iftoggle{showsolutions}{
	\begin{td-sol}[]\ %
        On peut faire une attaque par force brute. On prend
        \((m, t)\) au hasard et \(m\) admet au moins une étiquette.
        \begin{equation*}
            \bb P(\mathsf{Vrfy}_k(m,t) = 1) \geq \frac{1}{2^{\mathsf{O}(\log n)}} = \frac{1}{\mathsf{poly}(n)}.
        \end{equation*}
    \end{td-sol}
}{}

% ----- Consignes exo 2 ----- %
\begin{td-exo}[]\, % 2
	\begin{enumerate}
        \item Ecrivez formellement l'expérience correspondant
        à l'infalsifiabilité forte d'un MAC à partir de celle
        de l'infalsifiabilité.

        \item Prouvez la proposition vue en cours:
        \begin{proposition}
            Un MAC déterministe sûr avec vérification
            canonique est fortement sûr.
        \end{proposition}
    \end{enumerate}
\end{td-exo}

% ----- Solutions exo 2 ----- %
\iftoggle{showsolutions}{
	\begin{td-sol}[]\ %
        \begin{enumerate}
            \item \, % TODO formellement

            \item Il y a 2 cas de réussite pour l'adversaire:
            \begin{enumerate}
                \item Il produit \((m,t)\) tel que \(\forall (m',t')\in Q\) avec \(m'\neq m\),
                \begin{equation*}
                    \bb P(\mathsf{Vrfy}_k(m,t) = 1) \leq \mathsf{negl}(n).
                \end{equation*}
                car MAC est sûr.

                \item Il produit \((m,t)\) tel que \(\exists (m,t')\in Q\) avec \(t'\neq t\),
                \begin{equation*}
                    \bb P(\mathsf{Vrfy}_k(m,t) = 1) = 0.
                \end{equation*}

            \end{enumerate}
            Donc 
            \begin{equation*}
                \bb P_{\text{tot}} = \bb P_1 + \bb P_2 \leq \mathsf{negl}(n).
            \end{equation*}
            Donc le MAC est fortement sûr.
        \end{enumerate}
    \end{td-sol}
}{}

% ----- Consignes exo 3 ----- %
\begin{td-exo}[] % 3
	Considérons le MAC suivant pour des messages de longueur
    \(l(n) = 2n-2\) utilisant une PRF \(F\).

    En entrée, un message \(m_0 || m_1\) avec \(|m_0| = |m_1| = n-1\)
    et une clé \(k \in {\{0,1\}}^n\), le MAC produit
    \begin{equation*}
        t = F_k(0 || m_0) || F_k(1 || m_1).
    \end{equation*}
    La vérification est canonique. Le MAC est-il sûr?
\end{td-exo}

% ----- Solutions exo 3 ----- %
\iftoggle{showsolutions}{
	\begin{td-sol}[]\ %
        On a
        \begin{equation*}
            \mathsf{Mac}_k(m_0 || m_1) = F_k(0 || m_0) || F_k(1 || m_1).
        \end{equation*}
        
        L'adversaire peut utiliser la stratégie suivante:
        \begin{enumerate}
            \item Il demande l'étiquette de \(m_0 || m_{i\neq 1}\) à
            l'oracle et reçoit \(t_1 || t_2\).
            \item Il demande l'étiquette de \(m_{j\neq 0} || m_1\) à
            l'oracle et reçoit \(t_3 || t_4\).
            \item Il produit \((m_0 || m_1, t_1 || t_4)\).
        \end{enumerate}
    \end{td-sol}
}{}

% ----- Consignes exo 4 ----- %
\begin{td-exo}[] % 4
	Montrez que les MAC suivants ne sont pas sûrs même s'ils sont utilisés
    pour authentifier des messages de longueur fixe. Dans chaque cas, \(F\) 
    est une PRF, la clé \(k \in {\{0,1\}}^n\) est choisie au hasard et
    \(\langle i \rangle\) désigne l'entier \(i\) encodé sous forme de 
    chaîne de \(\frac n2\) bits.
    \begin{enumerate}
        \item L'étiquette de \(m = m_1,\ldots,m_l\) avec \(m_i\in {\{0,1\}}^{n}\)
        est \(t \coloneqq F_k(m_1)\oplus \cdots \oplus F_k(m_l)\).

        \item L'étiquette de \(m = m_1,\ldots,m_l\) avec \(m_i\in {\{0,1\}}^{\frac n2}\)
        est \(t \coloneqq F_k(\langle 1 \rangle || m_1) \oplus \cdots \oplus F_k(\langle l \rangle || m_l)\).

        \item L'étiquette de \(m = m_1,\ldots,m_l\) avec \(m_i\in {\{0,1\}}^{\frac n2}\)
        est \(r, t\) où \(r\) est choisi uniformément dans \({\{0,1\}}^{n}\) et
        \begin{equation*}
            t \coloneqq F_k(r)\oplus F_k(\langle 1 \rangle || m_1) \oplus \cdots \oplus F_k(\langle l \rangle || m_l).
        \end{equation*}
    \end{enumerate}
\end{td-exo}

% ----- Solutions exo 4 ----- %
\iftoggle{showsolutions}{
	\begin{td-sol}[]\ %
        \begin{enumerate}
            \item Pour \(m = m_1,\ldots,m_l\) on a
            \begin{equation*}
                \mathsf{Mac}_k(m) = F_k(m_1)\oplus \cdots \oplus F_k(m_l).
            \end{equation*}
            où \(| \mathsf{Mac}_k(m) | = n \neq |m|\).
            L'adversaire peut utiliser la stratégie suivante:
            \begin{enumerate}
                \item Il donne \(m_1 || m_2\) à l'oracle et reçoit \(t=F_k(m_1)\oplus F_k(m_2)\).
                avec \(m_1\neq m_2\).
                \item Il produit \(((m_2 || m_1), t)\) 
            \end{enumerate}
            Ainsi,
            \begin{equation*}
                \bb P(\mathsf{Vrfy}_k(m_2 || m_1, t) = 1) = 1.
            \end{equation*}

            \item Pour \(m = m_1,\ldots,m_l\) on a
            \begin{equation*}
                \mathsf{Mac}_k(m) = \bigoplus_{i=1}^l F_k(\langle i \rangle || m_i).
            \end{equation*}
            où \(| \mathsf{Mac}_k(m) | = n \neq |m|\).
            L'adversaire peut utiliser la stratégie suivante dans le cas \(l=2\):
            \begin{enumerate}
                \item Il donne \(m_1 || m_2\) à l'oracle et reçoit \(t_1=F_k(\langle 1 \rangle || m_1)\oplus F_k(\langle 2 \rangle || m_2)\).
                \item Il donne \(m_3 || m_4\) à l'oracle et reçoit \(t_2=F_k(\langle 1 \rangle || m_3)\oplus F_k(\langle 2 \rangle || m_4)\).
                \item Enfin, il donne \(m_1 || m_4\) à l'oracle et reçoit \(t_3=F_k(\langle 1 \rangle || m_1)\oplus F_k(\langle 2 \rangle || m_4)\).
                \item Alors, il obtient
                \begin{equation*}
                    t_1 \oplus t_2 \oplus t_3 = F_k(\langle 1 \rangle || m_3) \oplus F_k(\langle 2 \rangle || m_2).
                \end{equation*}
                donc il peut identifier \(m_3 || m_2\) avec cette étiquette.
            \end{enumerate}
            Donc 
            \begin{equation*}
                \bb P(\mathsf{Vrfy}_k(m_3 || m_2, t_1 \oplus t_2 \oplus t_3) = 1) = 1.
            \end{equation*}
            
        \end{enumerate}
    \end{td-sol}
}{}

% ----- Consignes exo 5 ----- %
\begin{td-exo}[] % 5
	Prouvez que les modifications suivantes du CBC-MAC ne permettent 
    pas d'obtenir un MAC sûr (même pour des messages de longueur fixe):
    \begin{enumerate}
        \item \(\mathsf{Mac}\) produit tous les blocs \(t_1,\ldots,t_l\) 
        au lieu de seulement \(t_l\). La vérification ne controle
        que \(t_l\).

        \item Au lieu de \(t_0 = 0^n\), un bloc initial aléatoire
        \(t_0\) est utilisé à chaque fois qu'un message est authentifié.
        Ainsi, pour le message \(t_0,m_1,\ldots,m_l\), affichez
        \((t_0, t_l)\) comme étiquette. La vérification est canonique.
    \end{enumerate}
\end{td-exo}

% ----- Solutions exo 5 ----- %
\iftoggle{showsolutions}{
	\begin{td-sol}[]\ %
        % TODO finir le 5
    \end{td-sol}
}{}

% % ----- Consignes exo 1 ----- %
% \begin{td-exo}[] % 1
% 	% TODO add content
% \end{td-exo}

% % ----- Solutions exo 1 ----- %
% \iftoggle{showsolutions}{
% 	\begin{td-sol}[]\ %
%         % TODO finir le 1
%     \end{td-sol}
% }{}

