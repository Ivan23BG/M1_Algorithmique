% NeuroSync Pitch - Clean Beamer presentation (slides only)
% Filename: NeuroSync_Pitch_Clean.tex
% Compile with: pdflatex (or use Overleaf)
%

\documentclass[10pt]{beamer}
\usetheme{Madrid}
\usecolortheme{orchid}
\usepackage[T1]{fontenc}
\usepackage[utf8]{inputenc}
\usepackage{graphicx}
\usepackage{booktabs}
\usepackage{amsmath}
\usepackage{multicol}
\usepackage{xcolor}
\usepackage{tikz}
\setbeamertemplate{navigation symbols}{} % remove nav symbols
\setbeamertemplate{caption}[numbered]

% Metadata
\title{NeuroSync}
\subtitle{Think faster. Work smarter.}
\author{Ivan Lejeune}
\date{October 7, 2025}

\begin{document}

% Title slide
\begin{frame}
    \titlepage{}
\end{frame}

% Slide 1: Hook
\begin{frame}{Imagine your computer could tell when you are tired}
    % \begin{center}
    %     \includegraphics[width=0.85\textwidth]{focus_intro.png}
    % \end{center}
    \vspace{0.3cm}
    \begin{itemize}
        \item NeuroSync --- a wearable AI assistant that helps manage focus and recovery.
    \end{itemize}
\end{frame}

% Slide 2: The problem
\begin{frame}{The problem}
    \begin{itemize}
        \item Rising distraction and burnout in tech and academic settings.
        \item Reduced uninterrupted focus time and lower productivity.
    \end{itemize}
    \vfill
    % \begin{center}
    %     \includegraphics[width=0.7\textwidth]{focus_graph.png}
    % \end{center}
\end{frame}

% Slide 3: Our solution
\begin{frame}{Our solution}
    \begin{columns}
        \column{0.55\textwidth}
        \begin{itemize}
            \item Comfortable wearable that monitors cognitive load and stress markers.
            \item On-device machine learning recommends work and rest cycles.
            \item Simple mobile dashboard with a productivity timeline.
        \end{itemize}
        \column{0.45\textwidth}
        % \begin{center}
        %     \includegraphics[width=0.9\textwidth]{neurosync_mockup.png}
        % \end{center}
    \end{columns}
\end{frame}

% Slide 4: How it works
\begin{frame}{How it works}
    % \begin{center}
    %     \includegraphics[width=0.9\textwidth]{how_it_works.png}
    % \end{center}
    \vspace{0.2cm}
    \begin{itemize}
        \item Sensors 
        \item Local ML
        \item Automated environment adjustments.
    \end{itemize}
\end{frame}

% Slide 5: Market and impact
\begin{frame}{Market and impact}
    \begin{itemize}
        \item Primary users: developers, masters students, remote teams.
        \item Secondary users: knowledge workers and creative professionals.
        \item Positioning: privacy-first, student-friendly pricing.
    \end{itemize}
    \vfill
    % \begin{center}
    %     \includegraphics[width=0.65\textwidth]{market_chart.png}
    % \end{center}
\end{frame}

% Slide 6: Traction and next steps
\begin{frame}{Traction and next steps}
    \begin{itemize}
        \item Pilot study planned at two universities.
        \item Early prototype: 12 testers, measurable focus improvements.
        \item Next: hardware iteration, model refinement, MVP release.
    \end{itemize}
    \vfill
    % \begin{center}
    %     \includegraphics[width=0.75\textwidth]{roadmap_timeline.png}
    % \end{center}
\end{frame}

% Slide 7: Business model
\begin{frame}{Business model}
    \begin{columns}
        \column{0.5\textwidth}
        \begin{itemize}
            \item \(1000\) devices sold in year 1 at \(€150\) each.
            \item Subscription for advanced analytics and coaching: \(€5\)/month.
            \item Partnerships with universities and tech firms.
            \item Integrations with calendars and development tools.
        \end{itemize}
        \column{0.5\textwidth}
        % \begin{center}
        %     \includegraphics[width=0.9\textwidth]{business_model.png}
        % \end{center}
    \end{columns}
\end{frame}

% Slide 8: Mini use case
\begin{frame}{Mini use case --- Alice the developer}
    % \begin{center}
    %     \includegraphics[width=0.8\textwidth]{alice_usecase.png}
    % \end{center}
    \vspace{0.2cm}
    \begin{itemize}
        \item Suggested 50-minute focus block, micro-break for breathing, notification dimming.
        \item Outcome: completed tasks with lower stress.
    \end{itemize}
\end{frame}

% Slide 9: Thank you
\begin{frame}{Thank you}
    % \begin{center}
    %     \includegraphics[width=0.9\textwidth]{closing_background.png}
        \vspace{0.4cm}
        \centering
        \Large{Questions?}
    % \end{center}
\end{frame}

\end{document}
