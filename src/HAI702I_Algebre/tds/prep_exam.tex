% ----- Consignes exo 3 ----- %
\begin{td-exo}[] \,\\% 3
	On se place dans \(\bb R^3\). L'objectif est de trouver une matrice 
    de rotation qui transforme le plan \((P)\) d'équation \(x + 2y + 2z = 0\) en
    le plan horizontal \((H)\) (d'équation \(z = 0\)).
    \begin{enumerate}
        \item Donnez un vecteur \(n\) normal au plan \((P)\), de norme 1.
        \item Trouver un vecteur \(u\) de coordonnées \((a,b,0)\),
        de norme 1 et orthogonal à \(n\).
        \item Complétez \((n,u)\) en une base orthonormale directe 
        \((n,u,v)\) de \(\bb R^3\).
        \item Soit \(M\) la matrice dont les colonnes sont \((u,v,n)\). 
        Justifiez le fait que \((u,v,n)\) est également une base orthonormale directe.
        \item On rappelle qu'une matrice de rotation \(Q\) est une matrice 
        orthogonale de déterminant 1 (donc telle que \(Q^T Q = Q Q^T = I\)
        et \(\det(Q) = 1\)). Montrez que \(M\) transforme le vecteur \((0,0,1)\)
        en \(n\). En déduire une matrice de rotation transformant \((P)\) en \((H)\).
    \end{enumerate}
\end{td-exo}

% ----- Solutions exo 3 ----- %
\iftoggle{showsolutions}{
	\begin{td-sol}[]\ % 3
		\begin{enumerate}
            \item Un vecteur normal au plan \((P)\) est un vecteur \(k = (a,b,c)\) 
            tel que 
            \begin{equation*}
                a + 2b + 2c = 0.
            \end{equation*}
            Ici, on prend \(k = (1, 2, 2)\). Normalisons-le:
            \begin{equation*}
                \|k\| = \sqrt{1^2 + 2^2 + 2^2} = 3,
            \end{equation*}
            donc le vecteur normal unitaire est
            \begin{equation*}
                n = \frac{1}{3}(1, 2, 2).
            \end{equation*}

            \item On cherche un vecteur \(k = (a,b,0)\) orthogonal à \(n\), soit:
            \begin{equation*}
                k \cdot n = \frac{1}{3}(a + 2b) = 0 \implies a + 2b = 0.
            \end{equation*}
            Choix simple: \(a = 2, b = -1\), donc \(k = (2,-1,0)\).  
            Normalisons-le:
            \begin{equation*}
                \|k\| = \sqrt{2^2 + {(-1)}^2} = \sqrt{5}, \quad
                u = \frac{1}{\sqrt{5}}(2,-1,0).
            \end{equation*}

            \item Complétons en une base orthonormale directe avec le produit vectoriel:
            \begin{equation*}
                \begin{aligned}
                    v & = n \times u \\
                    & = \frac{1}{3\sqrt{5}}
                    \begin{vmatrix}
                        e_1 & e_2 & e_3 \\
                        1 & 2 & 2 \\
                        2 & -1 & 0
                    \end{vmatrix} \\
                    & = \frac{1}{3\sqrt{5}} \big( (2 \cdot 0 - 2 \cdot (-1)) e_1
                    - (1 \cdot 0 - 2 \cdot 2) e_2
                    + (1 \cdot (-1) - 2 \cdot 2) e_3 \big) \\
                    & = \frac{1}{3\sqrt{5}} (2, 4, -5). 
                \end{aligned}
            \end{equation*}

            \item La matrice \(M\) dont les colonnes sont \((u,v,n)\) est
            \begin{equation*}
                M = 
                \begin{pmatrix}
                    2/\sqrt{5} & 2/(3\sqrt{5}) & 1/3 \\
                    -1/\sqrt{5} & 4/(3\sqrt{5}) & 2/3 \\
                    0 & -5/(3\sqrt{5}) & 2/3
                \end{pmatrix}.
            \end{equation*}
            Tous les vecteurs sont unitaires et orthogonaux, et \(u \times v = n\) assure la directivité.

            \item Une matrice de rotation \(Q\) doit vérifier \(Q^T Q = I\) et \(\det(Q)=1\).  
            Ici, \(M\) envoie \((0,0,1)\) sur \(n\), donc elle envoie \(H\) sur \(P\).  
            Pour envoyer \(P\) sur \(H\), il faut prendre l’inverse, soit \(Q = M^T\):
            \begin{equation*}
                Q = M^T = 
                \begin{pmatrix}
                    2/\sqrt{5} & -1/\sqrt{5} & 0 \\
                    2/(3\sqrt{5}) & 4/(3\sqrt{5}) & -5/(3\sqrt{5}) \\
                    1/3 & 2/3 & 2/3
                \end{pmatrix}.
            \end{equation*}
            Cette matrice est bien orthogonale, de déterminant 1, et envoie le plan \((P)\) sur le plan horizontal \((H)\).
        \end{enumerate}
	\end{td-sol}
}{}
