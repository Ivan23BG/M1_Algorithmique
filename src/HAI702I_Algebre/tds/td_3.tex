Les exercices ou les questions marqués d'une étoile ne sont pas prioritaires.
% ----- Consignes exo 1 ----- %
\begin{td-exo}[Vrai ou Faux] \,\\% 1
	\begin{enumerate}
		\item En dimension finie, un endomorphisme admet un nombre fini
		de vecteurs propres.

		\item Si \(A\) est diagonalisable, alors \(A^2\) l'est aussi.

		\item Si \(A^2\) est diagonalisable, alors \(A\) l'est aussi.

		\item Tout endomorphisme d'un espace vectoriel réel de 
		dimension impaire admet au moins une valeur propre.

		\item La somme de deux matrices diagonalisables est diagonalisable.
	\end{enumerate}
\end{td-exo}

% ----- Solutions exo 1 ----- %
\iftoggle{showsolutions}{
	\begin{td-sol}[]\ %
		\begin{enumerate}
			\item Faux, on peut multiplier un vecteur propre par un scalaire.

			\item Vrai, si \(A = PDP^{-1}\) avec \(D\) diagonale, alors
			\(A^2 = PD^2P^{-1}\) est aussi diagonale.

			\item Faux, la matrice 
			\begin{equation*}
				A = \begin{pmatrix}
					0 & 1 \\
					0 & 0
				\end{pmatrix}
			\end{equation*}
			n'est pas diagonalisable mais \(A^2 = 0\) l'est. Montrons-le:
			\begin{equation*}
				\begin{aligned}
					\mathsf{Det}(A - \lambda I_2) = \lambda^2
				\end{aligned}
			\end{equation*}
			La seule valeur propre est donc \(0\) et le sous-espace
			associé est de dimension \(1\leq 2\) car \(E_0(e_2)=0\).
		\end{enumerate}
	\end{td-sol}
}{}
