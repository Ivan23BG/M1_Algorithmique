% ----- Consignes exo 30 ----- %
\begin{td-exo}[Propriétés liées à la mesure au pire cas]\, % 30 
	\begin{enumerate}
		\item Soit \(\Pi\) un problème de maximisation appartenant à la classe \(\mathcal{NPO}\) ayant un ratio \(\delta\) pour la mesure.
		Montrer que cela implique que le ratio dans le pire cas est de \(\delta\).

		\item Sur le problème du \textsc{TSP}, considérons une instance \(I = (K_n, \overset{\rightarrow}d)\) avec \(\overset{\rightarrow}d\) le vecteur des arêtes-distances.
		Alors, pour n'importe quelle transformation affine
		\begin{equation*}
			\overset{\rightarrow}d \coloneqq \gamma \cdot \overset{\rightarrow}d + \mu \cdot \overset{\rightarrow}1
		\end{equation*}
		avec \(\gamma,\mu \in \bb Q\) produit une approximation différentielle équivalente à celle obtenue pour \(I\).
	\end{enumerate}
\end{td-exo}

% ----- Solutions exo 30 ----- %
\iftoggle{showsolutions}{ 
	\begin{td-sol}[]\ % 30
		
	\end{td-sol}
}{}
