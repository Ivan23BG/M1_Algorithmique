% ----- Consignes exo 2 ----- %
\begin{td-exo}[Autour de \textsc{Satisfaisabilité}]\, % 2 
	\vspace{-6mm}
\begin{algorithm}[H]
    \caption{\textsc{Non Egal Satisfaisabilité} (NAESAT)}
    \begin{algorithmic}
        \Require{Une formule conjonctive \(\varphi\) sur \(n\) variables et \(m\) clauses}
        \Ensure{Existe-t-il une affectation de valeurs de vérité aux variables qui satisfasse \(\varphi\) tel que chaque clause a un littéral vrai et un littéral faux?}
    \end{algorithmic}
\end{algorithm}
	
	Montrer que \textsc{Non Egal Satisfaisabilité} est \(\mathcal{NP}\)-\textit{complet}. 
	La preuve se fera à partir de \textsc{Satisfaisabilité}.
\end{td-exo}

% ----- Solutions exo 2 ----- %
\iftoggle{showsolutions}{ 
	\begin{td-sol}[]\ % 2
		Le problème de \textsc{Non Egal Satisfaisabilité} est dans \(\mathcal{NP}\) car on peut vérifier en temps polynomial si deux formules booléennes sont non équivalentes. 
		Pour montrer qu'il est \(\mathcal{NP}\)-complet, on effectue une réduction à partir du problème de \textsc{Satisfaisabilité}. 
		On montre que si on pouvait résoudre \textsc{Non Egal Satisfaisabilité} en temps polynomial, alors on pourrait résoudre \textsc{Satisfaisabilité} en temps polynomial, ce qui contredit l'hypothèse que \(\mathcal{P} \neq \mathcal{NP}\). 
		Par conséquent, \textsc{Non Egal Satisfaisabilité} est \(\mathcal{NP}\)-complet.

		Soit \(\varphi\) une formule conjonctive de \(3\)-\textsc{SAT}. 
		On construit une formule \(\psi\) pour le problème de \textsc{Non Egal Satisfaisabilité} de la manière suivante:
		\begin{equation*}
			\psi_i = \varphi_i \land (x_1 \lor \ldots \lor x_n) \land (\lnot x_1 \lor \ldots \lor \lnot x_n), \quad \forall i \in \{1, \ldots, m\}
		\end{equation*}

		Ensuite, on pose \(\psi = \bigwedge_{i=1}^m \psi_i\).
		
		Si \(\varphi\) est satisfaisable, alors il existe une affectation de valeurs de vérité aux variables qui satisfait \(\varphi\). 
		Cette même affectation satisfait également \(\psi\) car les clauses supplémentaires garantissent que chaque clause de \(\psi\) a un littéral vrai et un littéral faux.
		
		Inversement, si \(\psi\) est satisfaisable, alors il existe une affectation de valeurs de vérité aux variables qui satisfait \(\psi\). 
		Cette affectation doit également satisfaire \(\varphi\) (car si elle satisfait chaque clause de \(\psi\)), alors elle satisfait également chaque clause de \(\varphi\). 
		
		Par conséquent, \(\varphi\) est satisfaisable si et seulement si \(\psi\) est satisfaisable, ce qui montre que le problème de \textsc{Non Egal Satisfaisabilité} est \(\mathcal{NP}\)-complet (car il est au moins aussi dur que \textsc{Satisfaisabilité}, qui est lui-même \(\mathcal{NP}\)-complet).
	\end{td-sol}
}{}
