% ----- Consignes exo 1 ----- %
\begin{td-exo}[Quelques preuves de NP-complétude autour de 3-SAT]\, % 1 
	\begin{algorithm}[H]
    \caption{\(3\)-SAT}
    \begin{algorithmic}
        \Require{Une formule \(\varphi(x_1,\ldots,x_n)\) en 3-CNF}
        \Ensure{Existe-t-il une affectation des variables \((x_1,\ldots,x_n)\) satisfaisant \(\varphi\)?}
    \end{algorithmic}
\end{algorithm}
	\input{../assets/tikz/td1_ex1_2.tex}
	\vspace{-6mm}
\begin{algorithm}[H]
    \caption{\(\lnot3\)-SAT}
    \begin{algorithmic}
        \Require{Une formule \(\varphi(x_1,\ldots,x_n)\) en 3-CNF}
        \Ensure{Existe-t-il une affectation des variables \((x_1,\ldots,x_n)\) qui ne satisfait pas \(\varphi\)?}
    \end{algorithmic}
\end{algorithm}

	\begin{enumerate}
		\item Montrer que \(\lnot3\)-SAT appartient à la classe \(P\).
		\item Montrer que \textsc{Half} \(3\)-SAT est NP-complet. 
		Que dire du problème \textsc{Half} \(3\)-SAT dans le cas où le nombre de clauses satisfaites est au moins de \(50\%\)?
	\end{enumerate}
\end{td-exo}

% ----- Solutions exo 1 ----- %
\iftoggle{showsolutions}{ 
	\begin{td-sol}[]\ % 1
		\begin{enumerate}
			\item On rappelle que tous les litéraux sont positifs, il suffit alors de tout mettre à faux pour trouver une solution, donc le problème appartient à la classe \(P\).
			\item On fait une réduction de \(3\)-SAT vers \textsc{Half} \(3\)-SAT.\@

			On commence par faire la chose suivante:
			\begin{equation*}
				I \rightsquigarrow I + \underbrace{\left( x \lor y \lor z \right)}_{m\text{ copies}},\quad x,y,z\notin I
			\end{equation*}
			Si on a oui à \(3\)-SAT alors on a \(m\) clauses satisfaites. On en déduit que sur \(2m\) clauses, il suffit de mettre \(x,y,z\) à faux et alors 
			on a bien \(50\%\) des clauses qui sont satisfaites par la solution à \(3\)-SAT.\@

			Dans l'autre sens, si on a une solution à \textsc{Half} \(3\)-SAT on a plusieurs cas:
			\begin{itemize}
				\item Premier cas, on satisfait une clause dans \(I\). Alors toutes les \(m\) clauses sont validées dans \(I\) et 
				on a une solution à \(3\)-SAT.\@
				\item Second cas, on ne satisfait aucune clause de \(I\). Mais alors si dans \(I\) il n'y a aucune variable négative,
				on peut résoudre \(3\)-SAT en mettant tout à positif, de même dans le cas où il n'y a aucune variable positive. Il y 
				a donc forcément une variable présente de manière positive et négative dans \(I\). Alors, au moins une clause est satisfaite.
				Donc on est dans le premier cas et forcément les \(m\) clauses satisfaites sont dans \(I\).
			\end{itemize}

			Le problème de au moins \(50\%\) est facile. Il suffit de prendre une affectation aléatoire (ou son complémentaire).
		\end{enumerate}
	\end{td-sol}
}{}
