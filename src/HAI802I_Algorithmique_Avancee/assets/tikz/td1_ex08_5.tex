\ffigbox[\FBwidth]{%
\caption{\centering Application de la construction à l'instance donnée avec un exemple de coloration}\label{fig:td1_ex08_5}
}{
    \fbox{
        \begin{tikzpicture}[scale=1, main node/.style={circle, draw, fill=blue!20, inner sep=1pt, font=\scriptsize, minimum size=5mm, text=black}]
            % les sommets x_i
            \node[main node] (x1) at (-1,0) {\(x_1\)};
            \node[main node, fill=green!20] (bx1) at (-1,-0.8) {\(\ol{x_1}\)};

            \node[main node] (x2) at (-1,-1.8) {\(x_2\)};
            \node[main node, fill=green!20] (bx2) at (-1,-2.6) {\(\ol{x_2}\)};

            \node[main node] (x3) at (-1,-3.6) {\(x_3\)};
            \node[main node, fill=green!20] (bx3) at (-1,-4.4) {\(\ol{x_3}\)};

            \node[main node] (x4) at (-1,-5.4) {\(x_4\)};
            \node[main node, fill=green!20] (bx4) at (-1,-6.2) {\(\ol{x_4}\)};

            \node[main node] (x5) at (-1,-7.2) {\(x_5\)};
            \node[main node, fill=green!20] (bx5) at (-1,-8) {\(\ol{x_5}\)};
            

            % construction des triangles y_i
            \node[main node, fill=green!20] (y11) at (1, 0) {\(y_{11}\)};
            \node[main node] (y12) at (1, -1) {\(y_{12}\)};
            \node[main node, fill=red!20] (y13) at (1, -2) {\(y_{13}\)};
            \node[main node, fill=red!20] (y14) at (1.5, -0.5) {\(y_{14}\)};
            \node[main node, fill=green!20] (y15) at (1.5, -1.5) {\(y_{15}\)};
            \node[main node] (y16) at (2, -2) {\(y_{16}\)};
            % et encore 2 triangles similaires pour les autres clauses en dessous des premiers
            \node[main node] (y21) at (1, -3) {\(y_{21}\)};
            \node[main node, fill=green!20] (y22) at (1, -4) {\(y_{22}\)};
            \node[main node, fill=red!20] (y23) at (1, -5) {\(y_{23}\)};
            \node[main node, fill=red!20] (y24) at (1.5, -3.5) {\(y_{24}\)};
            \node[main node, fill=green!20] (y25) at (1.5, -4.5) {\(y_{25}\)};
            \node[main node] (y26) at (2, -5) {\(y_{26}\)};
            % rajout des deux v_i
            \node[main node, fill=red!20] (v1) at (3, 2) {\(v_1\)};
            \node[main node, fill=green!20] (v2) at (4, 2) {\(v_2\)};
            

            % les aretes
            % on connecte les x_i aux bx_i
            \foreach \i in {1,2,3,4,5} {
                \draw (x\i) to (bx\i);
            }

            % on connecte les y_i entre eux
            \foreach \i in {1,2} {
                \draw (y\i1) to (y\i2);
                \draw (y\i1) to (y\i4);
                \draw (y\i2) to (y\i4);
                \draw (y\i3) to (y\i5);
                \draw (y\i3) to (y\i6);
                \draw (y\i4) to (y\i5);
                \draw (y\i5) to (y\i6);
            }

            % on connecte les v_i entre eux
            \draw (v1) to (v2);

            % left bus vertical pour collecter les xi
            \coordinate (lbt) at (-2,2);

            \foreach \i in {1,2,3,4,5} {
                \draw (x\i)  -- ($(x\i)+(-0.5,0.5)$)  -| (lbt);
                \draw (bx\i) -- ($(bx\i)+(-0.5,0.5)$) -| (lbt);
            }

            \draw (lbt) -- (v1);

            % right bus vertical pour collecter les y6
            \coordinate (rbt) at (3.5,0.5);

            \foreach \i in {1,2} {
                \draw
                    (y\i6) -- ($(y\i6)+(0.5,0.5)$) -| (rbt);
            }

            \draw (v1) -- (rbt);
            \draw (v2) -- (rbt);

            % connect the y_i to the x_i according to the clauses
            \draw (x1) to (y11);
            \draw (bx3) to (y12);
            \draw (bx4) to (y13);

            \draw (bx4) to (y21);
            \draw (x2) to (y22);
            \draw (bx1) to (y23);
            
        \end{tikzpicture}
    }
}