% ============================================================================== 
% GENERAL MATH SHORTCUTS
% ==============================================================================

\newcommand{\abs}[1]{\left\lvert#1\right\rvert}  % Absolute value with scalable bars
\counterwithout{tdcounter}{section} 
\usetikzlibrary{shapes,arrows,positioning,calc,fit,backgrounds,angles,quotes}
% \newcommand{\becomes}{\begin{center}\(\downarrow\)\end{center}}


% ============================================================================== 
% JULIA CODE STUFF
% ==============================================================================


\usepackage{floatrow}
\usepackage{caption}
\usepackage{cases}


\usepackage{listings}
\usepackage{xcolor}

% ----- Julia language definition -----
\lstdefinelanguage{julia}{
    keywords={function, end, if, else, elseif, while, for, in, return, break, continue, struct, 
    mutable, using, import, module, export, const, let, global, local, abstract, typealias, 
    sin, atan, true, add},
    sensitive=true,
    comment=[l]{\#},
    morestring=[b]", % chktex 18
}
\definecolor{codebg}{RGB}{245,245,245}   % very light gray
\definecolor{codeborder}{RGB}{220,220,220} 
\definecolor{keywordcolor}{RGB}{0,0,150}
\definecolor{commentcolor}{RGB}{120,120,120}
\lstdefinestyle{julia-style}{
    language=julia, % new language
    basicstyle=\ttfamily\small,
    keywordstyle=\color{keywordcolor},
    commentstyle=\color{commentcolor},
    stringstyle=\color{verdant},
    backgroundcolor=\color{codebg},
    frame=single,
    framerule=0.5pt,
    rulecolor=\color{codeborder},
    tabsize=2,
    columns=fullflexible,
    keepspaces=true,
    showstringspaces=false,
}
% convenience macro
\newcommand{\juliaFile}[1]{\lstinputlisting[style=julia-style]{#1}}

% custom bash environment with same style as julia
\lstdefinelanguage{bash}{
    keywords={sudo, apt-get, install, update, upgrade, cd, ls, mkdir, rm, rmdir, touch, nano, vim, cat, echo, pwd, cp, mv},
    sensitive=true,
    comment=[l]{\#},
    morestring=[b]", % chktex 18
}
\lstdefinestyle{bash-style}{
    language=bash,
    basicstyle=\ttfamily\small,
    keywordstyle=\color{keywordcolor},
    commentstyle=\color{commentcolor},
    backgroundcolor=\color{codebg},
    frame=single,
    framerule=0.5pt,
    rulecolor=\color{codeborder},
    tabsize=2,
    columns=fullflexible,
    keepspaces=true,
    showstringspaces=false,
}
% convenience macro
\newcommand{\bashFile}[1]{\lstinputlisting[style=bash-style]{#1}}

% ============================================================================== 
% ALGORITHM STUFF
% ==============================================================================
\usepackage[noend]{algpseudocode}
\DeclareCaptionType{algorithm}[Algorithme][Liste des Algorithmes]
\captionsetup[algorithm]{justification=raggedright,singlelinecheck=false}

\newcommand*\Let[2]{\State #1 $\gets$ #2} % chktex 1
\algrenewcommand\alglinenumber[1]{
    {\sf\footnotesize\addfontfeatures{Colour=888888,Numbers=Monospaced}#1}}
\algrenewcommand\algorithmicrequire{\textbf{Input:}}
\algrenewcommand\algorithmicensure{\textbf{Question:}}
