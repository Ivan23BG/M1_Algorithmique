\makeatletter

% ==============================================================================
% TABLE OF CONTENTS FORMATTING
% ==============================================================================

% Adjusts indentation and spacing for different section levels in the TOC.
% \dottedcontents{suprasection}[3.5em]{}{2.3em}{1pc}    % Suprasection: extra indent for visual hierarchy
% \dottedcontents{section}[5.8em]{}{2.8em}{1pc}         % Section: increased indent for clarity
% \dottedcontents{subsection}[9.0em]{}{3.2em}{1pc}      % Subsection: further indent for structure
% \dottedcontents{subsubsection}[12.0em]{}{3.8em}{1pc}  % Subsubsection: deepest indent for detail

% ============================================================================== 
% GENERAL MATH SHORTCUTS
% ==============================================================================

\newcommand{\abs}[1]{\left\lvert#1\right\rvert}  % Absolute value with scalable bars
\newcommand{\Tau}{\mathcal{T}}                   % Calligraphic Tau, often used for topology

% ============================================================================== 
% SUBJECT-SPECIFIC THEOREM ENVIRONMENTS
% ==============================================================================

% --- Numbered theorem environments (per subsection) ---
% Uses colored left border and custom counter for clarity in notes.
\theoremstyle{default}
\mytheorem{definition}{Définition}{algebraic-amber}{subsection}{o}  % Definitions: amber, numbered per subsection
\mytheorem{exs}{Exemples}{matrix-mist}{subsection}{o}               % Examples: mist, numbered per subsection

% --- Unnumbered theorem environments ---
% Used for remarks, interpretations, and other statements that don't require numbering.
\theoremstyle{nonum}
\mytheorem{interp}{Interprétation}{verdant}{}{}    % Interpretation: green, unnumbered
\mytheorem{proposition}{Proposition}{verdant}{}{}  % Proposition: green, unnumbered
\mytheorem{notation}{Notation}{verdant}{}{}        % Notation: green, unnumbered
\mytheorem{example}{Exemple}{matrix-mist}{}{}      % Example: mist, unnumbered
\mytheorem{exo}{Exercice}{matrix-mist}{}{}         % Exercise: mist, unnumbered
\mytheorem{exercice}{Exercice}{matrix-mist}{}{}    % Exercise (alt): mist, unnumbered
\mytheorem{theorem}{Théorème}{astral}{}{}          % Theorem: blue, unnumbered
\mytheorem{lemma}{Lemme}{astral}{}{}               % Lemma: blue, unnumbered
\mytheorem{remark}{Remarque}{verdant}{}{}          % Remark: green, unnumbered

\makeatother

\usepackage{minted}