% ----- Consignes exo 1 ----- %
\begin{td-exo}[Manipuler les définitions]\ % 1
	\begin{enumerate}
		\item Rappelez ce qu'est une fonction calculable.
		Montrez que les fonctions suivantes sont calculables:
		\begin{itemize}
			\item \(f\ \colon\ x\mapsto 2x\),
			\item \(g\ \colon\ \langle x,y\rangle\mapsto x+y\),
			\item \(\upmodels\), la fonction définie nulle part.
		\end{itemize}

		\item Rappelez ce qu'est un ensemble récursif.
		Montrez que les ensembles suivants sont récursifs:
		\begin{itemize}
			\item l'ensemble de tous les entiers,
			\item l'ensemble vide,
			\item l'ensemble des nombres pairs.
		\end{itemize}

		\item Montrez que si \(E\) est un ensemble récursif, alors son complémentaire \(\overline{E}\) l'est aussi.

		\item Rappelez ce qu'est un ensemble énumérable. Ecrivez un programme qui converge 
		si et seulement si son entrée est paire. En déduire que l'ensemble des nombres pairs est énumérable.

		\item Montrez que si \(E\) est un ensemble récursif, alors \(E\) et \(\overline{E}\) sont énumérables.
	\end{enumerate}
\end{td-exo}

% ----- Solutions exo 1 ----- %
\iftoggle{showsolutions}{
	\begin{td-sol}[]\ %
		\begin{enumerate}
			\item Une fonction (partielle) est calculable (ou récursive) si il existe un programme qui la calcule.
			\begin{itemize}
				\item \(f\) est calculable car il peut être défini par le programme suivant:
				\begin{verbatim}
					x |-> return 2 * x 
				\end{verbatim}
				\item \(g\) est calculable car il peut être défini par le programme suivant:
				\begin{verbatim}
					z |-> return pi1(z) + pi2(z)
				\end{verbatim}
				\item \(\upmodels\) est calculable car il peut être défini par le programme suivant:
				\begin{verbatim}
					x |-> while true do {}, return 0
				\end{verbatim}
			\end{itemize}
			\item Un ensemble \(A\subseteq \N\) est récursif (ou décidable) si sa fonction caractéristique \(\chi_A\) est calculable.
			\begin{itemize}
				\item L'ensemble de tous les entiers est récursif car sa fonction caractéristique peut être définie par le programme suivant:
				\begin{verbatim}
					x |-> return 1
				\end{verbatim}
				\item L'ensemble vide est récursif car sa fonction caractéristique peut être définie par le programme suivant:
				\begin{verbatim}
					x |-> return 0
				\end{verbatim}
				\item L'ensemble des nombres pairs est récursif car sa fonction caractéristique peut être définie par le programme suivant:
				\begin{verbatim}
					x |-> if x mod 2 == 0 then return 1 else return 0
				\end{verbatim}
			\end{itemize}

			\item Il suffit d'inverser les valeurs de la fonction caractéristique.

			\item On utilise la fonction
			\begin{verbatim}
				x |-> if x mod 2 == 0 then return 0 else \upmodels(x)
			\end{verbatim}
			qui converge si et seulement si son entrée est paire.

			\item Si \(E\) est récursif, alors sa fonction caractéristique \(\chi_E\) est calculable.
			Donc on peut définir deux programmes:
			\begin{verbatim}
				x |-> if chi_E(x) == 1 then return 0 else \upmodels(x)
			\end{verbatim}
			et
			\begin{verbatim}
				x |-> if chi_E(x) == 0 then return 0 else \upmodels(x)
			\end{verbatim}
			qui convergent respectivement si et seulement si \(x\in E\) et \(x\in \overline{E}\).
		\end{enumerate}
	\end{td-sol}
}{}


% ----- Consignes exo 2 ----- %
\begin{td-exo}[Propriétés de cloture des ensembles récursifs] % 2
	Nous prouvons dans cet exercice que la classe des ensembles récursifs est close
	par union, intersection, produit cartésien et complémentaire.
	La situation est un peu différente pour les ensembles énumérables.
	Soient \(A\) et \(B\) deux ensembles récursifs.
	\begin{enumerate}
		\item Montrer que \(A\cup B\) est récursif.
		\item Montrer que \(A\cap B\) est récursif.
		\item Montrer que \(A\times B = \left\{\langle x,y\rangle, x\in A \wedge y\in B\right\}\) est récursif.
		\item Montrer que \(\ol A\) est récursif (c.f.\ exercice 1).
		\item Montrer que \(A \setminus B\) est récursif.
	\end{enumerate}
\end{td-exo}

% ----- Solutions exo 2 ----- %
\iftoggle{showsolutions}{
	\begin{td-sol}[]\ %
		\begin{enumerate}
			\item On utilise la fonction caractéristique:
			\begin{verbatim}
				x |-> if chi_A(x) == 1 or chi_B(x) == 1 then return 1 else return 0
			\end{verbatim}
			\item On utilise la fonction caractéristique:
			\begin{verbatim}
				x |-> if chi_A(x) == 1 and chi_B(x) == 1 then return 1 else return 0
			\end{verbatim}
			\item On utilise la fonction caractéristique:
			\begin{verbatim}
				z |-> if chi_A(pi1(z)) == 1 and chi_B(pi2(z)) == 1 then return 1 else return 0
			\end{verbatim}
			\item Voir exercice 1.
			\item On utilise la fonction caractéristique:
			\begin{verbatim}
				x |-> if chi_A(x) == 1 and chi_B(x) == 0 then return 1 else return 0
			\end{verbatim}
		\end{enumerate}
	\end{td-sol}
}{}


% ----- Consignes exo 3 ----- %
\begin{td-exo}[\emph{step}, convergence, projections et énumération] % 3
	Considérons le programme suivant (remarquez qu'il s'appelle \(a\)
	et son entrée s'appelle \(y\)):
	\begin{verbatim}
		[a|y]:
			z <- 0
			while step <y,pi1(z),pi2(z)> == 0 repeat
				z <- z + 1
			return 17
	\end{verbatim}
	\begin{enumerate}
		\item Soit \(b\) un programme qui s'arrete sur au moins une entrée. Que vaut \([a\mid b]\)?
		\item Que vaut \([a\mid b]\) si \(b\) est un programme qui ne s'arrête jamais?
		\item Montrez que l'ensemble des programmes qui s'arrêtent sur au moins une entrée est énumérable.
	\end{enumerate}
\end{td-exo}

% ----- Solutions exo 3 ----- %
\iftoggle{showsolutions}{
	\begin{td-sol}[]\ %
		\begin{enumerate}
			\item On note \(c\) une entrée telle que \([b\mid c]\downarrow\) (elle existe par hypothèse).
			Alors, il existe \(t\) tel que \(\text{Step}\langle b,c,t\rangle = 1\).
			Donc, pour \(z = \langle c,t\rangle\), on a \([a\mid z]\downarrow\).
			Donc \([a\mid b]\downarrow\) et vaut 17.

			\item Le programme \(a\) ne s'arrête jamais.

			\item Le programme \(a\) correspond exactement au programme dont le domaine est l'ensemble des programmes 
			qui s'arrêtent sur au moins une entrée. Donc cet ensemble est énumérable.
		\end{enumerate}
	\end{td-sol}
}{}


% ----- Consignes exo 4 ----- %
\begin{td-exo}[Ensembles énumérables] % 4
	On veut montrer que les propositions suivantes sont équivalentes:
	\begin{enumerate}[label=(\roman*)]
		\item \(E\) est \defemph{énumérable} (rappel de la définition) si
		\begin{equation*}
			\exists a, E = W_a = \left\{x, [a\mid x]\downarrow\right\}
		\end{equation*}

		\item \(E\) admet une \defemph{fonction d'énumération} calculable:
		\begin{equation*}
			\exists b, E = \text{Im}[b\mid \cdot] = \left\{x, \exists y, [b\mid y] = x\right\}
		\end{equation*}

		\item \(E\) est vide ou admet une \defemph{fonction d'énumération totale} calculable:
		\begin{equation*}
			E = \emptyset \text{ ou } \exists c, E = \text{Im}[c\mid \cdot] = \left\{x, \exists y, [c\mid y] = x\right\}
		\end{equation*}
	\end{enumerate}
	\begin{itemize}
		\item Montrer que (i) \(\Rightarrow\) (iii).
		\item Montrer que (iii) \(\Rightarrow\) (ii).
		\item Montrer que (ii) \(\Rightarrow\) (i).
	\end{itemize}
\end{td-exo}

% ----- Solutions exo 4 ----- %
\iftoggle{showsolutions}{
	\begin{td-sol}[]\ %
		\begin{itemize}
			\item (i) \(\Rightarrow\) (iii):
			Si \(E = \emptyset\), c'est trivial.
			Sinon, \(E\) contient au moins un élément qu'on note \(a_0\). On utilise le programme suivant:
			\begin{verbatim}
				enum: z |->
				if step <a,pi1(z),pi2(z)> == 0 then
					return pi1(z)
				else
					return a0
			\end{verbatim}
		\end{itemize}
	\end{td-sol}
}{}


% ----- Consignes exo 5 ----- %
\begin{td-exo}[Ensembles énumérables --- mieux comprendre] % 5
Exercise 2 content
\end{td-exo}

% ----- Solutions exo 5 ----- %
\iftoggle{showsolutions}{
	\begin{td-sol}[]\ %
		Exercice solution
	\end{td-sol}
}{}