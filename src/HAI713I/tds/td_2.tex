% ----- Consignes exo 1 ----- %
\begin{td-exo}[Ensembles énumérables] % 1
	On veut montrer que les propositions suivantes sont équivalentes:
	\begin{enumerate}[label=(\roman*)] % chktex 36
		\item \(E\) est \defemph{énumérable} (rappel de la définition) si
		\begin{equation*}
			\exists a, E = W_a = \left\{x, [a\mid x]\downarrow\right\}
		\end{equation*}

		\item \(E\) admet une \defemph{fonction d'énumération} calculable:
		\begin{equation*}
			\exists b, E = \text{Im}[b\mid \cdot] = \left\{x, \exists y, [b\mid y] = x\right\}
		\end{equation*}

		\item \(E\) est vide ou admet une \defemph{fonction d'énumération totale} calculable:
		\begin{equation*}
			E = \emptyset \text{ ou } \exists c, E = \text{Im}[c\mid \cdot] = \left\{x, \exists y, [c\mid y] = x\right\}
		\end{equation*}
	\end{enumerate}
	\begin{itemize}
		\item Montrer que (i) \(\Rightarrow\) (iii).
		\item Montrer que (iii) \(\Rightarrow\) (ii).
		\item Montrer que (ii) \(\Rightarrow\) (i).
	\end{itemize}
\end{td-exo}

% ----- Solutions exo 1 ----- %
\iftoggle{showsolutions}{
	\begin{td-sol}[]\ %
		\begin{itemize}
			\item (i) \(\Rightarrow\) (iii):
			Si \(E\) est vide, la propriété est vérifiée.
			Sinon, \(\exists\alpha\in E\). On définit le programme suivant:
			\begin{equation*}
				\begin{aligned}
					b\ :\ z \mapsto \,\,
					& \text{if Step}\langle a,\pi_1(z),\pi_2(z)\rangle \neq 0 \text{ then } \\
					& \qquad \text{return } \pi_1(z) \\
					& \text{else } \\
					& \qquad \text{return } \alpha\\
					& \text{end if}
				\end{aligned}
			\end{equation*}
			Montrons que \(\text{Im}[b\mid \cdot] = E\). Si \(e\in E\)
			alors \(\exists t\) tel que \(\text{Step}\langle a,e,t\rangle \neq 0\).
			Donc \([b\mid \langle e,t\rangle] = e\) pour tout \(e\in E\), d'où \(E= \text{Im}[b\mid \cdot]\).

			\item (iii) \(\Rightarrow\) (ii):
			Si \(E\) est vide, \(b\) qui diverge partout convient.
			Sinon, il existe une fonction d'énumération totale \(c\) de \(E\) calculable
			et donc une fonction d'énumération \(b=c\) de \(E\) calculable.

			\item (ii) \(\Rightarrow\) (i):
			Soit \(b\) une fonction d'énumération de \(E\) calculable.
			On définit le programme suivant:
			\begin{equation*}
				\begin{aligned}
					a\ :\ x \mapsto \,\,
					& z \leftarrow 0 \\
					& \text{while Step}\langle b, \pi_1(z), \pi_2(z)\rangle \neq x+1 \text{ do } \\
					& \qquad z \leftarrow z + 1 \\
					& \text{end while} \\
					& \text{return } 34
				\end{aligned}
			\end{equation*}
			Ce programme s'arrête sur \(x\) si et seulement si \(x\in E\).
			Donc \(E = W_a\).
		\end{itemize}
	\end{td-sol}
}{}


% ----- Consignes exo 2 ----- %
\begin{td-exo}[Ensembles énumérables --- mieux comprendre]\ % 2
	\begin{enumerate}
		\item Montrez que si \(E\) est un ensemble énumérable \emph{infini} alors
		il admet une fonction d'énumération totale bijective.

		\item Soit \(E\) un ensemble infini. Montrez que \(E\) est récursif si et seulement si
		il admet une fonction d'énumération croissante.

		\item Soit \(E\) un ensemble infini. Montrez que \(E\) est récursif si et seulement si
		il admet une fonction d'énumération strictement croissante.
	\end{enumerate}
\end{td-exo}

% ----- Solutions exo 2 ----- %
\iftoggle{showsolutions}{
	\begin{td-sol}[]\ %
		\begin{enumerate}
			\item On crée le programme suivant:
			\begin{equation*}
				\begin{aligned}
					b\ :\ n \mapsto \,\,
					& J \leftarrow \emptyset \\
					& k \leftarrow 0 \\
					& \text{While } \n{J} \leq n \text{ do } \\
					& \qquad J \leftarrow J \cup \left\{[b \mid k]\right\} \\
					& \qquad k \leftarrow k + 1 \\
					& \text{end while} \\
					& \text{Return } [b \mid k]
				\end{aligned}
			\end{equation*}
			Ce programme s'arrête toujours car \(E\) est infini
			et il rend tous les éléments de \(E\) sans répétition.

			\item On écrit le programme suivant pour montrer le sens indirect:
			\begin{equation*}
				\begin{aligned}
					d\ :\ x \mapsto \,\,
					& z \leftarrow 0 \\
					& \text{while } [b \mid z] < x \text{ do } \\
					& \qquad z \leftarrow z + 1 \\
					& \text{end while} \\
					& \text{if } [b \mid z] = x \text{ then return } 1 \text{ else return } 0 \text{ end if}
				\end{aligned}
			\end{equation*}
		\end{enumerate}
	\end{td-sol}
}{}


% ----- Consignes exo 3 ----- %
\begin{td-exo}[Enumeration des fonctions totales] % 3
	Nous allons montrer dans cet exercice qu'il n'est pas possible d'avoir un système de
	programmation \og{} raisonnable \fg{} où les programmes s'arrêtent toujours.
	Supposons que ce système existe et notons \([x\ \mid\ y]'\) le résultat du calcul
	du \(x\)-ième programme sur l'entrée \(y\).

	\begin{enumerate}
		\item On suppose que dans un de nos programmes on peut
		en appeler un autre et que la fonction successeur soit calculable.
		Montrez que \(g:x\mapsto [x|x]'+1\) est calculable dans ce système.
		On notera \(n\) son numéro: \([n\mid \cdot]' = g(\cdot)\).

		\item Que vaut \(g(n)\)? En déduire qu'un tel système n'existe pas.
	\end{enumerate}
\end{td-exo}

% ----- Solutions exo 3 ----- %
\iftoggle{showsolutions}{
	\begin{td-sol}[]\ %
		
	\end{td-sol}
}{}


% ----- Consignes exo 4 ----- %
\begin{td-exo}[Ensembles énumérables --- clôture] % 4
	Nous prouvons dans cet exercice que la classe des ensembles énumérables
	est bien close par union, intersection et produit cartésien. Soient
	\(A\) et \(B\) deux ensembles énumérables.
	\begin{enumerate}
		\item Montrez que \(A\cup B\) est énumérable.
		\item Montrez que \(A\cap B\) est énumérable.
		\item Montrez que \(A\times B = \{\langle x,y\rangle \mid x\in A, y\in B\}\) est énumérable.
	\end{enumerate}
\end{td-exo}

% ----- Solutions exo 4 ----- %
\iftoggle{showsolutions}{
	\begin{td-sol}[]\ %
		
	\end{td-sol}
}{}