% ----- Consignes exo 1 ----- %
\begin{td-exo}[] % 1
    Combien y-a-t'il de graphes simples non isomorphes avec 4 sommets? Dessiner chacun de ces graphes.
\end{td-exo}

% ----- Solutions exo 1 ----- %
\iftoggle{showsolutions}{
	\begin{td-sol}[] % 1
		Il y a exactement 11 graphes simples non isomorphes avec 4 sommets, les voici:
        \begin{center}
            \foreach \n in {1,...,11} { %chktex 1 %chktex 11
            \begin{minipage}[c][2cm][c]{0.16\textwidth}
                \centering
                \input{./assets/td_1_ex_1_\n.tikz}
            \end{minipage}
                \ifnum\n=4 \\[4mm]\fi
                \ifnum\n=8 \\[4mm]\fi
            }
        \end{center}
	\end{td-sol}
}{}

% ----- Consignes exo 2 ----- %
\begin{td-exo}[Exercice 2] % 2
    Calculez les paramètres \(n\) (nombre de sommets), 
\(m\) (nombre d'arcs ou d'arêtes), 
\(\delta\) (degré min), 
\(\Delta\) (degré max) et 
\(D\) (diamètre) des graphes suivants:

\begin{enumerate}
    \item \(B_d\) (les arbres binomiaux de dimension \(d\)),
    \item \(C_n\) (cycle à \(n\) sommets),
    \item \(K_n\) (graphe complet à \(n\) sommets),
    \item \(GR_{p\times q}\) (grille \(p\times q\)),
    \item \(TR_{p\times q}\) (tore \(p\times q\)),
    \item \(H_d\) (hypercube de dimension \(d\)).
\end{enumerate}

Dessinez les graphes suivants:
\begin{enumerate}
    \item \(B_3\)
    \item \(K_5\)
    \item \(GR_{4\times 4}\)
    \item \(TR_{4\times 4}\)
    \item \(H_2\)
    \item \(H_3\)
    \item \(H_4\)
\end{enumerate}

\begin{rappel}
    Le \defemph{diamètre} \(D\) d'un graphe est la plus grande des plus courtes distances entre deux sommets quelconques.

    Par exemple, dans un arbre, le diamètre est la longueur du chemin le plus long entre deux feuilles.
\end{rappel}
\end{td-exo}

% ----- Solutions exo 2 ----- %
\iftoggle{showsolutions}{
    \begin{td-sol}[] %
        Les paramètres des graphes sont les suivants:

        \begin{tabular}{c|c|c|c|c|c} % chktex 44
            Graphe & \(n\) & \(m\) & \(\delta\) & \(\Delta\) & \(D\) \\
            \hline % chktex 44
            \(B_d\) & \(2^d\) & \(2^d-1\) & \(1\) & \(3\) & \(d\) \\
            \(C_n\) & \(n\) & \(n\) & \(2\) & \(2\) & \(\left\lfloor \frac n2 \right\rfloor \) \\
            \(K_n\) & \(n\) & \(\frac{n(n-1)}2\) & \(n-1\) & \(n-1\) & \(1\) \\
            \(GR_{p\times q}\) & \(pq\) & \(2pq - p - q\) & \(2\) & \(4\) & \(p+q-2\) \\
            \(TR_{p\times q}\) & \(pq\) & \(2pq\) & \(4\) & \(4\) & \(\left\lfloor \frac p2 \right\rfloor + \left\lfloor \frac q2\right\rfloor \) \\
            \(H_d\) & \(2^d\) & \(d\cdot2^{d-1}\) & \(d\) & \(d\) & \(d\) \\
        \end{tabular}
    \end{td-sol}
}{}



% ----- Consignes exo 10 ----- %
\begin{td-exo}[Exercice 10] % 10
Un \(n\)-cube ou (hypercube de dimension \(n\)) est un graphe dont les sommets représentent les éléments de \({\{0,1\}}^n\)
et deux sommets sont adjacents si et seulement si les \(n\)-uplets correspondants diffèrent en exactement une composante. 
Montrer que:
\begin{enumerate}
    \item Le \(n\)-cube possède \(2^n\) sommets,
    \item Le \(n\)-cube est \(n\)-régulier,
    \item Le nombre d'arêtes est \(n\cdot 2^{n-1}\).
\end{enumerate}
Représenter le 1-cube, le 2-cube et le 3-cube
\end{td-exo}

% ----- Solutions exo 10 ----- %
\iftoggle{showsolutions}{
	\begin{td-sol}[] %
        \begin{enumerate}
            \item Il y a exactement \(2^n\) \(n\)-uplets dans \({\{0,1\}}^n\) donc un \(n\)-cube a exactement \(2^n\) sommets.
            \item Il y a exactement \(n\) \(n\)-uplets qui diffèrent d'un sommet en exactement une composante (il suffit d'inverser une de ses composantes).
            \item Chaque sommet a \(n\) voisins et chaque arête est comptée 2 fois donc \(n*2^n / 2 = n2^{n-1}\).
            \item le 1 cube est une ligne reliant 2 points le 2 un carré et le 3 un cube. %chktex 17, faire la representation graphique
        \end{enumerate}
	\end{td-sol}
}{}


% ----- Consignes exo 7 ----- %
\begin{td-exo}[Exercice 7] % 7
Les nombres \(\delta(G)\) et \(\Delta(G)\) représentent respectivement 
les degrés minimum et maximum d'un graphe 
\(G=(X,E)\) (\(X\) représente l'ensemble des sommets et \(E\) celui des arêtes) où \(n=\n X\) et \(m = \n E\).
Montrer que \(\delta(G) \leq 2\frac mn \leq \Delta (G)\).
\end{td-exo}

% ----- Solutions exo 7 ----- %
\iftoggle{showsolutions}{
	\begin{td-sol}[] %
		On sait que pour tout sommet \(d\), on a
        \begin{equation*}
            \delta \leq d_i \leq \Delta
        \end{equation*}
        Si on fait la somme des distances pour tous les sommets, on obtient alors
        \begin{equation*}
            \delta \leq 2m/n \leq \Delta
        \end{equation*}
	\end{td-sol}   % A VERIFIER
}{}





% ----- Consignes exo 8 ----- %
\begin{td-exo}[Exercice 8] % 8
    Montrer que si un graphe biparti \(G = (X_1, X_2, E)\) est \(k\)-régulier (avec \(k>0\)) alors \(\n X_1 = \n X_2\).
\end{td-exo}

% ----- Solutions exo 8 ----- %
\iftoggle{showsolutions}{
	\begin{td-sol}[] %
		Il y a \(n_1\) sommets à gauche et \(n_2\) à droite. Il y a donc \(kn_1\) arêtes qui sortent
        de la partie à gauche et qui vont à droite, et vice versa. Alors \(kn_1 = kn_2\) et donc \(n_1=n_2\).
	\end{td-sol}
}{}


% ----- Consignes exo 9 ----- %
\begin{td-exo}[Exercice 9] % 9
    Montrer que tout graphe simple possède au moins deux sommets de même degré
\end{td-exo}

% ----- Solutions exo 9 ----- %
\iftoggle{showsolutions}{
	\begin{td-sol}[] %
		Supposons que tous les sommets ont des degrés différents. Alors, comme le degré maximal est \(n-1\), on a
        \begin{equation*}
            d(x_n) = n-1, d(x_{n-1})=n-2,\ldots,x_2=1,x_1=0
        \end{equation*}
        Or, \(x_n\) est relié à tous les sommets, donc en particulier à \(x_1\), mais celui-ci est de degré 0. 
        Il y a donc contradiction et il existe au moins deux sommets de même degré
	\end{td-sol}
}{}


% ----- Consignes exo 11 ----- %
\begin{td-exo}[Exercice 11] % 11
    Montrer que tout arbre d'ordre \(n>1\) a au moins 2 sommets pendants (un sommet pendant est un sommet de degré 1).
\end{td-exo}

% ----- Solutions exo 11 ----- %
\iftoggle{showsolutions}{
	\begin{td-sol}[] %
		Raisonner sur le degré des sommets sur la chaine maximale (les sommets les plus distants).
        On peut aussi raisonner sur la somme des degrés et montrer qu'il faut forcement 2 de degre 1 pour la formule
	\end{td-sol}
}{}


% ----- Consignes exo 16 ----- %
\begin{td-exo}[Exercice 16] % 16
    Soient \(G_1 = (X_1, A_1)\) et \(G_2 = (X_2, A_2)\) les graphes suivants:

    % inserer graphes ici

    Repondre aux questions suivantes:
    \begin{enumerate}
        \item Donner \(\Gamma^+(x)\) (la liste des successeurs), \(\Gamma^-(x)\) (la liste des prédécesseurs),
        \(\Gamma(x),d^+(x),d^-(x) \) et \(d(x)\) pour tout \(x\in X_2\).

        \item Donner les matrices d'adjacence de \(G_1\) et \(G_2\).

        \item Etant donné un graphe orienté \(G = (X,A)\) ayant \(n\) sommets et \(m\) arcs, 
        sa matrice d'incidence est une matrice \(n\times m\), notée \(P = (p_{ie})\) telle que \(p_{ie} = 1\)
        si et seulement si le sommet \(i\) est l'origine de l'arc \(e\), \(p_{ie} = -1\) si et seulement si le
        sommet \(i\) est l'extrêmité de l'arc \(e\) et \(p_{ie} = 0\) sinon. Donner la matrice d'incidence
        des graphes \(G_1\) et \(G_2\).

        \item Représenter les deux graphes par leurs listes d'adjacence (les sommets d'une liste d'adjacence
        sont rangés consécutivement dans un tableau.)
    \end{enumerate}
\end{td-exo}

% ----- Solutions exo 16 ----- %
\iftoggle{showsolutions}{
	\begin{td-sol}[] %
		A completer
	\end{td-sol}
}{}


% ----- Consignes exo 17 ----- %
\begin{td-exo}[Exercice 17] % 17
    a completer, penser à rajouter les packages pour faire des algorithmes
\end{td-exo}

% ----- Solutions exo 2 ----- %
\iftoggle{showsolutions}{
	\begin{td-sol}[] %
		Exercice solution
	\end{td-sol}
}{}


% ----- Consignes exo 18 ----- %
\begin{td-exo}[Notions de base] % 18
    Soit la matrice binaire ou matrice d'adjacence \(M\) associée au graphe orienté \(G = (S, U)\) telle que
    \begin{equation*}
        M = \begin{pmatrix}
            0&1&0&1&0\\
            0&0&1&0&0\\
            0&0&0&0&1\\
            0&0&1&0&1\\
            0&0&0&0&0\\
        \end{pmatrix}
    \end{equation*}\ %
    \begin{enumerate}
        \item Tracer le graphe représentatif de cette matrice.
        \item Donner la matrice d'incidence sommets-arcs de ce graphe.
        \item Calculer \(M^2, M^3, M^4\). Que pouvez-vous en dire?
        \item Calculer 
        \begin{equation*}
            A = I + M + M^2 + M^3 + M^4.
        \end{equation*}
        Interpréter A.
        \item Appliquer l'algorithme de Roy Warshall. Que constatez-vous?
    \end{enumerate}
\end{td-exo}

% ----- Solutions exo 18 ----- %
\iftoggle{showsolutions}{
	\begin{td-sol}[] % 18
		Exercice solution
	\end{td-sol}
}{}


% ----- Consignes exo 2 ----- %
\begin{td-exo}[Optional title 2] % 2
Exercise 2 content
\end{td-exo}

% ----- Solutions exo 2 ----- %
\iftoggle{showsolutions}{
	\begin{td-sol}[] %
		Exercice solution
	\end{td-sol}
}{}


% ----- Consignes exo 2 ----- %
\begin{td-exo}[Optional title 2] % 2
Exercise 2 content
\end{td-exo}

% ----- Solutions exo 2 ----- %
\iftoggle{showsolutions}{
	\begin{td-sol}[] %
		Exercice solution
	\end{td-sol}
}{}