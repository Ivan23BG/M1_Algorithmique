% ----- Consignes exo 4 ----- %
\begin{td-exo}[Résolution en utilisant la méthode de séparation et évaluation]\,\\ % 4 
	On considère le problème linéaire en nombres entiers ci-dessous:
	\begin{equation*}
		PL_0 = 
		\begin{cases}
			2x_1 + x_2 \leq 6\\
			2x_1 + 3x_2 \leq 9\\
			x_1,x_2 \in\bb N\\
			\max(z) = 3x_1 + 4x_2
		\end{cases}
	\end{equation*}
	\begin{enumerate}
		\item Résoudre le programme linéaire en nombres entiers ci-dessus par la méthode d'évaluation et de séparation tel que le premier branchement se fera sur la variable \(x_2\).
		\item Résoudre le programme linéaire en nombres entiers ci-dessus par la méthode d'évaluation et de séparation tel que le premier branchement se fera sur la variable \(x_1\).
		\item Donner la signifaction géométrique du premier branchement sur la variable \(x_2\). Quels sont les principes vus en cours que l'on retrouve lors de l'interprétation géométrique?
	\end{enumerate}
\end{td-exo}

% ----- Solutions exo 4 ----- %
\iftoggle{showsolutions}{ 
	\begin{td-sol}[]\ % 4
		\begin{enumerate}
			\item L'arbre de branchement sur \(x_2\) est le suivant:

			\vspace{2mm}
\ffigbox[\FBwidth]{%
\caption{\centering Branch and bound}\label{fig:td1_ex04_f1}
}{
    \fbox{
        \begin{tikzpicture}[
            main node/.style={
                draw,
                rounded rectangle,
                rounded corners=3pt,
                fill=blue!20,
                inner sep=4pt,
                font=\scriptsize,
                align=center,
                text=black
            },
            solution/.style={
                draw=red,
                very thick,
                fill=red!8
            },
            optimal/.style={
                draw=green,
                very thick,
                fill=green!8
            },
            invalid/.style={
                draw=gray,
                dashed,
                text=gray,
                fill=gray!10
            },
        ]

        % root
        \node[main node] (s00) {
            \(z^* = 12.75\)\\
            \(x^* = (2.25,1.5)\)
        };

        % children (relative placement) -- tier 1
        \node[main node, below right=2cm and 2cm of s00] (s12) {
            \(z^* = 11.5\)\\
            \(x^* = \left(2.5,1\right)\)
        };

        \node[main node, below left=2cm and 2cm of s00] (s11) {
            \(z^* = 12.5\)\\
            \(x^* = (3,2)\)
        };

        % children (relative placement) -- tier 2
        \node[main node, below left=2cm and 1cm of s11] (s21) {
            \(z^* = 12.33\)\\
            \(x^* = \left(1,\frac73\right)\)
        };

        \node[main node, invalid, below right=2cm and 1cm of s11] (s22) {
            Impossible
        };
        
        \node[main node, solution, below left=2cm and 1cm of s12] (s23) {
            \(z^* = 10\)\\
            \(x^* = \left(2, 1\right)\)
        };

        \node[main node, solution, below right=2cm and 1cm of s12] (s24) {
            \(z^* = 9\)\\
            \(x^* = \left(3, 0\right)\)
        };

        % children (relative placement) -- tier 3
        \node[main node, solution, below left=2cm and .6cm of s21] (s31) {
            \(z^* = 11\)\\
            \(x^* = \left(1,2\right)\)
        };

        \node[main node, optimal, below right=2cm and .6cm of s21] (s32) {
            \(z^* = 12\)\\
            \(x^* = (0,3)\)
        };

        % edges
        \draw[-{Stealth}] (s00) -- node[above left] {\(x_2 \leq 1\)} (s11);
        \draw[-{Stealth}] (s00) -- node[above right] {\(x_2 \geq 2\)} (s12);
        
        \draw[-{Stealth}] (s11) -- node[above left] {\(x_1 \leq 1\)} (s21);
        \draw[-{Stealth}] (s11) -- node[above right] {\(x_1 \geq 2\)} (s22);
        
        \draw[-{Stealth}] (s12) -- node[above left] {\(x_1 \leq 2\)} (s23);
        \draw[-{Stealth}] (s12) -- node[above right] {\(x_1 \geq 3\)} (s24);
        
        \draw[-{Stealth}] (s21) -- node[above left] {\(x_2 \leq 2\)} (s31);
        \draw[-{Stealth}] (s21) -- node[above right] {\(x_2 \geq 3\)} (s32);

        \end{tikzpicture}
    }
}

			\item L'arbre de branchement sur \(x_1\) est le suivant:

			\vspace{2mm}
\ffigbox[\FBwidth]{%
\caption{\centering Branch and bound}\label{fig:td1_ex04_f2}
}{
    \fbox{
        \begin{tikzpicture}[
            main node/.style={
                draw,
                rounded rectangle,
                rounded corners=3pt,
                fill=blue!20,
                inner sep=4pt,
                font=\scriptsize,
                align=center,
                text=black
            },
            solution/.style={
                draw=red,
                very thick,
                fill=red!8
            },
            optimal/.style={
                draw=green,
                very thick,
                fill=green!8
            }
        ]

        % root
        \node[main node] (s00) {
            \(z^* = 12.75\)\\
            \(x^* = (2.25,1.5)\)
        };

        % children (relative placement) -- tier 1
        \node[main node, below left=2cm and 2cm of s00] (s11) {
            \(z^* = 12.66\)\\
            \(x^* = \left(2,\frac53\right)\)
        };

        \node[main node, solution, below right=2cm and 2cm of s00] (s12) {
            \(z^* = 9\)\\
            \(x^* = (3,0)\)
        };

        % children (relative placement) -- tier 2
        \node[main node, below left=2cm and 2cm of s11] (s21) {
            \(z^* = 10\)\\
            \(x^* = \left(2,1\right)\)
        };

        \node[main node, optimal, below right=2cm and 2cm of s11] (s22) {
            \(z^* = 12\)\\
            \(x^* = (0,3)\)
        };

        % edges
        \draw[-{Stealth}] (s00) -- node[above left] {\(x_1 \leq 2\)} (s11);
        \draw[-{Stealth}] (s00) -- node[above right] {\(x_1 \geq 3\)} (s12);
        
        \draw[-{Stealth}] (s11) -- node[above left] {\(x_2 \leq 1\)} (s21);
        \draw[-{Stealth}] (s11) -- node[above right] {\(x_2 \geq 2\)} (s22);

        \end{tikzpicture}
    }
}

			% \item et un test

			% \vspace{2mm}
\ffigbox[\FBwidth]{%
\caption{\centering Branch and bound}\label{fig:test}
}{
    \fbox{
        \begin{tikzpicture}[
            main node/.style={
                draw, 
                rounded rectangle, 
                rounded corners=3pt, 
                fill=blue!20, 
                inner sep=4pt, 
                font=\scriptsize, 
                align=center,
                text=black
            },
            solution/.style={draw=red, very thick, fill=red!8},
            optimal/.style={draw=green, very thick, fill=green!8},
            invalid/.style={draw=gray, dashed, text=gray, fill=gray!10},

            edge from parent/.style={-{Stealth}},
            level distance=18mm,
            sibling distance=14mm
            ]

            \node[main node] {
            \(z^* = 12.75\)\\
            \(x^* = (2.25,1.5)\)
            }
            child { node[main node] {
                \(z^* = 12.5\)\\
                \(x^* = (3,2)\)
            }
            edge from parent node[above left] {\(x_2 \le 1\)}

            child { node[main node] {
                \(z^* = 12.33\)\\
                \(x^* = \left(1,\frac73\right)\)
                }
                edge from parent node[above left] {\(x_1 \le 1\)}

                child { node[main node, solution] {
                    \(z^* = 11\)\\
                    \(x^* = (1,2)\)
                }
                edge from parent node[above left] {\(x_2 \le 2\)}
                }

                child { node[main node, optimal] {
                    \(z^* = 12\)\\
                    \(x^* = (0,3)\)
                }
                edge from parent node[above right] {\(x_2 \ge 3\)}
                }
            }

            child { node[main node, invalid] {Impossible}
                edge from parent node[above right] {\(x_1 \ge 2\)}
            }
            }
            child { node[main node] {
                \(z^* = 11.5\)\\
                \(x^* = (2.5,1)\)
            }
            edge from parent node[above right] {\(x_2 \ge 2\)}

            child { node[main node, solution] {
                \(z^* = 10\)\\
                \(x^* = (2,1)\)
                }
                edge from parent node[above left] {\(x_1 \le 2\)}
            }

            child { node[main node, solution] {
                \(z^* = 9\)\\
                \(x^* = (3,0)\)
                }
                edge from parent node[above right] {\(x_1 \ge 3\)}
            }
            };

        \end{tikzpicture}

    }
}
		\end{enumerate}

		
	\end{td-sol}
}{}
