% ----- Consignes exo 35 ----- %
\begin{td-exo}[Le problème du flot maximum versus la coupe minimum]\, % 35 
	\begin{enumerate}
		\item Questions préliminaires
		\begin{enumerate}
			\item Rappeler le principe des algorithmes principaux pour résoudre le problème du flot maximum.
			\item Dans toute solution optimale, que dire des arcs retours? 
			Et des arcs avant appartenant à une coupe minimale?
		\end{enumerate}
		\item Modélisation et propriétés.
		On considère dans la partie qui suit le graphe orienté pondéré donné par la figure~\ref{fig:td4_ex35_f1} suivante:

		\ffigbox[\FBwidth]{%
\caption{\centering Un réseau de flot}\label{fig:td4_ex35_f1}
}{
    \fbox{
        \begin{tikzpicture}[scale=1, main node/.style={circle, draw, fill=blue!20, inner sep=1pt, font=\scriptsize, minimum size=6mm, text=black}]
            % les sommets initiaux
            \node[main node] (s) at (0,0) {\(s\)};
            \node[main node] (a) at (2,2) {\(a\)};
            \node[main node] (b) at (2,-2) {\(b\)};
            \node[main node] (c) at (4,0) {\(c\)};
            \node[main node] (t) at (6,0) {\(t\)};

            % les arcs avec capacités
            \draw[-{Stealth}] (s) to node[above left] {\(2\)} (a);
            \draw[-{Stealth}] (s) to node[below left] {\(1\)} (b);
            \draw[-{Stealth}] (a) to node[above] {\(2\)} (c);
            \draw[-{Stealth}] (b) to node[below right] {\(1\)} (c);
            \draw[-{Stealth}] (c) to node[below] {\(2\)} (t);
        \end{tikzpicture}
    }
}

		\begin{enumerate}
			\item Formuler le problème du flot maximum par un programme linéaire noté \(P_0\) pour le réseau donné par la figure~\ref{fig:td4_ex35_f1}.
			Formuler le problème du flot maximum par un programme linéaire \(\mathcal{P}\).

			\item Donner le programme dual de \(P_0\) et de \(\mathcal{P}\).
			Quelle est la structure de la matrice des contraintes pour le dual?

			\item Utiliser les écarts complémentaires pour retrouver un théorème de \textsc{Min-Max}.
		\end{enumerate}
	\end{enumerate}
\end{td-exo}

% ----- Solutions exo 35 ----- %
\iftoggle{showsolutions}{ 
	\begin{td-sol}[]\ % 35
		
	\end{td-sol}
}{}
