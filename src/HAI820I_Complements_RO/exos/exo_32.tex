% ----- Consignes exo 32 ----- %
\begin{td-exo}[Le problème du transversal dans un graphe]\,\\ % 32 
	On rappelle la définition suivante:
	\vspace{2mm}
	\begin{definition}
		Un \defemph{transversal} (couverture) est un ensemble de sommets tel que pour chaque arete du graphe, au moins une de ses deux extremités appartient au transversal.
	\end{definition}
	\vspace{2mm}
	On considère le graphe donné par la figure~\ref{fig:td4_ex31_f1}.
	Donner le programme linéaire en nombres entiers qui permet de trouver un transversal de taille minimale dans ce graphe.
\end{td-exo}

% ----- Solutions exo 32 ----- %
\iftoggle{showsolutions}{ 
	\begin{td-sol}[]\ % 32
		
	\end{td-sol}
}{}
