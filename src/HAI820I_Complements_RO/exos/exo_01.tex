% ----- Consignes exo 1 ----- %
\begin{td-exo}[Matrice totalement unimodulaire]\,\\ % 1 
	Est-ce que les matrices suivantes sont totalement unimodulaires?
	\begin{enumerate}
		\item La matrice \(A\) suivante:
		\begin{equation*}
			A = 
			\begin{pmatrix}
				1 & -1 & 1\\
				-1 & 1 & 0
			\end{pmatrix}
		\end{equation*}
		\item La matrice \(B\) suivante:
		\begin{equation*}
			B = 
			\begin{pmatrix}
				1 & 1 & 0 & 0 & 1\\
				0 & 1 & 1 & 1 & 0\\
				-1 & 0 & 0 & 0 & -1\\
				0 & 0 & -1 & 0 & 0
			\end{pmatrix}
		\end{equation*}
		\item La matrice \(C\) suivante:
		\begin{equation*}
			C = 
			\begin{pmatrix}
				1 & -1\\
				1 & 1
			\end{pmatrix}
		\end{equation*}
		\item La matrice \(D\) suivante:
		\begin{equation*}
			D = 
			\begin{pmatrix}
				1 & 1 & 0\\
				0 & 1 & 1\\
				1 & 0 & 1
			\end{pmatrix}
		\end{equation*}

		\item Propriétés: est-ce que les matrices suivantes sont totalement unimodulaires 
		sachant que \(A\) est totalement unimodulaire?

		\begin{enumerate}
			\item La matrice \(-A\),
			\item la transposée \(A^T\),
			\item la matrice \([A, I]\),
			\item la matrice \([A, -A]\).
		\end{enumerate}

		\item Considérons la matrice \(E\) définie de la manière suivante:
		\begin{equation*}
			E = 
			\begin{pmatrix}
				1 & -1 & 1\\
				1 & 1 & 0
			\end{pmatrix}
		\end{equation*}
		et soit \(b = \begin{pmatrix}
			2 & 1
		\end{pmatrix}\).

		Est-ce que \(E\) est totalement unimodulaire? Trouver les deux solutions à valeurs entières.
	\end{enumerate}
\end{td-exo}

% ----- Solutions exo 1 ----- %
\iftoggle{showsolutions}{ 
	\begin{td-sol}[]\ % 1
		
	\end{td-sol}
}{}
