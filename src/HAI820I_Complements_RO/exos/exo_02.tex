% ----- Consignes exo 2 ----- %
\begin{td-exo}[Solutions entières versus solutions réelles]\, % 2 
	Donner les solutions réelles et entières des problèmes suivants.
	\begin{enumerate}
		\item Un premier LP\@:
		\begin{equation*}
			PL_1 = 
			\begin{cases}
				2x_1 - 2x_2 \geq 1\\
				x_1 + x_2 \geq 1\\
				x_2 \leq 2\\
				x_1,x_2 \in \bb R^+\\
				\min(z) = 3x_1 - x_2
			\end{cases}
		\end{equation*}

		\item Un second LP\@:
		\begin{equation*}
			PL_2 = 
			\begin{cases}
				10x_1 + 7x_2 \leq 29\\
				x_1,x_2 \in \bb R^+\\
				\max(z) = 300x_1 + 205x_2
			\end{cases}
		\end{equation*}

		\item Un troisième LP\@:
		\begin{equation*}
			PL_3 = 
			\begin{cases}
				-3x_1 + 4x_2 \geq 6\\
				2x_1 + 2x_2 \geq 11\\
				4x_1 - 5x_2 \leq 10\\
				x_1,x_2 \in \bb R^+\\
				\min(z) = -3x_1 - 2x_2
			\end{cases}
		\end{equation*}
	\end{enumerate}
\end{td-exo}

% ----- Solutions exo 2 ----- %
\iftoggle{showsolutions}{
	\begin{td-sol}[]\ % 2
		Les solutions sont
		\begin{enumerate}
			\item On commence avec les valeurs \((\frac34, \frac14)\) ce qui nous donne un \(z_{LP}^\ast = 2\).
			En ILP cela donne \((1,0)\) pour \(z_{ILP}^\ast = 3\).
			L'écart total est de 1 ce qui semble raisonnable
		\end{enumerate}
	\end{td-sol}
}{}
