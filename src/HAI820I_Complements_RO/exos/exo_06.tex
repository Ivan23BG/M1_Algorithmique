% ----- Consignes exo 6 ----- %
\begin{td-exo}[Contraintes avec une seule variable]\,\\ % 6 
	On considère un problème de programmation linéaire standard, à une seule contrainte, défini par
	\begin{equation*}
		\max \sum_{j=1}^{n} u_j x_j
	\end{equation*}
	avec 
	\begin{equation*}
		\sum_{j=1}^{n} v_j x_j \leq V,\text{ et } x_j\geq0,\quad\text{pour } 1 \leq j \leq n
	\end{equation*}
	Tous les coefficient \(u_j\) et \(v_j\) sont supposés strictement positifs et l'on suppose les variables classées par rapports utilité/volume décroissants ou, pour rester dans un formalisme plus mathématique, suivant les valeurs décroissantes des rapports \(\frac{u_j}{v_j}\).
	Montrer que la variable \(x_1\) est entrante et que, en la faisant entrer en base, on atteint l'optimum de l'objectif en une seule étape.
	Exprimer la valeur de l'objectif en fonction des différents coefficients.
\end{td-exo}

% ----- Solutions exo 6 ----- %
\iftoggle{showsolutions}{ 
	\begin{td-sol}[]\ % 6
		
	\end{td-sol}
}{}
