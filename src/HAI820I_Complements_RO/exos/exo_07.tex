% ----- Consignes exo 7 ----- %
\begin{td-exo}[Le problème du voyageur du commerce symétrique]\, % 7 
	\begin{enumerate}
		\item Appliquer la méthode par séparation et évaluation décrite en cours pour résoudre le problème du voyageur de commerce dans le graphe \(K_6\) dont la matrice des poids est la suivante:

		\begin{center}
			\begin{tabular}{c|cccccc} % chktex 44
				\ & \(x\) & \(y\) & \(z\) & \(t\) & \(u\) & \(v\)\\
				\hline % chktex 44
				\(x\) & \(\infty\) & \(4\) & \(7\) & \(2\) & \(5\) & \(4\)\\
				\(y\) & \(4\) & \(\infty\) & \(3\) & \(2\) & \(1\) & \(2\)\\
				\(z\) & \(7\) & \(3\) & \(\infty\) & \(2\) & \(6\) & \(3\)\\
				\(t\) & \(2\) & \(2\) & \(2\) & \(\infty\) & \(5\) & \(3\)\\
				\(u\) & \(5\) & \(1\) & \(6\) & \(5\) & \(\infty\) & \(2\)\\
				\(v\) & \(4\) & \(2\) & \(3\) & \(3\) & \(2\) & \(\infty\)\\
			\end{tabular}
		\end{center}
		
		\item Que faudrait-il faire si on voulait connaitre tous les cycles de poids minimum?
	\end{enumerate}
\end{td-exo}

% ----- Solutions exo 7 ----- %
\iftoggle{showsolutions}{ 
	\begin{td-sol}[]\ % 7
		\begin{enumerate}
			\item On commence par calculer l'ACPM sur \(G \setminus \{x\}\):\\
			Les arêtes de poids minimal qui permettent de connecter tous les sommets sont:
			\begin{itemize}
				\item \(y-u\) de poids \(1\),
				\item \(y-t\) de poids \(2\),
				\item \(z-t\) de poids \(2\),
				\item \(v-u\) de poids \(2\),
				\item \(z-v\) de poids \(3\).
			\end{itemize}
			Le poids total de l'ACPM est \(10\).

			Ensuite on reconnecte \(x\) avec les arêtes les moins couteuses: \\
			Les deux plus petites arêtes partant de \(x\) sont:
			\begin{itemize}
				\item \(x-t = 2\),
				\item \(x-y = 4\).
			\end{itemize}

			Ainsi, le poids total du 1-arbre résultant est:
			\begin{equation*}
				LB_0 = 10 + 2 + 4 = 16
			\end{equation*}
			qui constitue notre \textbf{borne inférieure initiale}.

			On vérifie maintenant si la solution trouvée est réalisable (est-elle un cycle Hamiltonien):\\
			On calcule le degré de chaque sommet dans ce 1-arbre:
			\begin{itemize}
				\item \(\deg(x) = 2\),
				\item \(\deg(y) = 3\),
				\item \(\deg(z) = 2\),
				\item \(\deg(t) = 3\),
				\item \(\deg(u) = 2\),
				\item \(\deg(v) = 2\).
			\end{itemize}
			Certains sommets ont un degré supérieur à 2, donc ce 1-arbre n'est pas une solution réalisable (pas de cycle hamiltonien).

			On branche alors sur un des sommets de degré \(>2\), par exemple \(y\), et on crée des sous-problèmes en excluant successivement une arête incidente à \(y\).

			\begin{enumerate}
				\item \textbf{Exclusion de \(y-u\):} 
				On recalcule le 1-arbre compatible avec cette contrainte. 
				Les deux arêtes les moins coûteuses de \(x\) restent \(x-t\) et \(x-y\). 
				Le nouvel arbre a toujours un poids:
				\begin{equation*}
					LB = 10 + 2 + 4 = 16
				\end{equation*}
				Les degrés deviennent:
				\begin{itemize}
					\item \(\deg(y) = 2\), \(\deg(t) = 3\), \(\deg(x)=2\), \(\deg(z)=2\), \(\deg(u)=2\), \(\deg(v)=2\)
				\end{itemize}
				On branche maintenant sur \(t\) (degré \(>2\)).

				\item \textbf{Exclusion de \(y-t\):} 
				Recalcul du 1-arbre compatible:
				\begin{equation*}
					LB = 16
				\end{equation*}
				Les degrés deviennent
				\begin{itemize}
					\item \(\deg(t)=2\), \(\deg(y)=2\), \(\deg(x)=2\), \(\deg(z)=2\), \(\deg(u)=2\), \(\deg(v)=2\)
				\end{itemize}
				Cette configuration donne un cycle hamiltonien réalisable!  

				\textbf{Étape 5 — Borne supérieure (solution réalisable):} \\
				Un cycle hamiltonien réalisable correspondant est:
				\begin{equation*}
					x \to t \to z \to v \to u \to y \to x
				\end{equation*}
				de poids:
				\begin{equation*}
					2 + 2 + 3 + 2 + 1 + 4 = 14
				\end{equation*}

				\item \textbf{Exclusion de \(x-y\):} 
				Recalcul du 1-arbre compatible, poids \(LB = 16\); on branche encore sur \(t\) ou \(v\).  
				Toutes ces branches auront un \(LB \ge 16\), donc elles peuvent être éliminées par notre borne supérieure.
			\end{enumerate}

			On conclut:\\
			Les sous-problèmes dont la borne inférieure est \(\ge UB=14\) sont éliminés. La solution trouvée est donc optimale.
			\begin{equation*}
				\text{Cycle optimal: } x \to t \to z \to v \to u \to y \to x, \quad \text{poids } 14.
			\end{equation*}

			\item Pour connaître tous les cycles de poids minimum il faudrait continuer la procédure de séparation et d'évaluation, mais couper les branches seulement en cas d'inégalité stricte.
		\end{enumerate}
	\end{td-sol}
}{}
