% ----- Consignes exo 7 ----- %
\begin{td-exo}[Le problème du voyageur du commerce symétrique]\, % 7 
	\begin{enumerate}
		\item Appliquer la méthode par séparation et évaluation décrite en cours pour résoudre le problème du voyageur de commerce dans le graphe \(K_6\) dont la matrice des poids est la suivante:

		\begin{tabular}{c|cccccc} % chktex 44
			\ & \(x\) & \(y\) & \(z\) & \(t\) & \(u\) & \(v\)\\
			\hline % chktex 44
			\(x\) & \(\infty\) & \(4\) & \(7\) & \(2\) & \(5\) & \(4\)\\
			\(y\) & \(4\) & \(\infty\) & \(3\) & \(2\) & \(1\) & \(2\)\\
			\(z\) & \(7\) & \(3\) & \(\infty\) & \(2\) & \(6\) & \(3\)\\
			\(t\) & \(2\) & \(2\) & \(2\) & \(\infty\) & \(5\) & \(3\)\\
			\(u\) & \(5\) & \(1\) & \(6\) & \(5\) & \(\infty\) & \(2\)\\
			\(v\) & \(4\) & \(2\) & \(3\) & \(3\) & \(2\) & \(\infty\)\\
		\end{tabular}

		\item Que faudrait-il faire si on voulait connaitre tous les cycles de poids minimum?
	\end{enumerate}
\end{td-exo}

% ----- Solutions exo 7 ----- %
\iftoggle{showsolutions}{ 
	\begin{td-sol}[]\ % 7
		
	\end{td-sol}
}{}
