% ----- Consignes exo 3 ----- %
\begin{td-exo}[\textsc{Sac a dos} en variables entières]\, % 3 
	On étudie le problème \textsc{Sac a dos} en variables entières, positives, pas nécessairement \(\{0, 1\}\), de la forme:
	\begin{equation*}
		\begin{cases}
			\max_{z \in {\bb N}^N} \sum_{i=1}^N u_i z_i\\
			z \leq b\\
			\sum_{i=1}^N p_i z_i \leq M
		\end{cases}
	\end{equation*}
	où les \(u_i\in\bb N\) représentent l'utilité des objets de type \(i\), les \(p_i\) leur poids, les \(b_i\) la quantité disponible (avec \(b_i \in\bb N\cup \{+\infty\}\)), \(M\) le poids maximum admissible du sac à dos et \(N\) le nombre de types d'objets considérés.
	Le relaché continu est défini comme suit:
	\begin{equation*}
		\begin{cases}
			\max_{z \in {\bb R}^N_+} \sum_{i=1}^N u_i z_i\\
			z \leq b\\
			\sum_{i=1}^N p_i z_i \leq M
		\end{cases}
	\end{equation*}
	\begin{enumerate}
		\item Montrer qu'on peut changer les \(b_i\) en \(\min\{b_i, \lfloor M/p_i \rfloor\}, \forall i\) sans changer la solution du problème en valeurs entières.
		On appellera cette opération la mise à jour des quantités disponibles.

		\item On suppose sans perte de généralité les ratios \(u_i/p_i\) triés par ordre décroissant.
		Montrer qu'on peut calculer simplement la solution du relaché continu.
		On se contentera d'une preuve dans le cas où les ratios \(u_i/p_i\) sont strictement décroissants.

		\item On considère le cas particulier (avec donc ici \(b_i = +\infty,\forall i\)):
		\begin{equation*}
			\begin{cases}
				\max_{\bb N^3_+} 4z_1 + 5z_2 + 2 z_3\\
				4z_1 + 7z_2 + 3z_3 \leq 19
			\end{cases}
		\end{equation*}

		\begin{enumerate}
			\item Effectuer la mise à jour des quantités disponibles.

			\item Expliciter la solution du relaché continu (avec les nouvelles bornes).

			\item Calculer la valeur associée.
		\end{enumerate}

		\item On introduit la variable d'écart \(z_4 \in\bb N\) telle que 
		\begin{equation*}
			4z_1 + 7z_2 + 3z_3 + z_4 = 19
		\end{equation*}
		et la variable d'écart à la borne supérieure \(\ol{z_1} = b_1 - z_1 = 4 - z_1\).
		Vérifier que 
		\begin{equation*}
			-4 \ol{z_1} + 7 z_2 + 3 z_3 + z_4 = 3
		\end{equation*}
		Appliquer la technique de coupe de Gomory à l'égalité précédente pour obtenir la relation satisfaite par les points entiers admissibles pour le problème à valeurs entières, et qui élimine la solution du relaxé continu.

		\item On considère le problème \(PR'\) obtenu en ajoutant au problème relaché continu la coupe de Gomory.
		Montrer que cette dernière contrainte est active en tout solution de \(PR'\) et calculer la valeur et les solutions de \(PR'\).
		Que peut-on en conclure?

		\item Comment aurait-on pu résoudre le problème du \textsc{Sac a dos} par une approche de séparation et évaluation.
		On étudiera le cas de minorantes obtenues en résolvant le problème relaché continu avec branchement sur les valeurs fractionnaires.
	\end{enumerate}
\end{td-exo}

% ----- Solutions exo 3 ----- %
\iftoggle{showsolutions}{ 
	\begin{td-sol}[]\ % 3
		
	\end{td-sol}
}{}
