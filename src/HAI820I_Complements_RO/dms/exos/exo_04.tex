% ----- Consignes exo 4 ----- %
\begin{td-exo}[Comparaisons branch and bound and branch and cut]\, % 4 
	Nous considérons le programme linéaire suivant:
	\begin{equation*}
		PL_0 = 
		\begin{cases}
			\max z(x_1, x_2) = 2x_1 + x_2 \\
			2x_1 + 5x_2 \leq 17\\
			3x_1 + 2x_2 \leq 10\\
			x_1, x_2 \geq 0
		\end{cases}
	\end{equation*}
	\begin{enumerate}
		\item Représenter graphiquement le polytope associé aux équations de \(PL_0\).

		\item Résoudre graphiquement \(PL_0\).

		\item Résoudre par la méthode du simplexe.
		Vous donnerez les tableaux nécessaires à la résolution.

		\item Maintenant nous cherchons une solution à valeur entière:

		\begin{enumerate}
			\item Vous donnerez la solution optimale en utilisant la méthode de Branch and Bound.
			Vous donnerez l'arbre et vous préciserez toutes les étapes.

			\item Vous donnerez la solution optimale en utilisant la méthode des coupes de Dantzig.
			Vous préciserez à chaque fois la contrainte rajoutée et vous l'exprimerez en fonctions des variables \(x_1\) et \(x_2\).

			\item Vous donnerez la solution optimale en utilisant la méthode des coupes de Gomory.
			Vous préciserez à chaque fois la contrainte rajoutée et vous l'exprimerez en fonctions des variables \(x_1\) et \(x_2\).

			\item Vous donnerez la solution optimale en utilisant la méthode des coupes de Chvatal-Gomory.
			Vous préciserez à chaque fois la contrainte rajoutée et vous l'exprimerez en fonctions des variables \(x_1\) et \(x_2\).
		\end{enumerate}
	\end{enumerate}
\end{td-exo}

% ----- Solutions exo 4 ----- %
\iftoggle{showsolutions}{ 
	\begin{td-sol}[]\ % 4
		\begin{enumerate}
			\item Le programme linéaire peut être représenté graphiquement comme dans la figure~\ref{fig:dm1_ex04_f1} ci-dessous:

			\vspace{2mm}
\ffigbox[\FBwidth]{%
\caption{\centering Représentation du problème \(PL_0\) dans \(\bb R^2\)}\label{fig:dm1_ex04_f1}
}{
    \fbox{
        \begin{tikzpicture}
            \begin{axis}[
                % axis lines=middle,
                xlabel={\(x_1\)},
                ylabel={\(x_2\)},
                xmin=0, xmax=4,
                ymin=0, ymax=4,
                grid=both,
                axis equal image, % <-- orthonormal grid
                width=10cm,
                height=7cm,
                legend pos=outer north east, % <-- legend outside
                legend cell align=left,
                legend image post style={fill opacity=0.45},
                legend style={fill=pagebg, draw=pagetext}
            ]

            % ----- Constraint lines -----

            % 2x1 + 5x2 = 17 -> x2 = (17 - 2x1)/5
            \addplot[domain=0:4, very thick, color=blue] {(17 - 2*x)/5};
            \addlegendentry{\(2x_1 + 5x_2 = 17\)}

            % 3x1 + 2x2 = 10 -> x2 = (10 - 3x1)/2
            \addplot[domain=0:4, very thick, color=orange] {(10 - 3*x)/2};
            \addlegendentry{\(3x_1 + 2x_2 = 10\)}

            % ----- Feasible region -----
            \addplot[
                fill=cyan!30,
                opacity=0.45,
                draw=none,
                area legend
            ] coordinates {
                (0, 0)
                (0, 17/5)
                (32/22, 31/11)
                (10/3, 0)
            };
            \addlegendentry{Région réalisable}

            % ----- Optimal real solution -----
            \addplot[
                only marks,
                mark=*,
                mark size=2.5pt,
                color=red!80!black
            ] coordinates {(10/3, 0)};
            \addlegendentry{Solution réelle}

            % % ----- Optimal integer solution -----
            % \addplot[
            %     only marks,
            %     mark=square*,
            %     mark size=2.5pt,
            %     color=purple
            % ] coordinates {(1,0)};
            % \addlegendentry{Solution entière}
            
            % % Draw the objective level line
            % z = 2x1 + x2 = 2*(10/3) + 0 = 20/3
            \addplot[dashed, thick, yellow, domain=0:4] {20/3 - 2*x};
            \addlegendentry{Ligne de niveau \(z=\frac{20}{3}\)}
            \end{axis}
        \end{tikzpicture}
    }
}

			\item On peut lire sur la figure~\ref{fig:dm1_ex04_f1} que la solution optimale est \(x^\ast = \left(\frac{10}{3}, 0\right)\).

			\item Résolvons par la méthode du simplexe. 
			On commence par mettre le problème sous forme standard:
			\begin{equation*}
				PL_0 = 
				\begin{cases}
					\max z(x_1, x_2) = 2x_1 + x_2 \\
					2x_1 + 5x_2 + x_3 = 17\\
					3x_1 + 2x_2 + x_4 = 10\\
					x_1, x_2, x_3, x_4 \geq 0
				\end{cases}
			\end{equation*}
			On construit ensuite le tableau initial:
			\begin{center}
				\begin{tabular}{|ccc|cccc|} % chktex 44
					\hline  % chktex 44
					\, & \, &\(c\)&\(2\)&\(1\)&\(0\)&\(0\)\\
					\hline % chktex 44
					\multicolumn{1}{|c|}{\(c^J\)}& \multicolumn{2}{c|}{variables de base}&\(x_1\)&\(x_2\)&\(x_3\)&\(x_4\)\\
					\hline % chktex 44
					\multicolumn{1}{|c|}{\(0\)}& \multicolumn{1}{c|}{\(x_1^{1} = x_3\)}&\(17\)&\(2\)&\(5\)&\(1\)&\(0\)\\
					\hline % chktex 44
					\multicolumn{1}{|c|}{\(0\)}& \multicolumn{1}{c|}{\(x_2^{1} = x_4\)}&\(10\)&\(3\)&\(2\)&\(0\)&\(1\)\\
					\hline % chktex 44
					\multicolumn{1}{|c|}{} & \(z(x)\) & \multicolumn{1}{|c|}{\(0\)}& \(-2\) & \(-1\) & \(0\) & \(0\)\\
					\hline % chktex 44
				\end{tabular}
			\end{center}

			On fait rentrer \(x_1\) dans la base, et on sort \(x_4\):
			\begin{center}
				\begin{tabular}{|ccc|cccc|} % chktex 44
					\hline  % chktex 44
					\, & \, &\(c\)&\(2\)&\(1\)&\(0\)&\(0\)\\
					\hline % chktex 44
					\multicolumn{1}{|c|}{\(c^J\)}& \multicolumn{2}{c|}{variables de base}&\(x_1\)&\(x_2\)&\(x_3\)&\(x_4\)\\
					\hline % chktex 44
					\multicolumn{1}{|c|}{\(0\)}& \multicolumn{1}{c|}{\(x_1^{2} = x_3\)}&\(\frac{31}{3}\)&\(0\)&\(\frac{11}{3}\)&\(1\)&\(-\frac{2}{3}\)\\
					\hline % chktex 44
					\multicolumn{1}{|c|}{\(2\)}& \multicolumn{1}{c|}{\(x_2^{2} = x_1\)}&\(\frac{10}{3}\)& \(1\) & \(\frac{2}{3}\) & \(0\) & \(\frac{1}{3}\)\\
					\hline % chktex 44
					\multicolumn{1}{|c|}{} & \(z(x)\) & \multicolumn{1}{|c|}{\(\frac{20}{3}\)}& \(0\) & \(\frac{1}{3}\) & \(0\) & \(\frac{2}{3}\)\\
					\hline % chktex 44
				\end{tabular}
			\end{center}
			Tous les coefficients de la ligne \(z(x)\) sont positifs, on a donc trouvé la solution optimale \(x^\ast = \left(\frac{10}{3}, 0\right)\) avec \(z(x^\ast) = \frac{20}{3}\).

			\item On passe à la résolution du problème à valeurs entières.

			\begin{enumerate}
				\item On commence par faire le branchement sur \(x_1\):
				\begin{itemize}
					\item Cas \(x_1 \leq 3\):
					Le problème peut alors être représenté graphiquement comme dans la figure~\ref{fig:dm1_ex04_f2} ci-dessous:
					
					\vspace{2mm}
\ffigbox[\FBwidth]{%
\caption{\centering Représentation du problème \(PL_0\) avec \(x_1 \leq 3\)}\label{fig:dm1_ex04_f2}
}{
    \fbox{
        \begin{tikzpicture}
            \begin{axis}[
                % axis lines=middle,
                xlabel={\(x_1\)},
                ylabel={\(x_2\)},
                xmin=0, xmax=4,
                ymin=0, ymax=4,
                grid=both,
                axis equal image, % <-- orthonormal grid
                width=10cm,
                height=7cm,
                legend pos=outer north east, % <-- legend outside
                legend cell align=left,
                legend image post style={fill opacity=0.45},
                legend style={fill=pagebg, draw=pagetext}
            ]

            % ----- Constraint lines -----

            % 2x1 + 5x2 = 17 -> x2 = (17 - 2x1)/5
            \addplot[domain=0:4, very thick, color=blue] {(17 - 2*x)/5};
            \addlegendentry{\(2x_1 + 5x_2 = 17\)}

            % 3x1 + 2x2 = 10 -> x2 = (10 - 3x1)/2
            \addplot[domain=0:4, very thick, color=orange] {(10 - 3*x)/2};
            \addlegendentry{\(3x_1 + 2x_2 = 10\)}

            % manually draw line x1 = 3
            \addplot[
                very thick,
                color=green!60!black
            ] coordinates {(3,0) (3,4)};
            \addlegendentry{\(x_1 = 3\)}

            % ----- Feasible region -----
            \addplot[
                fill=cyan!30,
                opacity=0.45,
                draw=none,
                area legend
            ] coordinates {
                (0, 0)
                (0, 17/5)
                (32/22, 31/11)
                (3, 1/2)
                (3, 0)
            };
            \addlegendentry{Région réalisable}

            % ----- Optimal real solution -----
            \addplot[
                only marks,
                mark=*,
                mark size=2.5pt,
                color=red!80!black
            ] coordinates {(3, 0.5)};
            \addlegendentry{Solution réelle}

            % % ----- Optimal integer solution -----
            % \addplot[
            %     only marks,
            %     mark=square*,
            %     mark size=2.5pt,
            %     color=purple
            % ] coordinates {(1,0)};
            % \addlegendentry{Solution entière}
            
            % % Draw the objective level line
            \addplot[dashed, thick, yellow, domain=0:4] {6.5 - 2*x};
            \addlegendentry{Ligne de niveau \(z=6.5\)}
            \end{axis}
        \end{tikzpicture}
    }
}

					En résolvant ce problème (graphiquement, par le simplexe ou avec un solveur), on trouve la solution optimale \(x^\ast = \left(3, \frac{1}{2}\right)\) avec \(z(x^\ast) = 6.5\).

					On peut ensuite faire le branchement sur \(x_2\):
					\begin{itemize}
						\item Cas \(x_2 \leq 0\):
						Le problème peut être représenté graphiquement comme dans la figure~\ref{fig:dm1_ex04_f3} ci-dessous:

						\input{../assets/tikz/dm1_ex04_f3.tex}

						En résolvant ce problème on trouve la solution optimale \(x^\ast = (3, 0)\) avec \(z(x^\ast) = 6\).
						C'est une solution entière, on peut donc la garder en tant que solution réalisable pour le problème à valeurs entières.
						\begin{equation*}
							x^\ast_{\bb N} = (3, 0), z(x^\ast_{\bb N}) = 6
						\end{equation*}

						\item Cas \(x_2 \geq 1\):
						Le problème peut être représenté graphiquement comme dans la figure~\ref{fig:dm1_ex04_f4} ci-dessous:

						\input{../assets/tikz/dm1_ex04_f4.tex}

						En résolvant ce problème on trouve la solution optimale \(x^\ast = \left(\frac{8}{3}, 1\right)\) avec \(z(x^\ast) = \frac{19}{3}\).
						Comme \(x_1\) n'est pas entier, on peut faire le branchement sur \(x_1\):

						\begin{itemize}
							\item Cas \(x_1 \leq 2\):
							Le problème peut être représenté graphiquement comme dans la figure~\ref{fig:dm1_ex04_f5} ci-dessous:

							\vspace{2mm}
\ffigbox[\FBwidth]{%
\caption{\centering Représentation du problème \(PL_0\) avec les contraintes \(x_1 \leq 2\) et \(x_2 \geq 1\)}\label{fig:dm1_ex04_f5}
}{
    \fbox{
        \begin{tikzpicture}
            \begin{axis}[
                xlabel={\(x_1\)},
                ylabel={\(x_2\)},
                xmin=0, xmax=4,
                ymin=0, ymax=4,
                grid=both,
                axis equal image,
                width=10cm,
                height=7cm,
                legend pos=outer north east,
                legend cell align=left,
                legend image post style={fill opacity=0.45},
                legend style={fill=pagebg, draw=pagetext}
            ]

            % ----- Constraint lines -----

            % 2x1 + 5x2 = 17 -> x2 = (17 - 2x1)/5
            \addplot[domain=0:4, very thick, color=blue] {(17 - 2*x)/5};
            \addlegendentry{\(2x_1 + 5x_2 = 17\)}

            % 3x1 + 2x2 = 10 -> x2 = (10 - 3x1)/2
            \addplot[domain=0:4, very thick, color=orange] {(10 - 3*x)/2};
            \addlegendentry{\(3x_1 + 2x_2 = 10\)}

            % x2 = 1
            \addplot[domain=0:4, very thick, color=red] {1};
            \addlegendentry{\(x_2 = 1\)}

            % manually draw line x1 = 3
            \addplot[
                very thick,
                color=green!60!black
            ] coordinates {(2,0) (2,4)};
            \addlegendentry{\(x_1 = 2\)}

            % ----- Feasible region -----
            \addplot[
                fill=cyan!30,
                opacity=0.45,
                draw=none,
                area legend
            ] coordinates {
                (0, 1)
                (0, 17/5)
                (32/22, 31/11)
                (2, 2)
                (2, 1)
            };
            \addlegendentry{Région réalisable}

            % ----- Optimal real solution -----
            \addplot[
                only marks,
                mark=*,
                mark size=2.5pt,
                color=red!80!black
            ] coordinates {(2, 2)};
            \addlegendentry{Solution réelle}
            
            
            % % Draw the objective level line
            \addplot[dashed, thick, yellow, domain=0:4] {6 - 2*x};
            \addlegendentry{Ligne de niveau \(z=6\)}
            \end{axis}
        \end{tikzpicture}
    }
}

							En résolvant ce problème on trouve la solution optimale \(x^\ast = (2, 2)\) avec \(z(x^\ast) = 6\).
							C'est une solution entière, on peut donc la garder en tant que solution réalisable pour le problème à valeurs entières.
							\begin{equation*}
								x^\ast_{\bb N} = (2, 2), z(x^\ast_{\bb N}) = 6
							\end{equation*}

							\item Cas \(x_1 \geq 3\):
							Le système de contraintes est alors incompatible, il n'y a pas de solution réalisable.
						\end{itemize}
					\end{itemize}

					\item Cas \(x_1 \geq 4\):
					Le système de contraintes est alors incompatible, il n'y a pas de solution réalisable.
				\end{itemize}
				On a traité tous les cas, on trouve que les solutions optimales à valeurs entières sont \(x^\ast_{\bb N} = (3, 0)\) et \(x^\ast_{\bb N} = (2, 2)\) avec \(z(x^\ast_{\bb N}) = 6\).
				L'arbre de branchement est représenté dans la figure~\ref{fig:dm1_ex04_f6} ci-dessous:
				
				\vspace{2mm}
\ffigbox[\FBwidth]{%
\caption{\centering Arbre de branchement du problème \(PL_0\)}\label{fig:dm1_ex04_f6}
}{
    \fbox{
        \begin{tikzpicture}[
            main node/.style={
                draw,
                rounded rectangle,
                rounded corners=3pt,
                fill=blue!20,
                inner sep=4pt,
                font=\scriptsize,
                align=center,
                text=black
            },
            solution/.style={
                draw=red,
                very thick,
                fill=red!8
            },
            optimal/.style={
                draw=green,
                very thick,
                fill=green!8
            },
            invalid/.style={
                draw=gray,
                dashed,
                text=gray,
                fill=gray!10
            },
        ]

        % root
        \node[main node] (s00) {
            \(z^* = \frac{20}{3}\)\\
            \(x^* = \left(\frac{10}{3}, 0\right)\)
        };

        % children (relative placement) -- tier 1
        \node[main node, below left=2cm and 2cm of s00] (s11) {
            \(z^* = \frac{13}{2}\)\\
            \(x^* = \left(3, \frac{1}{2}\right)\)
        };

        \node[main node, invalid, below right=2cm and 2cm of s00] (s12) {
            Impossible
        };

        % children (relative placement) -- tier 2
        \node[main node, optimal, below left=2cm and 1cm of s11] (s21) {
            \(z^* = 6\)\\
            \(x^* = \left(3, 0\right)\)
        };

        \node[main node, below right=2cm and 1cm of s11] (s22) {
            \(z^* = \frac{19}{3}\)\\
            \(x^* = \left(\frac{8}{3}, 1\right)\)
        };

        % children (relative placement) -- tier 3, we're now below s22
        \node[main node, optimal, below left=2cm and 1cm of s22] (s31) {
            \(z^* = 6\)\\
            \(x^* = \left(2, 2\right)\)
        };

        \node[main node, invalid, below right=2cm and 1cm of s22] (s32) {
            Impossible
        };

        % edges
        \draw[-{Stealth}] (s00) -- node[above left] {\(x_1 \leq 3\)} (s11);
        \draw[-{Stealth}] (s00) -- node[above right] {\(x_1 \geq 4\)} (s12);
        
        \draw[-{Stealth}] (s11) -- node[above left] {\(x_2 \leq 0\)} (s21);
        \draw[-{Stealth}] (s11) -- node[above right] {\(x_2 \geq 1\)} (s22);

        \draw[-{Stealth}] (s22) -- node[above left] {\(x_1 \leq 2\)} (s31);
        \draw[-{Stealth}] (s22) -- node[above right] {\(x_1 \geq 3\)} (s32);
        \end{tikzpicture}
    }
}

				On conclut cette recherche avec la solution optimale à valeurs entières \(x^\ast_{\bb N} = (3, 0)\) ou \(x^\ast_{\bb N} = (2, 2)\) avec \(z(x^\ast_{\bb N}) = 6\).

				On remarquera d'ailleurs qu'on aurait pu arrêter le branchement dès la première solution entière trouvée, puisque \(z_{\bb N}^\ast = \lfloor z^\ast_{\bb R} \rfloor = 6\).

				\item On passe à la méthode des coupes de Dantzig.
				On rappelle brièvement la coupe en question: 
				\begin{equation*}
					\sum_{j\in N} x_j \geq 1
				\end{equation*}
				pour \(N\) l'ensemble des variables hors base.

				La solution optimale du problème est \(x^\ast = \left(\frac{10}{3}, 0\right)\) avec \(z(x^\ast) = \frac{20}{3}\).
				Les variables hors base sont \(x_2\) et \(x_4\), on a donc \(N = \{2, 4\}\).
				La coupe de Dantzig est alors:
				\begin{equation*}
					x_2 + x_4 \geq 1
				\end{equation*}
				En ajoutant cette contrainte au problème, on trouve la solution
				\begin{equation*}
					x^\ast = \left(\frac{8}{3}, 1, \frac{20}{3}, 0\right), z(x^\ast) = \frac{19}{3}
				\end{equation*}

				La variable hors base est \(x_4\), on rajoute la contrainte \(x_4 \geq 1\) et on trouve la solution
				\begin{equation*}
					x^\ast = (3, 0, 11, 1), z(x^\ast) = 6
				\end{equation*}

				C'est une solution entière, on a donc trouvé la solution optimale à valeurs entières \(x^\ast_{\bb N} = (3, 0)\) avec \(z(x^\ast_{\bb N}) = 6\).
				
				\item On passe à la méthode des coupes de Gomory.
				On reprend le dernier tableau du simplexe pour le problème:
				\begin{center}
					\begin{tabular}{|ccc|cccc|} % chktex 44
						\hline  % chktex 44
						\, & \, &\(c\)&\(2\)&\(1\)&\(0\)&\(0\)\\
						\hline % chktex 44
						\multicolumn{1}{|c|}{\(c^J\)}& \multicolumn{2}{c|}{variables de base}&\(x_1\)&\(x_2\)&\(x_3\)&\(x_4\)\\
						\hline % chktex 44
						\multicolumn{1}{|c|}{\(0\)}& \multicolumn{1}{c|}{\(x_1^{2} = x_3\)}&\(\frac{31}{3}\)&\(0\)&\(\frac{11}{3}\)&\(1\)&\(-\frac{2}{3}\)\\
						\hline % chktex 44
						\multicolumn{1}{|c|}{\(2\)}& \multicolumn{1}{c|}{\(x_2^{2} = x_1\)}&\(\frac{10}{3}\)& \(1\) & \(\frac{2}{3}\) & \(0\) & \(\frac{1}{3}\)\\
						\hline % chktex 44
						\multicolumn{1}{|c|}{} & \(z(x)\) & \multicolumn{1}{|c|}{\(\frac{20}{3}\)}& \(0\) & \(\frac{1}{3}\) & \(0\) & \(\frac{2}{3}\)\\
						\hline % chktex 44
					\end{tabular}
				\end{center}

				On reprend la ligne de \(x_2^2\) pour construire la coupe de Gomory:
				\begin{equation*}
					\begin{aligned}
						&\left\langle \frac{2}{3} \right\rangle x_2 + \left\langle \frac{1}{3} \right\rangle x_4 = \left\langle \frac{10}{3} \right\rangle\\
						\iff & \frac{2}{3} x_2 + \frac{1}{3} x_4 \geq \frac{1}{3}\\
						\iff & 2 x_2 + x_4 \geq 1
					\end{aligned}
				\end{equation*}
				On remarque d'ailleurs qu'utiliser la ligne de \(x_1^2\) aurait donné la même contrainte.

				On a 
				\begin{equation*}
					x_4 = 10 - 3x_1 - 2x_2
				\end{equation*}
				ce qui donne
				\begin{equation*}
					\begin{aligned}
						&2 x_2 + 10 - 3x_1 - 2x_2 \geq 1\\
						\iff & -3 x_1 \geq -9\\
						\iff & x_1 \leq 3
					\end{aligned}
				\end{equation*}
				On ajoute cette contrainte au problème et on reprend encore une fois le dernier tableau du simplexe:
				\begin{center}
					\begin{tabular}{|ccc|ccccc|} % chktex 44
						\hline  % chktex 44
						\, & \, &\(c\)&\(2\)&\(1\)&\(0\)&\(0\)&\(0\)\\
						\hline % chktex 44
						\multicolumn{1}{|c|}{\(c^J\)}& \multicolumn{2}{c|}{variables de base}&\(x_1\)&\(x_2\)&\(x_3\)&\(x_4\)&\(x_5\)\\
						\hline % chktex 44
						\multicolumn{1}{|c|}{\(0\)}& \multicolumn{1}{c|}{\(x_1^{3} = x_3\)}&\(8.5\)&\(0\)&\(0\)&\(1\)&\(-2.5\)&\(5.5\)\\
						\hline % chktex 44
						\multicolumn{1}{|c|}{\(1\)}& \multicolumn{1}{c|}{\(x_2^{3} = x_2\)}&\(0.5\)& \(0\) & \(1\) & \(0\) & \(0.5\) & \(-1.5\)\\
						\hline % chktex 44
						\multicolumn{1}{|c|}{\(2\)}& \multicolumn{1}{c|}{\(x_3^{3} = x_1\)}&\(3\)& \(1\) & \(0\) & \(0\) & \(0\) & \(1\)\\
						\hline % chktex 44
						\multicolumn{1}{|c|}{} & \(z(x)\) & \multicolumn{1}{|c|}{\(6.5\)}& \(0\) & \(0\) & \(0\) & \(0.5\) & \(0.5\)\\
						\hline % chktex 44
					\end{tabular}
				\end{center}
				On reprend la ligne de \(x_1^3\) (ou celle de \(x_2^3\)) pour construire la coupe de Gomory suivante:
				\begin{equation*}
					\begin{aligned}
						0.5x_4 + 0.5 x_5 \geq 0.5
						\iff & x_4 + x_5 \geq 1\\
						\iff & 10 - 3x_1 - 2x_2 + (3 - x_1) \geq 1\\
						\iff & -4 x_1 - 2 x_2 \geq -12\\
						\iff & 2 x_1 + x_2 \leq 6
					\end{aligned}
				\end{equation*}
				On ajoute cette contrainte au problème et on trouve la solution optimale \(x^\ast = (3, 0)\) avec \(z(x^\ast) = 6\).
				C'est une solution entière, on a donc trouvé la solution optimale à valeurs entières \(x^\ast_{\bb N} = (3, 0)\) avec \(z(x^\ast_{\bb N}) = 6\).

				\item On passe à la méthode des coupes de Chvatal-Gomory.
				On veut trouver \(u = (u_1, u_2) \in \bb R^2_+\) tel que 
				\begin{equation*}
					\begin{aligned}
						\left( 2 u_1 + 3 u_2 \right) x_1 + \left( 5 u_1 + 2 u_2 \right) x_2 &\leq \lfloor 17 u_1 + 10 u_2 \rfloor\\
					\end{aligned}
				\end{equation*}
				avec \(2 u_1 + 3 u_2 \in \bb Z\) et \(5 u_1 + 2 u_2 \in \bb Z\).

				En particulier, on veut invalider la solution réelle optimale \(x^\ast = \left(\frac{10}{3}, 0\right)\).
				Cela revient à chercher \(u\) tel que
				\begin{equation*}\label{eq:dm1_ex04_eq1}
					\begin{aligned}
						&\left( 2 u_1 + 3 u_2 \right) \frac{10}{3} > \lfloor 17 u_1 + 10 u_2 \rfloor\\
						\iff & 20 u_1 + 30 u_2 > \lfloor 51 u_1 + 30 u_2 \rfloor
					\end{aligned}
				\end{equation*}
				Or \(u\) ne peut pas satisfaire l'inégalité~\eqref{eq:dm1_ex04_eq1} tout en maintenant les contraintes d'intégralité sur les coefficients de la coupe.

				Ainsi il faut d'abord rajouter une contrainte au problème pour faire en sorte que \(x^\ast\) ne soit plus réalisable, par exemple \(x_1 \leq 3\), ensuite on peut trouver une coupe de Chvatal-Gomory qui élimine la solution optimale du problème modifié.
			\end{enumerate}
		\end{enumerate}
	\end{td-sol}
}{}
