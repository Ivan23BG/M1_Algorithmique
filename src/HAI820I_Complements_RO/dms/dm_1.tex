\section{Partie théorique}\label{sec:s_1}

\subsection{Polyèdre entier}\label{subsec:ss1_1}
% ----- Consignes exo 1 ----- %
\begin{td-exo}[Représentation et facettes]\,\\ % 1 
	Soit \(P_{\varepsilon}\) le polyèdre défini par les inégalités linéaires suivantes:
    \begin{equation*}
        (P_{\varepsilon}) = 
        \begin{cases}
            x_2 \leq 3\\
            \varepsilon x_1 + (2 -\varepsilon) x_2 \leq 4\\
            x_i \geq 0,\forall i \in \{1, 2\}
        \end{cases}
    \end{equation*}
    \begin{enumerate}
        \item Illustrer le polyèdre \(P_\varepsilon\) et les inégalités dans le plan pour \(\varepsilon = 1\) et \(\varepsilon = -1\).

        \item Soient \(\varepsilon = 3\) et le polyèdre entier \(P_I = \conv(P_3 \cap \bb Z^2)\).
        Dessiner \(P_I\) et donner une représentation (extérieure) minimale de \(P_I\).
    \end{enumerate}
\end{td-exo}

% ----- Solutions exo 1 ----- %
\iftoggle{showsolutions}{ 
	\begin{td-sol}[]\ % 1
		\begin{enumerate}
            \item Pour \(\varepsilon = -1\), le problème \(P_\varepsilon\) devient le suivant:
            \begin{equation*}
                P_\varepsilon = P_{-1} = 
                \begin{cases}
                    x_2 \leq 3\\
                    - x_1 + 3 x_2 \leq 4\\
                    x_i \geq 0,\forall i \in \{1, 2\}
                \end{cases}
            \end{equation*}
            On peut représenter ce problème dans le plan comme suit:

            \vspace{2mm}
\ffigbox[\FBwidth]{%
\caption{\centering Représentation du problème \(P_\varepsilon\) dans \(\bb R^2\) pour \(\varepsilon = -1\)}\label{fig:dm1_ex01_f1}
}{
    \fbox{
        \begin{tikzpicture}
            \begin{axis}[
                % axis lines=middle,
                xlabel={\(x_1\)},
                ylabel={\(x_2\)},
                xmin=0, xmax=6.5,
                ymin=0, ymax=4,
                grid=both,
                axis equal image, % <-- orthonormal grid
                width=10cm,
                height=7cm,
                legend pos=outer north east, % <-- legend outside
                legend cell align=left,
                legend image post style={fill opacity=0.45},
                legend style={fill=pagebg, draw=pagetext}
            ]

            % ----- Constraint lines -----

            % x2 = 3  -> x2 = 3
            \addplot[domain=0:6.5, very thick, color=blue] {3};
            \addlegendentry{\(x_2 = 3\)}

            % -x1 + 3x2 = 4 -> x2 = (4 + x1)/3
            \addplot[domain=0:6.5, very thick, color=orange] {4/3 + x/3};
            \addlegendentry{\(-x_1 + 3x_2 = 4\)}

            % % x2 = 2
            % \addplot[domain=0:3, very thick, color=green!60!black] {2};
            % \addlegendentry{\(x_2 = 2\)}

            % ----- Feasible region -----
            \addplot[
                fill=cyan!30,
                opacity=0.45,
                draw=none,
                area legend
            ] coordinates {
                (0, 0)
                (0, 4/3)
                (5, 3)
                (6.5, 3)
                (6.5, 0)
            };
            \addlegendentry{Région réalisable}

            % % ----- Optimal real solution -----
            % \addplot[
            %     only marks,
            %     mark=*,
            %     mark size=2.5pt,
            %     color=red!80!black
            % ] coordinates {(0.75,0.25)};
            % \addlegendentry{Solution réelle}

            % % ----- Optimal integer solution -----
            % \addplot[
            %     only marks,
            %     mark=square*,
            %     mark size=2.5pt,
            %     color=purple
            % ] coordinates {(1,0)};
            % \addlegendentry{Solution entière}
            
            % % Draw the objective level line
            % \addplot[dashed, thick, yellow, domain=0:2] {3*x - 2};
            % \addlegendentry{Ligne de niveau \(z=2\)}

            % Draw the direction of decrease (vector)
            % \draw[->, thick, red!70!black] 
            %     (axis cs:0.75,0.25) -- (axis cs:0.75-0.6,0.25+0.2) 
            %     node[above left] {$(-3,1)$};
            \end{axis}
        \end{tikzpicture}
    }
}

            Quelques remarques par rapport à la figure:
            \begin{itemize}
                \item Tout d'abord on peut remarquer que l'ensemble des points dans le polyèdre est non borné en \(x_1\).
                \item Les points extrêmes de ce polyèdre sont:
                \begin{equation*}
                    (0, 0), (0, 4/3), (5, 3), (6.5, 3), (6.5, 0)
                \end{equation*}
            \end{itemize}

            Regardons maintenant le cas \(\varepsilon = 1\). Le problème \(P_\varepsilon\) devient le suivant:
            \begin{equation*}
                P_\varepsilon = P_{1} = 
                \begin{cases}
                    x_2 \leq 3\\
                    x_1 + x_2 \leq 4\\
                    x_i \geq 0,\forall i \in \{1, 2\}
                \end{cases}
            \end{equation*}
            On peut représenter ce problème dans le plan comme suit:

            \vspace{2mm}
\ffigbox[\FBwidth]{%
\caption{\centering Représentation du problème \(P_\varepsilon\) dans \(\bb R^2\) pour \(\varepsilon = 1\)}\label{fig:dm1_ex01_f2}
}{
    \fbox{
        \begin{tikzpicture}
            \begin{axis}[
                xlabel={\(x_1\)},
                ylabel={\(x_2\)},
                xmin=0, xmax=5,
                ymin=0, ymax=5,
                grid=both,
                axis equal image,
                width=10cm,
                height=7cm,
                legend pos=outer north east,
                legend cell align=left,
                legend image post style={fill opacity=0.45},
                legend style={fill=pagebg, draw=pagetext}
            ]

            % ----- Constraint lines -----

            % x2 = 3  -> x2 = 3
            \addplot[domain=0:5, very thick, color=blue] {3};
            \addlegendentry{\(x_2 = 3\)}

            % x1 + x2 = 4 -> x2 = 4 - x1
            \addplot[domain=0:5, very thick, color=orange] {4 - x};
            \addlegendentry{\(x_1 + x_2 = 4\)}

            % ----- Feasible region -----
            \addplot[
                fill=cyan!30,
                opacity=0.45,
                draw=none,
                area legend
            ] coordinates {
                (0, 0)
                (0, 3)
                (1, 3)
                (4, 0)
            };
            \addlegendentry{Région réalisable}
            \end{axis}
        \end{tikzpicture}
    }
}

            Quelques remarques par rapport à la figure:
            \begin{itemize}
                \item Tout d'abord on peut remarquer que l'ensemble des points dans le polyèdre est complètement borné cette fois-ci.
                Contrairement au cas \(\varepsilon = -1\), on ne peut pas choisir n'importe quelle valeur pour \(x_1\).
                \item Les points extrêmes de ce polyèdre sont:
                \begin{equation*}
                    (0, 0), (0, 3), (1, 3), (4, 0)
                \end{equation*}
                \item Ces deux problèmes, même s'ils viennent tous deux de \(P_\varepsilon\) ont des représentations dans l'espace bien différentes, cela est notamment dû à l'impact qu'a le choix de \(\varepsilon\) sur le signe des variables \(x_1\) et \(x_2\) dans la seconde contrainte. C'est ce qui décide \og{}l'angle de la pente\fg{} de cette contrainte et qui permet donc de borner ou non le polyèdre.
            \end{itemize}

            \item On fixe désormais \(\varepsilon = 3\). On commence par représenter le problème \(P_3\) dans l'espace comme suit:

            \vspace{2mm}
\ffigbox[\FBwidth]{%
\caption{\centering Représentation du problème \(P_\varepsilon\) dans \(\bb R^2\) pour \(\varepsilon = 3\)}\label{fig:dm1_ex01_f3}
}{
    \fbox{
        \begin{tikzpicture}
            \begin{axis}[
                xlabel={\(x_1\)},
                ylabel={\(x_2\)},
                xmin=0, xmax=3,
                ymin=0, ymax=4,
                grid=both,
                axis equal image,
                width=10cm,
                height=7cm,
                legend pos=outer north east,
                legend cell align=left,
                legend image post style={fill opacity=0.45},
                legend style={fill=pagebg, draw=pagetext}
            ]

            % ----- Constraint lines -----

            % x2 = 3  -> x2 = 3
            \addplot[domain=0:3, very thick, color=blue] {3};
            \addlegendentry{\(x_2 = 3\)}

            % 3x1 - x2 = 4 -> x2 = 3x1 - 4
            \addplot[domain=0:3, very thick, color=orange] {3*x - 4};
            \addlegendentry{\(3x_1 - x_2 = 4\)}

            % ----- Feasible region -----
            \addplot[
                fill=cyan!30,
                opacity=0.45,
                draw=none,
                area legend
            ] coordinates {
                (0, 0)
                (0, 3)
                (7/3, 3)
                (4/3, 0)
            };
            \addlegendentry{Région réalisable}
            \end{axis}
        \end{tikzpicture}
    }
}

            Ensuite, on veut contraindre cet ensemble de points pour se retrouver uniquement avec des valeurs entières. 
            Cela donne le graphe suivant:

            \vspace{2mm}
\ffigbox[\FBwidth]{%
\caption{\centering Représentation de l'ensemble de points \(P_3 \cap \bb Z^2\) dans \(\bb R^2\)}\label{fig:dm1_ex01_f4}
}{
    \fbox{
        \begin{tikzpicture}
            \begin{axis}[
                xlabel={\(x_1\)},
                ylabel={\(x_2\)},
                xmin=0, xmax=3,
                ymin=0, ymax=4,
                grid=both,
                axis equal image,
                width=10cm,
                height=7cm,
                legend pos=outer north east,
                legend cell align=left,
                legend image post style={fill opacity=0.45},
                legend style={fill=pagebg, draw=pagetext}
            ]

            % ----- Constraint lines -----

            % x2 = 3  -> x2 = 3
            \addplot[domain=0:3, very thick, color=blue] {3};
            \addlegendentry{\(x_2 = 3\)}

            % 3x1 - x2 = 4 -> x2 = 3x1 - 4
            \addplot[domain=0:3, very thick, color=orange] {3*x - 4};
            \addlegendentry{\(3x_1 - x_2 = 4\)}

            % ----- Feasible region -----
            \addplot[
                fill=cyan!30,
                opacity=0.45,
                draw=none,
                area legend
            ] coordinates {
                (0, 0)
                (0, 3)
                (7/3, 3)
                (4/3, 0)
            };
            \addlegendentry{Région réalisable de \(P_3\)}

            % ----- Integer points inside the feasible region -----
            \foreach \x in {0,1}{
                \foreach \y in {0,1,2,3}{
                    \addplot[
                        only marks,
                        mark=*,
                        mark size=2.2pt,
                        color=green!70
                    ] coordinates {(\x,\y)};
                }
            }
            \addplot[
                only marks,
                mark=*,
                mark size=2.2pt,
                color=green!70
            ] coordinates {(2,2)};
            \addplot[
                only marks,
                mark=*,
                mark size=2.2pt,
                color=green!70
            ] coordinates {(2,3)};
            \addlegendentry{Ensemble de points dans \(P_3 \cap \bb Z^2\)}
            \end{axis}
        \end{tikzpicture}
    }
}

            On veut maintenant trouver l'enveloppe convexe de cet ensemble de points ce qui donne le graphe suivant:

            \input{../assets/tikz/dm1_ex01_f5.tex}

            Enfin, on peut nettoyer un peu la figure pour voir plus clairement l'enveloppe convexe:

            \vspace{2mm}
\ffigbox[\FBwidth]{%
\caption{\centering Représentation de l'enveloppe convexe de \(P_3 \cap \bb Z^2\) dans \(\bb R^2\) nettoyée}\label{fig:dm1_ex01_f6}
}{
    \fbox{
        \begin{tikzpicture}
            \begin{axis}[
                % axis lines=middle,
                xlabel={\(x_1\)},
                ylabel={\(x_2\)},
                xmin=0, xmax=3,
                ymin=0, ymax=4,
                grid=both,
                axis equal image, % <-- orthonormal grid
                width=10cm,
                height=7cm,
                legend pos=outer north east, % <-- legend outside
                legend cell align=left,
                legend image post style={fill opacity=0.45},
                legend style={fill=pagebg, draw=pagetext}
            ]

            % ----- Constraint lines -----

            % % x2 = 3  -> x2 = 3
            % \addplot[domain=0:3, very thick, color=blue] {3};
            % \addlegendentry{\(x_2 = 3\)}

            % % 3x1 - x2 = 4 -> x2 = 3x1 - 4
            % \addplot[domain=0:3, very thick, color=orange] {3*x - 4};
            % \addlegendentry{\(3x_1 - x_2 = 4\)}

            % % x2 = 2
            % \addplot[domain=0:3, very thick, color=green!60!black] {2};
            % \addlegendentry{\(x_2 = 2\)}

            % % ----- Feasible region -----
            % \addplot[
            %     fill=cyan!30,
            %     opacity=0.45,
            %     draw=none,
            %     area legend
            % ] coordinates {
            %     (0, 0)
            %     (0, 3)
            %     (7/3, 3)
            %     (4/3, 0)
            % };
            % \addlegendentry{Région réalisable de \(P_3\)}

            % ----- Integer points inside the feasible region -----
            \foreach \x in {0,1}{
                \foreach \y in {0,1,2,3}{
                    \addplot[
                        only marks,
                        mark=*,
                        mark size=2.2pt,
                        color=green!70,
                        forget plot
                    ] coordinates {(\x,\y)};
                }
            }
            \addplot[
                only marks,
                mark=*,
                mark size=2.2pt,
                color=green!70,
                forget plot
            ] coordinates {(2,2)};
            \addplot[
                only marks,
                mark=*,
                mark size=2.2pt,
                color=green!70
            ] coordinates {(2,3)};
            \addlegendentry{Ensemble de points dans \(P_3 \cap \bb Z^2\)}
            
            % ----- Smaller Feasible region -----
            \addplot[
                fill=red!30,
                opacity=0.45,
                draw=none,
                area legend
            ] coordinates {
                (0, 0)
                (0, 3)
                (2, 3)
                (2, 2)
                (1, 0)
            };
            \addlegendentry{Enveloppe convexe de \(P_3 \cap \bb Z^2\)}

            % % ----- Optimal real solution -----
            % \addplot[
            %     only marks,
            %     mark=*,
            %     mark size=2.5pt,
            %     color=red!80!black
            % ] coordinates {(0.75,0.25)};
            % \addlegendentry{Solution réelle}

            % % ----- Optimal integer solution -----
            % \addplot[
            %     only marks,
            %     mark=square*,
            %     mark size=2.5pt,
            %     color=purple
            % ] coordinates {(1,0)};
            % \addlegendentry{Solution entière}
            
            % % Draw the objective level line
            % \addplot[dashed, thick, yellow, domain=0:2] {3*x - 2};
            % \addlegendentry{Ligne de niveau \(z=2\)}

            % Draw the direction of decrease (vector)
            % \draw[->, thick, red!70!black] 
            %     (axis cs:0.75,0.25) -- (axis cs:0.75-0.6,0.25+0.2) 
            %     node[above left] {$(-3,1)$};
            \end{axis}
        \end{tikzpicture}
    }
}

            On voit alors que ce polyèdre \(P_I\) est clairement défini par les trois contraintes suivantes:
            \begin{equation*}
                (P_I) = 
                \begin{cases}
                    x_2 \leq 3\\
                    x_1 \leq 2\\
                    2x_1 - x_2 \leq 2\\
                    x_i \geq 0,\ \forall i \in \{1, 2\}
                \end{cases}
            \end{equation*}
            ce qui donne le graphe suivant:

            \vspace{2mm}
\ffigbox[\FBwidth]{%
\caption{\centering Représentation du polyèdre \(P_I\) dans \(\bb R^2\)}\label{fig:dm1_ex01_f7}
}{
    \fbox{
        \begin{tikzpicture}
            \begin{axis}[
                xlabel={\(x_1\)},
                ylabel={\(x_2\)},
                xmin=0, xmax=3,
                ymin=0, ymax=4,
                grid=both,
                axis equal image,
                width=10cm,
                height=7cm,
                legend pos=outer north east,
                legend cell align=left,
                legend image post style={fill opacity=0.45},
                legend style={fill=pagebg, draw=pagetext}
            ]

            % ----- Constraint lines -----

            % x2 = 3  -> x2 = 3
            \addplot[domain=0:3, very thick, color=blue] {3};
            \addlegendentry{\(x_2 = 3\)}
            
            % x1 = 2
            \addplot[
                very thick,
                color=orange
            ] coordinates {(2,0) (2,4)};
            \addlegendentry{\(x_1 = 2\)}

            % 2x1 - x2 = 2 -> x2 = 2x1 - 2
            \addplot[domain=0:3, very thick, color=green!60!black] {2*x - 2};
            \addlegendentry{\(2x_1 - x_2 = 2\)}
            
            % ----- Smaller Feasible region -----
            \addplot[
                fill=red!30,
                opacity=0.45,
                draw=none,
                area legend
            ] coordinates {
                (0, 0)
                (0, 3)
                (2, 3)
                (2, 2)
                (1, 0)
            };
            \addlegendentry{Enveloppe convexe de \(P_I\)}
            \end{axis}
        \end{tikzpicture}
    }
}

            Quelques remarques sur \(P_I\):
            \begin{itemize}
                \item Comme pour le cas \(\varepsilon = 1\), l'ensemble des points du polyèdre est borné.
                Plus fort que ca, les différents points du polyèdre sont bornés par des points entiers.
                \item Les points extrêmes de ce polyèdre sont:
                \begin{equation*}
                    (0, 0), (0, 3), (2, 3), (2, 2), (1, 0)
                \end{equation*}
                \item Tous les points extrêmes de \(P_I\) sont entiers.
                \item Le polyèdre \(P_I\) est entièrement défini par 3 inégalités.
            \end{itemize}
        \end{enumerate}
	\end{td-sol}
}{}


\subsection{Optimisation combinatoire}\label{subsec:ss1_2}
% ----- Consignes exo 2 ----- %
\begin{td-exo}[Bornes géométriques pour le problème du \textsc{Voyageur de Commerce}]\,\\ % 2 
    On considère le problème du \textsc{Voyageur de Commerce} sur un graphe complet non-orienté formé de \(n\) villes.
    On note \(E\) l'ensemble des arêtes, formé des \(n(n-1)/2\) parties \(\{i,j\}\) de deux éléments distincts de \(\{1,\ldots,n\}\).
    On note \(d_{ij}\) la distance de la ville \(i\) à la ville \(j\).
    On rappelle qu'à un tour, on associe un vecteur \(x\in \bb R^E\) tel que \(x_{ij} = 1\) si l'arête \({i,j}\) appartient au tour et \(x_{ij}=0\) sinon.
    On considère le problème linéaire \(P_1\) (sans contraintes d'intégrité) suivant:
    \begin{equation*}
        (P_1) = 
        \begin{cases}
            \min \sum_{\{i,j\}\in E} d_{ij} x_{ij},\ x\in\bb R^E\\
            \sum_{j:\{i,j\}\in E} x_{ij} = 2,\ \forall i \in \{1,\ldots,n\}\\
            x_{ij} \geq 0,\ \forall \{i,j\} \in E
        \end{cases}
    \end{equation*}
\end{td-exo}

% ----- Solutions exo 2 ----- %
\iftoggle{showsolutions}{ 
	\begin{td-sol}[]\ % 2
		
	\end{td-sol}
}{}


% ----- Consignes exo 3 ----- %
\begin{td-exo}[]\,\\ % 3 
	
\end{td-exo}

% ----- Solutions exo 3 ----- %
\iftoggle{showsolutions}{ 
	\begin{td-sol}[]\ % 3
		
	\end{td-sol}
}{}

\subsection{Comparaison de méthodes}\label{subsec:ss1_3}
% ----- Consignes exo 4 ----- %
\begin{td-exo}[]\,\\ % 4 
	
\end{td-exo}

% ----- Solutions exo 4 ----- %
\iftoggle{showsolutions}{ 
	\begin{td-sol}[]\ % 4
		
	\end{td-sol}
}{}


% ----- Consignes exo 5 ----- %
\begin{td-exo}[]\,\\ % 5 
	
\end{td-exo}

% ----- Solutions exo 5 ----- %
\iftoggle{showsolutions}{ 
	\begin{td-sol}[]\ % 5
		
	\end{td-sol}
}{}


% ----- Consignes exo 6 ----- %
\begin{td-exo}[]\,\\ % 6 
	
\end{td-exo}

% ----- Solutions exo 6 ----- %
\iftoggle{showsolutions}{ 
	\begin{td-sol}[]\ % 6
		
	\end{td-sol}
}{}
