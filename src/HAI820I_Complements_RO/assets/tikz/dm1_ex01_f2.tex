\vspace{2mm}
\ffigbox[\FBwidth]{%
\caption{\centering Représentation du problème \(P_\varepsilon\) dans \(\bb R^2\) pour \(\varepsilon = 1\)}\label{fig:dm1_ex01_f2}
}{
    \fbox{
        \begin{tikzpicture}
            \begin{axis}[
                xlabel={\(x_1\)},
                ylabel={\(x_2\)},
                xmin=0, xmax=5,
                ymin=0, ymax=5,
                grid=both,
                axis equal image,
                width=10cm,
                height=7cm,
                legend pos=outer north east,
                legend cell align=left,
                legend image post style={fill opacity=0.45},
                legend style={fill=pagebg, draw=pagetext}
            ]

            % ----- Constraint lines -----

            % x2 = 3  -> x2 = 3
            \addplot[domain=0:5, very thick, color=blue] {3};
            \addlegendentry{\(x_2 = 3\)}

            % x1 + x2 = 4 -> x2 = 4 - x1
            \addplot[domain=0:5, very thick, color=orange] {4 - x};
            \addlegendentry{\(x_1 + x_2 = 4\)}

            % ----- Feasible region -----
            \addplot[
                fill=cyan!30,
                opacity=0.45,
                draw=none,
                area legend
            ] coordinates {
                (0, 0)
                (0, 3)
                (1, 3)
                (4, 0)
            };
            \addlegendentry{Région réalisable}
            \end{axis}
        \end{tikzpicture}
    }
}