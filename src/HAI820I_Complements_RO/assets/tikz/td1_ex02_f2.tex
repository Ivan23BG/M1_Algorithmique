\vspace{2mm}
\ffigbox[\FBwidth]{%
\caption{\centering Représentation du problème \(PL_2\) dans \(\bb R^2\)}\label{fig:td1_ex02_f2}
}{
    \fbox{
        \begin{tikzpicture}
            \begin{axis}[
                axis lines=middle,
                xlabel={\(x_1\)},
                ylabel={\(x_2\)},
                xmin=0, xmax=4,
                ymin=0, ymax=5,
                grid=both,
                axis equal image,
                width=10cm,
                height=7cm,
                legend pos=outer north east,
                legend cell align=left,
                legend image post style={fill opacity=0.45},
                legend style={fill=pagebg, draw=pagetext}
            ]

            % ----- Constraint line: 10x1 + 7x2 = 29 -----
            \addplot[domain=0:2.9, very thick, color=blue] { (29 - 10*x)/7 };
            \addlegendentry{\(10x_1 + 7x_2 = 29\)}

            % ----- Feasible region -----
            \addplot[
                fill=cyan!35,
                opacity=0.45,
                draw=none,
                area legend
            ] coordinates {
                (0,0)
                (2.9,0)
                (0,4.14)
            };
            \addlegendentry{Région réalisable}

            % ----- Real optimum -----
            \addplot[only marks, mark=*, mark size=3pt, color=red!80!black] coordinates {(2.9,0)};
            \addlegendentry{Solution réelle \((2.9,0)\)}

            % ----- Integer optimum -----
            \addplot[only marks, mark=square*, mark size=3pt, color=purple] coordinates {(0,4)};
            \addlegendentry{Solution entière \((0,4)\)}
            
            % Level line through real optimum
            \addplot[dashed, thick, yellow, domain=0:4.5] {-1.46*(x - 2.9)};
            \addlegendentry{Ligne de niveau \(z = 870\)}


            \end{axis}
        \end{tikzpicture}
    }
}