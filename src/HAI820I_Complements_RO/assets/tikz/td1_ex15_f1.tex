\vspace{2mm}
\ffigbox[\FBwidth]{%
\caption{\centering Représentation du problème \(PL\) initial}\label{fig:td1_ex15_f1}
}{
    \fbox{
        \begin{tikzpicture}
            \begin{axis}[
                xlabel={\(x_1\)},
                ylabel={\(x_2\)},
                xmin=0, xmax=4,
                ymin=0, ymax=4,
                grid=both,
                axis equal image,
                width=10cm,
                height=7cm,
                legend pos=outer north east,
                legend cell align=left,
                legend image post style={fill opacity=0.45},
                legend style={fill=pagebg, draw=pagetext}
            ]

            % ----- Constraint lines -----
            % just have to express x2 in terms of x1 for each constraint
            % x + y = 2 -> y = 2 - x

            % 2x1 + 5x2 = 17 -> x2 = (17 - 2x1)/5
            \addplot[domain=0:4, very thick, color=blue] {(17 - 2*x)/5};
            \addlegendentry{\(2x_1 + 5x_2 = 17\)}

            % 3x1 + 2x2 = 10 -> x2 = (10 - 3x1)/2
            \addplot[domain=0:4, very thick, color=orange] {(10 - 3*x)/2};
            \addlegendentry{\(3x_1 + 2x_2 = 10\)}

            % first cut
            % 6x1 + 3x2 = 19 -> x2 = (19 - 6x1)/3
            \addplot[domain=0:4, very thick, color=green] {(19 - 6*x)/3};
            \addlegendentry{\(6x_1 + 3x_2 = 19\)}

            % second cut
            % 30x1 + 12x2 = 94 -> x2 = (94 - 30x1)/12
            \addplot[domain=0:4, very thick, color=red] {(94 - 30*x)/12};
            \addlegendentry{\(30x_1 + 12x_2 = 94\)}

            % third cut
            % 66x1 + 27x2 = 206 -> x2 = (206 - 66x1)/27
            \addplot[domain=0:4, very thick, color=purple] {(206 - 66*x)/27};
            \addlegendentry{\(66x_1 + 27x_2 = 206\)}

            % optimal cut
            % x2 = -2x1 + 3
            \addplot[domain=0:4, very thick, dashed, color=magenta] {-2*x + 6};
            \addlegendentry{\(x_2 = -2x_1 + 6\)}

            % ----- Feasible region -----
            \addplot[
                fill=cyan!30,
                opacity=0.45,
                draw=none,
                area legend
            ] coordinates {
                (0,0)
                (0,17/5)
                (16/11,31/11)
                (10/3,0)
            };
            \addlegendentry{Région réalisable initiale}

            % ----- Optimal real solution -----
            \addplot[
                only marks,
                mark=*,
                mark size=2.5pt,
                color=red!80!black
            ] coordinates {(10/3,0)};
            \addlegendentry{Solution réelle}

            % ----- Optimal integer solution -----
            \addplot[
                only marks,
                mark=square*,
                mark size=2.5pt,
                color=purple
            ] coordinates {(3,0)};
            \addlegendentry{Solution entière}
            
            % ----- Level line for the objective function -----
            % z is a vector, the level line is orthogonal to the vector of coefficients of the objective function
            % 2x1 + x2 -> z = 2x1 + x2
            % we can express x2 in terms of z and x1: x2 = z - 2x1
            
            \addplot[dashed, thick, yellow, domain=0:4] {20/3 - 2*x};
            \addlegendentry{Ligne de niveau \(z=\frac{20}{3}\)}
            \end{axis}
        \end{tikzpicture}
    }
}