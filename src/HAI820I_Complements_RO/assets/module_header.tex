\makeatletter

% ==============================================================================
% TABLE OF CONTENTS FORMATTING
% ==============================================================================

% Adjusts indentation and spacing for different section levels in the TOC.

% ============================================================================== 
% GENERAL MATH SHORTCUTS
% ==============================================================================

\newcommand{\abs}[1]{\left\lvert#1\right\rvert}  % Absolute value with scalable bars
\newcommand{\Tau}{\mathcal{T}}                   % Calligraphic Tau, often used for topology

% ============================================================================== 
% SUBJECT-SPECIFIC THEOREM ENVIRONMENTS
% ==============================================================================

% --- Numbered theorem environments (per subsection) ---
% Uses colored left border and custom counter for clarity in notes.
\theoremstyle{default}
\mytheorem{definition}{Définition}{algebraic-amber}{subsection}{o}  % Definitions: amber, numbered per subsection
\mytheorem{exs}{Exemples}{matrix-mist}{subsection}{o}               % Examples: mist, numbered per subsection

% --- Unnumbered theorem environments ---
% Used for remarks, interpretations, and other statements that don't require numbering.
\theoremstyle{nonum}
\mytheorem{interp}{Interprétation}{verdant}{}{}    % Interpretation: green, unnumbered
\mytheorem{proposition}{Proposition}{verdant}{}{}  % Proposition: green, unnumbered
\mytheorem{notation}{Notation}{verdant}{}{}        % Notation: green, unnumbered
\mytheorem{example}{Exemple}{matrix-mist}{}{}      % Example: mist, unnumbered
\mytheorem{exo}{Exercice}{matrix-mist}{}{}         % Exercise: mist, unnumbered
\mytheorem{exercice}{Exercice}{matrix-mist}{}{}    % Exercise (alt): mist, unnumbered
\mytheorem{theorem}{Théorème}{astral}{}{}          % Theorem: blue, unnumbered
\mytheorem{lemma}{Lemme}{astral}{}{}               % Lemma: blue, unnumbered
\mytheorem{remark}{Remarque}{verdant}{}{}          % Remark: green, unnumbered

\makeatother

\counterwithout{tdcounter}{section} 

\usetikzlibrary{shapes,arrows,positioning,calc,fit,backgrounds,angles,quotes}

\newcommand{\becomes}{\begin{center}\(\downarrow\)\end{center}}


\usepackage{floatrow}
\usepackage{caption}
\usepackage{cases}


\usepackage{listings}
\usepackage{xcolor}

% ----- Julia language definition -----
\lstdefinelanguage{julia}{
    keywords={function, end, if, else, elseif, while, for, in, return, break, continue, struct, 
    mutable, using, import, module, export, const, let, global, local, abstract, typealias, 
    sin, atan, true, add},
    sensitive=true,
    comment=[l]{\#},
    morestring=[b]", % chktex 18
}
\definecolor{codebg}{RGB}{245,245,245}   % very light gray
\definecolor{codeborder}{RGB}{220,220,220} 
\definecolor{keywordcolor}{RGB}{0,0,150}
\definecolor{commentcolor}{RGB}{120,120,120}
\lstdefinestyle{julia-style}{
    language=julia, % new language
    basicstyle=\ttfamily\small,
    keywordstyle=\color{keywordcolor},
    commentstyle=\color{commentcolor},
    stringstyle=\color{verdant},
    backgroundcolor=\color{codebg},
    frame=single,
    framerule=0.5pt,
    rulecolor=\color{codeborder},
    tabsize=2,
    columns=fullflexible,
    keepspaces=true,
    showstringspaces=false,
}
% convenience macro
\newcommand{\juliaFile}[1]{\lstinputlisting[style=julia-style]{#1}}

% custom bash environment with same style as julia
\lstdefinelanguage{bash}{
    keywords={sudo, apt-get, install, update, upgrade, cd, ls, mkdir, rm, rmdir, touch, nano, vim, cat, echo, pwd, cp, mv},
    sensitive=true,
    comment=[l]{\#},
    morestring=[b]", % chktex 18
}
\lstdefinestyle{bash-style}{
    language=bash,
    basicstyle=\ttfamily\small,
    keywordstyle=\color{keywordcolor},
    commentstyle=\color{commentcolor},
    backgroundcolor=\color{codebg},
    frame=single,
    framerule=0.5pt,
    rulecolor=\color{codeborder},
    tabsize=2,
    columns=fullflexible,
    keepspaces=true,
    showstringspaces=false,
}
% convenience macro
\newcommand{\bashFile}[1]{\lstinputlisting[style=bash-style]{#1}}