\section{Convexité: ensembles et fonctions}\label{subsec:ss_1}

% ----- Consignes exo 1 ----- %
\begin{td-exo}[Représentation et facettes]\,\\ % 1 
	Soit \(P_{\varepsilon}\) le polyèdre défini par les inégalités linéaires suivantes:
    \begin{equation*}
        (P_{\varepsilon}) = 
        \begin{cases}
            x_2 \leq 3\\
            \varepsilon x_1 + (2 -\varepsilon) x_2 \leq 4\\
            x_i \geq 0,\forall i \in \{1, 2\}
        \end{cases}
    \end{equation*}
    \begin{enumerate}
        \item Illustrer le polyèdre \(P_\varepsilon\) et les inégalités dans le plan pour \(\varepsilon = 1\) et \(\varepsilon = -1\).

        \item Soient \(\varepsilon = 3\) et le polyèdre entier \(P_I = \conv(P_3 \cap \bb Z^2)\).
        Dessiner \(P_I\) et donner une représentation (extérieure) minimale de \(P_I\).
    \end{enumerate}
\end{td-exo}

% ----- Solutions exo 1 ----- %
\iftoggle{showsolutions}{ 
	\begin{td-sol}[]\ % 1
		\begin{enumerate}
            \item Pour \(\varepsilon = -1\), le problème \(P_\varepsilon\) devient le suivant:
            \begin{equation*}
                P_\varepsilon = P_{-1} = 
                \begin{cases}
                    x_2 \leq 3\\
                    - x_1 + 3 x_2 \leq 4\\
                    x_i \geq 0,\forall i \in \{1, 2\}
                \end{cases}
            \end{equation*}
            On peut représenter ce problème dans le plan comme suit:

            \vspace{2mm}
\ffigbox[\FBwidth]{%
\caption{\centering Représentation du problème \(P_\varepsilon\) dans \(\bb R^2\) pour \(\varepsilon = -1\)}\label{fig:dm1_ex01_f1}
}{
    \fbox{
        \begin{tikzpicture}
            \begin{axis}[
                % axis lines=middle,
                xlabel={\(x_1\)},
                ylabel={\(x_2\)},
                xmin=0, xmax=6.5,
                ymin=0, ymax=4,
                grid=both,
                axis equal image, % <-- orthonormal grid
                width=10cm,
                height=7cm,
                legend pos=outer north east, % <-- legend outside
                legend cell align=left,
                legend image post style={fill opacity=0.45},
                legend style={fill=pagebg, draw=pagetext}
            ]

            % ----- Constraint lines -----

            % x2 = 3  -> x2 = 3
            \addplot[domain=0:6.5, very thick, color=blue] {3};
            \addlegendentry{\(x_2 = 3\)}

            % -x1 + 3x2 = 4 -> x2 = (4 + x1)/3
            \addplot[domain=0:6.5, very thick, color=orange] {4/3 + x/3};
            \addlegendentry{\(-x_1 + 3x_2 = 4\)}

            % % x2 = 2
            % \addplot[domain=0:3, very thick, color=green!60!black] {2};
            % \addlegendentry{\(x_2 = 2\)}

            % ----- Feasible region -----
            \addplot[
                fill=cyan!30,
                opacity=0.45,
                draw=none,
                area legend
            ] coordinates {
                (0, 0)
                (0, 4/3)
                (5, 3)
                (6.5, 3)
                (6.5, 0)
            };
            \addlegendentry{Région réalisable}

            % % ----- Optimal real solution -----
            % \addplot[
            %     only marks,
            %     mark=*,
            %     mark size=2.5pt,
            %     color=red!80!black
            % ] coordinates {(0.75,0.25)};
            % \addlegendentry{Solution réelle}

            % % ----- Optimal integer solution -----
            % \addplot[
            %     only marks,
            %     mark=square*,
            %     mark size=2.5pt,
            %     color=purple
            % ] coordinates {(1,0)};
            % \addlegendentry{Solution entière}
            
            % % Draw the objective level line
            % \addplot[dashed, thick, yellow, domain=0:2] {3*x - 2};
            % \addlegendentry{Ligne de niveau \(z=2\)}

            % Draw the direction of decrease (vector)
            % \draw[->, thick, red!70!black] 
            %     (axis cs:0.75,0.25) -- (axis cs:0.75-0.6,0.25+0.2) 
            %     node[above left] {$(-3,1)$};
            \end{axis}
        \end{tikzpicture}
    }
}

            Quelques remarques par rapport à la figure:
            \begin{itemize}
                \item Tout d'abord on peut remarquer que l'ensemble des points dans le polyèdre est non borné en \(x_1\).
                \item Les points extrêmes de ce polyèdre sont:
                \begin{equation*}
                    (0, 0), (0, 4/3), (5, 3), (6.5, 3), (6.5, 0)
                \end{equation*}
            \end{itemize}

            Regardons maintenant le cas \(\varepsilon = 1\). Le problème \(P_\varepsilon\) devient le suivant:
            \begin{equation*}
                P_\varepsilon = P_{1} = 
                \begin{cases}
                    x_2 \leq 3\\
                    x_1 + x_2 \leq 4\\
                    x_i \geq 0,\forall i \in \{1, 2\}
                \end{cases}
            \end{equation*}
            On peut représenter ce problème dans le plan comme suit:

            \vspace{2mm}
\ffigbox[\FBwidth]{%
\caption{\centering Représentation du problème \(P_\varepsilon\) dans \(\bb R^2\) pour \(\varepsilon = 1\)}\label{fig:dm1_ex01_f2}
}{
    \fbox{
        \begin{tikzpicture}
            \begin{axis}[
                xlabel={\(x_1\)},
                ylabel={\(x_2\)},
                xmin=0, xmax=5,
                ymin=0, ymax=5,
                grid=both,
                axis equal image,
                width=10cm,
                height=7cm,
                legend pos=outer north east,
                legend cell align=left,
                legend image post style={fill opacity=0.45},
                legend style={fill=pagebg, draw=pagetext}
            ]

            % ----- Constraint lines -----

            % x2 = 3  -> x2 = 3
            \addplot[domain=0:5, very thick, color=blue] {3};
            \addlegendentry{\(x_2 = 3\)}

            % x1 + x2 = 4 -> x2 = 4 - x1
            \addplot[domain=0:5, very thick, color=orange] {4 - x};
            \addlegendentry{\(x_1 + x_2 = 4\)}

            % ----- Feasible region -----
            \addplot[
                fill=cyan!30,
                opacity=0.45,
                draw=none,
                area legend
            ] coordinates {
                (0, 0)
                (0, 3)
                (1, 3)
                (4, 0)
            };
            \addlegendentry{Région réalisable}
            \end{axis}
        \end{tikzpicture}
    }
}

            Quelques remarques par rapport à la figure:
            \begin{itemize}
                \item Tout d'abord on peut remarquer que l'ensemble des points dans le polyèdre est complètement borné cette fois-ci.
                Contrairement au cas \(\varepsilon = -1\), on ne peut pas choisir n'importe quelle valeur pour \(x_1\).
                \item Les points extrêmes de ce polyèdre sont:
                \begin{equation*}
                    (0, 0), (0, 3), (1, 3), (4, 0)
                \end{equation*}
                \item Ces deux problèmes, même s'ils viennent tous deux de \(P_\varepsilon\) ont des représentations dans l'espace bien différentes, cela est notamment dû à l'impact qu'a le choix de \(\varepsilon\) sur le signe des variables \(x_1\) et \(x_2\) dans la seconde contrainte. C'est ce qui décide \og{}l'angle de la pente\fg{} de cette contrainte et qui permet donc de borner ou non le polyèdre.
            \end{itemize}

            \item On fixe désormais \(\varepsilon = 3\). On commence par représenter le problème \(P_3\) dans l'espace comme suit:

            \vspace{2mm}
\ffigbox[\FBwidth]{%
\caption{\centering Représentation du problème \(P_\varepsilon\) dans \(\bb R^2\) pour \(\varepsilon = 3\)}\label{fig:dm1_ex01_f3}
}{
    \fbox{
        \begin{tikzpicture}
            \begin{axis}[
                xlabel={\(x_1\)},
                ylabel={\(x_2\)},
                xmin=0, xmax=3,
                ymin=0, ymax=4,
                grid=both,
                axis equal image,
                width=10cm,
                height=7cm,
                legend pos=outer north east,
                legend cell align=left,
                legend image post style={fill opacity=0.45},
                legend style={fill=pagebg, draw=pagetext}
            ]

            % ----- Constraint lines -----

            % x2 = 3  -> x2 = 3
            \addplot[domain=0:3, very thick, color=blue] {3};
            \addlegendentry{\(x_2 = 3\)}

            % 3x1 - x2 = 4 -> x2 = 3x1 - 4
            \addplot[domain=0:3, very thick, color=orange] {3*x - 4};
            \addlegendentry{\(3x_1 - x_2 = 4\)}

            % ----- Feasible region -----
            \addplot[
                fill=cyan!30,
                opacity=0.45,
                draw=none,
                area legend
            ] coordinates {
                (0, 0)
                (0, 3)
                (7/3, 3)
                (4/3, 0)
            };
            \addlegendentry{Région réalisable}
            \end{axis}
        \end{tikzpicture}
    }
}

            Ensuite, on veut contraindre cet ensemble de points pour se retrouver uniquement avec des valeurs entières. 
            Cela donne le graphe suivant:

            \vspace{2mm}
\ffigbox[\FBwidth]{%
\caption{\centering Représentation de l'ensemble de points \(P_3 \cap \bb Z^2\) dans \(\bb R^2\)}\label{fig:dm1_ex01_f4}
}{
    \fbox{
        \begin{tikzpicture}
            \begin{axis}[
                xlabel={\(x_1\)},
                ylabel={\(x_2\)},
                xmin=0, xmax=3,
                ymin=0, ymax=4,
                grid=both,
                axis equal image,
                width=10cm,
                height=7cm,
                legend pos=outer north east,
                legend cell align=left,
                legend image post style={fill opacity=0.45},
                legend style={fill=pagebg, draw=pagetext}
            ]

            % ----- Constraint lines -----

            % x2 = 3  -> x2 = 3
            \addplot[domain=0:3, very thick, color=blue] {3};
            \addlegendentry{\(x_2 = 3\)}

            % 3x1 - x2 = 4 -> x2 = 3x1 - 4
            \addplot[domain=0:3, very thick, color=orange] {3*x - 4};
            \addlegendentry{\(3x_1 - x_2 = 4\)}

            % ----- Feasible region -----
            \addplot[
                fill=cyan!30,
                opacity=0.45,
                draw=none,
                area legend
            ] coordinates {
                (0, 0)
                (0, 3)
                (7/3, 3)
                (4/3, 0)
            };
            \addlegendentry{Région réalisable de \(P_3\)}

            % ----- Integer points inside the feasible region -----
            \foreach \x in {0,1}{
                \foreach \y in {0,1,2,3}{
                    \addplot[
                        only marks,
                        mark=*,
                        mark size=2.2pt,
                        color=green!70
                    ] coordinates {(\x,\y)};
                }
            }
            \addplot[
                only marks,
                mark=*,
                mark size=2.2pt,
                color=green!70
            ] coordinates {(2,2)};
            \addplot[
                only marks,
                mark=*,
                mark size=2.2pt,
                color=green!70
            ] coordinates {(2,3)};
            \addlegendentry{Ensemble de points dans \(P_3 \cap \bb Z^2\)}
            \end{axis}
        \end{tikzpicture}
    }
}

            On veut maintenant trouver l'enveloppe convexe de cet ensemble de points ce qui donne le graphe suivant:

            \input{../assets/tikz/dm1_ex01_f5.tex}

            Enfin, on peut nettoyer un peu la figure pour voir plus clairement l'enveloppe convexe:

            \vspace{2mm}
\ffigbox[\FBwidth]{%
\caption{\centering Représentation de l'enveloppe convexe de \(P_3 \cap \bb Z^2\) dans \(\bb R^2\) nettoyée}\label{fig:dm1_ex01_f6}
}{
    \fbox{
        \begin{tikzpicture}
            \begin{axis}[
                % axis lines=middle,
                xlabel={\(x_1\)},
                ylabel={\(x_2\)},
                xmin=0, xmax=3,
                ymin=0, ymax=4,
                grid=both,
                axis equal image, % <-- orthonormal grid
                width=10cm,
                height=7cm,
                legend pos=outer north east, % <-- legend outside
                legend cell align=left,
                legend image post style={fill opacity=0.45},
                legend style={fill=pagebg, draw=pagetext}
            ]

            % ----- Constraint lines -----

            % % x2 = 3  -> x2 = 3
            % \addplot[domain=0:3, very thick, color=blue] {3};
            % \addlegendentry{\(x_2 = 3\)}

            % % 3x1 - x2 = 4 -> x2 = 3x1 - 4
            % \addplot[domain=0:3, very thick, color=orange] {3*x - 4};
            % \addlegendentry{\(3x_1 - x_2 = 4\)}

            % % x2 = 2
            % \addplot[domain=0:3, very thick, color=green!60!black] {2};
            % \addlegendentry{\(x_2 = 2\)}

            % % ----- Feasible region -----
            % \addplot[
            %     fill=cyan!30,
            %     opacity=0.45,
            %     draw=none,
            %     area legend
            % ] coordinates {
            %     (0, 0)
            %     (0, 3)
            %     (7/3, 3)
            %     (4/3, 0)
            % };
            % \addlegendentry{Région réalisable de \(P_3\)}

            % ----- Integer points inside the feasible region -----
            \foreach \x in {0,1}{
                \foreach \y in {0,1,2,3}{
                    \addplot[
                        only marks,
                        mark=*,
                        mark size=2.2pt,
                        color=green!70,
                        forget plot
                    ] coordinates {(\x,\y)};
                }
            }
            \addplot[
                only marks,
                mark=*,
                mark size=2.2pt,
                color=green!70,
                forget plot
            ] coordinates {(2,2)};
            \addplot[
                only marks,
                mark=*,
                mark size=2.2pt,
                color=green!70
            ] coordinates {(2,3)};
            \addlegendentry{Ensemble de points dans \(P_3 \cap \bb Z^2\)}
            
            % ----- Smaller Feasible region -----
            \addplot[
                fill=red!30,
                opacity=0.45,
                draw=none,
                area legend
            ] coordinates {
                (0, 0)
                (0, 3)
                (2, 3)
                (2, 2)
                (1, 0)
            };
            \addlegendentry{Enveloppe convexe de \(P_3 \cap \bb Z^2\)}

            % % ----- Optimal real solution -----
            % \addplot[
            %     only marks,
            %     mark=*,
            %     mark size=2.5pt,
            %     color=red!80!black
            % ] coordinates {(0.75,0.25)};
            % \addlegendentry{Solution réelle}

            % % ----- Optimal integer solution -----
            % \addplot[
            %     only marks,
            %     mark=square*,
            %     mark size=2.5pt,
            %     color=purple
            % ] coordinates {(1,0)};
            % \addlegendentry{Solution entière}
            
            % % Draw the objective level line
            % \addplot[dashed, thick, yellow, domain=0:2] {3*x - 2};
            % \addlegendentry{Ligne de niveau \(z=2\)}

            % Draw the direction of decrease (vector)
            % \draw[->, thick, red!70!black] 
            %     (axis cs:0.75,0.25) -- (axis cs:0.75-0.6,0.25+0.2) 
            %     node[above left] {$(-3,1)$};
            \end{axis}
        \end{tikzpicture}
    }
}

            On voit alors que ce polyèdre \(P_I\) est clairement défini par les trois contraintes suivantes:
            \begin{equation*}
                (P_I) = 
                \begin{cases}
                    x_2 \leq 3\\
                    x_1 \leq 2\\
                    2x_1 - x_2 \leq 2\\
                    x_i \geq 0,\ \forall i \in \{1, 2\}
                \end{cases}
            \end{equation*}
            ce qui donne le graphe suivant:

            \vspace{2mm}
\ffigbox[\FBwidth]{%
\caption{\centering Représentation du polyèdre \(P_I\) dans \(\bb R^2\)}\label{fig:dm1_ex01_f7}
}{
    \fbox{
        \begin{tikzpicture}
            \begin{axis}[
                xlabel={\(x_1\)},
                ylabel={\(x_2\)},
                xmin=0, xmax=3,
                ymin=0, ymax=4,
                grid=both,
                axis equal image,
                width=10cm,
                height=7cm,
                legend pos=outer north east,
                legend cell align=left,
                legend image post style={fill opacity=0.45},
                legend style={fill=pagebg, draw=pagetext}
            ]

            % ----- Constraint lines -----

            % x2 = 3  -> x2 = 3
            \addplot[domain=0:3, very thick, color=blue] {3};
            \addlegendentry{\(x_2 = 3\)}
            
            % x1 = 2
            \addplot[
                very thick,
                color=orange
            ] coordinates {(2,0) (2,4)};
            \addlegendentry{\(x_1 = 2\)}

            % 2x1 - x2 = 2 -> x2 = 2x1 - 2
            \addplot[domain=0:3, very thick, color=green!60!black] {2*x - 2};
            \addlegendentry{\(2x_1 - x_2 = 2\)}
            
            % ----- Smaller Feasible region -----
            \addplot[
                fill=red!30,
                opacity=0.45,
                draw=none,
                area legend
            ] coordinates {
                (0, 0)
                (0, 3)
                (2, 3)
                (2, 2)
                (1, 0)
            };
            \addlegendentry{Enveloppe convexe de \(P_I\)}
            \end{axis}
        \end{tikzpicture}
    }
}

            Quelques remarques sur \(P_I\):
            \begin{itemize}
                \item Comme pour le cas \(\varepsilon = 1\), l'ensemble des points du polyèdre est borné.
                Plus fort que ca, les différents points du polyèdre sont bornés par des points entiers.
                \item Les points extrêmes de ce polyèdre sont:
                \begin{equation*}
                    (0, 0), (0, 3), (2, 3), (2, 2), (1, 0)
                \end{equation*}
                \item Tous les points extrêmes de \(P_I\) sont entiers.
                \item Le polyèdre \(P_I\) est entièrement défini par 3 inégalités.
            \end{itemize}
        \end{enumerate}
	\end{td-sol}
}{} %

% % ----- Consignes exo 2 ----- %
\begin{td-exo}[Autour de \textsc{Satisfaisabilité}]\, % 2 
	\vspace{-6mm}
\begin{algorithm}[H]
    \caption{\textsc{Non Egal Satisfaisabilité} (NAESAT)}
    \begin{algorithmic}
        \Require{Une formule conjonctive \(\varphi\) sur \(n\) variables et \(m\) clauses}
        \Ensure{Existe-t-il une affectation de valeurs de vérité aux variables qui satisfasse \(\varphi\) tel que chaque clause a un littéral vrai et un littéral faux?}
    \end{algorithmic}
\end{algorithm}
	
	Montrer que \textsc{Non Egal Satisfaisabilité} est \(\mathcal{NP}\)-\textit{complet}. 
	La preuve se fera à partir de \textsc{Satisfaisabilité}.
\end{td-exo}

% ----- Solutions exo 2 ----- %
\iftoggle{showsolutions}{ 
	\begin{td-sol}[]\ % 2
		Le problème de \textsc{Non Egal Satisfaisabilité} est dans \(\mathcal{NP}\) car on peut vérifier en temps polynomial si deux formules booléennes sont non équivalentes. 
		Pour montrer qu'il est \(\mathcal{NP}\)-complet, on effectue une réduction à partir du problème de \textsc{Satisfaisabilité}. 
		On montre que si on pouvait résoudre \textsc{Non Egal Satisfaisabilité} en temps polynomial, alors on pourrait résoudre \textsc{Satisfaisabilité} en temps polynomial, ce qui contredit l'hypothèse que \(\mathcal{P} \neq \mathcal{NP}\). 
		Par conséquent, \textsc{Non Egal Satisfaisabilité} est \(\mathcal{NP}\)-complet.

		Soit \(\varphi\) une formule conjonctive de \(3\)-\textsc{SAT}. 
		On construit une formule \(\psi\) pour le problème de \textsc{Non Egal Satisfaisabilité} de la manière suivante:
		\begin{equation*}
			\psi_i = \varphi_i \land (x_1 \lor \ldots \lor x_n) \land (\lnot x_1 \lor \ldots \lor \lnot x_n), \quad \forall i \in \{1, \ldots, m\}
		\end{equation*}

		Ensuite, on pose \(\psi = \bigwedge_{i=1}^m \psi_i\).
		
		Si \(\varphi\) est satisfaisable, alors il existe une affectation de valeurs de vérité aux variables qui satisfait \(\varphi\). 
		Cette même affectation satisfait également \(\psi\) car les clauses supplémentaires garantissent que chaque clause de \(\psi\) a un littéral vrai et un littéral faux.
		
		Inversement, si \(\psi\) est satisfaisable, alors il existe une affectation de valeurs de vérité aux variables qui satisfait \(\psi\). 
		Cette affectation doit également satisfaire \(\varphi\) (car si elle satisfait chaque clause de \(\psi\)), alors elle satisfait également chaque clause de \(\varphi\). 
		
		Par conséquent, \(\varphi\) est satisfaisable si et seulement si \(\psi\) est satisfaisable, ce qui montre que le problème de \textsc{Non Egal Satisfaisabilité} est \(\mathcal{NP}\)-complet (car il est au moins aussi dur que \textsc{Satisfaisabilité}, qui est lui-même \(\mathcal{NP}\)-complet).
	\end{td-sol}
}{}
 %

% % ----- Consignes exo 3 ----- %
\begin{td-exo}[Quelques preuves de NP-complétude pour \textsc{Steiner tree}]\,\\ % 3 
	\input{../assets/tikz/td1_ex03_1.tex}
	\input{../assets/tikz/td1_ex03_2.tex}
	\input{../assets/tikz/td1_ex03_3.tex}

	\begin{enumerate}
		\item Montrer que \textsc{Steiner Tree} est NP-complet.
		Procéder à une réduction à partir de \textsc{Exact Cover by 3-Sets}.

		\item Montrer que \textsc{Steiner Tree} est NP-complet par une réduction à partir de \textsc{Vertex Cover}.
	\end{enumerate}
\end{td-exo}

% ----- Solutions exo 3 ----- %
\iftoggle{showsolutions}{ 
	\begin{td-sol}[]\ % 3
		
	\end{td-sol}
}{}
 %

% % ----- Consignes exo 4 ----- %
\begin{td-exo}[]\,\\ % 4 
	Coder dans Sagemath les opérations d'addition et de multiplication de deux polynomes qui sont donnés par la liste de leurs coefficients. 
	Le premier élément de la liste correspondra à l'élément constant du polynome.
	Vous testerez vos fonctions sur des polynomes aléatoires de degrés dans \(\left\{10, 100, 1000\right\}\) avec les domaines de coefficients suivants: \(\bb{GF}(17), \bb Z, \bb Z/100\bb Z\).
	A chaque fois vous prendrez soin de faire des opérations impliquant aussi bien des polynomes de degrés identiques que des polynomes de degrés différents.
	Vous devrez comparer les résultats de vos calculs avec celui fait par les opérations natives d'addition et de multiplication de polynomes dans Sage.
\end{td-exo}

% ----- Solutions exo 4 ----- %
\iftoggle{showsolutions}{ 
	\begin{td-sol}[]\ % 4
		On utilise le code suivant:

		\SageFile{../tps/exo04_f01.py}{Fonction d'addition de polynomes}{lst:ex04_f01}

	\end{td-sol}
}{}
 %

% % ----- Consignes exo 5 ----- %
\begin{td-exo}[title]\,\\ % 5 
	
\end{td-exo}

% ----- Solutions exo 5 ----- %
\iftoggle{showsolutions}{ 
	\begin{td-sol}[]\ % 5
		
	\end{td-sol}
}{}
 %

% % ----- Consignes exo 6 ----- %
\begin{td-exo}[Contraintes avec une seule variable]\,\\ % 6 
	On considère un problème de programmation linéaire standard, à une seule contrainte, défini par
	\begin{equation*}
		\max \sum_{j=1}^{n} u_j x_j
	\end{equation*}
	avec 
	\begin{equation*}
		\sum_{j=1}^{n} v_j x_j \leq V,\text{ et } x_j\geq0,\quad\text{pour } 1 \leq j \leq n
	\end{equation*}
	Tous les coefficient \(u_j\) et \(v_j\) sont supposés strictement positifs et l'on suppose les variables classées par rapports utilité/volume décroissants ou, pour rester dans un formalisme plus mathématique, suivant les valeurs décroissantes des rapports \(\frac{u_j}{v_j}\).
	Montrer que la variable \(x_1\) est entrante et que, en la faisant entrer en base, on atteint l'optimum de l'objectif en une seule étape.
	Exprimer la valeur de l'objectif en fonction des différents coefficients.
\end{td-exo}

% ----- Solutions exo 6 ----- %
\iftoggle{showsolutions}{ 
	\begin{td-sol}[]\ % 6
		Comme les variables sont classées par rapports utilité/volume décroissants, la variable \(x_1\) a le rapport le plus élevé \(\frac{u_1}{v_1}\). En faisant entrer \(x_1\) en base, on maximise immédiatement l'utilité par unité de volume consommé.

		En entrant \(x_1\) en base, on peut allouer autant que possible à \(x_1\) sans dépasser la contrainte de volume. La quantité maximale de \(x_1\) que l'on peut allouer est donnée par
		\begin{equation*}
			x_1 = \frac{V}{v_1}
		\end{equation*}
		sous réserve que \(x_1 \leq 1\) (si \(x_1\) est une variable binaire, sinon elle peut être continue).

		L'objectif atteint en faisant entrer \(x_1\) en base est alors
		\begin{equation*}
			u_1 x_1 = u_1 \frac{V}{v_1} = V \frac{u_1}{v_1}
		\end{equation*}

		Cette solution est optimale car toute autre variable \(x_j\) avec \(j > 1\) aurait un rapport utilité/volume inférieur à celui de \(x_1\), ce qui signifie que pour chaque unité de volume consommée, on obtiendrait moins d'utilité que si on avait alloué ce volume à \(x_1\).
	\end{td-sol}
}{}
 %

% % ----- Consignes exo 7 ----- %
\begin{td-exo}[Fonction convexe]\ % 7
	\begin{enumerate}
		\item Est-ce qu'une combinaison linéaire à coefficients positifs de fonctions convexes est convexe?

		\item Est-ce que le produit de deux fonctions convexes est convexe?

		\item Si \(f_1\) et \(f_2\) sont deux fonctions convexes, est-ce que \(\max\left(f_1,f_2\right)\) est convexe?

		\item Montrer que la fonction \(f\ \colon\ x\mapsto x^2\) est une fonction convexe sur \(\bb R\).
	\end{enumerate}
\end{td-exo}

% ----- Solutions exo 7 ----- %
\iftoggle{showsolutions}{
	\begin{td-sol}[]\ %
		\begin{enumerate}
			\item Oui. On pose \(g(x) = \sum_{i\in I} \alpha_i f_i(x)\). Alors
			\begin{equation*}
				\begin{aligned}
					g(\lambda x + (1-\lambda) y) 
					& = \sum_{i\in I} \alpha_i f_i(\lambda x + (1-\lambda) y) \\
					& \leq \sum_{i\in I} \alpha_i \left(\lambda f_i(x) + (1-\lambda) f_i(y)\right) \\
					& = \lambda g(x) + (1-\lambda) g(y).
				\end{aligned}
			\end{equation*}

			\item Non. Par exemple, \(f_1(x) = x\) et \(f_2(x) = x^2\) sont convexes mais \(f_1(x) f_2(x) = x^3\) n'est pas convexe.

			\item Oui. On pose \(g(x) = \max\left(f_1(x), f_2(x)\right)\). Alors
			\begin{equation*}
				\begin{aligned}
					g(\lambda x + (1-\lambda) y) 
					& = \max\left(f_1(\lambda x + (1-\lambda) y), f_2(\lambda x + (1-\lambda) y)\right) \\
					& \leq \max\left(\lambda f_1(x) + (1-\lambda) f_1(y), \lambda f_2(x) + (1-\lambda) f_2(y)\right) \\
					& \leq \lambda \max\left(f_1(x), f_2(x)\right) + (1-\lambda) \max\left(f_1(y), f_2(y)\right) \\
					& = \lambda g(x) + (1-\lambda) g(y).
				\end{aligned}
			\end{equation*}

			\item Soit \(x,y \in \bb R\) et \(\lambda \in [0,1]\). Alors
			\begin{equation*}
				\begin{aligned}
					f(\lambda x + (1-\lambda) y) 
					& = {(\lambda x + (1-\lambda) y)}^2 \\
					& = \lambda^2 x^2 + {(1-\lambda)}^2 y^2 + 2\lambda(1-\lambda) xy \\
					& \iff \lambda^2 x^2 + {(1-\lambda)}^2 y^2 + 2\lambda(1-\lambda) xy - \lambda x^2 - (1-\lambda) y^2 \leq 0 \\
					& \iff \lambda (1-\lambda)\left(\frac{\lambda}{1-\lambda} x^2 + \frac{1-\lambda}{\lambda} y^2 + 2xy - \frac{x^2}{1-\lambda} - \frac{y^2}{\lambda}\right) \leq 0 \\
					& \iff \lambda (1-\lambda)\left(-{\left(x - y\right)}^2\right) \leq 0.
				\end{aligned}
			\end{equation*}
			Or tous les termes sont positifs sauf le dernier. Donc l'inégalité est vérifiée.
		\end{enumerate}
	\end{td-sol}
}{}
 %

% % ----- Consignes exo 8 ----- %
\begin{td-exo}[Simples inégalités valides]\,\\ % 8 
	Considérons l'ensemble suivant:
	
	\begin{equation*}
		X = \{x_i \in \{0,1\}:\ 3x_1 - 4x_2 + 2x_3 + x_5 \leq -2 \}
	\end{equation*}

	Donner deux équations valides.
\end{td-exo}

% ----- Solutions exo 8 ----- %
\iftoggle{showsolutions}{ 
	\begin{td-sol}[]\ % 8
		
	\end{td-sol}
}{}
 %

% \section{Divers formulations}

% % ----- Consignes exo 9 ----- %
\begin{td-exo}[Reformulation en programme linéaire]\,\\% 9
	Reformuler les problèmes suivants sous forme de programme linéaire.
	\begin{enumerate}
		\item \begin{equation*}
			\begin{cases}
				\min z = 2x_1 + 3 \n{x_2 - 10} \\
				\n{x_1 + 2} + \n {x_2} \leq 5\\
				2x_1 + x_2 \leq 4
			\end{cases}
		\end{equation*}
	\end{enumerate}

	\item Soit un ensemble d'inégalités linéaires 
	\begin{equation*}
		a_i^T x \leq b_i, \quad i=1,\ldots,m,a_i\in\bb R^n,b_i\in\bb R.
	\end{equation*}
	Formuler un modèle (uniquement des contraintes sans fonction objectif) pour lequel un point \(x\in\bb N^n\)
	satisfait au moins \(k\) des \(m\) contraintes (\(k\leq m\)) entiers de plus satisfaite
	\begin{equation*}
		0\leq x_j \leq M,\quad \forall j \in \{1,\ldots,m\}.
	\end{equation*}
\end{td-exo}

% ----- Solutions exo 9 ----- %
\iftoggle{showsolutions}{
	\begin{td-sol}[]\ %
		A remplir %TODO solve exercise 9
	\end{td-sol}
}{}
 %

% % ----- Consignes exo 10 ----- %
\begin{td-exo}[Linéarisation de fonctions non linéaires] % 10
	A remplir %TODO write exercise 10
\end{td-exo}

% ----- Solutions exo 10 ----- %
\iftoggle{showsolutions}{
	\begin{td-sol}[]\ %
		A remplir %TODO solve exercise 10
	\end{td-sol}
}{}
 %

% % ----- Consignes exo 11 ----- %
\begin{td-exo}[Simple inégalité valide]\, % 11 
	Considérons la région entière \(X = P\cap \bb N^4\) avec

	\begin{equation*}
		P = \{x\in \bb R^4_+:\ 13x_1 + 20x_2 + 11x_3 + 6x_4 \geq 72\}
	\end{equation*}
	Donner une inégalité valide.
\end{td-exo}

% ----- Solutions exo 11 ----- %
\iftoggle{showsolutions}{ 
	\begin{td-sol}[]\ % 11
		On sait que les \(x_i\) sont des entiers positifs. 
		Par conséquent, on peut trouver une inégalité valide en arrondissant les coefficients de l'inégalité à l'entier supérieur le plus proche.
		Par exemple, on peut faire comme suit:
		\begin{equation*}
			\begin{aligned}
				13x_1 + 20x_2 + 11x_3 + 6x_4 \geq 72 
				& \implies \frac{13}{11}x_1 + \frac{20}{11}x_2 + x_3 + \frac{6}{11}x_4 \geq \frac{72}{11} \\
				&\implies 2x_1 + 2x_2 + x_3 + x_4 \geq 7
			\end{aligned}
		\end{equation*}
		On peut faire ce procédé pour toute valeur entière pour obtenir des inégalités valides différentes.
	\end{td-sol}
}{}
 %

% % ----- Consignes exo 12 ----- %
\begin{td-exo}[Inégalités valides]\,\\ % 12 
	Considérons une instance de \textsc{Bin Packing Problem}:

    \vspace{0.2cm}
    \vspace{-6mm}
\begin{algorithm}[H]
    \caption{\textsc{Bin Packing Problem}}
    \begin{algorithmic}
        \Require{Soient \(n\) nombres entiers \(a_1,\ldots,a_n\) et \(W, k \in \bb N^\ast\)}
        \Ensure{Existe-t-il une partition en \(k\) boites \(B_1,\ldots,B_k\) de capacité \(W\) telle que \(\forall j \in \{1,\ldots,k\}, \sum_{i \in B_j} a_i \leq W\)?}
    \end{algorithmic}
\end{algorithm}

	On considère l'instance suivante:
	\begin{itemize}
		\item La capacité des boites est \(W = 6\).
		\item Les objets à ranger sont \(a_1 = 2\), \(a_2 = 3\), \(a_3 = 3\), \(a_4 = 3\), \(a_5 = 4\), \(a_6 = 4\), \(a_7 = 5\).
	\end{itemize}


	\begin{enumerate}
		\item Donner un algorithme linéaire en nombres entiers qui modélise le \textsc{Bin Packing Problem}.

		\item Donner une solution optimale relaxée pour l'instance donnée. Montrer que cette solution est optimale.

		\item Proposer une borne inférieure pour toute solution optimale.

		\item Donner un ensemble de coupes pour l'instance.
	\end{enumerate}
\end{td-exo}

% ----- Solutions exo 12 ----- %
\iftoggle{showsolutions}{ 
	\begin{td-sol}[]\ % 12
		
	\end{td-sol}
}{}
 %

% % ----- Consignes exo 13 ----- %
\begin{td-exo}[Le polytope pour le \(0-1\) \textsc{Sac a Dos}]\,\\ % 13 
	Soit 
	\begin{equation*}
		S = \left\{ x \in \bb R^5\ \mid\ 79x_1 + 53x_2 + 53x_3 + 45x_4 + 45x_5 \leq 178\right\}.
	\end{equation*}
	avec \(x_i \in \{0,1\}\) pour \(i = 1, \ldots, 5\).

	\begin{enumerate}
		\item Donner les ensembles minimaux dépendants et donner les inégalités valides.
		\item Donner l'extension \(E(C)\) d'un ensemble minimal dépendant et les coupes associées.
	\end{enumerate}
\end{td-exo}

% ----- Solutions exo 13 ----- %
\iftoggle{showsolutions}{ 
	\begin{td-sol}[]\ % 13
		Les ensembles minimaux dépendants sont les suivants: 
		\begin{itemize} 
			\item \(C_1 = \{1, 2, 3\}\) avec \(a(C_1) = 185 > 178\), 
			\item \(C_2 = \{1, 2, 4, 5\}\) avec \(a(C_2) = 222 > 178\),
			\item \(C_3 = \{1, 3, 4, 5\}\) avec \(a(C_3) = 222 > 178\), 
			\item \(C_4 = \{2, 3, 4, 5\}\) avec \(a(C_4) = 196 > 178\). 
		\end{itemize}

		Les inégalités valides associées à ces ensembles sont les suivantes: 
		\begin{itemize} 
			\item \(x_1 + x_2 + x_3 \leq 2\), 
			\item \(x_1 + x_2 + x_4 + x_5 \leq 3\), 
			\item \(x_1 + x_3 + x_4 + x_5 \leq 3\), 
			\item \(x_2 + x_3 + x_4 + x_5 \leq 3\). 
		\end{itemize} 
		
		\item L'extension \(E(C)\) de ces ensembles est la même pour tous ces ensembles sauf pour \(C_4\) qui devient alors
		\begin{equation*}
			x_1 + x_2 + x_3 + x_4 + x_5 \leq 3. 
		\end{equation*}
	\end{td-sol}
}{}
 %

% % ----- Consignes exo 14 ----- %
\begin{td-exo}[Sur les coupes de Gomory]\,\\ % 14 
	Une coupe \(ax \leq \alpha\) est dite \og{}plus profonde\fg{} qu'une coupe \(a'x \leq \alpha'\) si
	\begin{equation*}
		\left\{x\in\mathcal{D}\ |\ a x \leq \alpha \right\} \subset \left\{x\in\mathcal{D}\ |\ a' x \leq \alpha' \right\},
	\end{equation*}
	\begin{enumerate}
		\item Expliquer le sens de ce terme.

		\item Essayer de déterminer la coupe la plus profonde dans les exercices~\ref{exos:exo_15} et~\ref{exos:exo_16} suivants.
	\end{enumerate}
\end{td-exo}

% ----- Solutions exo 14 ----- %
\iftoggle{showsolutions}{ 
	\begin{td-sol}[]\ % 14
		\begin{itemize}
			\item Une coupe \(ax \leq \alpha\) est dite plus profonde qu'une coupe \(a'x \leq \alpha'\) si elle élimine un plus grand nombre de points non entiers que la coupe \(a'x \leq \alpha'\). En d'autres termes, la coupe plus profonde est plus restrictive et réduit davantage l'ensemble des solutions possibles, ce qui peut conduire à une meilleure approximation de l'enveloppe convexe des solutions entières.

			\item On peut trouver (dans ce cas graphiquement) les coupes suivantes:
			\begin{itemize}
				\item Pour l'exercice~\ref{exos:exo_15}, la coupe optimale est \(x_2 \leq -2x_1 + 6\) (en magenta sur la figure~\ref{fig:td1_ex15_f1}).
				\item Pour l'exercice~\ref{exos:exo_16}, la coupe optimale est \(x_1 + x_2 \leq 3\) (en magenta sur la figure~\ref{fig:td1_ex16_f1}).
			\end{itemize}
		\end{itemize}
	\end{td-sol}
}{}
 %

% % ----- Consignes exo 15 ----- %
\begin{td-exo}[Sur les coupes de Gomory]\label{exos:exo_15}\,\\ % 15 
	On considère le programme linéaire suivant:
	\begin{equation*}
		PL = 
		\begin{cases}
			2x_1 + 5x_2 \leq 17 \\
			3x_1 + 2x_2 \leq 10 \\
			x_1, x_2 \in \bb N\\
			\max{(z)} = 2x_1 + x_2
		\end{cases}
	\end{equation*}

	\begin{enumerate}
		\item Résoudre le programme linéaire en nombres entiers par la méthode des coupes de Gomory.
		\item Résoudre le programme linéaire en nombres entiers par la méthode du Branch and Bound.
	\end{enumerate}
\end{td-exo}

% ----- Solutions exo 15 ----- %
\iftoggle{showsolutions}{ 
	\begin{td-sol}[]\, % 15
		On considère:
		\begin{equation*}
			PL = 
			\begin{cases}
				2x_1 + 5x_2 \leq 17 \\
				3x_1 + 2x_2 \leq 10 \\
				x_1,x_2 \in \bb N \\
				\max(z) = 2x_1 + x_2
			\end{cases}
		\end{equation*}

		On commence par résoudre la relaxation linéaire du programme linéaire en nombres entiers avec le simplexe.
		On obtient le tableau suivant:
		\begin{center}
			\begin{tabular}{|ccc|cccc|} % chktex 44
				\hline  % chktex 44
				\, & \, &\(c\)&\(2\)&\(1\)&\(0\)&\(0\)\\
				\hline % chktex 44
				\multicolumn{1}{|c|}{\(c^J\)}& \multicolumn{2}{c|}{variables de base}&\(x_1\)&\(x_2\)&\(x_3\)&\(x_4\)\\
				\hline % chktex 44
				\multicolumn{1}{|c|}{\(0\)}& \multicolumn{1}{c|}{\(x_1^{1} = x_3\)}&\(17\)&\(2\)&\(5\)&\(1\)&\(0\)\\
				\hline % chktex 44
				\multicolumn{1}{|c|}{\(0\)}& \multicolumn{1}{c|}{\(x_2^{1} = x_4\)}&\(10\)&\(3\)&\(2\)&\(0\)&\(1\)\\
				\hline % chktex 44
				\multicolumn{1}{|c|}{} & \(z(x)\) & \multicolumn{1}{|c|}{\(0\)}& \(-2\) & \(-1\) & \(0\) & \(0\)\\
				\hline % chktex 44
			\end{tabular}
		\end{center}

		On effectue le pivot, \(x_1\) rentre (coefficient le plus négatif) et \(x_4\) sort (rapport le plus petit).
		On obtient alors le tableau suivant:
		\begin{center}
			\begin{tabular}{|ccc|cccc|} % chktex 44
				\hline  % chktex 44
				\, & \, &\(c\)&\(2\)&\(1\)&\(0\)&\(0\)\\
				\hline % chktex 44
				\multicolumn{1}{|c|}{\(c^J\)}& \multicolumn{2}{c|}{variables de base}&\(x_1\)&\(x_2\)&\(x_3\)&\(x_4\)\\
				\hline % chktex 44
				\multicolumn{1}{|c|}{\(0\)}& \multicolumn{1}{c|}{\(x_1^{2} = x_3\)}& \(31/3\) & \(0\) & \(11/3\) & \(1\) & \(-2/3\)\\
				\hline % chktex 44
				\multicolumn{1}{|c|}{\(2\)}& \multicolumn{1}{c|}{\(x_2^{2} = x_1\)}& \(10/3\) & \(1\) & \(2/3\) & \(0\) & \(1/3\)\\
				\hline % chktex 44
				\multicolumn{1}{|c|}{} & \(z(x)\) & \multicolumn{1}{|c|}{\(20/3\)}& \(0\) & \(1/3\) & \(0\) & \(2/3\)\\
				\hline % chktex 44
			\end{tabular}
		\end{center}
		Tous les coefficients de la ligne \(z\) sont positifs, on a atteint l'optimum de la relaxation linéaire:
		\begin{equation*}
			x^\ast_{\bb R}=\left(\frac{10}{3}, 0\right), \quad z^\ast_{\bb R}=\frac{20}{3}.
		\end{equation*}

		\begin{enumerate}
			\item Appliquons la méthode des coupes de Gomory.

			La variable de base \(x_1\) est fractionnaire, on va construire une coupe de Gomory à partir de la ligne correspondante:
			\begin{equation*}
				x_1 = \frac{10}{3} - \frac{2}{3}x_2 - 0\cdot x_3 - \frac{1}{3}x_4.
			\end{equation*}
			En isolant les parties fractionnaires, on obtient la coupe de Gomory suivante:
			\begin{equation*}
				\frac{1}{3}x_2 + \frac{2}{3}x_4 \geq \frac{1}{3}.
			\end{equation*}
			En multipliant par 3, on peut réécrire la coupe de Gomory de la manière suivante:
			\begin{equation*}
				x_2 + 2x_4 \geq 1.
			\end{equation*}
			Mais on sait que
			\begin{equation*}
				x_4 = 10 - 2x_2 - 3x_1
			\end{equation*}
			et alors on obtient la coupe
			\begin{equation*}
				\begin{aligned}
					&x_2 + 2\cdot(10 - 2x_2 - 3x_1) \geq 1\\
					\iff & x_2 + 20 - 4x_2 - 6x_1 \geq 1\\
					\iff & -3x_2 - 6x_1 \geq -19\\
					\iff & 6x_1 + 3x_2 \leq 19
				\end{aligned}
			\end{equation*}
			En ajoutant cette coupe au programme linéaire, on obtient le programme linéaire suivant:
			\begin{equation*}
				\begin{cases}
					2x_1 + 5x_2 \leq 17 \\
					3x_1 + 2x_2 \leq 10 \\
					6x_1 + 3x_2 \leq 19 \\
					x_1, x_2, x_3, x_4 \in \bb N
				\end{cases}
			\end{equation*}
			En résolvant la relaxation linéaire de ce programme linéaire, la solution
			\begin{equation*}
				x^\ast_{\bb R} = (\frac{19}{6}, 0), \quad z^\ast_{\bb R} = \frac{19}{3}
			\end{equation*}

			On peut alors répéter ce procédé encore une fois pour obtenir la coupe suivante
			\begin{equation*}
				x_1 = \frac{19}{6} - \frac{1}{2}x_2 - 0\cdot x_3 - 0\cdot x_4 - \frac{1}{6}x_5
			\end{equation*}
			En isolant les parties fractionnaires, on obtient la coupe de Gomory suivante:
			\begin{equation*}
				\frac{1}{2}x_2 + \frac{5}{6}x_5 \geq \frac{1}{6}.
			\end{equation*}
			En multipliant par 6, on peut réécrire la coupe de Gomory de la manière suivante:
			\begin{equation*}
				3x_2 + 5x_5 \geq 1.
			\end{equation*}
			Mais on sait que
			\begin{equation*}
				x_5 = 19 - 6x_1 - 3x_2
			\end{equation*}
			et alors on obtient la coupe
			\begin{equation*}
				\begin{aligned}
					&3x_2 + 5\cdot(19 - 6x_1 - 3x_2) \geq 1\\
					\iff & 3x_2 + 95 - 30x_1 - 15x_2 \geq 1\\
					\iff & -12x_2 - 30x_1 \geq -94\\
					\iff & 30x_1 + 12x_2 \leq 94
				\end{aligned}
			\end{equation*}
			En ajoutant cette coupe au programme linéaire, on obtient le programme linéaire suivant:
			\begin{equation*}
				\begin{cases}
					2x_1 + 5x_2 \leq 17 \\
					3x_1 + 2x_2 \leq 10 \\
					6x_1 + 3x_2 \leq 19 \\
					30x_1 + 12x_2 \leq 94 \\
					x_1, x_2, x_3, x_4, x_5 \in \bb N
				\end{cases}
			\end{equation*}
			En résolvant la relaxation linéaire de ce programme linéaire, la solution
			\begin{equation*}
				x^\ast_{\bb R} = (3, \frac{1}{3}), \quad z^\ast_{\bb R} = \frac{19}{3}.
			\end{equation*}
			En répétant ce procédé encore une fois, on trouve la coupe suivante:
			\begin{equation*}
				x_2 = \frac{1}{3} - 0\cdot x_1 - 0\cdot x_3 - 0\cdot x_4 - \frac{5}{3}\cdot x_5 - \frac{1}{3}x_6
			\end{equation*}
			En isolant les parties fractionnaires, on obtient la coupe de Gomory suivante:
			\begin{equation*}
				\frac{1}{3}x_5 + \frac{2}{3}x_6 \geq \frac{1}{3}.
			\end{equation*}
			En multipliant par 3, on peut réécrire la coupe de Gomory de la manière suivante:
			\begin{equation*}
				x_5 + 2x_6 \geq 1.
			\end{equation*}
			Mais on sait que
			\begin{equation*}
				x_6 = 94 - 30x_1 - 12x_2 
			\end{equation*}
			et 
			\begin{equation*}
				x_5 = 19 - 6x_1 - 3x_2
			\end{equation*}
			et alors on obtient la coupe
			\begin{equation*}
				\begin{aligned}
					& (19 - 6x_1 - 3x_2) + 2\cdot(94 - 30x_1 - 12x_2) \geq 1\\
					\iff & 19 - 6x_1 - 3x_2 + 188 - 60x_1 - 24x_2 \geq 1\\
					\iff & -66x_1 - 27x_2 \geq -206\\
					\iff & 66x_1 + 27x_2 \leq 206
				\end{aligned}
			\end{equation*}
			En ajoutant cette coupe au programme linéaire, on obtient le programme linéaire suivant:
			\begin{equation*}
				\begin{cases}
					2x_1 + 5x_2 \leq 17 \\
					3x_1 + 2x_2 \leq 10 \\
					6x_1 + 3x_2 \leq 19 \\
					30x_1 + 12x_2 \leq 94 \\
					66x_1 + 27x_2 \leq 206 \\
					x_1, x_2, x_3, x_4, x_5, x_6 \in \bb N
				\end{cases}
			\end{equation*}
			En résolvant la relaxation linéaire de ce programme linéaire, la solution
			\begin{equation*}
				x^\ast_{\bb R} = (2.92, 0.5), \quad z^\ast_{\bb R} = 6.33.
			\end{equation*}

			Ce procédé peut ne jamais converger, on s'arrête ici par simplicité.
			Une représentation graphique du programme linéaire et des coupes associées est donnée ci-dessous en~\ref{fig:td1_ex15_f1}.

			\vspace{2mm}
\ffigbox[\FBwidth]{%
\caption{\centering Représentation du problème \(PL\) initial}\label{fig:td1_ex15_f1}
}{
    \fbox{
        \begin{tikzpicture}
            \begin{axis}[
                xlabel={\(x_1\)},
                ylabel={\(x_2\)},
                xmin=0, xmax=4,
                ymin=0, ymax=4,
                grid=both,
                axis equal image,
                width=10cm,
                height=7cm,
                legend pos=outer north east,
                legend cell align=left,
                legend image post style={fill opacity=0.45},
                legend style={fill=pagebg, draw=pagetext}
            ]

            % ----- Constraint lines -----
            % just have to express x2 in terms of x1 for each constraint
            % x + y = 2 -> y = 2 - x

            % 2x1 + 5x2 = 17 -> x2 = (17 - 2x1)/5
            \addplot[domain=0:4, very thick, color=blue] {(17 - 2*x)/5};
            \addlegendentry{\(2x_1 + 5x_2 = 17\)}

            % 3x1 + 2x2 = 10 -> x2 = (10 - 3x1)/2
            \addplot[domain=0:4, very thick, color=orange] {(10 - 3*x)/2};
            \addlegendentry{\(3x_1 + 2x_2 = 10\)}

            % first cut
            % 6x1 + 3x2 = 19 -> x2 = (19 - 6x1)/3
            \addplot[domain=0:4, very thick, color=green] {(19 - 6*x)/3};
            \addlegendentry{\(6x_1 + 3x_2 = 19\)}

            % second cut
            % 30x1 + 12x2 = 94 -> x2 = (94 - 30x1)/12
            \addplot[domain=0:4, very thick, color=red] {(94 - 30*x)/12};
            \addlegendentry{\(30x_1 + 12x_2 = 94\)}

            % third cut
            % 66x1 + 27x2 = 206 -> x2 = (206 - 66x1)/27
            \addplot[domain=0:4, very thick, color=purple] {(206 - 66*x)/27};
            \addlegendentry{\(66x_1 + 27x_2 = 206\)}

            % optimal cut
            % x2 = -2x1 + 3
            \addplot[domain=0:4, very thick, dashed, color=magenta] {-2*x + 6};
            \addlegendentry{\(x_2 = -2x_1 + 6\)}

            % ----- Feasible region -----
            \addplot[
                fill=cyan!30,
                opacity=0.45,
                draw=none,
                area legend
            ] coordinates {
                (0,0)
                (0,17/5)
                (16/11,31/11)
                (10/3,0)
            };
            \addlegendentry{Région réalisable initiale}

            % ----- Optimal real solution -----
            \addplot[
                only marks,
                mark=*,
                mark size=2.5pt,
                color=red!80!black
            ] coordinates {(10/3,0)};
            \addlegendentry{Solution réelle}

            % ----- Optimal integer solution -----
            \addplot[
                only marks,
                mark=square*,
                mark size=2.5pt,
                color=purple
            ] coordinates {(3,0)};
            \addlegendentry{Solution entière}
            
            % ----- Level line for the objective function -----
            % z is a vector, the level line is orthogonal to the vector of coefficients of the objective function
            % 2x1 + x2 -> z = 2x1 + x2
            % we can express x2 in terms of z and x1: x2 = z - 2x1
            
            \addplot[dashed, thick, yellow, domain=0:4] {20/3 - 2*x};
            \addlegendentry{Ligne de niveau \(z=\frac{20}{3}\)}
            \end{axis}
        \end{tikzpicture}
    }
}

			On peut alors voir graphiquement l'inéfficacité des coupes de Gomory, on peut aussi voir que la solution optimale en nombres entiers est \(x^\ast = (3, 0)\) avec \(z^\ast = 6\).

			\item Pour la méthode du Branch and Bound, on commence par résoudre la relaxation linéaire du programme linéaire en nombres entiers avec le simplexe.
			On obtient la même solution que précédemment:
			\begin{equation*}
				x^\ast_{\bb R} = \left(\frac{10}{3}, 0\right), \quad z^\ast_{\bb R} = \frac{20}{3}.
				\end{equation*}
			Comme \(x_1\) est fractionnaire, on va faire un branchement sur \(x_1\):
			\begin{itemize}
				\item Branchement \(x_1 \leq 3\): on ajoute la contrainte \(x_1 \leq 3\) au programme linéaire, on résout la relaxation linéaire de ce programme linéaire et on trouve la solution \(x^\ast_{\bb R} = (3, \frac{1}{2})\) avec \(z^\ast_{\bb R} = 6.5\). 

				Comme \(x_2\) est fractionnaire, on va faire un branchement sur \(x_2\):
				\begin{itemize}
					\item Branchement \(x_2 \leq 0\): on ajoute la contrainte \(x_2 \leq 0\) au programme linéaire, on résout la relaxation linéaire de ce programme linéaire et on trouve la solution \(x^\ast_{\bb R} = (3, 0)\) avec \(z^\ast_{\bb R} = 6\).
					C'est une solution entière, on la garde en mémoire et on continue à explorer les autres branches.
					\item Branchement \(x_2 \geq 1\): on ajoute la contrainte \(x_2 \geq 1\) au programme linéaire, on résout la relaxation linéaire de ce programme linéaire et on trouve la solution \(x^\ast_{\bb R} = (2.92, 1)\) avec \(z^\ast_{\bb R} = 6.84\). On peut alors faire un branchement sur \(x_1\):
					\begin{itemize}
						\item Branchement \(x_1 \leq 2\): on ajoute la contrainte \(x_1 \leq 2\) au programme linéaire, on résout la relaxation linéaire de ce programme linéaire et on trouve la solution \(x^\ast_{\bb R} = (2, 1)\) avec \(z^\ast_{\bb R} = 5\). C'est une solution entière inférieure à la solution entière trouvée précédemment, on ne la garde pas en mémoire et on continue à explorer les autres branches.
						\item Branchement \(x_1 \geq 3\): on ajoute la contrainte \(x_1 \geq 3\) au programme linéaire, on résout la relaxation linéaire de ce programme linéaire et on trouve la solution \(x^\ast_{\bb R} = (3, 0)\) avec \(z^\ast_{\bb R} = 6\). C'est la même solution entière que celle trouvée précédemment, on poursuit.
					\end{itemize}
				\end{itemize}

				\item Branchement \(x_1 \geq 4\): aucune solution réalisable, on ne garde pas cette branche en mémoire.
			\end{itemize}
			On a fini de faire les branchements, la solution optimale en nombres entiers est \(x^\ast = (3, 0)\) avec \(z^\ast = 6\).

			Une représentation de l'arbre de branchement est donnée ci-dessous en~\ref{fig:td1_ex15_f2}:
			
			\vspace{2mm}
\ffigbox[\FBwidth]{%
\caption{\centering Arbre de séparation et évaluation du problème en nombres entiers}\label{fig:td1_ex15_f2}
}{
    \fbox{
        \begin{tikzpicture}[
            main node/.style={
                draw,
                rounded rectangle,
                rounded corners=3pt,
                fill=blue!20,
                inner sep=4pt,
                font=\scriptsize,
                align=center,
                text=black
            },
            optimal/.style={
                draw=green!60!black,
                very thick,
                fill=green!10
            },
            pruned/.style={
                draw=red!60!black,
                dashed,
                text=red!60!black,
                fill=red!10
            },
        ]

            % ROOT
            \node[main node] (root) {
                Relaxation linéaire\\
                \(x^*_{\mathbb R}=\left(\frac{10}{3},0\right)\)\\
                \(z^*_{\mathbb R}=\frac{20}{3}\)
            };

            % LEVEL 1
            \node[main node, below left=3cm and 2cm of root] (A) {
                \(x_1 \le 3\)\\
                \(x^*_{\mathbb R}=(3,\frac{1}{2})\)\\
                \(z^*_{\mathbb R}=6.5\)
            };

            \node[main node, pruned, below right=3cm and 2cm of root] (B) {
                \(x_1 \ge 4\)\\
                Aucune solution\\
                Élagué
            };

            % LEVEL 2 under A
            \node[main node, optimal, below left=2.5cm and 1.5cm of A] (A1) {
                \(x_2 \le 0\)\\
                \(x^*=(3,0)\)\\
                \(z^*=6\)
            };

            \node[main node, below right=2.5cm and 1.5cm of A] (A2) {
                \(x_2 \ge 1\)\\
                \(x^*_{\mathbb R}=(2.92,1)\)\\
                \(z^*_{\mathbb R}=6.84\)
            };

            % LEVEL 3 under A2
            \node[main node, pruned, below left=2.5cm and 1cm of A2] (A21) {
                \(x_1 \le 2\)\\
                \(x^*=(2,1)\)\\
                \(z^*=5\)\\
                Non retenu
            };

            \node[main node, optimal, below right=2.5cm and 1cm of A2] (A22) {
                \(x_1 \ge 3\)\\
                \(x^*=(3,0)\)\\
                \(z^*=6\)
            };

            % EDGES
            \draw[-{Stealth}] (root) -- (A);
            \draw[-{Stealth}] (root) -- (B);

            \draw[-{Stealth}] (A) -- (A1);
            \draw[-{Stealth}] (A) -- (A2);

            \draw[-{Stealth}] (A2) -- (A21);
            \draw[-{Stealth}] (A2) -- (A22);

        \end{tikzpicture}
    }
}

		\end{enumerate}	
	\end{td-sol}
}{}

 %

% % ----- Consignes exo 16 ----- %
\begin{td-exo}[Algorithme de coupe de Gomory]\label{exos:exo_16}\,\\ % 16
	Donner la solution du programme linéaire suivant:
	\begin{equation*}
		PL = 
		\begin{cases}
			2x_1 + x_2 \leq 6 \\
			2x_1 + 3x_2 \leq 9 \\
			x_1, x_2 \in \bb N\\
			\max{(z)} = 3x_1 + 4x_2
		\end{cases}
	\end{equation*}
\end{td-exo}

% ----- Solutions exo 16 ----- %
\iftoggle{showsolutions}{ 
	\begin{td-sol}[]\ % 16
		
	\end{td-sol}
}{}
 %

% % ----- Consignes exo 17 ----- %
\begin{td-exo}[title]\, % 17 
	
\end{td-exo}

% ----- Solutions exo 17 ----- %
\iftoggle{showsolutions}{ 
	\begin{td-sol}[]\ % 17
		
	\end{td-sol}
}{}
 %

% % ----- Consignes exo 18 ----- %
\begin{td-exo}[title]\, % 18 
	
\end{td-exo}

% ----- Solutions exo 18 ----- %
\iftoggle{showsolutions}{ 
	\begin{td-sol}[]\ % 18
		
	\end{td-sol}
}{}
 %

% % ----- Consignes exo 19 ----- %
\begin{td-exo}[Enveloppe convexe et points extremes]\,\\ % 19 
	Soit \(S\) l'ensemble suivant:
	\begin{equation*}
		S = \{x \in \bb R^3\ |\ x_1+x_3 \leq 1, x_1-x_3 \leq 0, x_1 \geq 0\}
	\end{equation*}
	Donner \(\conv(S)\).
\end{td-exo}

% ----- Solutions exo 19 ----- %
\iftoggle{showsolutions}{ 
	\begin{td-sol}[]\ % 19
		
	\end{td-sol}
}{}


% % ----- Consignes exo 20 ----- %
\begin{td-exo}[title]\, % 20 
	
\end{td-exo}

% ----- Solutions exo 20 ----- %
\iftoggle{showsolutions}{ 
	\begin{td-sol}[]\ % 20
		
	\end{td-sol}
}{}


% % ----- Consignes exo 21 ----- %
\begin{td-exo}[Approximation pour \textsc{Maximum Independent}: algorithme glouton]\, % 21 
	% rajout algos et figures ici
	\begin{enumerate}
		\item Donner la complexité de l'algorithme et montrer que la solution donnée par l'algorithme est un stable.

		\item A partir de l'algorithme, proposer un graphe qui donne le pire cas.

		\item Soit \(G = (V, E)\) un graphe ayant \(n\) sommets et \(m\) arêtes, soit \(\delta = \frac mn\) la densité du graphe de \(G\).
		Montrer que toute solution \(S\) est d'au moins \(\frac{n}{2 \delta + 1}\).

		\item Montrer maintenant que le ratio est \(\delta + 1\).
	\end{enumerate}
\end{td-exo}

% ----- Solutions exo 21 ----- %
\iftoggle{showsolutions}{ 
	\begin{td-sol}[]\ % 21
		\begin{enumerate}
			\item % todo
			\item On propose le graphe suivant % add ref and figure here
			Où \(A = K_n, B = IS_n, C = \{x\}\) où \(C\) domine \(B\) et \(A\) et \(B\) sont complètement connectés. L'algorithme peut choisir \(x\) et ne pas choisir les sommets de \(B\) alors que la solution optimale est de choisir tous les sommets de \(B\).
			Ainsi on a bien un ratio de \(\frac{n}{2}\) qui peut être arbitrairement grand.

			\item Sachant que l'algorithme s'arrête, lorsqu'on a plus de sommets à traiter, on a:
			\begin{equation}\label{eq:td1_ex21_1}
				\sum_{i \in S} (d_i + 1) = n
			\end{equation}
			où \(d_i\) est le degré du sommet \(i\) dans le graphe restant.

			De plus, on a
			\begin{equation}\label{eq:td1_ex21_2}
				\sum_{i \in S} \frac{d_i (d_i + 1)}{2} \leq m
			\end{equation}

			En combinant les équations~\eqref{eq:td1_ex21_1} et~\eqref{eq:td1_ex21_2}, on a
			\begin{equation*}
				\begin{aligned}
					\sum_{i \in S} (d_i + 1) \cdot \sum_{i \in S} d_i(d_i + 1) 
					&\leq 2m + n\\
					&\leq 2 \delta m + n - n (2 S + 1)
				\end{aligned}
			\end{equation*}

			En appliquant l'inégalité de Cauchy-Schwarz avec \(a_k = (d_i + 1)\) et \(b_k = 1\), on a
			\begin{equation*}
				\begin{aligned}
					{\left(\sum_{i \in S} (d_i + 1) \cdot 1\right)}^2
					& \leq \left( \sum_{i \in S} {(d_i + 1)}^2 \right) \cdot \left( \sum_{i \in S} 1^2 \right)\\
					& \leq \left( \sum_{i \in S} {(d_i + 1)}^2 \right)
				\end{aligned}
			\end{equation*}
			et on a bien
			\begin{equation*}
				\frac{n}{2 \delta + 1} \leq |S|
			\end{equation*}
			
		\end{enumerate}
	\end{td-sol}
}{}


% % ----- Consignes exo 22 ----- %
\begin{td-exo}[Propriétés de l'enveloppe convexe]\,\\ % 22 
	Soit \(x_0\in \bb R^n\) et soit \(C \subseteq \bb R^n\) un convexe.
	Montrer que 
	\begin{equation*}
		\conv(C \cup \{x_0\}) = \{ \lambda x + (1-\lambda)x_0\ |\ x \in C, \lambda \in \ff{0,1} \}.
	\end{equation*}
\end{td-exo}

% ----- Solutions exo 22 ----- %
\iftoggle{showsolutions}{ 
	\begin{td-sol}[]\ % 22
		
	\end{td-sol}
}{}


% % ----- Consignes exo 23 ----- %
\begin{td-exo}[Enveloppe convexe]\,\\ % 23 
	Soit \(C\) le convexe défini comme étant l'enveloppe convexe de \(S\): \(C = \conv(S)\).
	Montrer que tout point extreme de \(C\) appartient à \(S\).
\end{td-exo}

% ----- Solutions exo 23 ----- %
\iftoggle{showsolutions}{ 
	\begin{td-sol}[]\ % 23
		
	\end{td-sol}
}{}


% % ----- Consignes exo 24 ----- %
\begin{td-exo}[Faces]\,\\ % 24 
	Soit \(P_\varepsilon\in \bb R^2\) le polyèdre défini par les inégalités lineaires suivantes:
	\begin{equation*}
		P_\varepsilon =
		\begin{cases}
			x_2 \geq 3\\
			\varepsilon x_1 + (2 - \varepsilon)x_2 \geq 4\\
			x_i \geq 0, \quad \forall i \in \{1,2\}
		\end{cases}
	\end{equation*}
	
	\begin{enumerate}
		\item Illustrer \(P_\varepsilon\) et les inégalités dans le plan pour \(\varepsilon = -1\) et \(\varepsilon = 1\).
		\item Quelles sont les facettes de \(P_1\)?
		\item Soit \(\varepsilon=3\) et le polyèdre entier \(P_I = \conv(P_\varepsilon \cap \bb Z^2)\).
		Dessiner \(P_I\) et donner les équations qui donnent l'enveloppe convexe.
	\end{enumerate}
\end{td-exo}

% ----- Solutions exo 24 ----- %
\iftoggle{showsolutions}{ 
	\begin{td-sol}[]\ % 24
		
	\end{td-sol}
}{}


% % ----- Consignes exo 25 ----- %
\begin{td-exo}[Algorithme glouton pour \textsc{SAT}]\, % 25 
	Montrer que l'algorithme % add ref to algo here
	admet une performance relative de \(\frac{1}{2}\).
	On pourra utiliser une preuve par réduction.
\end{td-exo}

% ----- Solutions exo 25 ----- %
\iftoggle{showsolutions}{ 
	\begin{td-sol}[]\ % 25
		On fait une preuve par récurrence.

		Dans le cas de base, le résultat est trivial.

		Supposons maintenant qu'on a \(n\) variables.
		Soit \(v\) la variable dans \(\varphi\) qui apparait le plus.
		Soit \(c_1\) le nombre de clauses où \(v\) apparaît positivement et \(c_2\) le nombre de clauses où \(v\) apparaît négativement.
		Sans perte de généralité, supposons que \(c_1 \geq c_2\).
		En assignant \(v\) à vrai, on satisfait au moins \(c_1\) clauses,
		il reste alors \(c - c_1\) clauses à satisfaire sur les \(n-1\) variables restantes, où \(c\) est le nombre de clauses satisfaites par une affectation optimale.
		Par hypothèse de récurrence, l'algorithme satisfait au moins \(\frac{1}{2}(c - c_1)\) clauses parmi les \(n-1\) variables restantes.
		Le nombre de clauses satisfaites au total par l'algorithme est alors au moins
		\begin{equation*}
			c_1 + \frac{1}{2}(c - c_1) = \frac{1}{2}c + \frac{1}{2}c_1 \geq \frac{1}{2}c.
		\end{equation*}
		
	\end{td-sol}
}{}


% % ----- Consignes exo 26 ----- %
\begin{td-exo}[title]\,\\ % 26 
	
\end{td-exo}

% ----- Solutions exo 26 ----- %
\iftoggle{showsolutions}{ 
	\begin{td-sol}[]\ % 26
		
	\end{td-sol}
}{}


% % ----- Consignes exo 27 ----- %
\begin{td-exo}[Encadrement de la valeur optimale]\,\\ % 27 
	Montrer qu'à chaque itération de la génération de colonnes on a
	\begin{equation*}
		\text{valeur}(PL) \leq \text{valeur}(PL_R) 
	\end{equation*}
	et
	\begin{equation*}
		\text{score}(PL_R) + k \cdot \text{cred} \leq \text{score}(PL)
	\end{equation*}
	où \(k\) est la valeur d'une solution optimale \(\sum_{j=1}^{n} x_j^\ast \leq k\) et \(\text{cred}\) est le cout reduit minimum à une itération donnée.
\end{td-exo}

% ----- Solutions exo 27 ----- %
\iftoggle{showsolutions}{ 
	\begin{td-sol}[]\ % 27
		
	\end{td-sol}
}{}


% % ----- Consignes exo 28 ----- %
\begin{td-exo}[title]\, % 28 
	
\end{td-exo}

% ----- Solutions exo 28 ----- %
\iftoggle{showsolutions}{ 
	\begin{td-sol}[]\ % 28
		
	\end{td-sol}
}{}


% % ----- Consignes exo 29 ----- %
\begin{td-exo}[title]\, % 29 
	
\end{td-exo}

% ----- Solutions exo 29 ----- %
\iftoggle{showsolutions}{ 
	\begin{td-sol}[]\ % 29
		
	\end{td-sol}
}{}


% % ----- Consignes exo 30 ----- %
\begin{td-exo}[Propriétés liées à la mesure au pire cas]\, % 30 
	\begin{enumerate}
		\item Soit \(\Pi\) un problème de maximisation appartenant à la classe \(\mathcal{NPO}\) ayant un ratio \(\delta\) pour la mesure.
		Montrer que cela implique que le ratio dans le pire cas est de \(\delta\).

		\item Sur le problème du \textsc{TSP}, considérons une instance \(I = (K_n, \overset{\rightarrow}d)\) avec \(\overset{\rightarrow}d\) le vecteur des arêtes-distances.
		Alors, pour n'importe quelle transformation affine
		\begin{equation*}
			\overset{\rightarrow}d \coloneqq \gamma \cdot \overset{\rightarrow}d + \mu \cdot \overset{\rightarrow}1
		\end{equation*}
		avec \(\gamma,\mu \in \bb Q\) produit une approximation différentielle équivalente à celle obtenue pour \(I\).
	\end{enumerate}
\end{td-exo}

% ----- Solutions exo 30 ----- %
\iftoggle{showsolutions}{ 
	\begin{td-sol}[]\ % 30
		
	\end{td-sol}
}{}


% % ----- Consignes exo 31 ----- %
\begin{td-exo}[Mesure différentielle: borne inférieure pour le problème du \textsc{Bin Packing}]\,\\ % 31 
	Montrer que \textsc{Bin Packing} ne peut être résolu par un algorithme polynomial à rapport différentiel minoré par \(\frac{n-1}{n}\).

	Pour cela, on pourra supposer l'existence d'un tel algorithme et arriver à une contradiction.
\end{td-exo}

% ----- Solutions exo 31 ----- %
\iftoggle{showsolutions}{ 
	\begin{td-sol}[]\ % 31
		On suppose qu'il existe un algorithme polynomial \(A\) pour \textsc{Bin Packing} tel que 
		\begin{equation*}
			\frac{n - A(I)}{n - \textsc{opt}(I)} \geq \frac{n-1}{n}
		\end{equation*}
		alors
		\begin{equation*}
			A(I) - \textsc{opt}(I) \leq 1 - \frac{\textsc{opt}(I)}{n} < 1
		\end{equation*}
		or \(A(I)\) et \(\textsc{opt}(I)\) sont des entiers, donc \(A(I) = \textsc{opt}(I)\) et \(A\) est un algorithme polynomial pour \textsc{Bin Packing}, ce qui est impossible.
	\end{td-sol}
}{}


% % ----- Consignes exo 32 ----- %
\begin{td-exo}[title]\,\\ % 32 
	
\end{td-exo}

% ----- Solutions exo 32 ----- %
\iftoggle{showsolutions}{ 
	\begin{td-sol}[]\ % 32
		
	\end{td-sol}
}{}


% % ----- Consignes exo 33 ----- %
\begin{td-exo}[title]\, % 33 
	
\end{td-exo}

% ----- Solutions exo 33 ----- %
\iftoggle{showsolutions}{ 
	\begin{td-sol}[]\ % 33
		
	\end{td-sol}
}{}


% % ----- Consignes exo 34 ----- %
\begin{td-exo}[Sur le problème du voyageur de commerce]\,\\ % 34 
	Le problème du voyageur de commerce consiste à effectuer un circuit Hamiltonien de cout minimum dans un graphe complet non orienté.
	Pour cela nous considérons \(n+1\) villes tel que le cout entre deux villes est donné par la matrice \(C = (c_{ij})\) où \(c_{ij}\) est le cout pour aller de la ville \(i\) à la ville \(j\).
	\begin{enumerate}
		\item Pourquoi le problème du voyageur de commerce est étudié dans le cadre d'un graphe complet valué et non dans un graphe quelconque?
		\item Rappeler la définition d'un circuit Hamiltonien.
		Quelles conséquences sur les degrés des sommets du circuit Hamiltonien?
		Modéliser ce problème par un programme linéaire en nombres entiers en justifiant les contraintes.
		\item Donner un exemple qui satisfait les contraintes du programme linéaire en nombres entiers mais qui n'est pas une solution réalisable du voyageur de commerce. 
		Quel est l'inconvénient des nouvelles contraintes? (Penser aux problèmes des sous-tours.)
		\item Nous ajoutons au programme linéaire en nombres entiers les contraintes suivantes:
		\begin{equation*}
			u_i - u_j + n x_{ij} \leq n-1, \quad \forall 1 \leq i \neq j \leq n
		\end{equation*}
		Montrer que les contraintes qu'on vient d'ajouter définissent bien le problème du voyageur de commerce.
		\item Comment appele-t-on ce type de programme linéaire?
	\end{enumerate}
\end{td-exo}

% ----- Solutions exo 34 ----- %
\iftoggle{showsolutions}{ 
	\begin{td-sol}[]\ % 34
		
	\end{td-sol}
}{}


% 

% ----- Consignes exo 35 ----- %
\begin{td-exo}[Contraintes serrées et solutions] % 35
	On considère
	\begin{equation*}
		P = \begin{cases}
			\max z &= 6x_1 + 5x_2\\
			x_1 + x_2 &\leq 8\\
			-2x_1 + 3x_2 &\leq 6\\
			x_1 - x_2 &\leq 2\\
			x_1,x_2 &\geq 0.
		\end{cases},\quad
		D = \begin{cases}
			\min z = 8u_1 + 6u_2 + 2u_3\\
			u_1 - 2u_2 + u_3 &\geq 6\\
			u_1 + 3u_2 - u_3 &\geq 5\\
			u_i &\geq 0,\ i=1,2,3.
		\end{cases}
	\end{equation*}
	En supposant que la solution optimale du primal est \(x=(5,3)\), donner 
	la solution du dual. Quelles sont les contraintes serrées pour le primal et le dual.
\end{td-exo}

% ----- Solutions exo 35 ----- %
\iftoggle{showsolutions}{
	\begin{td-sol}[]\ % 35
		Supposons que la solution optimale du primal est \(x=(5,3)\). Les contraintes donnent alors:
		\begin{equation*}
			\begin{aligned}
				5 + 3 \leq 8 &\iff 8 \leq 8\\
				-2\cdot 5 + 3\cdot 3 \leq 6 &\iff -1 \leq 6\\
				5-3 \leq 2 &\iff 2 \leq 2
			\end{aligned}
		\end{equation*}
		Les contraintes 1 et 3 sont serrées, la contrainte 2 ne l'est pas. Donc on sait que 
		\begin{equation*}
			u_1 > 0,\ u_2 = 0,\ u_3 > 0.
		\end{equation*}
		En rentrant cela dans le dual on obtient:
		\begin{equation*}
			\begin{cases}
				u_1 + u_3 \geq 6\\
				u_1 - u_3 \geq 5\\
			\end{cases}
		\end{equation*}
		Alors, la solution optimale est \((u_1 = \frac{11}2, u_2 = 0, u_3 = \frac12)\).
	\end{td-sol}
}{}






% % petite note sur max flot = min cut
% % matrice A - TU = A totalement unimodulaire
% % totalement unimodulaire voulant dire que tous les dets
% % de toutes les sous-matrices de A valent 0 ou 1 ou -1
% % en particulier on connait une matrice comme ca:
% % la matrice d'incidence d'un graphe oriente
% % elle a que des 0 partout sauf un 1 et un -1 par colonne
% % donc elle est TU.










% % % ----- Consignes exo xx ----- %
% % \begin{td-exo}[] % xx

% % \end{td-exo}

% % % ----- Solutions exo xx ----- %
% % \iftoggle{showsolutions}{
% % 	\begin{td-sol}[]\ % xx
		
% % 	\end{td-sol}
% % }{}
