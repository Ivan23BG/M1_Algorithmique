\section{Génération de colonnes}\label{sec:s_2}

Rappelons le principe de l'algorithme de la génération de colonnes:
\begin{itemize}
    \item Initialisation: on met quelques colonnes de \(PL\). Soit \(PL_R\).
    \item Itérations: 
    \begin{itemize}
        \item Résoudre \((PL_R)\).
        \item Calculer les couts réduits des variables de \((PL)\).
        \item Si toutes les variables de \((PL)\) ont un cout réduit \(\geq 0\) alors STOP \(((PL)\) est résolu \()\). 
        Sinon ajouter une variable de cout réduit minimum à \((PL_R)\) et résoudre à nouveau \((PL_R)\).
    \end{itemize}

    Pour calculer les couts réduits:
    
    \item on utilise les variables duales \(\mu \geq 0\) associées aux contraintes
    \item Couts réduits: \(c - \mu A\).
    \item Dimensions:
    \begin{itemize}
        \item \(c\) et \(\mu\) vecteurs lignes,
        \item \(c\) avec \(n\) colonnes et \(\mu\) avec \(m\) colonnes,
        \item \(A\) la matrice \(m\times n\)
        \item \(n\) le nombre de variables et \(m\) le nombre de contraintes.
    \end{itemize}
\end{itemize}


% ----- Consignes exo 26 ----- %
\begin{td-exo}[title]\,\\ % 26 
	
\end{td-exo}

% ----- Solutions exo 26 ----- %
\iftoggle{showsolutions}{ 
	\begin{td-sol}[]\ % 26
		
	\end{td-sol}
}{}


% ----- Consignes exo 27 ----- %
\begin{td-exo}[Encadrement de la valeur optimale]\,\\ % 27 
	Montrer qu'à chaque itération de la génération de colonnes on a
	\begin{equation*}
		\text{valeur}(PL) \leq \text{valeur}(PL_R) 
	\end{equation*}
	et
	\begin{equation*}
		\text{score}(PL_R) + k \cdot \text{cred} \leq \text{score}(PL)
	\end{equation*}
	où \(k\) est la valeur d'une solution optimale \(\sum_{j=1}^{n} x_j^\ast \leq k\) et \(\text{cred}\) est le cout reduit minimum à une itération donnée.
\end{td-exo}

% ----- Solutions exo 27 ----- %
\iftoggle{showsolutions}{ 
	\begin{td-sol}[]\ % 27
		
	\end{td-sol}
}{}


% ----- Consignes exo 28 ----- %
\begin{td-exo}[title]\, % 28 
	
\end{td-exo}

% ----- Solutions exo 28 ----- %
\iftoggle{showsolutions}{ 
	\begin{td-sol}[]\ % 28
		
	\end{td-sol}
}{}


