% ----- Consignes exo 6 ----- %
\begin{td-exo}[Contraintes avec une seule variable]\,\\ % 6 
	On considère un problème de programmation linéaire standard, à une seule contrainte, défini par
	\begin{equation*}
		\max \sum_{j=1}^{n} u_j x_j
	\end{equation*}
	avec 
	\begin{equation*}
		\sum_{j=1}^{n} v_j x_j \leq V,\text{ et } x_j\geq0,\quad\text{pour } 1 \leq j \leq n
	\end{equation*}
	Tous les coefficient \(u_j\) et \(v_j\) sont supposés strictement positifs et l'on suppose les variables classées par rapports utilité/volume décroissants ou, pour rester dans un formalisme plus mathématique, suivant les valeurs décroissantes des rapports \(\frac{u_j}{v_j}\).
	Montrer que la variable \(x_1\) est entrante et que, en la faisant entrer en base, on atteint l'optimum de l'objectif en une seule étape.
	Exprimer la valeur de l'objectif en fonction des différents coefficients.
\end{td-exo}

% ----- Solutions exo 6 ----- %
\iftoggle{showsolutions}{ 
	\begin{td-sol}[]\ % 6
		Comme les variables sont classées par rapports utilité/volume décroissants, la variable \(x_1\) a le rapport le plus élevé \(\frac{u_1}{v_1}\). En faisant entrer \(x_1\) en base, on maximise immédiatement l'utilité par unité de volume consommé.

		En entrant \(x_1\) en base, on peut allouer autant que possible à \(x_1\) sans dépasser la contrainte de volume. La quantité maximale de \(x_1\) que l'on peut allouer est donnée par
		\begin{equation*}
			x_1 = \frac{V}{v_1}
		\end{equation*}
		sous réserve que \(x_1 \leq 1\) (si \(x_1\) est une variable binaire, sinon elle peut être continue).

		L'objectif atteint en faisant entrer \(x_1\) en base est alors
		\begin{equation*}
			u_1 x_1 = u_1 \frac{V}{v_1} = V \frac{u_1}{v_1}
		\end{equation*}

		Cette solution est optimale car toute autre variable \(x_j\) avec \(j > 1\) aurait un rapport utilité/volume inférieur à celui de \(x_1\), ce qui signifie que pour chaque unité de volume consommée, on obtiendrait moins d'utilité que si on avait alloué ce volume à \(x_1\).
	\end{td-sol}
}{}
