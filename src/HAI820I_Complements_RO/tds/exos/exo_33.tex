% ----- Consignes exo 33 ----- %
\begin{td-exo}[Modéliser les problèmes suivants par un programme linéaire en nombre entiers]\, % 33 
	\begin{enumerate}
		\item L'arbre couvrant de poits minimum.
		Quel est le problème sur le nombre de contraintes?

		\item Le problème du plus court chemin entre un sommet \(s\) et un sommet \(t\).
		Soit \(G = (V, E)\) un graphe orienté avec des coûts \(c_{ij}\geq 0\) sur les arcs \((i, j) \in E\).
		Soit \(F = \left\{ P = (e_{j_1}, e_{j_2}, \ldots, e_{j_k}) \right\}\) une séquence d'arcs d'un plus court chemin entre \(s\) et \(t\) dans le graphe \(G\). On considère le graphe donné par la figure~\ref{fig:td4_ex33_f1} ci-dessous:
		
		\ffigbox[\FBwidth]{%
\caption{\centering Un graphe orienté pondéré}\label{fig:td4_ex33_f1}
}{
    \fbox{
        \begin{tikzpicture}[scale=1, main node/.style={circle, draw, fill=blue!20, inner sep=1pt, font=\scriptsize, minimum size=6mm, text=black}]
            % les sommets initiaux
            \node[main node] (s) at (0,0) {\(s\)};
            \node[main node] (a) at (2,2) {\(a\)};
            \node[main node] (b) at (2,-2) {\(b\)};
            \node[main node] (t) at (4,0) {\(t\)};

            % les arcs avec capacités
            \draw[-{Stealth}] (s) to node[above left] {\((e_1, 1)\)} (a);
            \draw[-{Stealth}] (s) to node[below left] {\((e_2, 2)\)} (b);
            \draw[-{Stealth}] (a) to node[above] {\((e_4, 3)\)} (t);
            \draw[-{Stealth}] (b) to node[below right] {\((e_5, 1)\)} (t);
            \draw[-{Stealth}] (a) to node[below] {\((e_3, 2)\)} (b);
        \end{tikzpicture}
    }
}
		
		\begin{enumerate}
			\item Donner sa matrice d'incidence \(A\).
			\item Donner le dual.
			\item Résoudre le problème par l'algorithme du simplexe.
			\item Considérons un vecteur \(f = (f_1, \ldots, f_n)\) avec \(f_i\) associé à l'arc \(e_i\) tel que \(f_i = 1\) si l'arc \(e_i\) appartient à un plus court chemin \(P\) entre \(s\) et \(t\), et \(f_i = 0\) sinon.
			\begin{enumerate}
				\item Que signifie \(\sum_{j=1}^m a_{ij} f_j = 1\)?
				\item Que signifie \(\sum_{j=1}^m a_{ij} f_j = -1\)?
				\item Que signifie \(\sum_{j=1}^m a_{ij} f_j = 0\)?
			\end{enumerate}
			\item Utiliser l'algorithme du dual simplexe.
		\end{enumerate}
	\end{enumerate}
\end{td-exo}

% ----- Solutions exo 33 ----- %
\iftoggle{showsolutions}{ 
	\begin{td-sol}[]\ % 33
		
	\end{td-sol}
}{}
