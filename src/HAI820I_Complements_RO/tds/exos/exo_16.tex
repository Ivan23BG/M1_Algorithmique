% ----- Consignes exo 16 ----- %
\begin{td-exo}[Algorithme de coupe de Gomory]\label{exos:exo_16}\,\\ % 16
	Donner la solution du programme linéaire suivant:
	\begin{equation*}
		PL = 
		\begin{cases}
			2x_1 + x_2 \leq 6 \\
			2x_1 + 3x_2 \leq 9 \\
			x_1, x_2 \in \bb N\\
			\max{(z)} = 3x_1 + 4x_2
		\end{cases}
	\end{equation*}
\end{td-exo}

% ----- Solutions exo 16 ----- %
\iftoggle{showsolutions}{ 
	\begin{td-sol}[]\ % 16
		On commence par résoudre la relaxation linéaire du programme linéaire en nombres entiers avec le simplexe.
		On obtient le tableau suivant:
		%
		\begin{center}
			\begin{tabular}{|ccc|cccc|} % chktex 44
				\hline  % chktex 44
				\, & \, &\(c\)&\(3\)&\(4\)&\(0\)&\(0\)\\
				\hline % chktex 44
				\multicolumn{1}{|c|}{\(c^J\)}& \multicolumn{2}{c|}{variables de base}&\(x_1\)&\(x_2\)&\(x_3\)&\(x_4\)\\
				\hline % chktex 44
				\multicolumn{1}{|c|}{\(0\)}& \multicolumn{1}{c|}{\(x_1^{1} = x_3\)}&\(6\)&\(2\)&\(1\)&\(1\)&\(0\)\\
				\hline % chktex 44
				\multicolumn{1}{|c|}{\(0\)}& \multicolumn{1}{c|}{\(x_2^{1} = x_4\)}&\(9\)&\(2\)&\(3\)&\(0\)&\(1\)\\
				\hline % chktex 44
				\multicolumn{1}{|c|}{} & \(z(x)\) & \multicolumn{1}{|c|}{\(0\)}& \(-3\) & \(-4\) & \(0\) & \(0\)\\
				\hline % chktex 44
			\end{tabular}
		\end{center}

		On rentre \(x_2\) en base et on sort \(x_4\) de base.
		On obtient le tableau suivant:
		\begin{center}
			\begin{tabular}{|ccc|cccc|} % chktex 44
				\hline  % chktex 44
				\, & \, &\(c\)&\(3\)&\(4\)&\(0\)&\(0\)\\
				\hline % chktex 44
				\multicolumn{1}{|c|}{\(c^J\)}& \multicolumn{2}{c|}{variables de base}&\(x_1\)&\(x_2\)&\(x_3\)&\(x_4\)\\
				\hline % chktex 44
				\multicolumn{1}{|c|}{\(0\)}& \multicolumn{1}{c|}{\(x_1^{1} = x_3\)}&\(3\)&\(4/3\)&\(0\)&\(1\)&\(-1/3\)\\
				\hline % chktex 44
				\multicolumn{1}{|c|}{\(4\)}& \multicolumn{1}{c|}{\(x_2^{1} = x_2\)}&\(3\)&\(2/3\)&\(1\)&\(0\)&\(1/3\)\\
				\hline % chktex 44
				\multicolumn{1}{|c|}{} & \(z(x)\) & \multicolumn{1}{|c|}{\(12\)}& \(-1/3\) & \(0\) & \(0\) & \(4/3\)\\
				\hline % chktex 44
			\end{tabular}
		\end{center}

		On rentre \(x_1\) en base et on sort \(x_3\) de base.
		On obtient le tableau suivant:
		\begin{center}
			\begin{tabular}{|ccc|cccc|} % chktex 44
				\hline  % chktex 44
				\, & \, &\(c\)&\(3\)&\(4\)&\(0\)&\(0\)\\
				\hline % chktex 44
				\multicolumn{1}{|c|}{\(c^J\)}& \multicolumn{2}{c|}{variables de base}&\(x_1\)&\(x_2\)&\(x_3\)&\(x_4\)\\
				\hline % chktex 44
				\multicolumn{1}{|c|}{\(3\)}& \multicolumn{1}{c|}{\(x_1^{1} = x_1\)}&\(9/4\)&\(1\)&\(0\)&\(3/4\)&\(-1/4\)\\
				\hline % chktex 44
				\multicolumn{1}{|c|}{\(4\)}& \multicolumn{1}{c|}{\(x_2^{1} = x_2\)}&\(3/2\)&\(0\)&\(1\)& \(-1/2\) & \(1/2\)\\
				\hline % chktex 44
				\multicolumn{1}{|c|}{} & \(z(x)\) & \multicolumn{1}{|c|}{\(51/4\)}& \(0\) & \(0\) & \(1/4\) & \(5/4\)\\
				\hline % chktex 44
			\end{tabular}
		\end{center}
		Tous les coefficients sont positifs donc on a trouvé notre solution optimale:
		\begin{equation*}
			x^\ast = \left(\frac{9}{4}, \frac{3}{2}\right), z^\ast = \frac{51}{4}.
		\end{equation*}

		Ce problème peut être représenté graphiquement comme suit dans la figure~\ref{fig:td1_ex16_f1}:

		\vspace{2mm}
\ffigbox[\FBwidth]{%
\caption{\centering Représentation du problème \(PL\) initial}\label{fig:td1_ex16_f1}
}{
    \fbox{
        \begin{tikzpicture}
            \begin{axis}[
                % axis lines=middle,
                xlabel={\(x_1\)},
                ylabel={\(x_2\)},
                xmin=0, xmax=4,
                ymin=0, ymax=4,
                grid=both,
                axis equal image, % <-- orthonormal grid
                width=10cm,
                height=7cm,
                legend pos=outer north east, % <-- legend outside
                legend cell align=left,
                legend image post style={fill opacity=0.45},
                legend style={fill=pagebg, draw=pagetext}
            ]

            % ----- Constraint lines -----
            % just have to express x2 in terms of x1 for each constraint
            % x + y = 2 -> y = 2 - x

            % 2x1 + 1x2 = 6 -> x2 = 6 - 2x1
            \addplot[domain=0:4, very thick, color=blue] {6 - 2*x};
            \addlegendentry{\(2x_1 + x_2 = 6\)}

            % 2x1 + 3x2 = 9 -> x2 = (9 - 2x1)/3
            \addplot[domain=0:4, very thick, color=orange] {(9 - 2*x)/3};
            \addlegendentry{\(2x_1 + 3x_2 = 9\)}

            % optimal cut
            % x2 = -2x1 + 3
            \addplot[domain=0:4, very thick, dashed, color=magenta] {(3 - x)};
            \addlegendentry{Coupe optimale \(x_1 + x_2 = 3\)}

            % ----- Feasible region -----
            \addplot[
                fill=cyan!30,
                opacity=0.45,
                draw=none,
                area legend
            ] coordinates {
                (0,0)
                (0,3)
                (9/4, 3/2)
                (3,0)
            };
            \addlegendentry{Région réalisable initiale}

            % ----- Feasible region 2 -----
            \addplot[
                fill=red!30,
                opacity=0.45,
                draw=none,
                area legend
            ] coordinates {
                (0,0)
                (0,3)
                (3,0)
            };
            \addlegendentry{Région réalisable après les coupes}

            % ----- Optimal real solution -----
            \addplot[
                only marks,
                mark=*,
                mark size=2.5pt,
                color=red!80!black
            ] coordinates {(9/4, 3/2)};
            \addlegendentry{Solution réelle}

            % ----- Optimal integer solution -----
            \addplot[
                only marks,
                mark=square*,
                mark size=2.5pt,
                color=purple
            ] coordinates {(0,3)};
            \addlegendentry{Solution entière}
            
            % % ----- Level line for the objective function -----
            % % z is a vector, the level line is orthogonal to the vector of coefficients of the objective function
            % % 2x1 + x2 -> z = 2x1 + x2
            % 3x1 + 4x2 -> z = 3x1 + 4x2
            % % we can express x2 in terms of z and x1: x2 = z - 2x1
            % x2 = (z - 3x1)/4
            
            \addplot[dashed, thick, yellow, domain=0:4] {51/16 - (3/4)*x};
            \addlegendentry{Ligne de niveau \(z=\frac{51}{4}\)}
            \end{axis}
        \end{tikzpicture}
    }
}

		La meilleure coupe ici est la coupe \(x_1 + x_2 \leq 3\) (en magenta sur la figure~\ref{fig:td1_ex16_f1}) qui élimine le point optimal de la relaxation linéaire \(x^\ast = (9/4, 3/2)\) tout en conservant les points entiers réalisables.

		La solution optimale du programme linéaire en nombres entiers devient alors \(x^\ast = (0,3)\) avec \(z^\ast = 12\).
	\end{td-sol}
}{}
