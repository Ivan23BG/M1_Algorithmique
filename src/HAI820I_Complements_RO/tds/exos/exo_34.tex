% ----- Consignes exo 34 ----- %
\begin{td-exo}[Sur le problème du voyageur de commerce]\,\\ % 34 
	Le problème du voyageur de commerce consiste à effectuer un circuit Hamiltonien de cout minimum dans un graphe complet non orienté.
	Pour cela nous considérons \(n+1\) villes tel que le cout entre deux villes est donné par la matrice \(C = (c_{ij})\) où \(c_{ij}\) est le cout pour aller de la ville \(i\) à la ville \(j\).
	\begin{enumerate}
		\item Pourquoi le problème du voyageur de commerce est étudié dans le cadre d'un graphe complet valué et non dans un graphe quelconque?
		\item Rappeler la définition d'un circuit Hamiltonien.
		Quelles conséquences sur les degrés des sommets du circuit Hamiltonien?
		Modéliser ce problème par un programme linéaire en nombres entiers en justifiant les contraintes.
		\item Donner un exemple qui satisfait les contraintes du programme linéaire en nombres entiers mais qui n'est pas une solution réalisable du voyageur de commerce. 
		Quel est l'inconvénient des nouvelles contraintes? (Penser aux problèmes des sous-tours.)
		\item Nous ajoutons au programme linéaire en nombres entiers les contraintes suivantes:
		\begin{equation*}
			u_i - u_j + n x_{ij} \leq n-1, \quad \forall 1 \leq i \neq j \leq n
		\end{equation*}
		Montrer que les contraintes qu'on vient d'ajouter définissent bien le problème du voyageur de commerce.
		\item Comment appele-t-on ce type de programme linéaire?
	\end{enumerate}
\end{td-exo}

% ----- Solutions exo 34 ----- %
\iftoggle{showsolutions}{ 
	\begin{td-sol}[]\ % 34
		
	\end{td-sol}
}{}
