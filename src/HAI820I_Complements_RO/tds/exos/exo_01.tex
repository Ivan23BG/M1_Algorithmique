% ----- Consignes exo 1 ----- %
\begin{td-exo}[Matrice totalement unimodulaire]\,\\ % 1 
	Est-ce que les matrices suivantes sont totalement unimodulaires?
	\begin{enumerate}
		\item La matrice \(A\) suivante:
		\begin{equation*}
			A = 
			\begin{pmatrix}
				1 & -1 & 1\\
				-1 & 1 & 0
			\end{pmatrix}
		\end{equation*}
		\item La matrice \(B\) suivante:
		\begin{equation*}
			B = 
			\begin{pmatrix}
				1 & 1 & 0 & 0 & 1\\
				0 & 1 & 1 & 1 & 0\\
				-1 & 0 & 0 & 0 & -1\\
				0 & 0 & -1 & 0 & 0
			\end{pmatrix}
		\end{equation*}
		\item La matrice \(C\) suivante:
		\begin{equation*}
			C = 
			\begin{pmatrix}
				1 & -1\\
				1 & 1
			\end{pmatrix}
		\end{equation*}
		\item La matrice \(D\) suivante:
		\begin{equation*}
			D = 
			\begin{pmatrix}
				1 & 1 & 0\\
				0 & 1 & 1\\
				1 & 0 & 1
			\end{pmatrix}
		\end{equation*}

		\item Propriétés: est-ce que les matrices suivantes sont totalement unimodulaires 
		sachant que \(A\) est totalement unimodulaire?

		\begin{enumerate}
			\item La matrice \(-A\),
			\item la transposée \(A^T\),
			\item la matrice \([A, I]\),
			\item la matrice \([A, -A]\).
		\end{enumerate}

		\item Considérons la matrice \(E\) définie de la manière suivante:
		\begin{equation*}
			E = 
			\begin{pmatrix}
				1 & -1 & 1\\
				1 & 1 & 0
			\end{pmatrix}
		\end{equation*}
		et soit \(b = \begin{pmatrix}
			2 & 1
		\end{pmatrix}\).

		Est-ce que \(E\) est totalement unimodulaire? Trouver les deux solutions à valeurs entières au problème \(EX = B\).
	\end{enumerate}
\end{td-exo}

% ----- Solutions exo 1 ----- %
\iftoggle{showsolutions}{ 
	\begin{td-sol}[]\, % 1
		\begin{enumerate}
			\item La matrice \(A\) est totalement unimodulaire car toutes ses matrices ont déterminant \(1,-1\) ou \(0\).
			\begin{equation*}
				\begin{vmatrix}
					1 & -1\\
					-1 & 1\\
				\end{vmatrix} = 0,\quad 
				\begin{vmatrix}
					1 & 1\\
					-1 & 0\\
				\end{vmatrix} = 1\quad 
				\begin{vmatrix}
					-1 & 1\\
					1 & 0\\
				\end{vmatrix} = -1
			\end{equation*}

			\item On peut regarder le critère de suffisance pour qu'une matrice soit TU.\@
			On met les lignes 1 et 3 ensemble dans \(A\) et les deux autres dans \(B\). Alors on a:
			\begin{equation*}
				A = 
				\begin{pmatrix}
					1 & 1 & 0 & 0 & 1\\
					-1 & 0 & 0 & 0 & -1\\
				\end{pmatrix}
				B = 
				\begin{pmatrix}
					0 & 1 & 1 & 1 & 0\\
					0 & 0 & -1 & 0 & 0
				\end{pmatrix}
			\end{equation*}
			On a bien respecté les critères, les valeurs de signes opposés dont dans le même sous-ensemble et les valeurs de signe identique sont bien dans les sous-ensembles opposés. Alors \(B\) est totalement unimodulaire.

			\item La matrice \(C\) n'est pas TU car son déterminant est \(2\).

			\item La matrice \(D\) n'est pas TU car son déterminant est \(2\).

			\item Si \(A\) est une matrice totalement unimodulaire alors:
			\begin{enumerate}
				\item la matrice \(-A\) est aussi TU,
				\item la matrice \(A^T\) est TU,
				\item en développant le long de la diagonale de \(I\) on retrouve bien que \([A, I]\) est TU,
				\item par les propositions ci-dessus on a bien \([A, -A]\) qui est TU.\@
			\end{enumerate}

			\item La matrice \(E\) n'est pas TU.\@ Le problème s'exprime comme:
			\begin{equation*}
				PL = 
				\begin{cases}
					x_1 - x_2 + x_3 = 2\\
					x_1 + x_2 = 1
				\end{cases}
			\end{equation*}
			Les solutions sont \(X_1 = (1, 0, 1)\) et \(X_2 = (0, 1, 3)\).
		\end{enumerate}
	\end{td-sol}
}{}
