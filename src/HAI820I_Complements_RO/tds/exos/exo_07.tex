% ----- Consignes exo 7 ----- %
\begin{td-exo}[Le problème du voyageur du commerce symétrique]\, % 7 
	\begin{enumerate}
		\item Appliquer la méthode par séparation et évaluation décrite en cours pour résoudre le problème du voyageur de commerce dans le graphe \(K_6\) dont la matrice des poids est la suivante:

		\begin{center}
			\begin{tabular}{c|cccccc} % chktex 44
				\ & \(x\) & \(y\) & \(z\) & \(t\) & \(u\) & \(v\)\\
				\hline % chktex 44
				\(x\) & \(\infty\) & \(4\) & \(7\) & \(2\) & \(5\) & \(4\)\\
				\(y\) & \(4\) & \(\infty\) & \(3\) & \(2\) & \(1\) & \(2\)\\
				\(z\) & \(7\) & \(3\) & \(\infty\) & \(2\) & \(6\) & \(3\)\\
				\(t\) & \(2\) & \(2\) & \(2\) & \(\infty\) & \(5\) & \(3\)\\
				\(u\) & \(5\) & \(1\) & \(6\) & \(5\) & \(\infty\) & \(2\)\\
				\(v\) & \(4\) & \(2\) & \(3\) & \(3\) & \(2\) & \(\infty\)\\
			\end{tabular}
		\end{center}
		
		\item Que faudrait-il faire si on voulait connaitre tous les cycles de poids minimum?
	\end{enumerate}
\end{td-exo}

% ----- Solutions exo 7 ----- %
\iftoggle{showsolutions}{ 
	\begin{td-sol}[]\ % 7
		\begin{enumerate}
			\item \textbf{1. Construction du 1-arbre initial}

			On considère le graphe complet \(G\) sur \(V=\{x,y,z,t,u,v\}\).
			On fixe le sommet \(x\) et on calcule un 1-arbre.

			On applique l'algorithme de Kruskal au graphe \(G\setminus\{x\}\) muni des poids initiaux.

			Les arêtes sélectionnées sont:
			\begin{equation*}
				y\!-\!u=1,\quad
				y\!-\!t=2,\quad
				y\!-\!v=2,\quad
				z\!-\!t=2.
			\end{equation*}

			Poids de l'ACPM:\@
			\begin{equation*}
				7.
			\end{equation*}

			On ajoute les deux arêtes incidentes à \(x\) de poids minimal:
			\begin{equation*}
				x\!-\!t=2,\qquad x\!-\!y=4.
			\end{equation*}

			On obtient le 1-arbre initial de poids
			\begin{equation*}
				LB_0 = 7+2+4=13.
			\end{equation*}

			Les degrés sont:
			\begin{equation*}
				\deg(x)=2,\;
				\deg(y)=4,\;
				\deg(t)=3,\;
				\deg(z)=1,\;
				\deg(u)=1,\;
				\deg(v)=1.
			\end{equation*}

			Ce n'est pas un cycle hamiltonien.

			On construit une solution réalisable:
			\begin{equation*}
				x\to t\to z\to y\to u\to v\to x
			\end{equation*}
			de poids
			\begin{equation*}
				14.
			\end{equation*}

			On fixe donc
			\begin{equation*}
				UB=14.
			\end{equation*}

			Comme \(LB_0<UB\), on applique la séparation.

			%%%%%%%%%%%%%%%%%%%%%%%%%%%%%%%%%%%%%%%%%%%%%%%%%%%%%%%%%%%%

			\textbf{Règle de branchement}

			Si le 1-arbre courant n'est pas hamiltonien, on choisit un sommet \(v\) tel que \(\deg(v)>2\). 
			On crée une branche pour chacune des arêtes incidentes à \(v\) dans le 1-arbre courant, en les excluant successivement.

			Toute exclusion est modélisée par une fonction de poids modifiée \(w^{(k)}\) où l'arête exclue reçoit le poids \(\infty\).
			Cette fonction est utilisée:
			\begin{itemize}
				\item pour l'ACPM sur \(G\setminus\{x\}\),
				\item pour le choix des deux arêtes incidentes à \(x\).
			\end{itemize}

			%%%%%%%%%%%%%%%%%%%%%%%%%%%%%%%%%%%%%%%%%%%%%%%%%%%%%%%%%%%%

			\textbf{Branchement au niveau racine}

			Dans le 1-arbre initial, \(\deg(y)=4>2\).
			On branche donc sur \(y\).

			Les arêtes incidentes à \(y\) dans le 1-arbre sont:
			\begin{equation*}
				y\!-\!u,\quad y\!-\!t,\quad y\!-\!v,\quad y\!-\!x.
			\end{equation*}

			On crée donc quatre branches.

			%%%%%%%%%%%%%%%%%%%%%%%%%%%%%%%%%%%%%%%%%%%%%%%%%%%%%%%%%%%%
			%%%%%%%%%%%%%%%%%%%%%%%%%%%%%%%%%%%%%%%%%%%%%%%%%%%%%%%%%%%%

			\textbf{Branche A:\@ exclusion de \(y\!-\!u\)}

			On impose \(w(y,u)=\infty\).

			ACPM sur \(G\setminus\{x\}\):

			\begin{equation*}
				y\!-\!t=2,\quad
				y\!-\!v=2,\quad
				z\!-\!t=2,\quad
				u\!-\!v=2.
			\end{equation*}

			Poids = 8.

			On ajoute \(x\!-\!t=2\) et \(x\!-\!y=4\).

			\begin{equation*}
				LB_A=8+2+4=14.
			\end{equation*}

			Comme \(LB_A=UB\), la branche ne peut être élaguée.

			Dans ce 1-arbre, \(\deg(y)=3>2\).
			On branche à nouveau sur \(y\).

			Les arêtes incidentes à \(y\) sont
			\begin{equation*}
				y\!-\!t,\; y\!-\!v,\; y\!-\!x.
			\end{equation*}

			%%%%%%%%%%%%%%%%%%%%%%%%%%%%%%%%%%%%%%%%%%%%%%%%%%%%%%%%%%%%

			\textbf{Branche A1:\@ exclusion de \(y\!-\!t\)}

			Il reste à \(y\) les arêtes \(y\!-\!v=2\) et \(y\!-\!z=3\).

			Dans tout arbre couvrant, \(y\) doit être incident à au moins une arête. 
			Toute ACPM contient donc une arête de poids au moins 3 incidente à \(y\).

			Les trois autres arêtes ont poids minimal total 6.

			Toute ACPM a donc poids \(\geq 9\).

			\begin{equation*}
				LB_{A1}\geq 9+2+4=15>UB.
			\end{equation*}

			Branche élaguée.

			%%%%%%%%%%%%%%%%%%%%%%%%%%%%%%%%%%%%%%%%%%%%%%%%%%%%%%%%%%%%

			\textbf{Branche A2:\@ exclusion de \(y\!-\!v\)}

			Raisonnement analogue:

			Toute ACPM contient une arête incidente à \(y\) de poids au moins 3.

			\begin{equation*}
				LB_{A2}\geq 9+2+4=15>UB.
			\end{equation*}

			Branche élaguée.

			%%%%%%%%%%%%%%%%%%%%%%%%%%%%%%%%%%%%%%%%%%%%%%%%%%%%%%%%%%%%

			\textbf{Branche A3:\@ exclusion de \(y\!-\!x\)}

			ACPM inchangé (poids 8).

			Les deux arêtes minimales incidentes à \(x\) deviennent \(x\!-\!t=2\) et \(x\!-\!v=4\).

			\begin{equation*}
				LB_{A3}=8+2+4=14.
			\end{equation*}

			Tous les sommets ont degré 2.

			On obtient un cycle hamiltonien de poids 14.

			Branche fermée

			%%%%%%%%%%%%%%%%%%%%%%%%%%%%%%%%%%%%%%%%%%%%%%%%%%%%%%%%%%%%
			%%%%%%%%%%%%%%%%%%%%%%%%%%%%%%%%%%%%%%%%%%%%%%%%%%%%%%%%%%%%

			\textbf{Branche B:\@ exclusion de \(y\!-\!t\)}

			On impose \(w(y,t)=\infty\).

			ACPM:\@
			\begin{equation*}
				y\!-\!u=1,\quad
				y\!-\!v=2,\quad
				z\!-\!t=2,\quad
				u\!-\!v=2.
			\end{equation*}

			Poids = 7.

			Ajout de \(x\!-\!t=2\) et \(x\!-\!y=4\).

			\begin{equation*}
				LB_B=13.
			\end{equation*}

			\(\deg(y)=3\).
			On branche sur \(y\).

			%%%%%%%%%%%%%%%%%%%%%%%%%%%%%%%%%%%%%%%%%%%%%%%%%%%%%%%%%%%%

			\textbf{Branche B1:\@ exclusion de \(y\!-\!u\)}

			Toute ACPM contient alors une arête incidente à \(y\) de poids \(\geq 3\).

			\begin{equation*}
				LB_{B1}\geq 15.
			\end{equation*}

			Élaguée.

			%%%%%%%%%%%%%%%%%%%%%%%%%%%%%%%%%%%%%%%%%%%%%%%%%%%%%%%%%%%%

			\textbf{Branche B2:\@ exclusion de \(y\!-\!v\)}

			Même raisonnement:

			\begin{equation*}
				LB_{B2}\geq 15.
			\end{equation*}

			Élaguée.

			%%%%%%%%%%%%%%%%%%%%%%%%%%%%%%%%%%%%%%%%%%%%%%%%%%%%%%%%%%%%

			\textbf{Branche B3:\@ exclusion de \(y\!-\!x\)}

			Les arêtes incidentes à \(x\) deviennent \(x\!-\!t=2\) et \(x\!-\!v=4\).

			On obtient un cycle hamiltonien de poids 14.

			Branche fermée.

			%%%%%%%%%%%%%%%%%%%%%%%%%%%%%%%%%%%%%%%%%%%%%%%%%%%%%%%%%%%%
			%%%%%%%%%%%%%%%%%%%%%%%%%%%%%%%%%%%%%%%%%%%%%%%%%%%%%%%%%%%%

			\textbf{Branche C:\@ exclusion de \(y\!-\!v\)}

			ACPM:\@
			\begin{equation*}
				y\!-\!u=1,\quad
				y\!-\!t=2,\quad
				z\!-\!t=2,\quad
				u\!-\!v=2.
			\end{equation*}

			Poids = 7.

			\begin{equation*}
				LB_C=13.
			\end{equation*}

			On branche sur \(y\).

			Les deux sous-branches \(y\!-\!u\) exclue et \(y\!-\!t\) exclue donnent \(LB\geq 15\), donc élaguées.

			Exclusion de \(y\!-\!x\) donne un cycle hamiltonien de poids 14.

			Branche fermée.

			%%%%%%%%%%%%%%%%%%%%%%%%%%%%%%%%%%%%%%%%%%%%%%%%%%%%%%%%%%%%
			%%%%%%%%%%%%%%%%%%%%%%%%%%%%%%%%%%%%%%%%%%%%%%%%%%%%%%%%%%%%

			\textbf{Branche D:\@ exclusion de \(y\!-\!x\)}

			ACPM initial inchangé (poids 7).

			Arêtes incidentes minimales à \(x\):
			\begin{equation*}
				x\!-\!t=2,\quad x\!-\!v=4.
			\end{equation*}

			\begin{equation*}
				LB_D=13.
			\end{equation*}

			On branche sur \(y\).

			Les exclusions successives de \(y\!-\!u\) et \(y\!-\!t\) donnent \(LB\geq 15\): branches élaguées.

			L'exclusion de \(y\!-\!v\) produit un cycle hamiltonien de poids 14.

			Branche fermée.

			%%%%%%%%%%%%%%%%%%%%%%%%%%%%%%%%%%%%%%%%%%%%%%%%%%%%%%%%%%%%
			%%%%%%%%%%%%%%%%%%%%%%%%%%%%%%%%%%%%%%%%%%%%%%%%%%%%%%%%%%%%

			\textbf{Conclusion globale}

			Toutes les branches de l'arbre de séparation ont été:

			\begin{itemize}
				\item soit élaguées par une borne strictement supérieure à 14,
				\item soit explorées complètement et conduisant à un cycle de poids 14.
			\end{itemize}

			Il n'existe donc aucun cycle hamiltonien de poids strictement inférieur à 14.

			\begin{equation*}
				\boxed{\text{Poids optimal } = 14.}
			\end{equation*}

			L'arbre de séparation est donné à la figure~\ref{fig:td1_ex07_f1} en annexe.

			\item Pour obtenir \emph{tous} les cycles optimaux, il faut poursuivre la séparation en n'élaguant que lorsque \(LB>UB\) (et non \(LB\geq UB\)), et enregistrer toutes les solutions de poids 14.
		\end{enumerate}
	\end{td-sol}
}{}

