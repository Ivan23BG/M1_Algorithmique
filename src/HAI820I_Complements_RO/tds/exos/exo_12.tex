% ----- Consignes exo 12 ----- %
\begin{td-exo}[Inégalités valides]\,\\ % 12 
	Considérons une instance de \textsc{Bin Packing Problem}:

    \vspace{0.2cm}
    \vspace{-6mm}
\begin{algorithm}[H]
    \caption{\textsc{Bin Packing Problem}}
    \begin{algorithmic}
        \Require{Soient \(n\) nombres entiers \(a_1,\ldots,a_n\) et \(W, k \in \bb N^\ast\)}
        \Ensure{Existe-t-il une partition en \(k\) boites \(B_1,\ldots,B_k\) de capacité \(W\) telle que \(\forall j \in \{1,\ldots,k\}, \sum_{i \in B_j} a_i \leq W\)?}
    \end{algorithmic}
\end{algorithm}

	On considère l'instance suivante:
	\begin{itemize}
		\item La capacité des boites est \(W = 6\).
		\item Les objets à ranger sont \(a_1 = 2\), \(a_2 = 3\), \(a_3 = 3\), \(a_4 = 3\), \(a_5 = 4\), \(a_6 = 4\), \(a_7 = 5\).
	\end{itemize}


	\begin{enumerate}
		\item Donner un algorithme linéaire en nombres entiers qui modélise le \textsc{Bin Packing Problem}.

		\item Donner une solution optimale relaxée pour l'instance donnée. Montrer que cette solution est optimale.

		\item Proposer une borne inférieure pour toute solution optimale.

		\item Donner un ensemble de coupes pour l'instance.
	\end{enumerate}
\end{td-exo}

% ----- Solutions exo 12 ----- %
\iftoggle{showsolutions}{ 
	\begin{td-sol}[]\ % 12
		\begin{enumerate}
			\item On peut modéliser le \textsc{Bin Packing Problem} de deux manières différentes:
			\begin{itemize}
				\item On utilise des variables binaires \(x_{ij}\) qui indiquent si l'objet \(a_i\) est placé dans la boîte \(j\), et des variables binaires \(y_j\) qui indiquent si la boîte \(j\) est utilisée. L'objectif est de minimiser le nombre de boîtes utilisées, soit:
				\begin{equation*}
					\begin{cases}
						\min \sum_j y_j \\
						\sum_j x_{ij} = 1, \ \forall i \in \{1, \ldots, 7\}\\
						\sum_i a_i x_{ij} \leq W y_j, \ \forall j \\
						x_{ij} \in \{0, 1\}, \ \forall i, j \\
						y_j \in \{0, 1\}, \ \forall j
					\end{cases} 
				\end{equation*}

				\item Une autre modélisation consiste à ne pas considérer les \(y_i\) comme des variables, cela donne alors:
				\begin{equation*}
					\begin{cases}
						\min k\\
						\sum_j x_{ij} = 1, \ \forall i \in \{1, \ldots, 7\}\\
						\sum_i a_i x_{ij} \leq W, \ \forall j \\
						x_{ij} \in \{0, 1\}, \ \forall i, j \\
					\end{cases} 
				\end{equation*}
			\end{itemize}

			\item Une solution optimale pour cette instance consiste à placer les éléments comme suit:
			\begin{itemize}
				\item \(a_7\) est dans la boite 1,
				\item \(a_6\) et \(a_1\) sont dans la boite 2,
				\item \(a_5\) est dans la boite 3, 
				\item \(a_4\) et \(a_3\) sont dans la boite 4, 
				\item \(a_2\) est dans la boite 5. 
			\end{itemize}
			Cette solution utilise 5 boites.
			Une borne inférieure pour toute solution optimale est donnée par \(\left\lceil \frac{\sum_i a_i}{W} \right\rceil = \left\lceil \frac{22}{6} \right\rceil = 4\). 
			Les éléments \(a_4,\ldots,a_7\) nécessitent chacun une boite, et les éléments \(a_2\) et \(a_3\) ne peuvent rentrer tous deux dans ces boites.
			Le nombre minimum de boites est alors de 5, et la solution proposée est optimale. 

			\item Une borne inférieure pour toute solution optimale est donnée par
			\begin{equation*}
				\left\lceil \frac{\sum_i a_i}{W} \right\rceil
			\end{equation*}

			\item Un ensemble de coupes pour l'instance donnée serait (pour tout \(j\)):
			\begin{equation*}
				\begin{aligned}
					&C_1 = \begin{cases}
						x_{7j} + x_{1j} \leq y_j\\
						x_{7j} + x_{2j} \leq y_j\\
						x_{7j} + x_{3j} \leq y_j\\
						x_{7j} + x_{4j} \leq y_j\\
						x_{7j} + x_{5j} \leq y_j\\
						x_{7j} + x_{6j} \leq y_j\\
					\end{cases},\qquad
					C_2 = \begin{cases}
						x_{6j} + x_{2j} \leq y_j\\
						x_{6j} + x_{3j} \leq y_j\\
						x_{6j} + x_{4j} \leq y_j\\
						x_{6j} + x_{5j} \leq y_j\\
					\end{cases},\\
					&C_3 = \begin{cases}
						x_{5j} + x_{2j} \leq y_j\\
						x_{5j} + x_{3j} \leq y_j\\
						x_{5j} + x_{4j} \leq y_j\\
					\end{cases},\qquad
					C_4 = \begin{cases}
						x_{4j} + x_{3j} + x_{2j} \leq 2y_j\\
						x_{4j} + x_{3j} + x_{1j} \leq 2y_j\\
						x_{4j} + x_{2j} + x_{1j} \leq 2y_j\\
					\end{cases},\\
				\end{aligned}
			\end{equation*}
		\end{enumerate}
	\end{td-sol}
}{}
