% ----- Consignes exo 12 ----- %
\begin{td-exo}[Inégalités valides]\,\\ % 12 
	Considérons une instance de \textsc{Bin Packing Problem}:

    \vspace{0.2cm}
    \vspace{-6mm}
\begin{algorithm}[H]
    \caption{\textsc{Bin Packing Problem}}
    \begin{algorithmic}
        \Require{Soient \(n\) nombres entiers \(a_1,\ldots,a_n\) et \(W, k \in \bb N^\ast\)}
        \Ensure{Existe-t-il une partition en \(k\) boites \(B_1,\ldots,B_k\) de capacité \(W\) telle que \(\forall j \in \{1,\ldots,k\}, \sum_{i \in B_j} a_i \leq W\)?}
    \end{algorithmic}
\end{algorithm}

	On considère l'instance suivante:
	\begin{itemize}
		\item La capacité des boites est \(W = 6\).
		\item Les objets à ranger sont \(a_1 = 2\), \(a_2 = 3\), \(a_3 = 3\), \(a_4 = 3\), \(a_5 = 4\), \(a_6 = 4\), \(a_7 = 5\).
	\end{itemize}


	\begin{enumerate}
		\item Donner un algorithme linéaire en nombres entiers qui modélise le \textsc{Bin Packing Problem}.

		\item Donner une solution optimale relaxée pour l'instance donnée. Montrer que cette solution est optimale.

		\item Proposer une borne inférieure pour toute solution optimale.

		\item Donner un ensemble de coupes pour l'instance.
	\end{enumerate}
\end{td-exo}

% ----- Solutions exo 12 ----- %
\iftoggle{showsolutions}{ 
	\begin{td-sol}[]\ % 12
		\begin{enumerate}
			\item On peut modéliser le \textsc{Bin Packing Problem} à l'aide de variables binaires \(x_{ij}\) qui indiquent si l'objet \(a_i\) est placé dans la boîte \(j\), et des variables binaires \(y_j\) qui indiquent si la boîte \(j\) est utilisée. L'objectif est de minimiser le nombre de boîtes utilisées, soit \(\sum_j y_j\), sous les contraintes suivantes:
			\begin{equation*}
				\begin{aligned}
					& \sum_j x_{ij} = 1 && \forall i \in \{1, \ldots, 7\} \quad \text{(chaque objet doit etre place dans exactement une boite)} \\
					& \sum_i a_i x_{ij} \leq W y_j && \forall j \quad \text{(la somme des poids dans chaque boite ne doit pas depasser la capacite)} \\
					& x_{ij} \in \{0, 1\} && \forall i, j \\
					& y_j \in \{0, 1\} && \forall j
				\end{aligned}
			\end{equation*}
		\end{enumerate}
	\end{td-sol}
}{}
