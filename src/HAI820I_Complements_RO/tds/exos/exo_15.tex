% ----- Consignes exo 15 ----- %
\begin{td-exo}[Sur les coupes de Gomory]\label{exos:exo_15}\,\\ % 15 
	On considère le programme linéaire suivant:
	\begin{equation*}
		PL = 
		\begin{cases}
			2x_1 + 5x_2 \leq 17 \\
			3x_1 + 2x_2 \leq 10 \\
			x_1, x_2 \in \bb N\\
			\max{(z)} = 2x_1 + x_2
		\end{cases}
	\end{equation*}

	\begin{enumerate}
		\item Résoudre le programme linéaire en nombres entiers par la méthode des coupes de Gomory.
		\item Résoudre le programme linéaire en nombres entiers par la méthode du Branch and Bound.
	\end{enumerate}
\end{td-exo}

% ----- Solutions exo 15 ----- %
\iftoggle{showsolutions}{ 
	\begin{td-sol}[]\, % 15
		On considère:
		\begin{equation*}
			PL = 
			\begin{cases}
				2x_1 + 5x_2 \leq 17 \\
				3x_1 + 2x_2 \leq 10 \\
				x_1,x_2 \in \bb N \\
				\max(z) = 2x_1 + x_2
			\end{cases}
		\end{equation*}

		On commence par résoudre la relaxation linéaire du programme linéaire en nombres entiers avec le simplexe.
		On obtient le tableau suivant:
		\begin{center}
			\begin{tabular}{|ccc|cccc|} % chktex 44
				\hline  % chktex 44
				\, & \, &\(c\)&\(2\)&\(1\)&\(0\)&\(0\)\\
				\hline % chktex 44
				\multicolumn{1}{|c|}{\(c^J\)}& \multicolumn{2}{c|}{variables de base}&\(x_1\)&\(x_2\)&\(x_3\)&\(x_4\)\\
				\hline % chktex 44
				\multicolumn{1}{|c|}{\(0\)}& \multicolumn{1}{c|}{\(x_1^{1} = x_3\)}&\(17\)&\(2\)&\(5\)&\(1\)&\(0\)\\
				\hline % chktex 44
				\multicolumn{1}{|c|}{\(0\)}& \multicolumn{1}{c|}{\(x_2^{1} = x_4\)}&\(10\)&\(3\)&\(2\)&\(0\)&\(1\)\\
				\hline % chktex 44
				\multicolumn{1}{|c|}{} & \(z(x)\) & \multicolumn{1}{|c|}{\(0\)}& \(-2\) & \(-1\) & \(0\) & \(0\)\\
				\hline % chktex 44
			\end{tabular}
		\end{center}

		On effectue le pivot, \(x_1\) rentre (coefficient le plus négatif) et \(x_4\) sort (rapport le plus petit).
		On obtient alors le tableau suivant:
		\begin{center}
			\begin{tabular}{|ccc|cccc|} % chktex 44
				\hline  % chktex 44
				\, & \, &\(c\)&\(2\)&\(1\)&\(0\)&\(0\)\\
				\hline % chktex 44
				\multicolumn{1}{|c|}{\(c^J\)}& \multicolumn{2}{c|}{variables de base}&\(x_1\)&\(x_2\)&\(x_3\)&\(x_4\)\\
				\hline % chktex 44
				\multicolumn{1}{|c|}{\(0\)}& \multicolumn{1}{c|}{\(x_1^{2} = x_3\)}& \(31/3\) & \(0\) & \(11/3\) & \(1\) & \(-2/3\)\\
				\hline % chktex 44
				\multicolumn{1}{|c|}{\(2\)}& \multicolumn{1}{c|}{\(x_2^{2} = x_1\)}& \(10/3\) & \(1\) & \(2/3\) & \(0\) & \(1/3\)\\
				\hline % chktex 44
				\multicolumn{1}{|c|}{} & \(z(x)\) & \multicolumn{1}{|c|}{\(20/3\)}& \(0\) & \(1/3\) & \(0\) & \(2/3\)\\
				\hline % chktex 44
			\end{tabular}
		\end{center}
		Tous les coefficients de la ligne \(z\) sont positifs, on a atteint l'optimum de la relaxation linéaire:
		\begin{equation*}
			x^\ast_{\bb R}=\left(\frac{10}{3}, 0\right), \quad z^\ast_{\bb R}=\frac{20}{3}.
		\end{equation*}

		\begin{enumerate}
			\item Appliquons la méthode des coupes de Gomory.

			La variable de base \(x_1\) est fractionnaire, on va construire une coupe de Gomory à partir de la ligne correspondante:
			\begin{equation*}
				x_1 = \frac{10}{3} - \frac{2}{3}x_2 - 0\cdot x_3 - \frac{1}{3}x_4.
			\end{equation*}
			En isolant les parties fractionnaires, on obtient la coupe de Gomory suivante:
			\begin{equation*}
				\frac{1}{3}x_2 + \frac{2}{3}x_4 \geq \frac{1}{3}.
			\end{equation*}
			En multipliant par 3, on peut réécrire la coupe de Gomory de la manière suivante:
			\begin{equation*}
				x_2 + 2x_4 \geq 1.
			\end{equation*}
			Mais on sait que
			\begin{equation*}
				x_4 = 10 - 2x_2 - 3x_1
			\end{equation*}
			et alors on obtient la coupe
			\begin{equation*}
				\begin{aligned}
					&x_2 + 2\cdot(10 - 2x_2 - 3x_1) \geq 1\\
					\iff & x_2 + 20 - 4x_2 - 6x_1 \geq 1\\
					\iff & -3x_2 - 6x_1 \geq -19\\
					\iff & 6x_1 + 3x_2 \leq 19
				\end{aligned}
			\end{equation*}
			En ajoutant cette coupe au programme linéaire, on obtient le programme linéaire suivant:
			\begin{equation*}
				\begin{cases}
					2x_1 + 5x_2 \leq 17 \\
					3x_1 + 2x_2 \leq 10 \\
					6x_1 + 3x_2 \leq 19 \\
					x_1, x_2, x_3, x_4 \in \bb N
				\end{cases}
			\end{equation*}
			En résolvant la relaxation linéaire de ce programme linéaire, la solution
			\begin{equation*}
				x^\ast_{\bb R} = (\frac{19}{6}, 0), \quad z^\ast_{\bb R} = \frac{19}{3}
			\end{equation*}

			On peut alors répéter ce procédé encore une fois pour obtenir la coupe suivante
			\begin{equation*}
				x_1 = \frac{19}{6} - \frac{1}{2}x_2 - 0\cdot x_3 - 0\cdot x_4 - \frac{1}{6}x_5
			\end{equation*}
			En isolant les parties fractionnaires, on obtient la coupe de Gomory suivante:
			\begin{equation*}
				\frac{1}{2}x_2 + \frac{5}{6}x_5 \geq \frac{1}{6}.
			\end{equation*}
			En multipliant par 6, on peut réécrire la coupe de Gomory de la manière suivante:
			\begin{equation*}
				3x_2 + 5x_5 \geq 1.
			\end{equation*}
			Mais on sait que
			\begin{equation*}
				x_5 = 19 - 6x_1 - 3x_2
			\end{equation*}
			et alors on obtient la coupe
			\begin{equation*}
				\begin{aligned}
					&3x_2 + 5\cdot(19 - 6x_1 - 3x_2) \geq 1\\
					\iff & 3x_2 + 95 - 30x_1 - 15x_2 \geq 1\\
					\iff & -12x_2 - 30x_1 \geq -94\\
					\iff & 30x_1 + 12x_2 \leq 94
				\end{aligned}
			\end{equation*}
			En ajoutant cette coupe au programme linéaire, on obtient le programme linéaire suivant:
			\begin{equation*}
				\begin{cases}
					2x_1 + 5x_2 \leq 17 \\
					3x_1 + 2x_2 \leq 10 \\
					6x_1 + 3x_2 \leq 19 \\
					30x_1 + 12x_2 \leq 94 \\
					x_1, x_2, x_3, x_4, x_5 \in \bb N
				\end{cases}
			\end{equation*}
			En résolvant la relaxation linéaire de ce programme linéaire, la solution
			\begin{equation*}
				x^\ast_{\bb R} = (3, \frac{1}{3}), \quad z^\ast_{\bb R} = \frac{19}{3}.
			\end{equation*}
			En répétant ce procédé encore une fois, on trouve la coupe suivante:
			\begin{equation*}
				x_2 = \frac{1}{3} - 0\cdot x_1 - 0\cdot x_3 - 0\cdot x_4 - \frac{5}{3}\cdot x_5 - \frac{1}{3}x_6
			\end{equation*}
			En isolant les parties fractionnaires, on obtient la coupe de Gomory suivante:
			\begin{equation*}
				\frac{1}{3}x_5 + \frac{2}{3}x_6 \geq \frac{1}{3}.
			\end{equation*}
			En multipliant par 3, on peut réécrire la coupe de Gomory de la manière suivante:
			\begin{equation*}
				x_5 + 2x_6 \geq 1.
			\end{equation*}
			Mais on sait que
			\begin{equation*}
				x_6 = 94 - 30x_1 - 12x_2 
			\end{equation*}
			et 
			\begin{equation*}
				x_5 = 19 - 6x_1 - 3x_2
			\end{equation*}
			et alors on obtient la coupe
			\begin{equation*}
				\begin{aligned}
					& (19 - 6x_1 - 3x_2) + 2\cdot(94 - 30x_1 - 12x_2) \geq 1\\
					\iff & 19 - 6x_1 - 3x_2 + 188 - 60x_1 - 24x_2 \geq 1\\
					\iff & -66x_1 - 27x_2 \geq -206\\
					\iff & 66x_1 + 27x_2 \leq 206
				\end{aligned}
			\end{equation*}
			En ajoutant cette coupe au programme linéaire, on obtient le programme linéaire suivant:
			\begin{equation*}
				\begin{cases}
					2x_1 + 5x_2 \leq 17 \\
					3x_1 + 2x_2 \leq 10 \\
					6x_1 + 3x_2 \leq 19 \\
					30x_1 + 12x_2 \leq 94 \\
					66x_1 + 27x_2 \leq 206 \\
					x_1, x_2, x_3, x_4, x_5, x_6 \in \bb N
				\end{cases}
			\end{equation*}
			En résolvant la relaxation linéaire de ce programme linéaire, la solution
			\begin{equation*}
				x^\ast_{\bb R} = (2.92, 0.5), \quad z^\ast_{\bb R} = 6.33.
			\end{equation*}
		\end{enumerate}	

		\vspace{2mm}
\ffigbox[\FBwidth]{%
\caption{\centering Représentation du problème \(PL\) initial}\label{fig:td1_ex15_f1}
}{
    \fbox{
        \begin{tikzpicture}
            \begin{axis}[
                xlabel={\(x_1\)},
                ylabel={\(x_2\)},
                xmin=0, xmax=4,
                ymin=0, ymax=4,
                grid=both,
                axis equal image,
                width=10cm,
                height=7cm,
                legend pos=outer north east,
                legend cell align=left,
                legend image post style={fill opacity=0.45},
                legend style={fill=pagebg, draw=pagetext}
            ]

            % ----- Constraint lines -----
            % just have to express x2 in terms of x1 for each constraint
            % x + y = 2 -> y = 2 - x

            % 2x1 + 5x2 = 17 -> x2 = (17 - 2x1)/5
            \addplot[domain=0:4, very thick, color=blue] {(17 - 2*x)/5};
            \addlegendentry{\(2x_1 + 5x_2 = 17\)}

            % 3x1 + 2x2 = 10 -> x2 = (10 - 3x1)/2
            \addplot[domain=0:4, very thick, color=orange] {(10 - 3*x)/2};
            \addlegendentry{\(3x_1 + 2x_2 = 10\)}

            % first cut
            % 6x1 + 3x2 = 19 -> x2 = (19 - 6x1)/3
            \addplot[domain=0:4, very thick, color=green] {(19 - 6*x)/3};
            \addlegendentry{\(6x_1 + 3x_2 = 19\)}

            % second cut
            % 30x1 + 12x2 = 94 -> x2 = (94 - 30x1)/12
            \addplot[domain=0:4, very thick, color=red] {(94 - 30*x)/12};
            \addlegendentry{\(30x_1 + 12x_2 = 94\)}

            % third cut
            % 66x1 + 27x2 = 206 -> x2 = (206 - 66x1)/27
            \addplot[domain=0:4, very thick, color=purple] {(206 - 66*x)/27};
            \addlegendentry{\(66x_1 + 27x_2 = 206\)}

            % optimal cut
            % x2 = -2x1 + 3
            \addplot[domain=0:4, very thick, dashed, color=magenta] {-2*x + 6};
            \addlegendentry{\(x_2 = -2x_1 + 6\)}

            % ----- Feasible region -----
            \addplot[
                fill=cyan!30,
                opacity=0.45,
                draw=none,
                area legend
            ] coordinates {
                (0,0)
                (0,17/5)
                (16/11,31/11)
                (10/3,0)
            };
            \addlegendentry{Région réalisable initiale}

            % ----- Optimal real solution -----
            \addplot[
                only marks,
                mark=*,
                mark size=2.5pt,
                color=red!80!black
            ] coordinates {(10/3,0)};
            \addlegendentry{Solution réelle}

            % ----- Optimal integer solution -----
            \addplot[
                only marks,
                mark=square*,
                mark size=2.5pt,
                color=purple
            ] coordinates {(3,0)};
            \addlegendentry{Solution entière}
            
            % ----- Level line for the objective function -----
            % z is a vector, the level line is orthogonal to the vector of coefficients of the objective function
            % 2x1 + x2 -> z = 2x1 + x2
            % we can express x2 in terms of z and x1: x2 = z - 2x1
            
            \addplot[dashed, thick, yellow, domain=0:4] {20/3 - 2*x};
            \addlegendentry{Ligne de niveau \(z=\frac{20}{3}\)}
            \end{axis}
        \end{tikzpicture}
    }
}
	\end{td-sol}
}{}

