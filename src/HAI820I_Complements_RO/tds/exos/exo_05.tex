% ----- Consignes exo 5 ----- %
\begin{td-exo}[Branch and bound et plus court chemin]\,\\ % 5 
	Nous considérons le graphe donné par la figure~\ref{fig:td1_ex4_f1} suivante:

    \vspace{0.2cm}
    \ffigbox[\FBwidth]{%
\caption{\centering Recherche d'un plus court chemin par branch and bound}\label{fig:td1_ex4_f1}
}{
    \fbox{
        \begin{tikzpicture}[scale=1, main node/.style={circle, draw, fill=blue!20, inner sep=1pt, font=\scriptsize, minimum size=6mm, text=black}]
            % les sommets initiaux
            \node[main node] (s) at (-3.5,0) {\(s\)};
            \node[main node] (t) at (3.5,0) {\(t\)};

            \node[main node] (a) at (-2,1.5) {\(a\)};
            \node[main node] (b) at (-2,0) {\(b\)};
            \node[main node] (c) at (-2,-1.5) {\(c\)};

            \node[main node] (d) at (0,1.5) {\(d\)};
            \node[main node] (e) at (0,0) {\(e\)};
            \node[main node] (f) at (0,-1.5) {\(f\)};

            \node[main node] (g) at (2,1.5) {\(g\)};
            \node[main node] (h) at (2,0) {\(h\)};
            \node[main node] (i) at (2,-1.5) {\(i\)};

            % les arcs avec capacités
            \draw[-{Stealth}] (s) to node[above left] {\(2\)} (a);
            \draw[-{Stealth}] (s) to node[above] {\(2\)} (b);
            \draw[-{Stealth}] (s) to node[above right] {\(4\)} (c);

            \draw[-{Stealth}] (a) to node[right] {\(3\)} (b);
            \draw[-{Stealth}] (a) to node[above] {\(7\)} (d);
            \draw[-{Stealth}] (a) to node[above right] {\(2\)} (e);

            \draw[-{Stealth}] (b) to node[above] {\(9\)} (e);
            \draw[-{Stealth}] (b) to node[above right] {\(2\)} (f);

            \draw[-{Stealth}] (c) to node[above] {\(2\)} (f);

            \draw[-{Stealth}] (d) to node[above] {\(1\)} (g);
            \draw[-{Stealth}] (d) to node[above right] {\(2\)} (h);

            \draw[-{Stealth}] (e) to node[right] {\(1\)} (f);
            \draw[-{Stealth}] (e) to node[above] {\(3\)} (h);
            
            \draw[-{Stealth}] (f) to node[below right] {\(5\)} (h);
            \draw[-{Stealth}] (f) to node[below right] {\(2\)} (i);
            
            \draw[-{Stealth}] (g) to node[below right] {\(4\)} (t);
            
            \draw[-{Stealth}] (h) to node[below right] {\(4\)} (t);
            
            \draw[-{Stealth}] (i) to node[below right] {\(2\)} (t);
            \draw[-{Stealth}] (i) to node[below right] {\(2\)} (t);
        \end{tikzpicture}
    }
}

	Donner deux stratégies en utilisant le principe du branch and bound pour résoudre le problème d'un plus court chemin.
	Appliquer vos stratégies sur le graphe de la figure~\ref{fig:td1_ex4_f1}.

\end{td-exo}

% ----- Solutions exo 5 ----- %
\iftoggle{showsolutions}{ 
	\begin{td-sol}[]\ % 5
		\begin{itemize}
			\item \textbf{Stratégie 1:} On peut utiliser une approche de type Dijkstra, où à chaque étape, on explore le nœud avec la plus petite distance cumulée depuis le point de départ. On maintient une liste des nœuds à explorer et une table des distances minimales trouvées jusqu'à présent.

			\item \textbf{Stratégie 2:} On peut également appliquer une variante du branch and bound comme suit:
			\begin{itemize}
				\item On commence par calculer une borne supérieure du coût du chemin, par exemple en exhibant un chemin quelconque du point de départ à la destination, ici \(s \to a \to e \to h \to t\) avec un coût de \(2 + 2 + 3 + 4 = 11\).
				\item Ensuite, on explore les branches du graphe en interdisant un arc à la fois, en recalculant les coûts minimaux pour chaque branche.
				\item Si une branche mène à un coût supérieur à la borne supérieure actuelle, on peut la couper (pruning).
				\item On continue ce processus jusqu'à ce que toutes les branches aient été explorées ou coupées, et la meilleure solution trouvée sera le plus court chemin.
			\end{itemize}
		\end{itemize}

		En appliquant ces stratégies sur le graphe de la figure~\ref{fig:td1_ex4_f1}, on trouve que le plus court chemin de \(s\) à \(t\) est \(s \to b \to f \to i \to t\) avec un coût total de \(8\).
	\end{td-sol}
}{}
