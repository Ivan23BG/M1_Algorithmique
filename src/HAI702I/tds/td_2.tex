
% ----- Consignes exo 4 ----- %
\begin{td-exo}[] \(\star\) \,\\% 4
	Dans l'espace muni d'un repère orthonormal. 
	On note \(\mathcal D\) la droite passant par le point \(A = (1, 3, -2)\) et
	de vecteur directeur \(u = (2, 1, 0)\), 
	\(\mathcal P\) le plan d'équation \(2x - 3y + 5z= 7\) et 
	\(M\) le point de coordonnées \((1, 2, 3)\).

	\begin{enumerate}
		\item Calculer la distance de \(M\) à la droite \(\mathcal D\).

		\item Calculer la distance de \(M\) au plan \(\mathcal P\).\\
		\textit{Indication: remarquer que le point \((1, 0, 1)\) appartient au plan \(\mathcal P\).}
	\end{enumerate}
\end{td-exo}

% ----- Solutions exo 4 ----- %
\iftoggle{showsolutions}{
	\begin{td-sol}[]\ %
        
        
