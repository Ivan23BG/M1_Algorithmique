Les exercices ou les questions marqués d'une étoile ne sont pas prioritaires.
% ----- Consignes exo 1 ----- %
\begin{td-exo}[Transformation de \(\bb R^2\)] \,\\% 1
	Ecrire pour chaque application linéaire ci dessous la matrice (dans la base canonique) de:
	\begin{enumerate}
		\item la rotation d'angle \(\theta\) et de centre \((0, 0)\),

		\item la projection sur la droite \(\vect\begin{pmatrix}
			a_1\\
			a_2
		\end{pmatrix}\),

		\item la symétrie par rapport à la droite \(\vect\begin{pmatrix}
			a_1\\
			a_2
		\end{pmatrix}\).
	\end{enumerate}
\end{td-exo}

% ----- Solutions exo 1 ----- %
\iftoggle{showsolutions}{
	\begin{td-sol}[]\ %
		\begin{enumerate}
			\item La matrice de la rotation d'angle \(\theta\) et de centre \((0, 0)\) est:
			\begin{equation*}
				R_\theta = 
				\begin{pmatrix}
					\cos(\theta) & -\sin(\theta)\\
					\sin(\theta) & \cos(\theta)
				\end{pmatrix}.
			\end{equation*}

			\item La matrice de la projection sur la droite \(\vect\begin{pmatrix}
				a_1\\
				a_2
			\end{pmatrix}\) est:
			\begin{equation*}
				P = \frac{1}{a_1^2 + a_2^2}
				\begin{pmatrix}
					a_1^2 & a_1 a_2\\
					a_1 a_2 & a_2^2
				\end{pmatrix}.
			\end{equation*}

			\item La matrice de la symétrie par rapport à la droite \(\vect\begin{pmatrix}
				a_1\\
				a_2
			\end{pmatrix}\) est:
			\begin{equation*}
				S = \frac{1}{a_1^2 + a_2^2}
				\begin{pmatrix}
					a_1^2 - a_2^2 & 2 a_1 a_2\\
					2 a_1 a_2 & a_2^2 - a_1^2
				\end{pmatrix}.
			\end{equation*}
		\end{enumerate}
	\end{td-sol}
}{}


% ----- Consignes exo 2 ----- %
\begin{td-exo}[] \,\\% 2
	Soit \(a,b\in\bb R^3\). On note \(a_{\perp b}\) le vecteur projeté de \(a\) sur le plan orthogonal à \(b\).
	\begin{enumerate}
		\item Exprimer \(a_{\perp b}\) en fonction de \(a\) et \(b\).
		\item Démontrer que \(a_{\perp b} = \frac{(b\wedge a)\wedge b}{\nn{b}^2}\).
		\item Trouver une matrice \(M \in\bb R^{3\times 3}\) telle que \(a_{\perp b} = M a\).
		Est-elle inversible?
	\end{enumerate}
\end{td-exo}

% ----- Solutions exo 2 ----- %
\iftoggle{showsolutions}{
	\begin{td-sol}[]\ %
		Non traité.
	\end{td-sol}
}{}


% ----- Consignes exo 3 ----- %
\begin{td-exo}[Inverser des matrices sans calculs] \,% 3
	\begin{enumerate}
		\item Soit
		\begin{equation*}
			A = 
			\begin{pmatrix}
				-1 & 1 & 1\\
				1 & -1 & 1\\
				1 & 1 & -1
			\end{pmatrix}.
		\end{equation*}
		Montrer que \(A^2 = 2I_3 - A\). En déduire que \(A\) est inversible et calculer \(A^{-1}\).

		\item Soit 
		\begin{equation*}
			A = 
			\begin{pmatrix}
				1 & 0 & 2\\
				0 & -1 & 1\\
				1 & -2 & 0
			\end{pmatrix}.
		\end{equation*}
		Calculer \(A^3-A\). En déduire que \(A\) est inversible et calculer \(A^{-1}\).

		\item Soit \(A\in \bb R^{n\times n}\) une matrice nilpotente, c'est-à-dire qu'il existe \(p\in\bb N\) tel que \(A^p = 0\).
		Démontrer que la matrice \(I_n - A\) est inversible et déterminer son inverse.
	\end{enumerate}
\end{td-exo}

% ----- Solutions exo 3 ----- %
\iftoggle{showsolutions}{
	\begin{td-sol}[]\ %
		\begin{enumerate}
			\item On calcule:
			\begin{equation*}
				A^2 = 
				\begin{pmatrix}
					-1 & 1 & 1\\
					1 & -1 & 1\\
					1 & 1 & -1
				\end{pmatrix}
				\begin{pmatrix}
					-1 & 1 & 1\\
					1 & -1 & 1\\
					1 & 1 & -1
				\end{pmatrix}
				=
				\begin{pmatrix}
					3 & -1 & -1\\
					-1 & 3 & -1\\
					-1 & -1 & 3
				\end{pmatrix}
				= 2I_3 - A.
			\end{equation*}
			Donc \(A^2 = 2I_3 - A\). Il en suit
			\begin{equation*}
				\begin{aligned}
					A^2 = 2I_3 - A
					&\iff A^2 + A = 2I_3\\
					&\iff \frac12(A^2 + A) = I_3\\
					&\iff A\left(\frac12(A + I_3)\right) = I_3.
				\end{aligned}
			\end{equation*}
			Donc \(A\) est inversible et
			\begin{equation*}
				\boxed{A^{-1} = \frac12(A + I_3)}.
			\end{equation*}

			\item On calcule:
			\begin{equation*}
				A^2 = 
				\begin{pmatrix}
					1 & 0 & 2\\
					0 & -1 & 1\\
					1 & -2 & 0
				\end{pmatrix}
				\begin{pmatrix}
					1 & 0 & 2\\
					0 & -1 & 1\\
					1 & -2 & 0
				\end{pmatrix}
				=
				\begin{pmatrix}
					3 & -4 & 2\\
					1 & -1 & -1\\
					1 & 2 & 0
				\end{pmatrix}.
			\end{equation*}
			\begin{equation*}
				A^3 = 
				\begin{pmatrix}
					1 & 0 & 2\\
					0 & -1 & 1\\
					1 & -2 & 0
				\end{pmatrix}
				\begin{pmatrix}
					3 & -4 & 2\\
					1 & -1 & -1\\
					1 & 2 & 0
				\end{pmatrix}
				=
				\begin{pmatrix}
					5 & 0 & 2\\
					0 & 3 & 1\\
					1 & -2 & 4
				\end{pmatrix}.
			\end{equation*}
			On remarque alors que
			\begin{equation*}
				\begin{aligned}
					A^3 - A
					&= 
					\begin{pmatrix}
						5 & 0 & 2\\
						0 & 3 & 1\\
						1 & -2 & 4
					\end{pmatrix}
					-
					\begin{pmatrix}
						1 & 0 & 2\\
						0 & -1 & 1\\
						1 & -2 & 0
					\end{pmatrix}\\
					&=
					\begin{pmatrix}
						4 & 0 & 0\\
						0 & 4 & 0\\
						0 & 0 & 4
					\end{pmatrix}
					= 4I_3.
				\end{aligned}
			\end{equation*}
			Comme précédemment, on en déduit que \(A\) est inversible et
			\begin{equation*}
				\boxed{A^{-1} = \frac14(A^2 - I_3)}.
			\end{equation*}

			\item Soit \(A\in \bb R^{n\times n}\) une matrice nilpotente. On cherche un inverse \(X\)
			de sorte que \((I_n - A)X = I_n\). On remarque que dans le cadre des réels,
			l'inverse de \(1 - x\) est \(\frac{1}{1 - x}\). Mais aussi:
			\begin{equation*}
				\frac{1}{1 - x} = 1 + x + x^2 + x^3 + \cdots = \sum_{k=0}^{\infty} x^k.
			\end{equation*}
			On pose alors \(Y\) un tel candidat:
			\begin{equation*}
				Y = I_n + A + A^2 + A^3 + \cdots = \sum_{k=0}^{p} A^k.
			\end{equation*}
			Où \(p\) est le plus petit entier tel que \(A^p = 0\). Il suit alors:
			\begin{equation*}
				\begin{aligned}
					(I_n - A)Y
					&= Y - AY\\
					&= \sum_{k=0}^{p} A^k - A \sum_{k=0}^{p} A^{k}\\
					&= \sum_{k=0}^{p} A^k - \sum_{k=0}^{p} A^{k+1}\\
					&= \sum_{k=0}^{p} A^k - \sum_{k=1}^{p+1} A^{k}\\
					&= \sum_{k=0}^{p} A^k - \sum_{k=1}^{p} A^{k} - A^{p+1}\\
					&= I_n + \sum_{k=1}^{p} A^k - \sum_{k=1}^{p} A^{k} - 0\\
					&= I_n.
				\end{aligned}
			\end{equation*}
			Notre candidat était bien choisi, donc \(I_n - A\) est inversible et
			\begin{equation*}
				\boxed{{(I_n - A)}^{-1} = \sum_{k=0}^{p} A^k}.
			\end{equation*}
		\end{enumerate}
	\end{td-sol}
}{}


% ----- Consignes exo 4 ----- %
\begin{td-exo}[Déterminant d'une matrice triangulaire] \(\star\) \,\\% 4
	Démontrer que le déterminant d'une matrice triangulaire est le produit de ses
	entrées diagonales.
\end{td-exo}

% ----- Solutions exo 4 ----- %
\iftoggle{showsolutions}{
	\begin{td-sol}[]\ %
		Non traité.
	\end{td-sol}
}{}


% ----- Consignes exo 5 ----- %
\begin{td-exo}[Inverser des matrices avec calculs] \,\\% 5
	A l'aide du pivot de Gauss, dire si les matrices suivantes sont inversibles et,
	le cas échéant, calculer leur inverse:
	\begin{enumerate}
		\item Soit
		\begin{equation*}
			A = 
			\begin{pmatrix}
				1 & 1 & 2\\
				1 & 2 & 1\\
				2 & 1 & 1
			\end{pmatrix}.
		\end{equation*}

		\item Soit
		\begin{equation*}
			B = 
			\begin{pmatrix}
				1 & 4 & 7\\
				2 & 5 & 8\\
				3 & 6 & 9
			\end{pmatrix}.
		\end{equation*}
	\end{enumerate}
\end{td-exo}

% ----- Solutions exo 5 ----- %
\iftoggle{showsolutions}{
	\begin{td-sol}[]\ %
		Non traité.
	\end{td-sol}
}{}


% ----- Consignes exo 6 ----- %
\begin{td-exo}[Décomposition d'une rotation] \(\star\) \,\\% 6
	On appelle cisaillement horizontal (\(x\)-shear) les transformations linéaires 
	de \(\bb R^2\) dont la matrice (dans les bases canoniques) est de la forme
	\begin{equation*}
		H_x = 
		\begin{pmatrix}
			1 & x\\
			0 & 1
		\end{pmatrix}
		\quad\text{avec } x\in\bb R.
	\end{equation*}
	On appelle cisaillement vertical (\(y\)-shear) les transformations linéaires
	de \(\bb R^2\) dont la matrice (dans les bases canoniques) est de la forme
	\begin{equation*}
		V_y = 
		\begin{pmatrix}
			1 & 0\\
			y & 1
		\end{pmatrix}
		\quad\text{avec } y\in\bb R.
	\end{equation*}
	
	\begin{enumerate}
		\item Représenter l'effet de ces transformations sur la base canonique.

		\item Soit \(R_{-\theta}\) la matrice de rotation d'angle \(-\theta\in\bb R\).
		Démontrer la décomposition suivante:
		\begin{equation*}
			R_{-\theta} = H_{-\tan(\frac{\theta}{2})} V_{-\sin(\theta)} H_{\tan(\frac{\theta}{2})}.
		\end{equation*}
	\end{enumerate}
\end{td-exo}

% ----- Solutions exo 6 ----- %
\iftoggle{showsolutions}{
	\begin{td-sol}[]\ %
		Non traité.
	\end{td-sol}
}{}


% ----- Consignes exo 7 ----- %
\begin{td-exo}[] \(\star\) \,\\% 7
	Soit \(A\) une matrice carrée à coefficients réels. Si \(A\) est inversible,
	est-ce que \(A^t\) est inversible? Si oui, quel est son inverse? Justifier.
\end{td-exo}

% ----- Solutions exo 7 ----- %
\iftoggle{showsolutions}{
	\begin{td-sol}[]\ %
		Non traité.
	\end{td-sol}
}{}