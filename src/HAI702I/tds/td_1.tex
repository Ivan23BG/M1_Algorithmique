Les exercices ou les questions marqués d'une étoile ne sont pas prioritaires.
% ----- Consignes exo 1 ----- %
\begin{td-exo}[] \,\\% 1
	Déterminer une base orthonormale directe done le premier vecteur est colinéaire au vecteur 
	\((1, 2, 2)\).
\end{td-exo}

% ----- Solutions exo 1 ----- %
\iftoggle{showsolutions}{
	\begin{td-sol}[]\ %
		On pose \(u = \frac{1}{3}(1, 2, 2)\), 
		c'est à dire celui imposé puis normalisé.

		Ensuite on veut \(v\) orthogonal à \(u\) et de norme \(1\).

		Il faut donc \(\langle u, v\rangle = 0\), soit \((0,1,-1)\).

		On le normalise en divisant par \(\sqrt{2}\): \(v = \frac{1}{\sqrt{2}}(0, 1, -1)\).

		Pour \(w\) on peut faire le produit vectoriel \(u \wedge v\):
		
		On trouve \(w = \frac{1}{3\sqrt{2}}(4, 1, 1)\).

		On pose alors notre base orthonormale directe:
		\begin{equation*}
			\mathcal B = \left(u, v, w\right) = \left(\frac{1}{3}(1, 2, 2), \frac{1}{\sqrt{2}}(0, 1, -1), \frac{1}{3\sqrt{2}}(4, 1, 1)\right).
		\end{equation*}
	\end{td-sol}
}{}


% ----- Consignes exo 2 ----- %
\begin{td-exo}[] \,\\% 2
	Pour quelles valeurs de \(a\) les vecteurs suivant sont-ils coplanaires?
	\begin{itemize}
		\item \((1, 0, a)\),

		\item \((a, 1, 0)\),

		\item \((0, a, 1)\).
	\end{itemize}
\end{td-exo}

% ----- Solutions exo 2 ----- %
\iftoggle{showsolutions}{
	\begin{td-sol}[]\ %
		On trouve assez vite que \(-1\) est solution.
		C'est la seule solution réelle, prouvable avec factorisation puis delta.
	\end{td-sol}
}{}


% ----- Consignes exo 3 ----- %
\begin{td-exo}[] \,\\% 3
	Soient \(u, v\) et \(w\) trois vecteurs de l'espace et \(a\in\bb R\).
	On considère l'équation vectorielle d'inconnue \(x\) suivante:
	\begin{equation*}
		u \wedge x = v
	\end{equation*}

	\begin{enumerate}
		\item Montrer que si l'équation admet une solution, alors \(u\) et \(v\) sont orthogonaux.
		On supposera dans la suite que \(u\) et \(v\) sont orthogonaux.

		\item Déterminer toutes les solutions colinéaires à \(u \wedge v\).

		\item En déduire toutes les solutions de l'équation.

		\item Déterminer les vecteurs solutions qui vérifient en outre \(\langle x, w \rangle = a\).
	\end{enumerate}
\end{td-exo}

% ----- Solutions exo 3 ----- %
\iftoggle{showsolutions}{
	\begin{td-sol}[]\ %
		
	\end{td-sol}
}{}


% ----- Consignes exo 4 ----- %
\begin{td-exo}[] \(\star\) \,\\% 4
	Dans l'espace muni d'un repère orthonormal. 
	On note \(\mathcal D\) la droite passant par le point \(A = (1, 3, -2)\) et
	de vecteur directeur \(u = (2, 1, 0)\), 
	\(\mathcal P\) le plan d'équation \(2x - 3y + 5z= 7\) et 
	\(M\) le point de coordonnées \((1, 2, 3)\).

	\begin{enumerate}
		\item Calculer la distance de \(M\) à la droite \(\mathcal D\).

		\item Calculer la distance de \(M\) au plan \(\mathcal P\).\\
		\textit{Indication: remarquer que le point \((1, 0, 1)\) appartient au plan \(\mathcal P\).}
	\end{enumerate}
\end{td-exo}

% ----- Solutions exo 4 ----- %
\iftoggle{showsolutions}{
	\begin{td-sol}[]\ %
		\begin{enumerate}
			\item La distance de \(M\) à la droite \(\mathcal D\) est donnée par la formule:
			\begin{equation*}
				d(M, \mathcal D) = \frac{\norm{AM \wedge u}}{\norm{u}} = \frac{\sqrt{30}}{\sqrt{5}} = \sqrt{6}.
			\end{equation*}
		\end{enumerate}
	\end{td-sol}
}{}


% ----- Consignes exo 5 ----- %
\begin{td-exo}[] \(\star\) \,\\% 5
	Déterminer la projection orthogonale \(\Delta'\) de la droite \(\Delta\) d'équation:
	\begin{equation*}
		\begin{cases}
			x = 1 + 2\lambda\\
			y = -1 + \lambda\\
			z = 2
		\end{cases}
	\end{equation*}
	dans le plan \(\mathcal P\) d'équation \(x + y + z = 1\).
\end{td-exo}

% ----- Solutions exo 5 ----- %
\iftoggle{showsolutions}{
	\begin{td-sol}[]\ %
		
	\end{td-sol}
}{}


% ----- Consignes exo 6 ----- %
\begin{td-exo}[] \(\star\) \,\\% 6
	Calculer l'équation de la sphère de centre \((1, 1, 1)\) et 
	dont le plan tangent est \(x + y + z = 2\).
\end{td-exo}

% ----- Solutions exo 6 ----- %
\iftoggle{showsolutions}{
	\begin{td-sol}[]\ %
		
	\end{td-sol}
}{}