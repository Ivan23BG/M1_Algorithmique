Les exercices ou les questions marqués d'une étoile ne sont pas prioritaires.
% ----- Consignes exo 1 ----- %
\begin{td-exo}[] \,\\% 1
	Déterminer une base orthonormale directe dont le premier vecteur est colinéaire au vecteur 
	\((1, 2, 2)\).
\end{td-exo}

% ----- Solutions exo 1 ----- %
\iftoggle{showsolutions}{
	\begin{td-sol}[]\ %
		On rappelle les définitions importantes:
		\begin{itemize}
			\item Deux vecteurs \(u\) et \(v\) sont orthogonaux si \(\langle u, v \rangle = 0\).

			\item Une base est orthonormale si ses vecteurs sont de norme \(1\) et deux à deux orthogonaux.

			\item Une base est directe si le produit vectoriel du premier vecteur par le deuxième donne le troisième.
		\end{itemize}

		Commençons par choisir notre premier vecteur \(u\).
		On veut \(u\) colinéaire à \((1, 2, 2)\) donc on a
		\begin{equation*}
			u = \lambda(1, 2, 2),\quad \lambda \in \bb R^*.
		\end{equation*}
		Ensuite on veut que \(u\) soit de norme \(1\):
		\begin{equation*}
			\begin{aligned}
				\nn{u} = 1 
				& \iff \sqrt{\sum_{i=1}^3 u_i^2} = 1\\
				& \iff \sum_{i=1}^3 u_i^2 = 1\\
				& \iff \lambda^2(1^2 + 2^2 + 2^2) = 1\\
				& \iff 9\lambda^2 = 1\\
				& \iff \lambda = \pm\frac{1}{3}.
			\end{aligned}
		\end{equation*}
		On prend \(\lambda = \frac{1}{3}\), donc
		\begin{equation*}
			\boxed{u = \frac{1}{3}(1, 2, 2)}.
		\end{equation*}

		Ensuite on veut choisir \(v\). Il faut que \(v\) soit orthogonal à \(u\):
		\begin{equation*}
			\langle u, v \rangle = 0.
		\end{equation*}
		Si on note \(v' = (x, y, z)\), on a:
		\begin{equation*}
			\begin{aligned}
				\langle u, v' \rangle = 0
				& \iff \frac{1}{3}(1, 2, 2) \cdot (x, y, z) = 0
				& \iff \frac{1}{3}(x + 2y + 2z) = 0.
				& \iff x + 2y + 2z = 0.
			\end{aligned}
		\end{equation*}

		Pour des questions de simplicité, on peut choisir \(x = 0, y = 1\) et \(z = -1\).
		Il faut ensuite normaliser \(v'\) pour obtenir \(v\).
		\begin{equation*}
			\nn{v'} = \sqrt{0^2 + 1^2 + {(-1)}^2} = \sqrt{2}.
		\end{equation*}
		On peut alors prendre \(\boxed{v = \frac{1}{\sqrt{2}}(0, 1, -1)}\).

		Enfin, pour \(w\), on peut faire le produit vectoriel \(u \wedge v\) pour être sûr que la base soit directe.
		Il y a plusieurs manières de faire le calcul, ici on utilise le déterminant:
		% description step by steop
		\begin{equation*}
			w = u \wedge v = 
			\begin{vmatrix}
				e_1 & e_2 & e_3\\
				\frac{1}{3} & \frac{2}{3} & \frac{2}{3}\\
				0 & \frac{1}{\sqrt{2}} & -\frac{1}{\sqrt{2}}
			\end{vmatrix}
			= e_1
			\begin{vmatrix}
				\frac{2}{3} & \frac{2}{3}\\
				\frac{1}{\sqrt{2}} & -\frac{1}{\sqrt{2}}
			\end{vmatrix}
			- e_2
			\begin{vmatrix}
				\frac{1}{3} & \frac{2}{3}\\
				0 & -\frac{1}{\sqrt{2}}
			\end{vmatrix}
			+ e_3
			\begin{vmatrix}
				\frac{1}{3} & \frac{2}{3}\\
				0 & \frac{1}{\sqrt{2}}
			\end{vmatrix}.
		\end{equation*}
		On trouve:
		\begin{equation*}
			\boxed{w = \frac{1}{3\sqrt{2}}(-4, 1, 1)}.
		\end{equation*}

		On a alors notre base orthonormale directe:
		\begin{equation*}
			\boxed{\mathcal B = \left(u, v, w\right) = \left(\frac{1}{3}(1, 2, 2), \frac{1}{\sqrt{2}}(0, 1, -1), \frac{1}{3\sqrt{2}}(-4, 1, 1)\right)}.
		\end{equation*}

		On peut vérifier rapidement qu'on a bien:
		\begin{itemize}
			\item \(\nn{u} = 1\), \(\nn{v} = 1\) et \(\nn{w} = 1\).

			\item \(\langle u, v \rangle = 0\), \(\langle u, w \rangle = 0\) et \(\langle v, w \rangle = 0\).

			\item \(u \wedge v = w\).
		\end{itemize}
	\end{td-sol}
}{}


% ----- Consignes exo 2 ----- %
\begin{td-exo}[] \,\\% 2
	Pour quelles valeurs de \(a\) les vecteurs suivant sont-ils coplanaires?
	\begin{itemize}
		\item \((1, 0, a)\),

		\item \((a, 1, 0)\),

		\item \((0, a, 1)\).
	\end{itemize}
\end{td-exo}

% ----- Solutions exo 2 ----- %
\iftoggle{showsolutions}{
	\begin{td-sol}[]\ %
		Trois vecteurs sont coplanaires si et seulement si la 
		matrice formée par ces vecteurs a un déterminant nul.
		Ici on a:
		\begin{equation*}
			\begin{vmatrix}
				1 & 0 & a\\
				a & 1 & 0\\
				0 & a & 1
			\end{vmatrix}
			= 1 + a^3.
		\end{equation*}
		On cherche donc les valeurs de \(a\) telles que:
		\begin{equation*}
			1 + a^3 = 0 \iff a^3 = -1 \iff a = -1.
		\end{equation*}
		Il n'y a pas d'autres solutions réelles. On peut le
		vérifier en factorisant \(a^3 + 1\):
		\begin{equation*}
			a^3 + 1 = (a + 1)(a^2 - a + 1).
		\end{equation*}
		où \(a^2 - a + 1\) n'a pas de racines réelles.

		Donc les vecteurs sont coplanaires si et seulement si \(\boxed{a = -1}\).
	\end{td-sol}
}{}


% ----- Consignes exo 3 ----- %
\begin{td-exo}[] \,\\% 3
	Soient \(u, v\) et \(w\) trois vecteurs de l'espace et \(a\in\bb R\).
	On considère l'équation vectorielle d'inconnue \(x\) suivante:
	\begin{equation*}
		u \wedge x = v
	\end{equation*}

	\begin{enumerate}
		\item Montrer que si l'équation admet une solution, alors \(u\) et \(v\) sont orthogonaux.
		On supposera dans la suite que \(u\) et \(v\) sont orthogonaux.

		\item Déterminer toutes les solutions colinéaires à \(u \wedge v\).

		\item En déduire toutes les solutions de l'équation.

		\item Déterminer les vecteurs solutions qui vérifient en outre \(\langle x, w \rangle = a\).
	\end{enumerate}
\end{td-exo}

% ----- Solutions exo 3 ----- %
\iftoggle{showsolutions}{
	\begin{td-sol}[]\ %
		
	\end{td-sol}
}{}


% ----- Consignes exo 4 ----- %
\begin{td-exo}[] \(\star\) \,\\% 4
	Dans l'espace muni d'un repère orthonormal. 
	On note \(\mathcal D\) la droite passant par le point \(A = (1, 3, -2)\) et
	de vecteur directeur \(u = (2, 1, 0)\), 
	\(\mathcal P\) le plan d'équation \(2x - 3y + 5z= 7\) et 
	\(M\) le point de coordonnées \((1, 2, 3)\).

	\begin{enumerate}
		\item Calculer la distance de \(M\) à la droite \(\mathcal D\).

		\item Calculer la distance de \(M\) au plan \(\mathcal P\).\\
		\textit{Indication: remarquer que le point \((1, 0, 1)\) appartient au plan \(\mathcal P\).}
	\end{enumerate}
\end{td-exo}

% ----- Solutions exo 4 ----- %
\iftoggle{showsolutions}{
	\begin{td-sol}[]\ %
		\begin{enumerate}
			\item Il y a plusieurs manières de calculer la distance d'un point à une droite.

			Première méthode: on cherche le point \(X\) de la droite \(\mathcal D\) tel que le segment \(MX\) soit orthogonal à la droite.
			Le point \(X\) de la droite \(\mathcal D\) s'écrit:
			\begin{equation*}
				X = A + \lambda u = (1, 3, -2) + \lambda(2, 1, 0) = (1 + 2\lambda, 3 + \lambda, -2).
			\end{equation*}
			On cherche \(\lambda\) tel que \(MX\) soit orthogonal à \(u\), c'est-à-dire:
			\begin{equation*}
				\begin{aligned}
					&\langle MX, u \rangle = 0\\
					\iff & \langle X - M, u \rangle = 0\\
					\iff & \langle (1 + 2\lambda - 1, 3 + \lambda - 2, -2 - 3), (2, 1, 0) \rangle = 0\\
					\iff & \langle (2\lambda, 1 + \lambda, -5), (2, 1, 0) \rangle = 0\\
					\iff & 4\lambda + 1 + \lambda = 0\\
					\iff & 5\lambda + 1 = 0\\
					\iff & \lambda = -\frac{1}{5}.
				\end{aligned}
			\end{equation*}
			On en déduit:
			\begin{equation*}
				X = \left(1 - \frac{2}{5}, 3 - \frac{1}{5}, -2\right) = \left(\frac{3}{5}, \frac{14}{5}, -2\right).
			\end{equation*}
			La distance cherchée est donc:
			\begin{equation*}
				\begin{aligned}
					d(M, \mathcal D) 
					&= \nn{MX} \\
					&= \sqrt{{\left(\frac{3}{5} - 1\right)}^2 + {\left(\frac{14}{5} - 2\right)}^2 + {(-2 - 3)}^2} \\
					&= \sqrt{\frac{4}{25} + \frac{16}{25} + 25} \\
					&= \sqrt{\frac{4 + 16 + 625}{25}} \\
					&= \sqrt{\frac{645}{25}} \\
					&= \frac{\sqrt{645}}{5}.
				\end{aligned}
			\end{equation*}

			Deuxième méthode: on utilise la formule de la distance d'un point à une droite:
			\begin{equation*}
				d(M, \mathcal D) = \frac{\nn{AM \wedge u}}{\nn{u}}.
			\end{equation*}
			On a:
			\begin{equation*}
				AM = M - A = (1 - 1, 2 - 3, 3 - (-2)) = (0, -1, 5).
			\end{equation*}
			Calculons le produit vectoriel:
			\begin{equation*}
				AM \wedge u = 
				\begin{vmatrix}
					e_1 & e_2 & e_3\\
					0 & -1 & 5\\
					2 & 1 & 0
				\end{vmatrix}
				= e_1
				\begin{vmatrix}
					-1 & 5\\
					1 & 0
				\end{vmatrix}
				- e_2
				\begin{vmatrix}
					0 & 5\\
					2 & 0
				\end{vmatrix}
				+ e_3
				\begin{vmatrix}
					0 & -1\\
					2 & 1
				\end{vmatrix}
				= -5e_1 + 10e_2 + 2e_3.
			\end{equation*}
			On en déduit:
			\begin{equation*}
				\nn{AM \wedge u} = \sqrt{{(-5)}^2 + 10^2 + 2^2} = \sqrt{25 + 100 + 4} = \sqrt{129}.
			\end{equation*}
			De plus:
			\begin{equation*}
				\nn{u} = \sqrt{2^2 + 1^2 + 0^2} = \sqrt{5}.
			\end{equation*}
			On trouve donc:
			\begin{equation*}
				\boxed{d(M, \mathcal D) = \frac{\sqrt{129}}{\sqrt{5}} = \frac{\sqrt{645}}{5}}.
			\end{equation*}

			\item De même, il y a plusieurs manières de calculer la distance d'un point à un plan.

			Première méthode: on cherche le point \(Y\) du plan \(\mathcal P\) tel que le segment \(MY\) soit orthogonal au plan.
			Le plan \(\mathcal P\) est orthogonal au vecteur normal \(n = (2, -3, 5)\). Ainsi, \(Y\) s'écrit:
			\begin{equation*}
				Y = M + \mu n = (1, 2, 3) + \mu(2, -3, 5) = (1 + 2\mu, 2 - 3\mu, 3 + 5\mu).
			\end{equation*}
			On cherche \(\mu\) tel que \(Y\) appartienne au plan, c'est-à-dire:
			\begin{equation*}
				\begin{aligned}
					&2(1 + 2\mu) - 3(2 - 3\mu) + 5(3 + 5\mu) = 7\\
					\iff & 2 + 4\mu - 6 + 9\mu + 15 + 25\mu = 7\\
					\iff & 38\mu + 11 = 7\\
					\iff & 38\mu = -4\\
					\iff & \mu = -\frac{2}{19}.
				\end{aligned}
			\end{equation*}
			On en déduit:
			\begin{equation*}
				\begin{aligned}
					Y 
					&= \left(1 - \frac{4}{19}, 2 + \frac{6}{19}, 3 - \frac{10}{19}\right)\\
					&= \left(\frac{15}{19}, \frac{44}{19}, \frac{47}{19}\right).
				\end{aligned}
			\end{equation*}
			La distance cherchée est donc:
			\begin{equation*}
				\begin{aligned}
					d(M, \mathcal P) 
					&= \nn{MY} \\
					&= \sqrt{{\left(\frac{15}{19} - 1\right)}^2 + {\left(\frac{44}{19} - 2\right)}^2 + {\left(\frac{47}{19} - 3\right)}^2} \\
					&= \sqrt{\frac{16}{361} + \frac{36}{361} + \frac{100}{361}} \\
					&= \sqrt{\frac{16 + 36 + 100}{361}} \\
					&= \sqrt{\frac{152}{361}} \\
					&= \frac{\sqrt{152}}{19}.
				\end{aligned}
			\end{equation*}

			Deuxième méthode: on utilise la formule de la distance d'un point à un plan:
			\begin{equation*}
				d(M, \mathcal P) = \frac{\n{ax_0 + by_0 + cz_0 + d}}{\sqrt{a^2 + b^2 + c^2}},
			\end{equation*}
			où \(ax + by + cz + d = 0\) est l'équation du plan et \(M = (x_0, y_0, z_0)\).
			Ici, \(a = 2\), \(b = -3\), \(c = 5\) et \(d = -7\).
			On a:
			\begin{equation*}
				\begin{aligned}
					d(M, \mathcal P) 
					&= \frac{\n{2 \cdot 1 - 3 \cdot 2 + 5 \cdot 3 - 7}}{\sqrt{2^2 + {(-3)}^2 + 5^2}}\\
					&= \frac{\n{2 - 6 + 15 - 7}}{\sqrt{4 + 9 + 25}}\\
					&= \frac{\n{4}}{\sqrt{38}}\\
					&= \frac{4}{\sqrt{38}} = \frac{4\sqrt{38}}{38} = \frac{2\sqrt{38}}{19} = \frac{\sqrt{152}}{19}.
				\end{aligned}
			\end{equation*}
			La distance cherchée est donc:
			\begin{equation*}
				\boxed{d(M, \mathcal P) = \frac{\sqrt{152}}{19}}.
			\end{equation*}
		\end{enumerate}
	\end{td-sol}
}{}


% ----- Consignes exo 5 ----- %
\begin{td-exo}[] \(\star\) \,\\% 5
	Déterminer la projection orthogonale \(\Delta'\) de la droite \(\Delta\) d'équation:
	\begin{equation*}
		\begin{cases}
			x = 1 + 2\lambda\\
			y = -1 + \lambda\\
			z = 2
		\end{cases}
	\end{equation*}
	dans le plan \(\mathcal P\) d'équation \(x + y + z = 1\).
\end{td-exo}

% ----- Solutions exo 5 ----- %
\iftoggle{showsolutions}{
	\begin{td-sol}[]\ %
		
	\end{td-sol}
}{}


% ----- Consignes exo 6 ----- %
\begin{td-exo}[] \(\star\) \,\\% 6
	Calculer l'équation de la sphère de centre \((1, 1, 1)\) et 
	dont le plan tangent est \(x + y + z = 2\).
\end{td-exo}

% ----- Solutions exo 6 ----- %
\iftoggle{showsolutions}{
	\begin{td-sol}[]\ %
		
	\end{td-sol}
}{}