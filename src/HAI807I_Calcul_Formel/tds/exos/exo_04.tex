% ----- Consignes exo 4 ----- %
\begin{td-exo}[]\,\\ % 4 
	Coder dans Sagemath les opérations d'addition et de multiplication de deux polynomes qui sont donnés par la liste de leurs coefficients. 
	Le premier élément de la liste correspondra à l'élément constant du polynome.
	Vous testerez vos fonctions sur des polynomes aléatoires de degrés dans \(\left\{10, 100, 1000\right\}\) avec les domaines de coefficients suivants: \(\bb{GF}(17), \bb Z, \bb Z/100\bb Z\).
	A chaque fois vous prendrez soin de faire des opérations impliquant aussi bien des polynomes de degrés identiques que des polynomes de degrés différents.
	Vous devrez comparer les résultats de vos calculs avec celui fait par les opérations natives d'addition et de multiplication de polynomes dans Sage.
\end{td-exo}

% ----- Solutions exo 4 ----- %
\iftoggle{showsolutions}{ 
	\begin{td-sol}[]\ % 4
		On utilise le code suivant pour additioner nos deux polynomes:

		\SageFile{../tps/exo04_f01.py}{Fonction d'addition de polynomes}{lst:ex04_f01}

		et celui-ci pour faire la multiplication:

		\SageFile{../tps/exo04_f02.py}{Fonction de multiplication de polynomes}{lst:ex04_f02}

		Remarquons que dans les deux fonctions on donne une variable \(R\) qui correspond à l'anneau dans lequel on travaille en entrée.
		Cela est surtout utile pour initialiser les coefficients de nos polynomes à zéro.

		Une fonction importante qu'on a utilisé dans les programmes précédent est celle de normalisation.
		Elle permet de retirer les zeros en trop après les opérations (par exemple \((X + 1) + (-X + 1)\) donnera \(2\) et pas \(0X + 2\)).

		\SageFile{../tps/exo04_f03.py}{Fonction de normalisation des resultats}{lst:ex04_f03}

		Enfin avec les fonctions suivantes on peut tester que nos fonctions font bien ce qu'elles devraient:

		\SageFile{../tps/exo04_f04.py}{Fonctions pour tester}{lst:ex04_f04}

		Et alors une fonction comme celle-ci nous confirme bien que nos fonctions effectuent les bonnes opérations:

		\SageFile{../tps/exo04_f05.py}{Lancement des tests --- Resultats positifs}{lst:ex04_f05}
	\end{td-sol}
}{}
