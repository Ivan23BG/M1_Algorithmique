% ----- Consignes exo 5 ----- %
\begin{td-exo}[]\,\\ % 5 
	Nous revenons ici sur l'analyse de la complexité de l'algorithme de Karatsuba pour des polynomes de \(A[X]\) avec \(A\) un anneau quelconque.
	\begin{enumerate}
		\item Donner le nombre exact de multiplications dans \(A\) qu'effectue au total l'algorithme de Karatsuba pour multiplier deux polynomes de taille \(n\) à coefficients dans \(A\).

		\item Calculer le nombre d'additions dans \(A\) effectuées au premier niveau récursif par l'algorithme de Karatsuba.

		\item Utiliser l'analyse de complexité de Karatsuba pour déduire une borne sur la complexité exacte en nombre d'opérations dans \(A\) (on ne souhaite plus avoir de constante \(c\) dans l'analyse).

		\item Déduire, par l'utilisation de sage, une taille de polynome à partir de laquelle il serait intéressant d'utiliser l'algorithme de Karatsuba plutôt que l'algorithme naïf de multiplication de polynomes.
	\end{enumerate}
\end{td-exo}

% ----- Solutions exo 5 ----- %
\iftoggle{showsolutions}{ 
	\begin{td-sol}[]\ % 5
		\begin{enumerate}
			\item 
		\end{enumerate}
	\end{td-sol}
}{}
