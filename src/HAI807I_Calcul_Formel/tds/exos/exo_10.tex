% ----- Consignes exo 10 ----- %
\begin{td-exo}[]\, % 10 
	\begin{enumerate}
		\item Soit \(\omega = 8\) une racine \(4\)-ième primitive de l'unité dans \(\bb{GF}(13)\).
		En utilisant sage, calculer la matrice de Vandermonde \(V\) associée à la transformée de Fourier discrète en \(\omega\).
		Trouver l'inverse de \(\omega\) et déduire la matrice de Vandermonde associée à la transformée de Fourier discrète en \(\omega^{-1}\).
		Vérifer l'égalité \(V_\omega \times V_{\omega^{-1}} = n\cdot I_n\).

		\item Trouver une valeur de \(\omega\) qui soit une racine \(5\)-ième de l'unité dans \(\bb{GF}(31)\) et calculer son inverse \(\omega^{-1}\).
		En utilisant les matrices de Vandermonde associées à \(\omega\) et \(\omega^{-1}\), calculer le produit des polynomes suivants dans \(\bb{GF}(13)[X]\) uniquement en faisant des produits matrices-vecteurs:
		\begin{itemize}
			\item \(P(X) = 11X + 1\) et \(Q(X) = 5X + 18\).
			\item \(P(X) = 16X^2 + 11X + 29\) et \(Q(X) = 5X^2 + 18\).
		\end{itemize}
	\end{enumerate}
\end{td-exo}

% ----- Solutions exo 10 ----- %
\iftoggle{showsolutions}{ 
	\begin{td-sol}[]\ % 10
		
	\end{td-sol}
}{}
