% ----- Consignes exo 7 ----- %
\begin{td-exo}[]\,\\ % 7 
	Dans un premier temps, vous allez coder l'algorithme de multiplication de polynomes de Karatsuba en faisant l'hypothèse que les polynomes d'entrée sont de taille \(2^k\) pour un entier \(k \geq 0\) quelconque.
	Votre fonction prendra en paramètre la liste des coefficients des deux polynomes à multiplier.
	Ces listes devront obligatoirement faire la meme taille.
	Dans un deuxieme temps, vous proposerez une amélioration du code qui permettra de prendre en parametre des polynomes de taille quelconque.
\end{td-exo}

% ----- Solutions exo 7 ----- %
\iftoggle{showsolutions}{ 
	\begin{td-sol}[]\ % 7
		On peut coder la multiplication de Karatsuba avec des polynomes de taille une puissance de 2 comme suit:

		\SageFile{../tps/exo07_f01.py}{Multiplication de Karatsuba avec \(\deg(P) = 2^k-1\)}{lst:ex07_f01}

		qo'on peut positivement tester avec le code qui suit:

		\SageFile{../tps/exo07_f02.py}{Test de la multiplication de Karatsuba}{lst:ex07_f02}
	\end{td-sol}
}{}
