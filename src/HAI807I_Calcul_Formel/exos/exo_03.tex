% ----- Consignes exo 3 ----- %
\begin{td-exo}[]\,\\ % 3 
	On travaille dans l'anneau de polynomes \((\bb Z/6\bb Z)[x]\).
	Les coefficients sont calculés modulo 6.
	Soient \(P(x) = 2x^2 + 3\) et \(Q(x) = 3x+1\)
	\begin{enumerate}
		\item \textbf{Anomalie du degré}
		\begin{enumerate}
			\item Quels sont les degrés de \(P\) et \(Q\)?

			\item Calculer le produit \(P(x)\cdot Q(x)\) en réduisant les coefficients modulo 6.

			\item Quel est le degré du polynome résultant? La propriété \(\deg(P\cdot Q) = \deg(P) + \deg(Q)\) est-elle vérifiée? Expliquez pourquoi.
		\end{enumerate}

		\item \textbf{Echec de la division euclidienne standard}. 
		On considère les polynomes \(F(x) = x^2+1\) et \(G(x) = 2x\).
		Essayez de poser la division euclidienne de \(F\) par \(G\).

		\textit{Indice}: Pour éliminer le terme en \(x^2\), vous devez multiplier \(2x\) par un coefficient \(k\) tel que \(2\cdot k = 1\pmod 6\). Un tel entier existe-t-il dans \(\bb Z/6\bb Z\)?
		Conclure.
	\end{enumerate}
\end{td-exo}

% ----- Solutions exo 3 ----- %
\iftoggle{showsolutions}{ 
	\begin{td-sol}[]\ % 3
		\begin{enumerate}
			\item Sur les degrés:
			\begin{enumerate}
				\item \(P\) est de degré 2 et \(Q\) de degré 1.

				\item On a
				\begin{equation*}
					\begin{aligned}
						P\cdot Q
						&= \left(2x^2 + 3\right) \cdot \left(3x + 1\right)&\pmod 6\\
						&= 6x^3 + 2x^2 + 9x + 3&\pmod 6\\
						&= 2x^2 + 3x + 3&\pmod 6
					\end{aligned}
				\end{equation*}

				\item Le degré du polynome résultant est 2. La propriété \(\deg(P\cdot Q) = \deg(P) + \deg(Q)\) n'est pas vérifiée car l'anneau n'est pas intègre (par exemple \(2\cdot 3 = 0\pmod 6\)).
			\end{enumerate}

			\item 2 ne possède pas d'inverse dans \(\bb Z/6\bb Z\) car \(2\wedge 6 = 2\). On ne peut donc pas poser la division euclidienne de \(F\) par \(G\) ici.
		\end{enumerate}
	\end{td-sol}
}{}