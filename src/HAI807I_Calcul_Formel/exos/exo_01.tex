% ----- Consignes exo 1 ----- %
\begin{td-exo}[Objectif]\, % 1 
	Identifier les propriétés d'un groupe (neutre, inverses, commutativité) à partir d'une table d'opérations symbolique.

	Soit \(G = \{\alpha, \beta, \gamma, \delta\}\) un ensemble muni d'une loi de composition interne \(*\).
	La table de l'opération est donnée ci-dessous (où l'entrée à la ligne \(i\) et la colonne \(j\) correspond à \(i*j\)):
	
	\vspace{2mm}
	\centering
	\begin{tabular}{|c|c|c|c|c|} % chktex 44
		\hline % chktex 44
		\(*\)&\(\alpha\)&\(\beta\)&\(\gamma\)&\(\delta\)\\
		\hline % chktex 44
		\(\alpha\)&\(\alpha\)&\(\beta\)&\(\gamma\)&\(\delta\)\\
		\hline % chktex 44
		\(\beta\)&\(\beta\)&\(\alpha\)&\(\delta\)&\(\gamma\)\\
		\hline % chktex 44
		\(\gamma\)&\(\gamma\)&\(\delta\)&\(\alpha\)&\(\beta\)\\
		\hline % chktex 44
		\(\delta\)&\(\delta\)&\(\gamma\)&\(\beta\)&\(\alpha\)\\
		\hline % chktex 44
	\end{tabular}

	\begin{enumerate}
		\item \textbf{Identification de l'élément neutre}: Quel est l'élément neutre \(e\) de ce groupe?
		Justifiez votre réponse en observant les lignes et colonnes du tableau.

		\item \textbf{Recherche d'inverses}: Pour chaque élément \(x\) de \(G\), trouver son inverse \(x^{-1}\) (l'unique élément tel que \(x * x^{-1} = e\)). Que remarquez-vous sur l'inverse de \(\gamma\)?

		\item \textbf{Commutativité}: Le groupe \((G, *)\) est-il abélien (commutatif)?

		\item \textbf{Résolution d'équation}: A l'aide de la table, résoudre l'équation suivante d'inconnue \(x\):
		\begin{equation*}
			\gamma * x = \delta
		\end{equation*}
	\end{enumerate}
\end{td-exo}

% ----- Solutions exo 1 ----- %
\iftoggle{showsolutions}{ 
	\begin{td-sol}[]\ % 1
		\begin{enumerate}
			\item La seule ligne (et colonne) qui ne modifie aucun élément est celle de \(\alpha\) donc \(e=\alpha\).

			\item On a
			\begin{equation*}
				\begin{cases}
					\alpha^{-1} &= \alpha\\
					\beta^{-1} &= \beta\\
					\gamma^{-1} &= \gamma\\
					\delta^{-1} &= \delta\\
				\end{cases}
			\end{equation*}
			On remarque que \(\gamma\) est son propre inverse.

			\item On a bien \(\forall x,y\in G, x*y = y*x\) car le tableau est symétrique par rapport à la diagonale.
			Donc \((G,*)\) est abélien.

			\item On cherche dans le tableau l'élément \(x\) tel que \(\gamma * x = \delta\). On trouve \(x=\beta\).
		\end{enumerate}
	\end{td-sol}
}{}
