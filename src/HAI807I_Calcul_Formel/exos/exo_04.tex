% ----- Consignes exo 4 ----- %
\begin{td-exo}[]\,\\ % 4 
	Coder dans Sagemath les opérations d'addition et de multiplication de deux polynomes qui sont donnés par la liste de leurs coefficients. 
	Le premier élément de la liste correspondra à l'élément constant du polynome.
	Vous testerez vos fonctions sur des polynomes aléatoires de degrés dans \(\left\{10, 100, 1000\right\}\) avec les domaines de coefficients suivants: \(\bb{GF}(17), \bb Z, \bb Z/100\bb Z\).
	A chaque fois vous prendrez soin de faire des opérations impliquant aussi bien des polynomes de degrés identiques que des polynomes de degrés différents.
	Vous devrez comparer les résultats de vos calculs avec celui fait par les opérations natives d'addition et de multiplication de polynomes dans Sage.
\end{td-exo}

% ----- Solutions exo 4 ----- %
\iftoggle{showsolutions}{ 
	\begin{td-sol}[]\ % 4
		On utilise le code suivant:

		\SageFile{../tps/exo04_f01.py}{Fonction d'addition de polynomes}{lst:ex04_f01}

	\end{td-sol}
}{}
