\iftoggle{showquestions}{
\section*{Instructions}

Ce devoir est à rendre avant le 12 décembre 2025 à 12h, soit par mail à l'adresse:
rodolphe.giroudeau@{}lirmm.fr,
soit en déposant votre devoir durant le cours
}{}

\section{Partie théorique}
% ----- Consignes exo 1 ----- %
\iftoggle{showquestions}{
    \begin{td-exo}[Algorithmes pour la programmation linéaire]\, % 1
        Considérons la formulation suivante:
        \begin{equation*}
            P_\beta =
            \begin{cases}
                \max z = 5x_1 + 2x_2\\
                6x_1 + x_2 \geq 6\\
                4x_1 + 4x_2 \geq 12\\
                x_1 + 2x_2 \geq 4\\
                x_i \geq 0,\quad \forall i \in \{1,2\}
            \end{cases}
        \end{equation*}\,
        \begin{enumerate}
            \item Résoudre le problème \(P_\beta\) par la méthode du big \(M\).
            \item Résoudre le problème \(P_\beta\) par la méthode à deux phases.
            \item \textbf{Difficile:}
            \begin{enumerate}
                \item Résoudre le problème \(P_\beta\) par la méthode dual-simplexe.
                \item Soit le programme linéaire \(P_\theta\)
                \begin{equation*}
                    P_\theta =
                    \begin{cases}
                        \max z = x_1 + 3x_2\\
                        x_1 + x_2 \geq 3\\
                        x_1 - 2x_2 \geq 5\\
                        -2x_1 + x_2 \leq 5\\
                        x_i \geq 0,\quad \forall i \in \{1,2\}
                    \end{cases}
                \end{equation*}
                Résoudre le problème \(P_\theta\) par la méthode dual-simplexe.
            \end{enumerate}
        \end{enumerate}
    \end{td-exo}
}{}

% ----- Solutions exo 1 ----- %
\begin{td-sol}[]\ %
    \begin{enumerate}
        \item Commençons par poser le problème sous forme standard:
        \begin{equation*}
            P_\beta =
            \begin{cases}
                \max z = 5x_1 + 2x_2 + 0x_3 + 0x_4 + 0x_5 - M\cdot(y_1 + y_2 + y_3)\\
                6x_1 + x_2 - x_3 + y_1 = 6\\
                4x_1 + 4x_2 - x_4 + y_2 = 12\\
                x_1 + 2x_2 - x_5 + y_3 = 4\\
                x_i\geq 0, y_j \geq 0,\quad \forall i \in \{1,\ldots,5\}, \forall j \in \{1,2,3\}.
            \end{cases}
        \end{equation*}
        Ensuite on construit notre tableau du simplexe:
        \begin{center}
            \begin{tabular}{|ccc|cccccccc|} % chktex 44
                \hline  % chktex 44
                & \ &\(c\)&\(5\)&\(2\)&\(0\)&\(0\)&\(0\)&\(-M\)&\(-M\)&\(-M\)\\
                \hline % chktex 44
                \multicolumn{1}{|c|}{\(c^J\)}& \multicolumn{2}{c|}{variables de base}&\(x_1\)&\(x_2\)&\(x_3\)&\(x_4\)&\(x_5\)&\(y_1\)&\(y_2\)&\(y_3\)\\
                \hline % chktex 44
                \multicolumn{1}{|c|}{\(-M\)}& \multicolumn{1}{c|}{\(x_1^{1}=y_1\)} &\(6\)&\(6\)&\(1\)&\(-1\)&\(0\)&\(0\)&\(1\)&\(0\)&\(0\)\\
                \hline % chktex 44
                \multicolumn{1}{|c|}{\(-M\)}& \multicolumn{1}{c|}{\(x_2^{1}=y_2\)} &\(12\)&\(4\)&\(4\)&\(0\)&\(-1\)&\(0\)&\(0\)&\(1\)&\(0\)\\
                \hline % chktex 44
                \multicolumn{1}{|c|}{\(-M\)}& \multicolumn{1}{c|}{\(x_3^{1}=y_3\)} &\(4\)&\(1\)&\(2\)&\(0\)&\(0\)&\(-1\)&\(0\)&\(0\)&\(1\)\\
                \hline % chktex 44
                \multicolumn{1}{|c|}{} &\(z(x)\)& \multicolumn{1}{|c|}{\(-22M\)} &\(-11M-5\)&\(-7M-2\)&\(M\)&\(M\)&\(M\)&\(0\)&\(0\)&\(0\)\\
                \hline % chktex 44
            \end{tabular}
        \end{center}
        et on déroule l'algorithme:
        \begin{itemize}
            \item on rentre \(x_1\),
            \item on sort \(y_1\) car \(1 < 3 < 4\),
        \end{itemize}
        \becomes{}
        \begin{center}
            \begin{tabular}{|ccc|cccccccc|} % chktex 44
                \hline  % chktex 44
                & \ &\(c\)&\(5\)&\(2\)&\(0\)&\(0\)&\(0\)&\(-M\)&\(-M\)&\(-M\)\\
                \hline % chktex 44
                \multicolumn{1}{|c|}{\(c^J\)}& \multicolumn{2}{c|}{variables de base}&\(x_1\)&\(x_2\)&\(x_3\)&\(x_4\)&\(x_5\)&\(y_1\)&\(y_2\)&\(y_3\)\\
                \hline % chktex 44
                \multicolumn{1}{|c|}{\(5\)}& \multicolumn{1}{c|}{\(x_1^{2}=x_1\)} &\(\vphantom{\dfrac16}1\)&\(1\)&\(\frac16\)&\(-\frac16\)&\(0\)&\(0\)&\(\frac16\)&\(0\)&\(0\)\\
                \hline % chktex 44
                \multicolumn{1}{|c|}{\(-M\)}& \multicolumn{1}{c|}{\(x_2^{2}=y_2\)} &\(\vphantom{\dfrac16}8\)&\(0\)&\(\frac{10}{3}\)&\(\frac46\)&\(-1\)&\(0\)&\(-\frac46\)&\(1\)&\(0\)\\
                \hline % chktex 44
                \multicolumn{1}{|c|}{\(-M\)}& \multicolumn{1}{c|}{\(x_3^{2}=y_3\)} &\(\vphantom{\dfrac16}3\)&\(0\)&\(\frac{11}{6}\)&\(\frac16\)&\(0\)&\(-1\)&\(-\frac16\)&\(0\)&\(1\)\\
                \hline % chktex 44
                \multicolumn{1}{|c|}{} &\(z(x)\)& \multicolumn{1}{|c|}{\(-11M+5\)} &\(\vphantom{\dfrac16}0\)&\(-\frac{31}6M-\frac7{12}\)&\(-\frac56M-\frac56\)&\(M\)&\(M\)&\(\frac{11}{6}M+\frac56\)&\(0\)&\(0\)\\
                \hline % chktex 44
            \end{tabular}
        \end{center}
        \begin{itemize}
            \item on rentre \(x_2\),
            \item on sort \(y_3\) car \(\frac{11}{2} < 6 < \frac{80}{3}\),
        \end{itemize}
        \becomes{}
        \begin{center}
            \begin{tabular}{|ccc|cccccccc|} % chktex 44
                \hline  % chktex 44
                & \ &\(c\)&\(5\)&\(2\)&\(0\)&\(0\)&\(0\)&\(-M\)&\(-M\)&\(-M\)\\
                \hline % chktex 44
                \multicolumn{1}{|c|}{\(c^J\)}& \multicolumn{2}{c|}{variables de base}&\(x_1\)&\(x_2\)&\(x_3\)&\(x_4\)&\(x_5\)&\(y_1\)&\(y_2\)&\(y_3\)\\
                \hline % chktex 44
                \multicolumn{1}{|c|}{\(5\)}& \multicolumn{1}{c|}{\(x_1^{3}=x_1\)} &\(\vphantom{\dfrac16}\frac{8}{11}\)&\(1\)&\(0\)&\(-\frac{2}{11}\)&\(0\)&\(\frac{1}{11}\)&\(\frac{2}{11}\)&\(0\)&\(-\frac{1}{11}\)\\
                \hline % chktex 44
                \multicolumn{1}{|c|}{\(-M\)}& \multicolumn{1}{c|}{\(x_2^{3}=y_2\)} &\(\vphantom{\dfrac16}\frac{28}{11}\)&\(0\)&\(0\)&\(\frac{4}{11}\)&\(-1\)&\(\frac{20}{11}\)&\(-\frac{4}{11}\)&\(1\)&\(-\frac{20}{11}\)\\
                \hline % chktex 44
                \multicolumn{1}{|c|}{\(2\)}& \multicolumn{1}{c|}{\(x_3^{3}=x_2\)} &\(\vphantom{\dfrac16}\frac{18}{11}\)&\(0\)&\(1\)&\(\frac{1}{11}\)&\(0\)&\(-\frac{6}{11}\)&\(-\frac{1}{11}\)&\(0\)&\(\frac{6}{11}\)\\
                \hline % chktex 44
                \multicolumn{1}{|c|}{} &\(z(x)\)& \multicolumn{1}{|c|}{\(-\frac{28}{11}M + \frac{76}{11}\)} &\(\vphantom{\dfrac16}0\)&\(0\)&\(-\frac{4}{11}M-\frac{8}{11}\)&\(-M\)&\(-\frac{20}{11}M-\frac{7}{11}\)&\(\frac{15}{11}M+\frac{8}{11}\)&\(0\)&\(\frac{31}{11}M+\frac{7}{11}\)\\
                \hline % chktex 44
            \end{tabular}
        \end{center}
        \begin{itemize}
            \item on rentre \(x_5\),
            \item on sort \(y_2\) car \(\frac{7}{5} < 8\) (l'autre rapport
            étant négatif on le compte pas),
        \end{itemize}
        \becomes{}
        \begin{center}
            \begin{tabular}{|ccc|cccccccc|} % chktex 44
                \hline  % chktex 44
                & \ &\(c\)&\(5\)&\(2\)&\(0\)&\(0\)&\(0\)&\(-M\)&\(-M\)&\(-M\)\\
                \hline % chktex 44
                \multicolumn{1}{|c|}{\(c^J\)}& \multicolumn{2}{c|}{variables de base}&\(x_1\)&\(x_2\)&\(x_3\)&\(x_4\)&\(x_5\)&\(y_1\)&\(y_2\)&\(y_3\)\\
                \hline % chktex 44
                \multicolumn{1}{|c|}{\(5\)}& \multicolumn{1}{c|}{\(x_1^{4}=x_1\)} &\(\vphantom{\dfrac16}\frac{3}{5}\)&\(1\)&\(0\)&\(-\frac{1}{5}\)&\(\frac{1}{20}\)&\(0\)&\(\frac{1}{5}\)&\(-\frac{1}{20}\)&\(0\)\\
                \hline % chktex 44
                \multicolumn{1}{|c|}{\(0\)}& \multicolumn{1}{c|}{\(x_2^{4}=x_5\)} &\(\vphantom{\dfrac16}\frac{7}{5}\)&\(0\)&\(0\)&\(\frac{1}{5}\)&\(-\frac{11}{20}\)&\(1\)&\(-\frac{1}{5}\)&\(\frac{11}{20}\)&\(-1\)\\
                \hline % chktex 44
                \multicolumn{1}{|c|}{\(2\)}& \multicolumn{1}{c|}{\(x_3^{4}=x_2\)} &\(\vphantom{\dfrac16}\frac{12}{5}\)&\(0\)&\(1\)&\(\frac{1}{5}\)&\(-\frac{3}{10}\)&\(0\)&\(-\frac{1}{5}\)&\(\frac{3}{10}\)&\(0\)\\
                \hline % chktex 44
                \multicolumn{1}{|c|}{} &\(z(x)\)& \multicolumn{1}{|c|}{\(\frac{39}{5}\)} &\(\vphantom{\dfrac16}0\)&\(0\)&\(-\frac{3}{5}\)&\(-\frac{7}{20}\)&\(0\)&\(M+\frac{3}{5}\)&\(M+\frac{7}{20}\)&\(M\)\\
                \hline % chktex 44
            \end{tabular}
        \end{center}
        A ce stade, il est clair que les variables artificielles ne pourront plus entrer en base.
        On peut donc les retirer et procéder sur le tableau réduit suivant:
        \begin{center}
            \begin{tabular}{|ccc|ccccc|} % chktex 44
                \hline  % chktex 44
                & \ &\(c\)&\(5\)&\(2\)&\(0\)&\(0\)&\(0\)\\
                \hline % chktex 44
                \multicolumn{1}{|c|}{\(c^J\)}& \multicolumn{2}{c|}{variables de base}&\(x_1\)&\(x_2\)&\(x_3\)&\(x_4\)&\(x_5\)\\
                \hline % chktex 44
                \multicolumn{1}{|c|}{\(5\)}& \multicolumn{1}{c|}{\(x_1^{4}=x_1\)} &\(\vphantom{\dfrac16}\frac{3}{5}\)&\(1\)&\(0\)&\(-\frac{1}{5}\)&\(\frac{1}{20}\)&\(0\)\\
                \hline % chktex 44
                \multicolumn{1}{|c|}{\(0\)}& \multicolumn{1}{c|}{\(x_2^{4}=x_5\)} &\(\vphantom{\dfrac16}\frac{7}{5}\)&\(0\)&\(0\)&\(\frac{1}{5}\)&\(-\frac{11}{20}\)&\(1\)\\
                \hline % chktex 44
                \multicolumn{1}{|c|}{\(2\)}& \multicolumn{1}{c|}{\(x_3^{4}=x_2\)} &\(\vphantom{\dfrac16}\frac{12}{5}\)&\(0\)&\(1\)&\(\frac{1}{5}\)&\(-\frac{3}{10}\)&\(0\)\\
                \hline % chktex 44
                \multicolumn{1}{|c|}{} &\(z(x)\)& \multicolumn{1}{|c|}{\(\frac{39}{5}\)} &\(\vphantom{\dfrac16}0\)&\(0\)&\(-\frac{3}{5}\)&\(-\frac{7}{20}\)&\(0\)\\
                \hline % chktex 44
            \end{tabular}
        \end{center}
        \begin{itemize}
            \item on rentre \(x_3\),
            \item on sort \(x_5\),
        \end{itemize}
        \becomes{}
        \begin{center}
            \begin{tabular}{|ccc|ccccc|} % chktex 44
                \hline  % chktex 44
                & \ &\(c\)&\(5\)&\(2\)&\(0\)&\(0\)&\(0\)\\
                \hline % chktex 44
                \multicolumn{1}{|c|}{\(c^J\)}& \multicolumn{2}{c|}{variables de base}&\(x_1\)&\(x_2\)&\(x_3\)&\(x_4\)&\(x_5\)\\
                \hline % chktex 44
                \multicolumn{1}{|c|}{\(5\)}& \multicolumn{1}{c|}{\(x_1^{5}=x_1\)} &\(\vphantom{\dfrac16}2\)&\(1\)&\(0\)&\(0\)&\(-\frac{1}{2}\)&\(1\)\\
                \hline % chktex 44
                \multicolumn{1}{|c|}{\(0\)}& \multicolumn{1}{c|}{\(x_2^{5}=x_3\)} &\(\vphantom{\dfrac16}7\)&\(0\)&\(0\)&\(1\)&\(-\frac{11}{4}\)&\(5\)\\
                \hline % chktex 44
                \multicolumn{1}{|c|}{\(2\)}& \multicolumn{1}{c|}{\(x_3^{5}=x_2\)} &\(\vphantom{\dfrac16}1\)&\(0\)&\(1\)&\(0\)&\(\frac{1}{4}\)&\(-1\)\\
                \hline % chktex 44
                \multicolumn{1}{|c|}{} &\(z(x)\)& \multicolumn{1}{|c|}{\(12\)} &\(\vphantom{\dfrac16}0\)&\(0\)&\(0\)&\(-2\)&\(3\)\\
                \hline % chktex 44
            \end{tabular}
        \end{center}
        Il ne reste probablement qu'une étape, faisons-la:
        \begin{itemize}
            \item on rentre \(x_4\),
            \item on sort \(x_2\), c'est la seule valeur positive,
        \end{itemize}
        \becomes{}
        \begin{center}
            \begin{tabular}{|ccc|ccccc|} % chktex 44
                \hline  % chktex 44
                & \ &\(c\)&\(5\)&\(2\)&\(0\)&\(0\)&\(0\)\\
                \hline % chktex 44
                \multicolumn{1}{|c|}{\(c^J\)}& \multicolumn{2}{c|}{variables de base}&\(x_1\)&\(x_2\)&\(x_3\)&\(x_4\)&\(x_5\)\\
                \hline % chktex 44
                \multicolumn{1}{|c|}{\(5\)}& \multicolumn{1}{c|}{\(x_1^{6}=x_1\)} &\(\vphantom{\dfrac16}4\)&\(1\)&\(2\)&\(0\)&\(0\)&\(-1\)\\
                \hline % chktex 44
                \multicolumn{1}{|c|}{\(0\)}& \multicolumn{1}{c|}{\(x_2^{6}=x_3\)} &\(\vphantom{\dfrac16}18\)&\(0\)&\(11\)&\(1\)&\(0\)&\(-6\)\\
                \hline % chktex 44
                \multicolumn{1}{|c|}{\(0\)}& \multicolumn{1}{c|}{\(x_3^{6}=x_4\)} &\(\vphantom{\dfrac16}4\)&\(0\)&\(4\)&\(0\)&\(1\)&\(-4\)\\
                \hline % chktex 44
                \multicolumn{1}{|c|}{} &\(z(x)\)& \multicolumn{1}{|c|}{\(20\)} &\(\vphantom{\dfrac16}0\)&\(8\)&\(0\)&\(0\)&\(-5\)\\
                \hline % chktex 44
            \end{tabular}
        \end{center}
        Dommage, on a encore un point pivot à faire rentrer (\(x_5\) a 
        valeur négative) donc on aimerait continuer de dérouler l'algorithme
        mais on a pas de critère pour trouver quelle variable sortir de base
        (en effet, toutes les valeurs dans la colonne du pivot sont négatives).
        
        On conclut donc que l'espace des solutions est non borné.
        C'est-à-dire qu'il n'existe pas \og{}un couple maximum\fg{}
        car on peut toujours en trouver un plus grand. Donc
        il n'y a pas un unique couple \((x_1, x_2)\) qui maximise \(z\).

    \end{enumerate}
\end{td-sol}

% ----- Consignes exo 2 ----- %
\iftoggle{showquestions}{
    \begin{td-exo}[Dualité]
        Considérez le programme linéaire le plus général envisageable donné ci-dessous:
        \begin{equation*}
            \begin{cases}
                \min z = c_1 x_1 + c_2 x_2 \\
                A_{11} x_1 + A_{12} x_2 \leq b_1 \\
                A_{21} x_1 + A_{22} x_2 = b_2 \\
                x_i \geq 0,\quad \forall i \in \{1,2\}
            \end{cases}
        \end{equation*}
        où \(A\) est une matrice \((m_1 + m_2) \times (n_1 + n_2)\) et \(c, x\in \bb R^{n_1 + n_2}\) et \(b \in \bb R^{m_1 + m_2}\).

        Caractériser le dual.
    \end{td-exo}
}{}

% ----- Solutions exo 2 ----- %
\begin{td-sol}[]\ %
    % TODO: completer solution exercice 2
\end{td-sol}

% ----- Consignes exo 3 ----- %
\iftoggle{showquestions}{
    \begin{td-exo}[Ensemble convexe]\,
        Soit \(C_1\) et \(C_2\) deux convexes de \(\bb R^{m+n}\). Montrer que l'ensemble 
        \begin{equation*}
            C = \left\{\left(x, y_1 + y_2\right) \mid x\in\bb R^m, y_1\in \bb R^n, y_2\in\bb R^n, (x,y_1)\in C_1, (x, y_2)\in C_2\right\}
        \end{equation*} est également convexe.
    \end{td-exo}
}{}

% ----- Solutions exo 3 ----- %
\begin{td-sol}[]\ %
    % TODO: completer solution exercice 3
\end{td-sol}

% ----- Consignes exo 4 ----- %
\iftoggle{showquestions}{
    \begin{td-exo}[Modélisation et dualité]
    Considérons un problème d'affectation avec \(m\) jobs et \(n\) travailleurs (\(n\geq m\)). Chaque job doit être affecté à exactement un travailleur.
    Soit \(p_{ij}\) le rendement obtenu si on affecte le job \(i\) au travailleur \(j\), où \(i\in\{1,\ldots,m\}\) et \(j\in\{1,\ldots,n\}\). On cherche une affectation qui maximise le rendement total.
    \begin{enumerate}
        \item Donner le programme linéaire.
        \item Donner la formulation du dual de ce problème.
    \end{enumerate}
    \end{td-exo}
}{}

% ----- Solutions exo 4 ----- %
\begin{td-sol}[]\ %
    % TODO: completer solution exercice 4
\end{td-sol}

% ----- Consignes exo 5 ----- %
\iftoggle{showquestions}{
    \begin{td-exo}[Programmation linéaire: Farkas]\,
        Considérons le programme linéaire suivant, qui dépend de \(\varepsilon\in\bb R\):
        \begin{equation*}
            \begin{cases}
                \min z = 4x_1 - 2x_2\\
                x_2\leq 3\\
                \varepsilon x_1 + (2-\varepsilon) x_2 \leq 4\\
                x_i \geq 0,\quad \forall i \in \{1,2\}
            \end{cases}
        \end{equation*}\,
        \begin{enumerate}
            \item Montrer que le problème est réalisable \(\forall \varepsilon\in\bb R\).
            \item Pour quelles valeurs de \(\varepsilon\) la valeur optimale est-elle non bornée?
        \end{enumerate}
    \end{td-exo}
}{}

% ----- Solutions exo 5 ----- %
\begin{td-sol}[]\ %
    % TODO: completer solution exercice 5
\end{td-sol}

% ----- Consignes exo 6 ----- %
\iftoggle{showquestions}{
    \begin{td-exo}[Résolution numérique]\,
        Résoudre le programme linéaire suivant par la méthode Primal-Dual:
        \begin{equation*}
            \text{Primal} = 
            \begin{cases}
                \min z(x_1, x_2, x_3) = 2x_1 + x_2 + 2x_3 + 8x_4\\
                2x_1 - x_2 + 3x_3 - 2x_4 = 3\\
                -x_1 + 3x_2 - 4x_3 = 1\\
                x_i \geq 0,\quad \forall i \in \{1,2,3,4\}
            \end{cases}
        \end{equation*}
    \end{td-exo}
}{}

% ----- Solutions exo 6 ----- %
\begin{td-sol}[]\ %
    % TODO: completer solution exercice 6
\end{td-sol}

\section{Partie pratique}

% ----- Consignes exo xx ----- %
\iftoggle{showquestions}{
    \begin{td-exo}[]

    \end{td-exo}
}{}

% ----- Solutions exo xx ----- %
\begin{td-sol}[]\ %
    % TODO: completer solution exercice xx
\end{td-sol}

% ----- Consignes exo xx ----- %
\iftoggle{showquestions}{
    \begin{td-exo}[]

    \end{td-exo}
}{}

% ----- Solutions exo xx ----- %
\begin{td-sol}[]\ %
    % TODO: completer solution exercice xx
\end{td-sol}

% ----- Consignes exo xx ----- %
\iftoggle{showquestions}{
    \begin{td-exo}[]

    \end{td-exo}
}{}

% ----- Solutions exo xx ----- %
\begin{td-sol}[]\ %
    % TODO: completer solution exercice xx
\end{td-sol}
