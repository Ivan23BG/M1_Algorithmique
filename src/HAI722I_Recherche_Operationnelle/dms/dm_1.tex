\iftoggle{showquestions}{
\section*{Instructions}

Ce devoir est à rendre avant le 12 décembre 2025 à 12h, soit par mail à l'adresse:
rodolphe.giroudeau@{}lirmm.fr,
soit en déposant votre devoir durant le cours
}{}

\section{Partie théorique}
% ----- Consignes exo 1 ----- %
\iftoggle{showquestions}{
    \begin{td-exo}[Algorithmes pour la programmation linéaire]\,\\ % 1
        Considérons la formulation suivante:
        \begin{equation*}
            P_\beta =
            \begin{cases}
                \max z = 5x_1 + 2x_2\\
                6x_1 + x_2 \geq 6\\
                4x_1 + 4x_2 \geq 12\\
                x_1 + 2x_2 \geq 4\\
                x_i \geq 0,\quad \forall i \in \{1,2\}
            \end{cases}
        \end{equation*}\,
        \begin{enumerate}
            \item Résoudre le problème \(P_\beta\) par la méthode du big \(M\).
            \item Résoudre le problème \(P_\beta\) par la méthode à deux phases.
            \item \textbf{Difficile:}
            \begin{enumerate}
                \item Résoudre le problème \(P_\beta\) par la méthode dual-simplexe.
                \item Soit le programme linéaire \(P_\theta\)
                \begin{equation*}
                    P_\theta =
                    \begin{cases}
                        \max z = x_1 + 3x_2\\
                        x_1 + x_2 \geq 3\\
                        x_1 - 2x_2 \geq 5\\
                        -2x_1 + x_2 \leq 5\\
                        x_i \geq 0,\quad \forall i \in \{1,2\}
                    \end{cases}
                \end{equation*}
                Résoudre le problème \(P_\theta\) par la méthode dual-simplexe.
            \end{enumerate}
        \end{enumerate}
    \end{td-exo}
}{}

% ----- Solutions exo 1 ----- %
\begin{td-sol}[]\ %
    \begin{enumerate}
        \item Commençons par poser le problème sous forme standard:
        \begin{equation*}
            P_\beta =
            \begin{cases}
                \max z = 5x_1 + 2x_2 + 0x_3 + 0x_4 + 0x_5 - M\cdot(y_1 + y_2 + y_3)\\
                6x_1 + x_2 - x_3 + y_1 = 6\\
                4x_1 + 4x_2 - x_4 + y_2 = 12\\
                x_1 + 2x_2 - x_5 + y_3 = 4\\
                x_i\geq 0, y_j \geq 0,\quad \forall i \in \{1,\ldots,5\}, \forall j \in \{1,2,3\}.
            \end{cases}
        \end{equation*}
        Ensuite on construit notre tableau du simplexe:
        \begin{center}
            \begin{tabular}{|ccc|cccccccc|} % chktex 44
                \hline  % chktex 44
                & \ &\(c\)&\(5\)&\(2\)&\(0\)&\(0\)&\(0\)&\(-M\)&\(-M\)&\(-M\)\\
                \hline % chktex 44
                \multicolumn{1}{|c|}{\(c^J\)}& \multicolumn{2}{c|}{variables de base}&\(x_1\)&\(x_2\)&\(x_3\)&\(x_4\)&\(x_5\)&\(y_1\)&\(y_2\)&\(y_3\)\\
                \hline % chktex 44
                \multicolumn{1}{|c|}{\(-M\)}& \multicolumn{1}{c|}{\(x_1^{1}=y_1\)} &\(6\)&\(6\)&\(1\)&\(-1\)&\(0\)&\(0\)&\(1\)&\(0\)&\(0\)\\
                \hline % chktex 44
                \multicolumn{1}{|c|}{\(-M\)}& \multicolumn{1}{c|}{\(x_2^{1}=y_2\)} &\(12\)&\(4\)&\(4\)&\(0\)&\(-1\)&\(0\)&\(0\)&\(1\)&\(0\)\\
                \hline % chktex 44
                \multicolumn{1}{|c|}{\(-M\)}& \multicolumn{1}{c|}{\(x_3^{1}=y_3\)} &\(4\)&\(1\)&\(2\)&\(0\)&\(0\)&\(-1\)&\(0\)&\(0\)&\(1\)\\
                \hline % chktex 44
                \multicolumn{1}{|c|}{} &\(z(x)\)& \multicolumn{1}{|c|}{\(-22M\)} &\(-11M-5\)&\(-7M-2\)&\(M\)&\(M\)&\(M\)&\(0\)&\(0\)&\(0\)\\
                \hline % chktex 44
            \end{tabular}
        \end{center}
        et on déroule l'algorithme:
        \begin{itemize}
            \item on rentre \(x_1\),
            \item on sort \(y_1\) car \(1 < 3 < 4\),
        \end{itemize}
        \becomes{}
        \begin{center}
            \begin{tabular}{|ccc|cccccccc|} % chktex 44
                \hline  % chktex 44
                & \ &\(c\)&\(5\)&\(2\)&\(0\)&\(0\)&\(0\)&\(-M\)&\(-M\)&\(-M\)\\
                \hline % chktex 44
                \multicolumn{1}{|c|}{\(c^J\)}& \multicolumn{2}{c|}{variables de base}&\(x_1\)&\(x_2\)&\(x_3\)&\(x_4\)&\(x_5\)&\(y_1\)&\(y_2\)&\(y_3\)\\
                \hline % chktex 44
                \multicolumn{1}{|c|}{\(5\)}& \multicolumn{1}{c|}{\(x_1^{2}=x_1\)} &\(\vphantom{\dfrac16}1\)&\(1\)&\(\frac16\)&\(-\frac16\)&\(0\)&\(0\)&\(\frac16\)&\(0\)&\(0\)\\
                \hline % chktex 44
                \multicolumn{1}{|c|}{\(-M\)}& \multicolumn{1}{c|}{\(x_2^{2}=y_2\)} &\(\vphantom{\dfrac16}8\)&\(0\)&\(\frac{10}{3}\)&\(\frac46\)&\(-1\)&\(0\)&\(-\frac46\)&\(1\)&\(0\)\\
                \hline % chktex 44
                \multicolumn{1}{|c|}{\(-M\)}& \multicolumn{1}{c|}{\(x_3^{2}=y_3\)} &\(\vphantom{\dfrac16}3\)&\(0\)&\(\frac{11}{6}\)&\(\frac16\)&\(0\)&\(-1\)&\(-\frac16\)&\(0\)&\(1\)\\
                \hline % chktex 44
                \multicolumn{1}{|c|}{} &\(z(x)\)& \multicolumn{1}{|c|}{\(-11M+5\)} &\(\vphantom{\dfrac16}0\)&\(-\frac{31}6M-\frac7{12}\)&\(-\frac56M-\frac56\)&\(M\)&\(M\)&\(\frac{11}{6}M+\frac56\)&\(0\)&\(0\)\\
                \hline % chktex 44
            \end{tabular}
        \end{center}
        \begin{itemize}
            \item on rentre \(x_2\),
            \item on sort \(y_3\) car \(\frac{11}{2} < 6 < \frac{80}{3}\),
        \end{itemize}
        \becomes{}
        \begin{center}
            \begin{tabular}{|ccc|cccccccc|} % chktex 44
                \hline  % chktex 44
                & \ &\(c\)&\(5\)&\(2\)&\(0\)&\(0\)&\(0\)&\(-M\)&\(-M\)&\(-M\)\\
                \hline % chktex 44
                \multicolumn{1}{|c|}{\(c^J\)}& \multicolumn{2}{c|}{variables de base}&\(x_1\)&\(x_2\)&\(x_3\)&\(x_4\)&\(x_5\)&\(y_1\)&\(y_2\)&\(y_3\)\\
                \hline % chktex 44
                \multicolumn{1}{|c|}{\(5\)}& \multicolumn{1}{c|}{\(x_1^{3}=x_1\)} &\(\vphantom{\dfrac16}\frac{8}{11}\)&\(1\)&\(0\)&\(-\frac{2}{11}\)&\(0\)&\(\frac{1}{11}\)&\(\frac{2}{11}\)&\(0\)&\(-\frac{1}{11}\)\\
                \hline % chktex 44
                \multicolumn{1}{|c|}{\(-M\)}& \multicolumn{1}{c|}{\(x_2^{3}=y_2\)} &\(\vphantom{\dfrac16}\frac{28}{11}\)&\(0\)&\(0\)&\(\frac{4}{11}\)&\(-1\)&\(\frac{20}{11}\)&\(-\frac{4}{11}\)&\(1\)&\(-\frac{20}{11}\)\\
                \hline % chktex 44
                \multicolumn{1}{|c|}{\(2\)}& \multicolumn{1}{c|}{\(x_3^{3}=x_2\)} &\(\vphantom{\dfrac16}\frac{18}{11}\)&\(0\)&\(1\)&\(\frac{1}{11}\)&\(0\)&\(-\frac{6}{11}\)&\(-\frac{1}{11}\)&\(0\)&\(\frac{6}{11}\)\\
                \hline % chktex 44
                \multicolumn{1}{|c|}{} &\(z(x)\)& \multicolumn{1}{|c|}{\(-\frac{28}{11}M + \frac{76}{11}\)} &\(\vphantom{\dfrac16}0\)&\(0\)&\(-\frac{4}{11}M-\frac{8}{11}\)&\(-M\)&\(-\frac{20}{11}M-\frac{7}{11}\)&\(\frac{15}{11}M+\frac{8}{11}\)&\(0\)&\(\frac{31}{11}M+\frac{7}{11}\)\\
                \hline % chktex 44
            \end{tabular}
        \end{center}
        \begin{itemize}
            \item on rentre \(x_5\),
            \item on sort \(y_2\) car \(\frac{7}{5} < 8\) (l'autre rapport
            étant négatif on le compte pas),
        \end{itemize}
        \becomes{}
        \begin{center}
            \begin{tabular}{|ccc|cccccccc|} % chktex 44
                \hline  % chktex 44
                & \ &\(c\)&\(5\)&\(2\)&\(0\)&\(0\)&\(0\)&\(-M\)&\(-M\)&\(-M\)\\
                \hline % chktex 44
                \multicolumn{1}{|c|}{\(c^J\)}& \multicolumn{2}{c|}{variables de base}&\(x_1\)&\(x_2\)&\(x_3\)&\(x_4\)&\(x_5\)&\(y_1\)&\(y_2\)&\(y_3\)\\
                \hline % chktex 44
                \multicolumn{1}{|c|}{\(5\)}& \multicolumn{1}{c|}{\(x_1^{4}=x_1\)} &\(\vphantom{\dfrac16}\frac{3}{5}\)&\(1\)&\(0\)&\(-\frac{1}{5}\)&\(\frac{1}{20}\)&\(0\)&\(\frac{1}{5}\)&\(-\frac{1}{20}\)&\(0\)\\
                \hline % chktex 44
                \multicolumn{1}{|c|}{\(0\)}& \multicolumn{1}{c|}{\(x_2^{4}=x_5\)} &\(\vphantom{\dfrac16}\frac{7}{5}\)&\(0\)&\(0\)&\(\frac{1}{5}\)&\(-\frac{11}{20}\)&\(1\)&\(-\frac{1}{5}\)&\(\frac{11}{20}\)&\(-1\)\\
                \hline % chktex 44
                \multicolumn{1}{|c|}{\(2\)}& \multicolumn{1}{c|}{\(x_3^{4}=x_2\)} &\(\vphantom{\dfrac16}\frac{12}{5}\)&\(0\)&\(1\)&\(\frac{1}{5}\)&\(-\frac{3}{10}\)&\(0\)&\(-\frac{1}{5}\)&\(\frac{3}{10}\)&\(0\)\\
                \hline % chktex 44
                \multicolumn{1}{|c|}{} &\(z(x)\)& \multicolumn{1}{|c|}{\(\frac{39}{5}\)} &\(\vphantom{\dfrac16}0\)&\(0\)&\(-\frac{3}{5}\)&\(-\frac{7}{20}\)&\(0\)&\(M+\frac{3}{5}\)&\(M+\frac{7}{20}\)&\(M\)\\
                \hline % chktex 44
            \end{tabular}
        \end{center}
        A ce stade, il est clair que les variables artificielles ne pourront plus entrer en base.
        On peut donc les retirer et procéder sur le tableau réduit suivant:
        \begin{center}
            \begin{tabular}{|ccc|ccccc|} % chktex 44
                \hline  % chktex 44
                & \ &\(c\)&\(5\)&\(2\)&\(0\)&\(0\)&\(0\)\\
                \hline % chktex 44
                \multicolumn{1}{|c|}{\(c^J\)}& \multicolumn{2}{c|}{variables de base}&\(x_1\)&\(x_2\)&\(x_3\)&\(x_4\)&\(x_5\)\\
                \hline % chktex 44
                \multicolumn{1}{|c|}{\(5\)}& \multicolumn{1}{c|}{\(x_1^{4}=x_1\)} &\(\vphantom{\dfrac16}\frac{3}{5}\)&\(1\)&\(0\)&\(-\frac{1}{5}\)&\(\frac{1}{20}\)&\(0\)\\
                \hline % chktex 44
                \multicolumn{1}{|c|}{\(0\)}& \multicolumn{1}{c|}{\(x_2^{4}=x_5\)} &\(\vphantom{\dfrac16}\frac{7}{5}\)&\(0\)&\(0\)&\(\frac{1}{5}\)&\(-\frac{11}{20}\)&\(1\)\\
                \hline % chktex 44
                \multicolumn{1}{|c|}{\(2\)}& \multicolumn{1}{c|}{\(x_3^{4}=x_2\)} &\(\vphantom{\dfrac16}\frac{12}{5}\)&\(0\)&\(1\)&\(\frac{1}{5}\)&\(-\frac{3}{10}\)&\(0\)\\
                \hline % chktex 44
                \multicolumn{1}{|c|}{} &\(z(x)\)& \multicolumn{1}{|c|}{\(\frac{39}{5}\)} &\(\vphantom{\dfrac16}0\)&\(0\)&\(-\frac{3}{5}\)&\(-\frac{7}{20}\)&\(0\)\\
                \hline % chktex 44
            \end{tabular}
        \end{center}
        \begin{itemize}
            \item on rentre \(x_3\),
            \item on sort \(x_5\),
        \end{itemize}
        \becomes{}
        \begin{center}
            \begin{tabular}{|ccc|ccccc|} % chktex 44
                \hline  % chktex 44
                & \ &\(c\)&\(5\)&\(2\)&\(0\)&\(0\)&\(0\)\\
                \hline % chktex 44
                \multicolumn{1}{|c|}{\(c^J\)}& \multicolumn{2}{c|}{variables de base}&\(x_1\)&\(x_2\)&\(x_3\)&\(x_4\)&\(x_5\)\\
                \hline % chktex 44
                \multicolumn{1}{|c|}{\(5\)}& \multicolumn{1}{c|}{\(x_1^{5}=x_1\)} &\(\vphantom{\dfrac16}2\)&\(1\)&\(0\)&\(0\)&\(-\frac{1}{2}\)&\(1\)\\
                \hline % chktex 44
                \multicolumn{1}{|c|}{\(0\)}& \multicolumn{1}{c|}{\(x_2^{5}=x_3\)} &\(\vphantom{\dfrac16}7\)&\(0\)&\(0\)&\(1\)&\(-\frac{11}{4}\)&\(5\)\\
                \hline % chktex 44
                \multicolumn{1}{|c|}{\(2\)}& \multicolumn{1}{c|}{\(x_3^{5}=x_2\)} &\(\vphantom{\dfrac16}1\)&\(0\)&\(1\)&\(0\)&\(\frac{1}{4}\)&\(-1\)\\
                \hline % chktex 44
                \multicolumn{1}{|c|}{} &\(z(x)\)& \multicolumn{1}{|c|}{\(12\)} &\(\vphantom{\dfrac16}0\)&\(0\)&\(0\)&\(-2\)&\(3\)\\
                \hline % chktex 44
            \end{tabular}
        \end{center}
        Il ne reste probablement qu'une étape, faisons-la:
        \begin{itemize}
            \item on rentre \(x_4\),
            \item on sort \(x_2\), c'est la seule valeur positive,
        \end{itemize}
        \becomes{}
        \begin{center}
            \begin{tabular}{|ccc|ccccc|} % chktex 44
                \hline  % chktex 44
                & \ &\(c\)&\(5\)&\(2\)&\(0\)&\(0\)&\(0\)\\
                \hline % chktex 44
                \multicolumn{1}{|c|}{\(c^J\)}& \multicolumn{2}{c|}{variables de base}&\(x_1\)&\(x_2\)&\(x_3\)&\(x_4\)&\(x_5\)\\
                \hline % chktex 44
                \multicolumn{1}{|c|}{\(5\)}& \multicolumn{1}{c|}{\(x_1^{6}=x_1\)} &\(\vphantom{\dfrac16}4\)&\(1\)&\(2\)&\(0\)&\(0\)&\(-1\)\\
                \hline % chktex 44
                \multicolumn{1}{|c|}{\(0\)}& \multicolumn{1}{c|}{\(x_2^{6}=x_3\)} &\(\vphantom{\dfrac16}18\)&\(0\)&\(11\)&\(1\)&\(0\)&\(-6\)\\
                \hline % chktex 44
                \multicolumn{1}{|c|}{\(0\)}& \multicolumn{1}{c|}{\(x_3^{6}=x_4\)} &\(\vphantom{\dfrac16}4\)&\(0\)&\(4\)&\(0\)&\(1\)&\(-4\)\\
                \hline % chktex 44
                \multicolumn{1}{|c|}{} &\(z(x)\)& \multicolumn{1}{|c|}{\(20\)} &\(\vphantom{\dfrac16}0\)&\(8\)&\(0\)&\(0\)&\(-5\)\\
                \hline % chktex 44
            \end{tabular}
        \end{center}
        Dommage, on a encore un point pivot à faire rentrer (\(x_5\) a 
        valeur négative) donc on aimerait continuer de dérouler l'algorithme
        mais on a pas de critère pour trouver quelle variable sortir de base
        (en effet, toutes les valeurs dans la colonne du pivot sont négatives).
        
        On conclut donc que l'espace des solutions est non borné.
        C'est-à-dire qu'il n'existe pas \og{}un couple maximum\fg{}
        car on peut toujours en trouver un plus grand. Donc
        il n'y a pas un unique couple \((x_1, x_2)\) qui maximise \(z\).

        \item Résolvons maintenant le problème \(P_\theta\) par la méthode à deux phases.

    \end{enumerate}
\end{td-sol}

% ----- Consignes exo 2 ----- %
\iftoggle{showquestions}{
    \begin{td-exo}[Dualité]\,\\
        Considérez le programme linéaire le plus général envisageable donné ci-dessous:
        \begin{equation*}
            \begin{cases}
                \min z = c_1 x_1 + c_2 x_2 \\
                A_{11} x_1 + A_{12} x_2 \leq b_1 \\
                A_{21} x_1 + A_{22} x_2 = b_2 \\
                x_i \geq 0,\quad \forall i \in \{1,2\}
            \end{cases}
        \end{equation*}
        où \(A\) est une matrice \((m_1 + m_2) \times (n_1 + n_2)\) et \(c, x\in \bb R^{n_1 + n_2}\) et \(b \in \bb R^{m_1 + m_2}\).

        Caractériser le dual.
    \end{td-exo}
}{}

% ----- Solutions exo 2 ----- %
\begin{td-sol}[]\ %
    % TODO: completer solution exercice 2
\end{td-sol}

% ----- Consignes exo 3 ----- %
\iftoggle{showquestions}{
    \begin{td-exo}[Ensemble convexe]\,\\
        Soit \(C_1\) et \(C_2\) deux convexes de \(\bb R^{m+n}\). Montrer que l'ensemble 
        \begin{equation*}
            C = \left\{\left(x, y_1 + y_2\right) \mid x\in\bb R^m, y_1\in \bb R^n, y_2\in\bb R^n, (x,y_1)\in C_1, (x, y_2)\in C_2\right\}
        \end{equation*} est également convexe.
    \end{td-exo}
}{}

% ----- Solutions exo 3 ----- %
\begin{td-sol}[]\ %
    Soient \((x^1, y^1)\) et \((x^2, y^2)\) deux points de \(C\). Par définition de \(C\), il existe \(y_1^1, y_2^1, y_1^2, y_2^2\) tels que
    \begin{equation*}
        (x^1, y_1^1) \in C_1, \quad (x^1, y_2^1) \in C_2, \quad (x^2, y_1^2) \in C_1, \quad (x^2, y_2^2) \in C_2,
    \end{equation*}
    et
    \begin{equation*}
        y^1 = y_1^1 + y_2^1, \quad y^2 = y_1^2 + y_2^2.
    \end{equation*}
    Soit \(\lambda \in [0,1]\). Considérons le point
    \begin{equation*}
        (x^\lambda, y^\lambda) = \lambda (x^1, y^1) + (1-\lambda)(x^2, y^2).
    \end{equation*}
    On a
    \begin{equation*}
        \begin{aligned}
            x^\lambda 
            &= \lambda x^1 + (1-\lambda)x^2, \\
            y^\lambda 
            &= \lambda y^1 + (1-\lambda)y^2 \\
            &= \lambda (y_1^1 + y_2^1) + (1-\lambda)(y_1^2 + y_2^2) \\
            &= (\lambda y_1^1 + (1-\lambda)y_1^2) + (\lambda y_2^1 + (1-\lambda)y_2^2).
        \end{aligned}
    \end{equation*}
    Notons \(y_1^\lambda = \lambda y_1^1 + (1-\lambda)y_1^2\) et \(y_2^\lambda = \lambda y_2^1 + (1-\lambda)y_2^2\). Par convexité de \(C_1\) et \(C_2\), on a
    \begin{equation*}
        (x^\lambda, y_1^\lambda) \in C_1, \quad (x^\lambda, y_2^\lambda) \in C_2.
    \end{equation*}
    Ainsi, par définition de \(C\), on a \((x^\lambda, y^\lambda) \in C\). Donc \(C\) est convexe.
\end{td-sol}

% ----- Consignes exo 4 ----- %
\iftoggle{showquestions}{
    \begin{td-exo}[Modélisation et dualité]\,\\
    Considérons un problème d'affectation avec \(m\) jobs et \(n\) travailleurs (\(n\geq m\)). Chaque job doit être affecté à exactement un travailleur.
    Soit \(p_{ij}\) le rendement obtenu si on affecte le job \(i\) au travailleur \(j\), où \(i\in\{1,\ldots,m\}\) et \(j\in\{1,\ldots,n\}\). On cherche une affectation qui maximise le rendement total.
    \begin{enumerate}
        \item Donner le programme linéaire.
        \item Donner la formulation du dual de ce problème.
    \end{enumerate}
    \end{td-exo}
}{}

% ----- Solutions exo 4 ----- %
\begin{td-sol}[]\ %
    \begin{enumerate}
        \item On introduit les variables \(x_{ij}\) qui valent 1 si le job \(i\) est affecté au travailleur \(j\), et 0 sinon. Le programme linéaire s'écrit alors:

        \begin{numcases}{P_0 = }
            \max z = \sum_{i=1}^m \sum_{j=1}^n p_{ij} x_{ij}\label{C1}\\ % chktex 1
            \sum_{j=1}^n x_{ij} = 1, \quad \forall i \in \{1,\ldots,m\}\label{C2}\\ % chktex 1
            \sum_{i=1}^m x_{ij} \leq 1, \quad \forall j \in \{1,\ldots,n\}\label{C3}\\ % chktex 1
            x_{ij} \in \{0,1\}, \quad \forall i,j\label{C4} % chktex 1
        \end{numcases}

        où les contraintes correspondent à:

        \begin{itemize}
            \item[\eqref{C1}] on maximise le rendement total
            \item[\eqref{C2}] chaque job est affecté à un travailleur
            \item[\eqref{C3}] chaque travailleur a au plus un job
            \item[\eqref{C4}] les variables sont \(0\) ou \(1\)
        \end{itemize}

        \item Pour obtenir la formulation du dual, on introduit les variables duales \(u_i\) (une par job) et \(v_j\) (une par travailleur). Le dual s'écrit alors:

        \begin{numcases}{D_0 = }
            \min w = \sum_{i=1}^m u_i + \sum_{j=1}^n v_j\label{D1}\\ % chktex 1
            u_i + v_j \geq p_{ij}, \quad \forall i,j\label{D2}\\ % chktex 1
            v_j \geq 0, \quad \forall j\label{D3} % chktex 1
        \end{numcases}

        où les contraintes correspondent à:

        \begin{itemize}
            \item[\eqref{D1}] on minimise la somme des coûts
            \item[\eqref{D2}] les coûts doivent couvrir les rendements
            \item[\eqref{D3}] les variables sont positives
        \end{itemize}

    \end{enumerate}
\end{td-sol}


% ----- Consignes exo 5 ----- %
\iftoggle{showquestions}{
    \begin{td-exo}[Programmation linéaire: Farkas]\,\\
        Considérons le programme linéaire suivant, qui dépend de \(\varepsilon\in\bb R\):
        \begin{equation*}
            \begin{cases}
                \min z = 4x_1 - 2x_2\\
                x_2\leq 3\\
                \varepsilon x_1 + (2-\varepsilon) x_2 \leq 4\\
                x_i \geq 0,\quad \forall i \in \{1,2\}
            \end{cases}
        \end{equation*}\,
        \begin{enumerate}
            \item Montrer que le problème est réalisable \(\forall \varepsilon\in\bb R\).
            \item Pour quelles valeurs de \(\varepsilon\) la valeur optimale est-elle non bornée?
        \end{enumerate}
    \end{td-exo}
}{}

% ----- Solutions exo 5 ----- %
\begin{td-sol}[]\ %
    % TODO: completer solution exercice 5
\end{td-sol}

% ----- Consignes exo 6 ----- %
\iftoggle{showquestions}{
    \begin{td-exo}[Résolution numérique]\,\\
        Résoudre le programme linéaire suivant par la méthode Primal-Dual:
        \begin{equation*}
            \text{Primal} = 
            \begin{cases}
                \min z(x_1, x_2, x_3) = 2x_1 + x_2 + 2x_3 + 8x_4\\
                2x_1 - x_2 + 3x_3 - 2x_4 = 3\\
                -x_1 + 3x_2 - 4x_3 = 1\\
                x_i \geq 0,\quad \forall i \in \{1,2,3,4\}
            \end{cases}
        \end{equation*}
    \end{td-exo}
}{}

% ----- Solutions exo 6 ----- %
\begin{td-sol}[]\ %
    % TODO: completer solution exercice 6
\end{td-sol}

\section{Partie pratique}

% ----- Consignes exo 7 ----- %
\iftoggle{showquestions}{
    \begin{td-exo}[Résolution d'un programme linéaire avec Julia]\,
        \begin{enumerate}
            \item Dans un premier temps nous allons regarder quelques commandes de base en Julia.
            \begin{enumerate}
                \item Pour lancer Julia:
                \bashFile{./julia_code/julia_launch.sh}

                \item Introduction à la syntaxe. La syntaxe est assez simple:
                \juliaFile{./julia_code/julia_syntax.jl}

                \item les fonctions sont bien définies:
                \juliaFile{./julia_code/julia_functions.jl}

                Renseignements obligatoires sur la syntaxe de Julia peuvent être trouvés sur les pages suivantes:
                \begin{itemize}
                    \item \url{https://julialang.org/}
                    \item \url{https://zestedesavoir.com/articles/78/a-la-decouverte-de-julia/}
                \end{itemize}

                \item Le package \texttt{JuMP} et le solveur \texttt{Cbe JuMP} sont des langages de modélisation pour l'optimisation mathématique intégré dans Julia.
                Pour l'utiliser il faut d'abord ajouter le package associé:
                \juliaFile{./julia_code/julia_jump_install.sh}

                Julia est un langage de modélisation et ne possède pas un solveur de programmation mathématique intégré. \texttt{CPLEX} est un solveur de programmation mathématique de haute performance, puissant pour la programmation linéaire, la programmation en nombres entiers et la programmation quadratique.
                Mais le solveur \texttt{CPLEX} n'est pas gratuit et requiert une license (facile à obtenir pour les étudiants mais le système est bridé).
                Nou sutiliserons donc plutôt \texttt{Cbc} qui est un solveur de programmation mathématique open-source moins puissant intégré dans Julia.
                Dans la suite, nous présentons les commandes pour utiliser \texttt{Cbc}.
                Pour utiliser un solveur à partir de Julia il faut ajouter le package associé.

                \item Création du modèle:

                Utiliser les commandes présentées dans cette section pour définir dans Julia un modèle associé au programme linéaire \(P_1\).
                Les variables \(x_1\) et \(x_2\) représentent respectivement la quantité de deux types différents de yaourt produits.
                La fonction objectif maximise le profit de la production:
                \juliaFile{./julia_code/julia_jump_model_P1.jl}

                \item Packages:
                \juliaFile{./julia_code/julia_jump_packages.jl}

                \item Définition du modèle:
                \juliaFile{./julia_code/julia_jump_define_model_P1.jl}

                \item Définition des variables, exemples de définition de variables:
                \juliaFile{./julia_code/julia_jump_variables_P1.jl}
                Ajouter à votre modèle les contraintes de \(P_1\).

                \item Pour résoudre le modèle, il faut exécuter:
                \juliaFile{./julia_code/julia_jump_solve_model_P1.jl}

                La valeur de la solution optimale du modèle, le temps d'exécution et la solutoin optimale, peuvent etre récupérés en utilisant les commandes suivantes:
                \juliaFile{./julia_code/julia_jump_solution_P1.jl}

                Voici un exemple, en supposant que \(x\) est le vecteur de variables, \(\text{profit}\) et \(\text{weight}\) sont des vecteurs de coefficients:
                \juliaFile{./julia_code/julia_jump_example_solution_P1.jl}
            \end{enumerate}
        \end{enumerate}
    \end{td-exo}
}{}

% ----- Solutions exo 7 ----- %
\begin{td-sol}[]\ %
    % TODO: completer solution exercice xx
\end{td-sol}

% ----- Consignes exo xx ----- %
\iftoggle{showquestions}{
    \begin{td-exo}[]\,\\

    \end{td-exo}
}{}

% ----- Solutions exo xx ----- %
\begin{td-sol}[]\ %
    % TODO: completer solution exercice xx
\end{td-sol}

% ----- Consignes exo xx ----- %
\iftoggle{showquestions}{
    \begin{td-exo}[]\,\\

    \end{td-exo}
}{}

% ----- Solutions exo xx ----- %
\begin{td-sol}[]\ %
    % TODO: completer solution exercice xx
\end{td-sol}
