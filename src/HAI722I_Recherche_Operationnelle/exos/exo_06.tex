% ----- Consignes exo 6 ----- %
\begin{td-exo}[Ensembles convexes]\ % 6
	\begin{enumerate}
		\item Soit \(C\) un convexe. Montrer que \(x\in C\) est un point extrême de \(C\)
		si et seulement si \(C\setminus\{x\}\) est convexe.

		\item A-t-on une caractérisation similaire pour une face de \(C\)?

		\item On considère dans \(\bb R^n\) les deux boules suivantes:
		\begin{itemize}
			\item \(\mathsf{B}_1=\{x = (x_1,\ldots,x_n)\in \bb R^n\ \mid\ \sum_{i=1}^\infty \n{x_i} \leq 1\}\)
			\item \(\mathsf{B}_\infty = \{x = (x_1,\ldots,x_n)\in \bb R^n\ \mid\ \max_{1\leq i\leq n} \n{x_i} \leq 1\}\)
		\end{itemize}
		Quels sont les points extrêmes de \(\mathsf{B}_1\) et \(\mathsf{B}_\infty\)?
	\end{enumerate}
\end{td-exo}

% ----- Solutions exo 6 ----- %
\iftoggle{showsolutions}{
	\begin{td-sol}[]\ %
		\begin{enumerate}
			\item Commencons par le sens direct:\\
			Soit \(x\in C\) un point extrême de \(C\). Montrons que \(C\setminus\{x\}\) est convexe.

			Soit \(y,z \in C\setminus\{x\}\) et \(\lambda \in \ff{0,1}\). Alors
			\begin{equation*}
				\lambda y + (1-\lambda) z \in C.
			\end{equation*}
			car \(C\) est convexe. Supposons par l'absurde que \(\lambda y + (1-\lambda) z = x\). 
			Alors \(x\) est une combinaison convexe de \(y\) et \(z\) avec \(\lambda \in \ff{0,1}\).
			Cela contredit le fait que \(x\) est un point extrême de \(C\). Donc la
			combinaison convexe \(\lambda y + (1-\lambda) z\) est dans \(C\setminus\{x\}\).
			Donc \(C\setminus\{x\}\) est convexe.

			Montrons maintenant le sens réciproque:\\
			Soit \(x\in C\) tel que \(C\setminus\{x\}\) est convexe. Montrons que \(x\) est un point extrême de \(C\).

			Supposons par l'absurde que \(x\) n'est pas un point extrême de \(C\). Alors, il existe \(y,z \in C\) et \(\lambda \in \oo{0,1}\) tels que
			\begin{equation*}
				x = \lambda y + (1-\lambda) z.
			\end{equation*}
			Comme \(y,z \in C\) et \(x\) est une combinaison convexe de \(y\) et \(z\), on a forcément \(y \neq x\) et \(z \neq x\).
			Donc \(y,z \in C\setminus\{x\}\). Comme \(C\setminus\{x\}\) est convexe, on a
			\begin{equation*}
				x = \lambda y + (1-\lambda) z \in C\setminus\{x\}.
			\end{equation*}
			C'est une contradiction. Donc \(x\) est un point extrême de \(C\).
		\end{enumerate}
	\end{td-sol}
}{}
