
% ----- Consignes exo 36 ----- %
\begin{td-exo}[Ecarts complémentaires et dualité] % 36
	Soit 
	\begin{equation*}
		PL = 
		\begin{cases}
			\max z(x_1,x_2) = 4x_1 + 5x_2\\
			2x_1 + x_2 \leq 800\\
			x_1 + 2x_2 \leq 700\\
			x_2 \leq 300\\
			x_1, x_2 \geq 0.
		\end{cases}
	\end{equation*}
	\begin{enumerate}
		\item Vérifier la faisabilité du système \(PL\).
		\item Résoudre le primal.
		\item Donner le programme dual.
		\item Vérifier les écarts complémentaires.
	\end{enumerate}
\end{td-exo}

% ----- Solutions exo 36 ----- %
\iftoggle{showsolutions}{
	\begin{td-sol}[]\ % 36
		\begin{enumerate}
			\item On sait que \((0,0,0)\) satisfait les contraintes donc
			il existe une solution au système \(PL\).

			On peut le vérifier avec le lemme de Farkas:
			\begin{equation*}
				Ax\leq b,\ x\geq 0\quad \mathsf{XOR}\quad \exists y\geq 0, yA\geq 0, yb<0
			\end{equation*}
			Or ici, il n'y a clairement pas de solution a \(yb<0\) car tous les \(y_i b_i\)
			sont positifs. Donc on a une solution à \(Ax\leq b\).

			\item On a le tableau initial suivant:
			
			\begin{center}
				\begin{tabular}{|ccc|ccccc|} % chktex 44
					\hline  % chktex 44
					\, & \, &\(c\)&\(4\)&\(5\)&\(0\)&\(0\)&\(0\)\\
					\hline % chktex 44
					\multicolumn{1}{|c|}{\(c^J\)}& \multicolumn{2}{c|}{variables de base}&\(x_1\)&\(x_2\)&\(x_3\)&\(x_4\)&\(x_5\)\\
					\hline % chktex 44
					\multicolumn{1}{|c|}{\(0\)}& \multicolumn{1}{c|}{\(x_1^{1}=x_3\)}&\(800\)&\(2\)&\(1\)&\(1\)&\(0\)&\(0\)\\
					\hline % chktex 44
					\multicolumn{1}{|c|}{\(0\)}& \multicolumn{1}{c|}{\(x_2^{1}=x_4\)}&\(700\)&\(1\)&\(2\)&\(0\)&\(1\)&\(0\)\\
					\hline % chktex 44
					\multicolumn{1}{|c|}{\(0\)}& \multicolumn{1}{c|}{\(x_3^{1}=x_5\)}&\(300\)&\(0\)&\(1\)&\(0\)&\(0\)&\(1\)\\
					\hline % chktex 44
					\multicolumn{1}{|c|}{} &\(z(x)\)& \multicolumn{1}{|c|}{\(0\)}&\(-4\)&\(-5\)&\(0\)&\(0\)&\(0\)\\
					\hline % chktex 44
				\end{tabular}
			\end{center}
			
			On fait rentrer \(x_2\) et on fait sortir \(x_5\):
			\begin{center}
				\begin{tabular}{|ccc|ccccc|} % chktex 44
					\hline  % chktex 44
					\, & \, &\(c\)&\(4\)&\(5\)&\(0\)&\(0\)&\(0\)\\
					\hline % chktex 44
					\multicolumn{1}{|c|}{\(c^J\)}& \multicolumn{2}{c|}{variables de base}&\(x_1\)&\(x_2\)&\(x_3\)&\(x_4\)&\(x_5\)\\
					\hline % chktex 44
					\multicolumn{1}{|c|}{\(0\)}& \multicolumn{1}{c|}{\(x_1^{2}=x_3\)}&\(500\)&\(2\)&\(0\)&\(1\)&\(0\)&\(-1\)\\
					\hline % chktex 44
					\multicolumn{1}{|c|}{\(0\)}& \multicolumn{1}{c|}{\(x_2^{2}=x_4\)}&\(100\)&\(1\)&\(0\)&\(0\)&\(1\)&\(-2\)\\
					\hline % chktex 44
					\multicolumn{1}{|c|}{\(5\)}& \multicolumn{1}{c|}{\(x_3^{2}=x_2\)}&\(300\)&\(0\)&\(1\)&\(0\)&\(0\)&\(1\)\\
					\hline % chktex 44
					\multicolumn{1}{|c|}{} &\(z(x)\)            & \multicolumn{1}{|c|}{\(1500\)}&\(-4\)&\(0\)&\(0\)&\(0\)&\(5\)\\
					\hline % chktex 44
				\end{tabular}
			\end{center}

			On fait rentrer \(x_1\) et on fait sortir \(x_4\):
			\begin{center}
				\begin{tabular}{|ccc|ccccc|} % chktex 44
					\hline  % chktex 44
					\, & \, &\(c\)&\(4\)&\(5\)&\(0\)&\(0\)&\(0\)\\
					\hline % chktex 44
					\multicolumn{1}{|c|}{\(c^J\)}& \multicolumn{2}{c|}{variables de base}&\(x_1\)&\(x_2\)&\(x_3\)&\(x_4\)&\(x_5\)\\
					\hline % chktex 44
					\multicolumn{1}{|c|}{\(0\)}& \multicolumn{1}{c|}{\(x_1^{3}=x_3\)}&\(300\)&\(0\)&\(0\)&\(1\)&\(-2\)&\(3\)\\
					\hline % chktex 44
					\multicolumn{1}{|c|}{\(4\)}& \multicolumn{1}{c|}{\(x_2^{3}=x_1\)}&\(100\)&\(1\)&\(0\)&\(0\)&\(1\)&\(-2\)\\
					\hline % chktex 44
					\multicolumn{1}{|c|}{\(5\)}& \multicolumn{1}{c|}{\(x_3^{3}=x_2\)}&\(300\)&\(0\)&\(1\)&\(0\)&\(0\)&\(1\)\\
					\hline % chktex 44
					\multicolumn{1}{|c|}{} &\(z(x)\)            & \multicolumn{1}{|c|}{\(1900\)}&\(0\)&\(0\)&\(0\)&\(4\)&\(-3\)\\
					\hline % chktex 44
				\end{tabular}
			\end{center}

			On fait rentrer \(x_5\) et on fait sortir \(x_3\):
			\begin{center}
				\begin{tabular}{|ccc|ccccc|} % chktex 44
					\hline  % chktex 44
					\, & \, &\(c\)&\(4\)&\(5\)&\(0\)&\(0\)&\(0\)\\
					\hline % chktex 44
					\multicolumn{1}{|c|}{\(c^J\)}& \multicolumn{2}{c|}{variables de base}&\(x_1\)&\(x_2\)&\(x_3\)&\(x_4\)&\(x_5\)\\
					\hline % chktex 44
					\multicolumn{1}{|c|}{\(0\)}& \multicolumn{1}{c|}{\(x_1^{4}=x_5\)}&\(100\)&\(0\)&\(0\)&\(\frac13\)&\(-\frac23\)&\(1\)\\
					\hline % chktex 44
					\multicolumn{1}{|c|}{\(4\)}& \multicolumn{1}{c|}{\(x_2^{4}=x_1\)}&\(300\)&\(1\)&\(0\)&\(\frac23\)&\(-\frac13\)&\(0\)\\
					\hline % chktex 44
					\multicolumn{1}{|c|}{\(5\)}& \multicolumn{1}{c|}{\(x_3^{4}=x_2\)}&\(200\)&\(0\)&\(1\)&\(-\frac13\)&\(\frac23\)&\(0\)\\
					\hline % chktex 44
					\multicolumn{1}{|c|}{} &\(z(x)\)            & \multicolumn{1}{|c|}{\(2200\)}&\(0\)&\(0\)&\(1\)&\(2\)&\(0\)\\
					\hline % chktex 44
				\end{tabular}
			\end{center}
			Tous les \(z_j-c_j\) sont positifs donc on a trouvé notre solution optimale:
			\begin{equation*}
				(x_1,x_2) = (300,200),\ z_{\max} = 2200.
			\end{equation*}

			En ecrivant 
			\begin{equation*}
				A = \begin{pmatrix}
					2 & 1\\
					1 & 2\\
					0 & 1
				\end{pmatrix},\quad b = \begin{pmatrix}
					800\\
					700\\
					300
				\end{pmatrix}
			\end{equation*}
			On peut observer qu'à l'optimum, les deux premières contraintes
			\begin{equation*}
				\begin{aligned}
					2x_1 + x_2 &= 800\\
					x_1 + 2x_2 &= 700
				\end{aligned}
			\end{equation*}
			sont serrées, donc \(y_1, y_2 > 0\) (c'est le cas ici).

			La troisième contrainte
			\begin{equation*}
				x_2 \leq 300
			\end{equation*}
			n'est pas serrée, donc \(y_3 = 0\) (c'est le cas ici).

			\item Le programme dual de \(PL\) est:
			\begin{equation*}
				\begin{cases}
					\min z(y_1,y_2,y_3) = 800y_1 + 700y_2 + 300y_3\\
					2y_1 + y_2 \geq 4\\
					y_1 + 2y_2 + y_3 \geq 5\\
					y_1, y_2, y_3 \geq 0.
				\end{cases}
			\end{equation*}

			On peut lire les valeurs de \(y\) dans le tableau, ici on a
			\begin{equation*}
				y=(1, 2, 0)
			\end{equation*}

			\item Vérifions les écarts complémentaires de manière plus rigoureuse:
			On pose
			\begin{equation*}
				A = \begin{pmatrix}
					2 & 1\\
					1 & 2\\
					0 & 1
				\end{pmatrix}
				,\quad
				B = \begin{pmatrix}
					800\\
					700\\
					300
				\end{pmatrix}
			\end{equation*}
			Avec \(x=(300, 200)\) solution optimale. On voit alors que 
			la première contrainte est serrée, la deuxième aussi mais pas la 
			troisième, donc on a la solution \(w=(w_1, w_2, w_3)\) avec 
			\begin{equation*}
				w_1\geq 0,\quad w_2\geq 0, \quad w_3 = 0.
			\end{equation*}
		\end{enumerate}
	\end{td-sol}
}{}
