% ----- Consignes exo 16 ----- %
\begin{td-exo}[Points extrêmes]\,\\ % 16
	Soit \(C\) le polyèdre convexe fermé de \(\bb R^2\) 
	décrit à l'aide des inégalités suivantes:
	\begin{equation*}
		\mathcal{P}_y =
		\begin{cases}
			x_1 + \frac83 x_2 \leq 4\\
			x_1 + x_2 \leq 2\\
			2x_1 \leq 3\\
			x_1 \geq 0\\
		\end{cases}
	\end{equation*}
	
	\begin{enumerate}
		\item Ecrire \(C\) sous la forme standard.
		\item Quels sont les points extrêmes de \(C\)?
	\end{enumerate}
\end{td-exo}

% ----- Solutions exo 16 ----- %
\iftoggle{showsolutions}{
	\begin{td-sol}[]\ %
		\begin{enumerate}
			\item On introduit les variables d'écart \(x_3, x_4, x_5\) pour obtenir la forme standard:
			\begin{equation*}
				\begin{cases}
					x_1 + \frac83 x_2 + x_3 = 4\\
					x_1 + x_2 + x_4 = 2\\
					2x_1 + x_5 = 3\\
					x_i \geq 0, \quad i=1,\ldots,5.
				\end{cases}
			\end{equation*}

			\item On écrit le problème matriciellement:
			\begin{equation*}
				\begin{cases}
					\begin{pmatrix}
						1 & \frac83 & 1 & 0 & 0 \\
						1 & 1 & 0 & 1 & 0 \\
						2 & 0 & 0 & 0 & 1
					\end{pmatrix}
				\end{cases}
			\end{equation*}
			On choisit ensuite 3 contraintes parmi les 5 et on résout les systèmes linéaires associés:
			\begin{itemize}
				\item Pour (3,4,5), on trouve \(x_1 = (0, 0, 4,2,3)\),
				\item Pour (1,2,3), on trouve \(x_2 = (\frac32, \frac12, \frac76, 0, 0)\),
			\end{itemize}
			et ainsi de suite. 
		\end{enumerate}
	\end{td-sol}
}{}
