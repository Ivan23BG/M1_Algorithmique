% ----- Consignes exo 15 ----- %
\begin{td-exo}[Points extrêmes et solutions]\,\\ % 15
	Soit le polyèdre \(P = \{x \in \bb R^3 \mid Ax \leq b\}\) avec
	\begin{equation*}
		A = \begin{pmatrix}
			2, 3, 6 \\
			-1, 0, 0 \\
			0, -1, 0 \\
			0, 0, -1 \\
		\end{pmatrix}, \quad
		b = \begin{pmatrix}
			6, 1, 0, 0
		\end{pmatrix}.
	\end{equation*}
	On supposera que le polyèdre est borné.

\end{td-exo}

% ----- Solutions exo 15 ----- %
\iftoggle{showsolutions}{
	\begin{td-sol}[]\ %
		Pour trouver les points extrêmes, on cherche à résoudre
		tous les arrangements possibles de 3 contraintes parmi les 4.
		On énumère les lignes qu'on choisit comme contraintes comme suit et on les résout dans l'ordre:
		\begin{equation*}
			(1,2,3), (1,2,4), (1,3,4), (2,3,4).
		\end{equation*}
		\begin{itemize}
			\item Pour (1,2,3), on résout:
			\begin{equation*}
				\begin{cases}
					2x_1 + 3x_2 + 6x_3 = 6\\
					-x_1 = 1\\
					-x_2 = 0
				\end{cases}
			\end{equation*}
			et on trouve \(x_1 = (-1, 0, \frac43)\).
			\item Pour (1,2,4), on trouve \(x_2 = (-1, \frac83, 0)\),
			\item Pour (1,3,4), on trouve \(x_3 = (3, 0, 0)\),
			\item Pour (2,3,4), on trouve \(x_4 = (-1, 0, 0)\).	
		\end{itemize}
		Les points extrêmes sont donc \(x_1, x_2, x_3\) et \(x_4\).
		Si on rajoute la contrainte \(x_i \geq 0\), on a seulement \(x_3\) qui reste.
	\end{td-sol}
}{}
