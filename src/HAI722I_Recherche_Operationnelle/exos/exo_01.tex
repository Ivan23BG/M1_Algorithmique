
% ----- Consignes exo 1 ----- %
\begin{td-exo}[Convexité]\, % 1 
	\begin{enumerate}
		\item Soit une famille (éventuellement infinie) d'inégalités linéaires
		\(a_i^T x \leq b_i,i\in I\). Soit \(C\) son ensemble de solutions.
		Montrer que \(C\) est convexe.

		\item Montrer que la boule fermée \(\mathsf{B}(a,r)\) est convexe
		pour tout \(a\in\bb R^n\) et \(r\in \bb R^+\).

		\item Soit \(S\subseteq\bb R^n\) et soit \(W\) l'ensemble de toutes les
		combinaisons convexes de points de \(S\). Montrer que \(W\) est convexe.

		\item Soit \(C\) un convexe. Montrer que 
		\begin{equation*}
			\bigcup_{0\leq\lambda\leq 1}\lambda C
		\end{equation*}
		est convexe.

		\item Une matrice \(A=(a_{ij})\) de dimension \(n\times n\) est bistochastique si 
		elle satisfait
		\begin{equation*}
			\begin{aligned}
				\forall i\in\{1,\ldots,n\}, & \sum_{j=1}^n a_{ij} = 1, \\
				\forall j\in\{1,\ldots,n\}, & \sum_{i=1}^n a_{ij} = 1, \\
				\forall (i,j)\in{\{1,\ldots,n\}}^2, & a_{ij}\geq 0.
			\end{aligned}
		\end{equation*}
		Une matrice de permutation \(P\) est une matrice bistochastique à valeurs 
		entières, c'est-à-dire que dans chaque ligne de \(P\) il y a un et un seul élément égal à 1,
		et les autres sont nuls. De même pour chaque colonne.
		\begin{enumerate}
			\item Montrer que pour toute matrice bistochastique \(A\), il existe
			une matrice de permutation \(P\) de même dimension telle que \(p_{ij}=0\)
			si \(a_{ij}=0\).

			\item Est-ce qu'une combinaison convexe de matrices de permutation est 
			une matrice bistochastique?

			\item Montrer que toute matrice bistochastique \(A\) est une combinaison convexe
			de matrices de permutation.

			\item Trouver la combinaison convexe pour la matrice \(A\) suivante:
			\begin{equation*}
				A = \begin{pmatrix}
					0.15 & 0.37 & 0 & 0.48 \\
					0.02 & 0.15 & 0.67 & 0.16 \\
					0.46 & 0.02 & 0.16 & 0.36 \\
					0.37 & 0.46 & 0.17 & 0
				\end{pmatrix}.
			\end{equation*}
		\end{enumerate}

		\item Soient maintenant \(C_1\) et \(C_2\) deux convexes disjoints et
		\begin{equation*}
			D_1 = \bigcup_{0\leq\lambda\leq 1}\lambda C_1, \quad i=1,2.
		\end{equation*}
		Montrer que l'un des deux convexes \(C_1\cap D_2\) ou \(C_2\cap D_1\) est vide.
	\end{enumerate}
\end{td-exo}

% ----- Solutions exo 1 ----- %
\iftoggle{showsolutions}{
	\begin{td-sol}[]\ %
		A remplir %TODO solve exercise 1
	\end{td-sol}
}{}
