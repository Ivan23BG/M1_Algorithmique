% ----- Consignes exo 50 ----- %
\begin{td-exo}[Un problème de découpe] % 50
	Soit une entreprise sidérurgique produisant des barres d'acier de longueur \(l\). 
    Elle doit satisfaire un carnet de commandes constitué de \(n_i\) barres de longueur \(L_i \leq l, i = 1, \ldots, m\).

    Le problème consiste à déterminer quelles découpes effectuer pour satisfaire le carnet de commande tout en minimisant la longueur totale des déchets.

    La modélisation classique, la plus simple, énumère toutes les découpes admissibles (ou du moins, parmi elles, les découpes intéressantes).
\end{td-exo}

% ----- Solutions exo 50 ----- %
\iftoggle{showsolutions}{
	\begin{td-sol}[]\ %
		On peut modéliser le problème comme suit:

        \begin{equation*}
            \begin{aligned}
                \min\quad & \sum_{j} c_j x_j\\
                \text{s.c.}\quad & n_i \leq t_{ij} x_j, \quad \forall i = 1, \ldots, m,\\
                                 & x_j \geq 0, \quad \forall j,
            \end{aligned}
        \end{equation*}
        où \(x_j\) est le nombre de barres découpées selon le schéma \(j\), \(t_{ij}\) est le nombre de barres de longueur \(L_i\) obtenues dans le schéma \(j\), et \(c_j\) est la longueur de déchets générée par le schéma \(j\).
	\end{td-sol}
}{}

