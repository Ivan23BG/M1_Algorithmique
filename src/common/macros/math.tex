\makeatletter

% ==============================================================================
% GENERAL FONT SHORTCUTS
% ==============================================================================

\newcommand{\scr}[1]{\mathscr{#1}}   % Shortcut for script font (calligraphic style)
\newcommand{\bb}[1]{\mathbb{#1}}     % Shortcut for blackboard bold font
\newcommand{\ol}[1]{\overline{#1}}   % Shortcut for overline
\newcommand{\ul}[1]{\underline{#1}}  % Shortcut for underline

% ==============================================================================
% INTERVAL NOTATION
% ==============================================================================

\newcommand{\oo}[1]{\mathopen{}\left]#1\right[\mathclose{}}   % Open interval: ]a, b[      % chktex 9 
\newcommand{\of}[1]{\mathopen{}\left]#1\right]\mathclose{}}   % Half-open interval: ]a, b] % chktex 9 chktex 10
\newcommand{\fo}[1]{\mathopen{}\left[#1\right[\mathclose{}}   % Half-open interval: [a, b[ % chktex 9 chktex 10 chktex 15
\newcommand{\ff}[1]{\mathopen{}\left[#1\right]\mathclose{}}   % Closed interval: [a, b]

% ==============================================================================
% SETS & GROUPS
% ==============================================================================

\newcommand{\R}{\mathbb{R}}                  % Real numbers
\newcommand{\Z}{\mathbb{Z}}                  % Integers
\newcommand{\N}{\mathbb{N}}                  % Natural numbers
\newcommand{\C}{\mathbb{C}}                  % Complex numbers
\newcommand{\Q}{\mathbb{Q}}                  % Rational numbers
\newcommand{\barR}{\overline{\bb{R}}}        % Extended real numbers (with infinities)
\newcommand{\glx}[1]{\mathsf{GL}_{#1}}       % General linear group of degree #1
\newcommand{\sub}{\subset}                   % Subset symbol
\newcommand{\restr}[2]{#1\mathop{}\!|_{#2}}  % Restriction of function #1 to set #2
\newcommand{\comp}[1]{{#1}^{\mathsf{c}}}     % Complement of set #1

% ==============================================================================
% DIFFERENTIAL & DERIVATIVE NOTATION
% ==============================================================================

\newcommand{\der}{\mathop{}\!{d}}                             % Differential operator
\newcommand{\p}{\mathop{}\!{\partial}}                        % Partial derivative operator
\providecommand{\dpar}[2]{\frac{\partial{#1}}{\partial{#2}}}  % Partial derivative

% ==============================================================================
% TOPOLOGY NOTATION
% ==============================================================================

\newcommand{\bolo}[2]{\mathsf{B}\left({#1, #2}\mathopen{}\right[\mathclose{}}  % Open ball: B(x, r)[   % chktex 9
\newcommand{\bolf}[2]{\mathsf{B}\left({#1, #2}\mathopen{}\right]\mathclose{}}  % Closed ball: B(x, r)] % chktex 9

% ==============================================================================
% LIMITS & NORM NOTATION
% ==============================================================================

\newcommand{\limi}{\underline{\lim}}                           % Lower limit (lim inf)
\newcommand{\lims}{\overline{\lim}}                            % Upper limit (lim sup)
\newcommand{\norm}{\mathcal{N}}                                % Norm symbol (calligraphic N)
\newcommand{\nn}[1]{\mathopen{}\left\|#1\right\|\mathclose{}}  % Double bar norm: ||x||
\newcommand{\n}[1]{\mathopen{}\left|#1\right|\mathclose{}}     % Single bar norm: |x|

% ==============================================================================
% OPERATORS & FUNCTIONS
% ==============================================================================

\providecommand{\1}{\mathds{1}}                   % Indicator function
\DeclareMathOperator{\im}{\mathsf{Im}}            % Imaginary part
\DeclareRobustCommand{\re}{\mathsf{Re}}           % Real part
\RedeclareMathOperator{\ker}{\mathsf{Ker}}        % Kernel of a map
\RedeclareMathOperator{\det}{\mathsf{det}}        % Determinant
\DeclareMathOperator{\vect}{\mathsf{Vect}}        % Vector space generated by a set
\DeclareMathOperator{\diam}{\mathsf{Diam}}        % Diameter of a set
\DeclareMathOperator{\orb}{\mathsf{orb}}          % Orbit
\DeclareMathOperator{\st}{\mathsf{st}}            % Standard part (nonstandard analysis)
\DeclareMathOperator{\spr}{\mathsf{SP_{\bb{R}}}}  % Real spectrum
\DeclareMathOperator{\aut}{\mathsf{Aut}}          % Automorphism group
\DeclareMathOperator{\bij}{\mathsf{Bij}}          % Bijection group
\DeclareMathOperator{\rank}{\mathsf{rank}}        % Rank of a matrix
\DeclareMathOperator{\tr}{\mathsf{tr}}            % Trace of a matrix
\DeclareMathOperator{\id}{\mathsf{Id}}            % Identity map
\DeclareMathOperator{\var}{\mathsf{Var}}          % Variance
\DeclareMathOperator{\cov}{\mathsf{Cov}}          % Covariance
\providecommand{\B}{\mathsf{B}}                   % Bold symbol (often used for balls or sets)

% ==============================================================================
% MISCELLANEOUS SHORTCUTS
% ==============================================================================

\newcommand{\one}{\mathds{1}}                  % Alternative indicator function
\newcommand{\from}{\ \colon\ }                 % Function definition symbol (f : X → Y)
\newcommand{\smol}[1]{\text{\scriptsize{#1}}}  % Typeset text in scriptsize (for indices, etc.)

\makeatother % chktex 17