% ==============================================================================
% FONT & LANGUAGE SETTINGS
% ==============================================================================

\usepackage[T1]{fontenc}         % T1 font encoding: better European character support
\usepackage[french]{babel}       % French language support: hyphenation, translations
\usepackage{lmodern}             % Latin Modern font: improved version of Computer Modern

% ==============================================================================
% LISTS & ENUMERATION
% ==============================================================================

\usepackage[shortlabels]{enumitem} % Customizable lists (itemize/enumerate)
\setlist[itemize,1]{label={\color{gray}\small \textbullet}} % Custom bullet for itemize

% ==============================================================================
% PAGE LAYOUT & SECTIONING
% ==============================================================================

\usepackage{fancyhdr}            % Custom headers and footers
\usepackage{titlesec}            % Section title formatting
\usepackage{titletoc}            % Table of contents formatting
\usepackage{sectsty}             % Section font customization

% ==============================================================================
% MATHEMATICS & SYMBOLS
% ==============================================================================

\usepackage{centernot}           % Centered negation symbols (e.g., \centernot\implies)
\usepackage{stmaryrd}            % Extra math symbols (e.g., \llbracket, \rrbracket)
\usepackage[overload]{abraces}   % Extensible braces for math expressions
\usepackage{latexsym}            % Standard LaTeX math symbols
\usepackage{amsmath}             % Advanced math typesetting (align, gather, etc.)
\usepackage{amsfonts}            % AMS math fonts (e.g., \mathbb)
\usepackage{amssymb}             % Additional AMS math symbols
\usepackage{amsthm}              % Theorem environments (definition, lemma, etc.)
\usepackage{mathtools}           % Extensions to amsmath (e.g., \coloneqq)
\usepackage{mathrsfs}            % Script font (\mathscr)
\usepackage{MnSymbol}            % More math symbols (arrows, delimiters, etc.)
\usepackage{dsfont}              % Double stroke font (\mathds)
\usepackage{etoolbox}            % Programming tools for conditional macros

% ==============================================================================
% GRAPHICS & PLOTTING
% ==============================================================================

\usepackage{tikz}                % Drawing graphics programmatically
\usepackage{pgfplots}            % Plotting graphs with TikZ
\pgfplotsset{compat=1.18}        % Set pgfplots compatibility version
\usetikzlibrary{positioning, shapes, arrows.meta}          % Arrow tips for TikZ diagrams

% ==============================================================================
% COLOR MANAGEMENT
% ==============================================================================

\usepackage{xcolor}              % Color support for text and graphics
\usepackage{colortbl}            % Color support for tables

% --- Custom color definitions ---
\definecolor{astral}{RGB}{46,116,181}             % Blue shade for headers
\definecolor{verdant}{RGB}{96,172,128}            % Green shade
\definecolor{algebraic-amber}{RGB}{255,179,102}   % Amber shade
\definecolor{calculus-coral}{RGB}{255,191,191}    % Coral shade
\definecolor{divergent-denim}{RGB}{130,172,211}   % Denim shade
\definecolor{matrix-mist}{RGB}{204,204,204}       % Misty gray
\definecolor{numeric-navy}{RGB}{204,204,204}      % Navy gray (same as mist)
\definecolor{quadratic-quartz}{RGB}{204,153,153}  % Quartz shade

% ==============================================================================
% FRAMED ENVIRONMENTS
% ==============================================================================

\usepackage[]{mdframed}           % Framed boxes for theorems, definitions, etc.

% ==============================================================================
% SECTIONING ENVIRONMENTS & TOC CUSTOMIZATION
% ==============================================================================

% --- Set up sectioning depth ---
\setcounter{secnumdepth}{5}       % Numbering depth for sections
\setcounter{tocdepth}{5}          % TOC depth

% --- Define suprasection (custom section above \section) ---
\titleclass{\suprasection}{straight}[\part]
\newcounter{suprasection}
\renewcommand\thesuprasection{\arabic{suprasection}}

% --- Format suprasection titles ---
\titleformat{\suprasection}
    {\color{astral}\normalfont\sffamily\bfseries\Large\filcenter}{\thesuprasection}{1em}{}
\titlespacing*{\suprasection}{0pt}{3.5ex plus 1ex minus .2ex}{2.3ex plus .2ex}

% --- TOC formatting: shift levels down ---
\dottedcontents{suprasection}[1.5em]{}{2.3em}{1pc}
\dottedcontents{section}[3.8em]{}{2.8em}{1pc}
\dottedcontents{subsection}[7.0em]{}{3.2em}{1pc}
\dottedcontents{subsubsection}[10.0em]{}{3.8em}{1pc}

% --- Add suprasection to TOC ---
\titlecontents{suprasection}
    [0em]
    {\addvspace{1pc}\bfseries}
    {\contentslabel{2em}}
    {\hspace*{-2em}}
    {\titlerule*[0.5pc]{.}\contentspage}

% ==============================================================================
% HYPERLINKS
% ==============================================================================

\usepackage{hyperref}             % Hyperlinks for cross-references and URLs

% ==============================================================================
% MISCELLANEOUS SETTINGS
% ==============================================================================

\setlength{\parindent}{0pt}       % Remove paragraph indentation

% --- Section header styling ---
\allsectionsfont{\color{astral}\normalfont\sffamily\bfseries} % Style all section headers

% ==============================================================================
% CUSTOM COMMANDS & MACROS
% ==============================================================================

% --- Emphasis in text ---
\providecommand{\defemph}[1]{{\sffamily\bfseries\color{astral}#1}}

% --- Utility commands ---
\newcommand{\skipline}{\vspace{\baselineskip}} % Skip a line
\newcommand{\noi}{\noindent}                   % No indentation
\newcommand{\ptr}{\(\triangleright\)}          % Triangle right symbol

% ==============================================================================
% ADVANCED MACROS & REDECLARATIONS
% ==============================================================================

% The following section provides a safe way to redeclare an existing math 
% operator command (like \sin, \ker, etc.) with a new name or formatting, 
% similar to \DeclareMathOperator, but without causing errors if the operator 
% already exists. It also supports starred and unstarred versions for different 
% formatting.

\makeatletter
\newcommand\RedeclareMathOperator{%
	\@ifstar{\def\rmo@s{m}\rmo@redeclare}{\def\rmo@s{o}\rmo@redeclare}%
}
\newcommand\rmo@redeclare[2]{%
	\begingroup \escapechar\m@ne\xdef\@gtempa{{\string#1}}\endgroup
	\expandafter\@ifundefined\@gtempa%
	{\@latex@error{\noexpand#1undefined}\@ehc}%
	\relax
	\expandafter\rmo@declmathop\rmo@s{#1}{#2}}

\newcommand\rmo@declmathop[3]{%
	\DeclareRobustCommand{#2}{\qopname\newmcodes@#1{#3}}%
}
\@onlypreamble\RedeclareMathOperator%
\makeatother

% End of header