% ----- Consignes exo 1 ----- %
\begin{td-exo}[Glouglouglou]\,\\ % 1 
	Dérouler à la main l'algorithme de ford-Fulkerson sur les deux réseaux de transport suivants afin de déterminer dans chaque cas un flot maximal ainsi qu'une coupe minimale (les valeurs notées sur les arcs sont les capacités correspondantes).
    Pour le second graphe, on commencera par une augmentation de flot le long du chemin \(sabdp\) suivi d'une augmentation de flot le long de \(scedbap\).

    % TODO Insert 2 graphs here
\end{td-exo}

% ----- Solutions exo 1 ----- %
\iftoggle{showsolutions}{
	\begin{td-sol}[]\,\\ %
		A remplir %TODO solve exercise 1
	\end{td-sol}
}{}


% ----- Consignes exo 2 ----- %
\begin{td-exo}[PL]\,\\ % 2 
	Soit \(\mathcal{N} = (D, s, p, c)\) un réseau de transport. Montrer qu'il est possible d'écrire un programme linéaire correspondant à la racherche d'un flot de valeur maximal dans \(\mathcal N\).
    On proposera un modèle utilisant un nombre polynomial (en la taille de \(D\)) de variables et de contraintes.
\end{td-exo}

% ----- Solutions exo 2 ----- %
\iftoggle{showsolutions}{
	\begin{td-sol}[]\,\\ %
		A remplir %TODO solve exercise 2
	\end{td-sol}
}{}


% ----- Consignes exo 3 ----- %
\begin{td-exo}[Décomposition d'un flot]\,\\ % 3 
	Soit \(f\) un flot à valeurs entières sur un réseau de transport \(\mathcal{N} = (D, s, p, c)\).
    On veut montrer qu'il existe dans \(D\) un multi-ensemble (un ensemble où les éléments peuvent apparaitre plusieurs fois) de chemins \(P_1,\ldots,P_k\) tous de \(s\) à \(p\) et un multi-ensemble de cycles \(C_1,\ldots,C_l\) tels que pour tout arc \(xy\) de \(D\) on ait \(f(xy) = \#\{i\in \{1,\ldots,k\}: xy\in P_i\} + \#\{j\in \{1,\ldots,l\}: xy\in C_j\}\).
    \begin{enumerate}
        \item Sur le premier graphe de l'exercice 1, % TODO add reference here
        on considère le flot \(f\) défini par \(f(ab) = 5, f(bc) = 2, f(bp) = 4, f(ca) = 2, f(cd) = 1, f(db) = 1, f(sa) = 3, f(sc) = 1\).
        Donner la décomposition voulue.

        \item Prouver le résultat attendu en raisonnant par récurrence sur \(\sum_{xy\in A(D)} f(xy)\).
    \end{enumerate}
\end{td-exo}

% ----- Solutions exo 3 ----- %
\iftoggle{showsolutions}{
	\begin{td-sol}[]\,\\ %
		A remplir %TODO solve exercise 3
	\end{td-sol}
}{}


% ----- Consignes exo 4 ----- %
\begin{td-exo}[Théorème de Hall]\,\\ % 4 
	Soit \(G = ((A, B), E)\) un graphe biparti (de bipartition \((A, B)\)). On souhaite montrer. % TODO FINIR
\end{td-exo}

% ----- Solutions exo 4 ----- %
\iftoggle{showsolutions}{
	\begin{td-sol}[]\,\\ %
		A remplir %TODO solve exercise 4
	\end{td-sol}
}{}


% ----- Consignes exo 5 ----- %
\begin{td-exo}[Convexité]\,\\ % 5 
	% fill
\end{td-exo}

% ----- Solutions exo 5 ----- %
\iftoggle{showsolutions}{
	\begin{td-sol}[]\,\\ %
		A remplir %TODO solve exercise 5
	\end{td-sol}
}{}


% ----- Consignes exo 6 ----- %
\begin{td-exo}[Convexité]\,\\ % 6 
	% fill
\end{td-exo}

% ----- Solutions exo 6 ----- %
\iftoggle{showsolutions}{
	\begin{td-sol}[]\,\\ %
		A remplir %TODO solve exercise 6
	\end{td-sol}
}{}


% ----- Consignes exo 7 ----- %
\begin{td-exo}[Convexité]\,\\ % 7 
	% fill
\end{td-exo}

% ----- Solutions exo 7 ----- %
\iftoggle{showsolutions}{
	\begin{td-sol}[]\,\\ %
		A remplir %TODO solve exercise 7
	\end{td-sol}
}{}