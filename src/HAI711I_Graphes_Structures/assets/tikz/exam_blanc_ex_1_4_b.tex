\ffigbox[\FBwidth]{%
\caption{\centering Réseau résiduel après avoir retiré un flot de 1 par \(sadp\) et un autre de 3 par \(sbep\)}\label{Fig:exam_blanc_ex_1_3_b}
}{
    \fbox{
        \begin{tikzpicture}[scale=1, main node/.style={circle, draw, fill=blue!20, inner sep=1pt, font=\scriptsize, minimum size=6mm}]
            % les sommets initiaux
            \node[main node] (s) at (-3,0) {\(s\)};
            \node[main node] (p) at (3,0) {\(p\)};

            \node[main node] (a) at (-1,2) {\(a\)};
            \node[main node] (b) at (-1,0) {\(b\)};
            \node[main node] (c) at (-1,-2) {\(c\)};

            \node[main node] (d) at (1,2) {\(d\)};
            \node[main node] (e) at (1,0) {\(e\)};
            \node[main node] (f) at (1,-2) {\(f\)};

            % les arcs avec capacités
            \draw[-{Stealth}, draw=red] (s) to node[above left] {\(1\)} (a);
            \draw[-{Stealth}, draw=blue] (s) to node[below left] {\(3\)} (c);

            \draw[-{Stealth}, draw=red, bend left] (a) to node[above] {\(2\)} (d);
            \draw[-{Stealth}, draw=red, bend left] (d) to node[above] {\(3\)} (a);

            \draw[-{Stealth}, draw=red, bend left] (b) to node[above] {\(1\)} (e);
            \draw[-{Stealth}, draw=red, bend left] (e) to node[above] {\(5\)} (b);
            \draw[-{Stealth}] (b) to node[left] {\(9\)} (f);

            \draw[-{Stealth}, draw=blue] (f) to node[below] {\(1\)} (c);

            \draw[-{Stealth}] (d) to node[left] {\(2\)} (b);

            \draw[-{Stealth}] (e) to node[right] {\(9\)} (c);
            \draw[-{Stealth}, draw=red] (e) to node[above] {\(4\)} (p);
        \end{tikzpicture}
    }
}