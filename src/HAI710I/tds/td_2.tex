Soient les structures et fonctions générales de recherche données en cours
permettant de parcourir les états d'un espace d'états par priorité croissante.
On précise que:
\begin{itemize}
	\item \(\text{Explorer}\) peut ré-explorer plusieurs fois un même état,
	\item \(\text{ExplorerOptimise}\) évite de ré-explorer un état déjà exploré
	et ne conserve dans la frontière qu'un noeud par état.
\end{itemize}


% ----- Consignes exo 1 ----- %
\begin{td-exo}[] % 1
	Préciser pour chacune des stratégies suivantes comment doivent être
	choisis les coûts et priorités \(c_0, c_{sn},p_0\) et \(p_{sn}\) des 
	deux fonctions \(\text{Explorer}\) et \(\text{ExplorerOptimise}\).
	\begin{itemize}
		\item Recherche par coût minimal (le coût étant défini
		comme la somme du coût des actions ayant conduit de l'état
		du noeud racine à l'état du noeud courant).

		\item La recherche gloutonne.

		\item La recherche avec l'algorithme \(A^\ast\).
	\end{itemize}
\end{td-exo}

% ----- Solutions exo 1 ----- %
\iftoggle{showsolutions}{
	\begin{td-sol}[]\ %
		On construit le tableau suivant:
		\begin{tabular}{|c|c|c|c|c|} %chktex 44
			\hline %chktex 44
			\ & \(c_0\) & \(p_0\) & \(c_{sn}\) & \(p_{sn}\)\\
			\hline %chktex 44
			coût min & 0 & \(k\) quelconque & \(c(n) + \text{cout}(a)\) & \(c(n) + \text{cout}(a)\)\\
			\hline %chktex 44
			glouton & 0 & \(k'\) quelconque & \(c(n) + \text{cout}(a)\) & \(h(a.\text{res}(x.\text{etat}))(x)\)\\
			\hline %chktex 44
			\(A^\ast\) & 0 & \(h(e_0)\) & \(c(n) + \text{cout}(a)\) & \(h(se) + c_{sn}\)\\
			\hline %chktex 44
		\end{tabular}

		Maintenant pour la recherche gloutonne:
		% TODO insert graph here
		% on commence avec |A, 366|^1
		% ensuite de 1 |Z, 374| -- |T, 329| -- |S, 253|^2
		% ensuite de 2 on remonte vers |A, 366| et sinon on redescend à |O, 380|, |R, 193|, |F, 176|^3
		% ensuite de 3 on remonte vers |S, 253| et sinon on redescend à |B, 0|^4 -> but

		% avec explorer optimise on obtient le meme graphe mais on supprime (barre) 
		% la deuxieme occurence de S at A.
		Le cout de cette solution est de 450 alors que la solution optimale est de 411 (à vérifier).

		Et enfin pour \(A^\ast\):
		% TODO insert graph here
		% on commence avec |A, 366|^1
		% ensuite de 1 |Z, 449| -- |T, 447| -- |S, 393|^2
		% ensuite de 2 on a |F, 415|^4 -- |R, 413|^3 -- |O, 671| et en remontant |A, 646| 
		% ensuite de 3 on a |S, 553| -- |P, 417|^5 -- |C, 526|
		% ensuite de 4 on a |S, 691| -- |B, 450|
		% ensuite de 5 on a |R, 607|, |C, 615|, |B, 518|^6 -> but
		% <- les graphes correspondent à l'exercice 2
	\end{td-sol}
}{}


% ----- Consignes exo 2 ----- %
\begin{td-exo}[] % 2
	Soit le problème de calcul d'une route de \emph{Arad} à \emph{Bucharest}
	(cf.\ document \og{}Problème de recherche de route\fg{} sur Moodle). Dessiner 
	les arbres de recherche correspondant à l'exécution de ces 3 stratégies
	en distinguant le cas où la fonction \(\text{Explorer}\) est utilisée
	de celui où l'on utilise la fonction \(\text{ExplorerOptimise}\).
\end{td-exo}

% ----- Solutions exo 2 ----- %
\iftoggle{showsolutions}{
	\begin{td-sol}[]\ %
		A faire %TODO sol exo 2
	\end{td-sol}
}{}


% ----- Consignes exo 3 ----- %
\begin{td-exo}[] % 3
	Même question de \emph{Iasi} à \emph{Fagaras}. On pourra
	prendre les distances à vol d'oiseau suivantes
	pour Fagaras:
	%TODO insert distances table here 
\end{td-exo}

% ----- Solutions exo 3 ----- %
\iftoggle{showsolutions}{
	\begin{td-sol}[]\ %
		Pour le glouton non optimisé on a:
		% TODO insert graph here
		% I, 220^1
		% N, 170^2 -- V, 260
		% I, 220^3
		% etc et on boucle entre I et N, donc glouton non optimisé n'est pas complet
		Pour le glouton optimisé on a:
		% I, 220^1
		% N, 170^2 -- V, 260^3
		% il se passe rien en 2
		% U, 230^4
		% B, 190^5 -- H, 310
		% G, 250 -- P, 110, F, 0^6 -> but
	\end{td-sol}
}{}


% ----- Consignes exo 4 ----- %
\begin{td-exo}[] % 4
	Dire pour chacune des deux versions de l'algorithme d'exploration,
	si les 3 stratégies précédentes sont complètes. Préciser éventuellement 
	sous quelles conditions elles le sont.
\end{td-exo}

% ----- Solutions exo 4 ----- %
\iftoggle{showsolutions}{
	\begin{td-sol}[]\ %
		Synthèse sur la complétude:
		% TODO inserer tableau ici
		%    | NO                                                                    | O
		% CM | si et seulement si \(\forall a, \text{cout}(a) \geq \varepsilon > 0\) | meme condition OU nombre d'etats fini
		% Gl | non complet                                                           | espace fini 
		% A* | meme que pour cout min                                                | meme que pour cout min
		où on suppose que:
		\begin{itemize}
			\item le cout de toute action est positif,
			\item \(\forall n, g(n) = \sum_{i\in\text{branche}(n)} \text{cout}(i)\),
			\item l'heuristique de tout etat est positif,
			\item l'heuristique de l'etat but est nulle.
		\end{itemize}

		Synthèse sur l'optimalité
		% TODO inserer tableau ici
		%    | NO                                                                    | O
		% CM | optimal                                                               | optimal
		% Gl | non optimal                                                           | non optimal 
		% A* | si et seulement si h est admissible                                   | si et seulement si h est monotone
		où on suppose qu'on a déjà la complétude
	\end{td-sol}
}{}


% ----- Consignes exo 5 ----- %
\begin{td-exo}[] % 5
	Une heuristique est dite \defemph{admissible} si pour tout état \(e\) elle
	ne sur-estime jamais le cout du chemin optimal de \(e\) à l'état but:
	\begin{equation*}
		\forall e, h(e) \leq g*(e)
	\end{equation*}
	où \(g*(e)\) dénote le cout d'un des chemins optimaux de
	\(e\) à l'état but le plus proche.

	On considère la stratégie \(A^\ast\) avec l'algorithme \(\text{Explorer}\).
	\begin{enumerate}
		\item Exhiber un exemple montrant que si l'heuristique utilisée n'est pas
		admissible alors la solution trouvée peut ne pas être optimale.

		\item Montrer que si l'heuristique est admissible alors la solution
		trouvée est optimale.

		\item Est-ce vrai pour l'algorithme \(\text{ExplorerOptimise}\)?
	\end{enumerate}
\end{td-exo}

% ----- Solutions exo 5 ----- %
\iftoggle{showsolutions}{
	\begin{td-sol}[]\ %
		\begin{enumerate}
			\item Il suffit de prendre le graphe suivant:
			% TODO insert graph here
			% A,B,C,D, A=I, {AB, AC, BD, CD}, couts:{2, 1, 1, 1}, heuristique: {0, 1, 5, 1}
			Clairement le chemin optimal est \((I,B,D)\) mais \(A^\ast\) fait \((I,C,D)\).

			\item Montrons par l'absurde que l'état but trouvé par \(A^\ast\) est une solution
			optimale, quand \(h\) est admissible.

			Supposons que le noeud but trouvé par \(A^\ast\) est non optimal. Alors, 
			\begin{equation*}
				g(b) > g^\ast
			\end{equation*}
			(le coût de \(b\) est supérieur au cout optimal). Soit \(n\) un noeud non exploré (mais généré
			et donc une feuille de l'arbre de recherche) sur le chemin conduisant à une solution optimale.
			Alors, \(g(n) \leq g^\ast\) car les couts sont croissant le long d'une branche.
			Donc \(g(n) < g(b)\) par hypothèse. De plus, on a \(h\) admissible, donc
			\begin{equation*}
				\forall e, h(e) \leq g^\ast(e)
			\end{equation*}
			On note \(e_n\) l'état du noeud \(n\). Alors
			\begin{equation*}
				h(e_n) \leq g^\ast(e_n)
			\end{equation*}
			Comme \(n\) est sur un chemin optimal vers un but, on a
			\begin{equation*}
				g(n) + g^\ast(e_n) = g^\ast
			\end{equation*}
			et donc
			\begin{equation*}
				f(n) = g(n) + h(n) \leq g(n) + g^\ast(e_n) = g^\ast
			\end{equation*}
			Or, comme \(b\) a été trouvé par \(A^\star\), on sait que
			\begin{equation*}
				f(b) \leq f(n) \leq g^\ast
			\end{equation*}
			mais \(f(b) > g^\ast\). Contradiction, donc \(b\) est optimal.
		\end{enumerate}
	\end{td-sol}
}{}


% ----- Consignes exo 6 ----- %
\begin{td-exo}[] % 6
	A faire %TODO consignes exo 6
\end{td-exo}

% ----- Solutions exo 6 ----- %
\iftoggle{showsolutions}{
	\begin{td-sol}[]\ %
		A faire %TODO sol exo 6
	\end{td-sol}
}{}


% ----- Consignes exo 7 ----- %
\begin{td-exo}[] % 7
	A faire %TODO consignes exo 7
\end{td-exo}

% ----- Solutions exo 7 ----- %
\iftoggle{showsolutions}{
	\begin{td-sol}[]\ %
		\begin{enumerate}
			\item L'espace d'états est le suivant:
			\begin{itemize}
				\item Les états sont des cases, 
				\item l'état initial est \((A,2)\) et 
				\item le but est \((D,4)\).
			\end{itemize}
			\item Voici le tableau des heuristiques:
			% TODO inserer tableau ici
			% x                          | A1 | A2   | ...  | A6 | B1 . B6 . D1 . C5 . D2 .
			% \(h_{\text{vol}}\)         | 3  | rc18 | rc13 | ...
			% \(h_{\text{manhattan}}\)   | 6  |  5   |   5  | ...
			Ces deux heuristiques sont bien admissibles et monotones.
			\item Clairement ici \(h_{\text{manhattan}}\) domine \(h_{\text{vol}}\)
		\end{enumerate}
	\end{td-sol}
}{}


% ----- Consignes exo 8 ----- %
\begin{td-exo}[] % 8
	A faire %TODO consignes exo 8
\end{td-exo}

% ----- Solutions exo 8 ----- %
\iftoggle{showsolutions}{
	\begin{td-sol}[]\ %
		A faire %TODO sol exo 8
	\end{td-sol}
}{}
