Soient les structures et fonctions générales de recherche données en cours
permettant de parcourir les états d'un espace d'états par priorité croissante.
On précise que:
\begin{itemize}
	\item \(\text{Explorer}\) peut ré-explorer plusieurs fois un même état,
	\item \(\text{ExplorerOptimise}\) évite de ré-explorer un état déjà exploré
	et ne conserve dans la frontière qu'un noeud par état.
\end{itemize}


% ----- Consignes exo 1 ----- %
\begin{td-exo}[] % 1
	Préciser pour chacune des stratégies suivantes comment doivent être
	choisis les coûts et priorités \(c_0, c_{sn},p_0\) et \(p_{sn}\) des 
	deux fonctions \(\text{Explorer}\) et \(\text{ExplorerOptimise}\).
	\begin{itemize}
		\item Recherche par coût minimal (le coût étant défini
		comme la somme du coût des actions ayant conduit de l'état
		du noeud racine à l'état du noeud courant).

		\item La recherche gloutonne.

		\item La recherche avec l'algorithme \(A^\ast\).
	\end{itemize}
\end{td-exo}

% ----- Solutions exo 1 ----- %
\iftoggle{showsolutions}{
	\begin{td-sol}[]\ %
		On construit le tableau suivant:
		\begin{tabular}{|c|c|c|c|c|} %chktex 44
			\hline %chktex 44
			\ & \(c_0\) & \(p_0\) & \(c_{sn}\) & \(p_{sn}\)\\
			\hline %chktex 44
			coût min & 0 & \(k\) quelconque & \(c(n) + \text{cout}(a)\) & \(c(n) + \text{cout}(a)\)\\
			\hline %chktex 44
			glouton & 0 & \(k'\) quelconque & \(c(n) + \text{cout}(a)\) & \(h(a.\text{res}(x.\text{etat}))(x)\)\\
			\hline %chktex 44
			\(A^\ast\) & 0 & \(h(e_0)\) & \(c(n) + \text{cout}(a)\) & \(h(se) + c_{sn}\)\\
			\hline %chktex 44
		\end{tabular}
	\end{td-sol}
}{}


% ----- Consignes exo 2 ----- %
\begin{td-exo}[] % 2
	Soit le problème de calcul d'une route de \emph{Arad} à \emph{Bucharest}
	(cf.\ document \og{}Problème de recherche de route\fg{} sur Moodle). Dessiner 
	les arbres de recherche correspondant à l'exécution de ces 3 stratégies
	en distinguant le cas où la fonction \(\text{Explorer}\) est utilisée
	de celui où l'on utilise la fonction \(\text{ExplorerOptimise}\).
\end{td-exo}

% ----- Solutions exo 2 ----- %
\iftoggle{showsolutions}{
	\begin{td-sol}[]\ %
		A faire %TODO sol exo 2
	\end{td-sol}
}{}


% ----- Consignes exo 3 ----- %
\begin{td-exo}[] % 3
	Même question de \emph{Iasi} à \emph{Fagaras}. On pourra
	prendre les distances à vol d'oiseau suivantes
	pour Fagaras:
	%TODO insert distances table here 
\end{td-exo}

% ----- Solutions exo 3 ----- %
\iftoggle{showsolutions}{
	\begin{td-sol}[]\ %
		A faire %TODO sol exo 3
	\end{td-sol}
}{}


% ----- Consignes exo 4 ----- %
\begin{td-exo}[] % 4
	Dire pour chacune des deux versions de l'algorithme d'exploration,
	si les 3 stratégies précédentes sont complètes. Préciser éventuellement 
	sous quelles conditions elles le sont.
\end{td-exo}

% ----- Solutions exo 4 ----- %
\iftoggle{showsolutions}{
	\begin{td-sol}[]\ %
		A faire %TODO sol exo 4
	\end{td-sol}
}{}


% ----- Consignes exo 5 ----- %
\begin{td-exo}[] % 5
	Une heuristique est dite \defemph{admissible} si pour tout état \(e\) elle
	ne sur-estime jamais le cout du chemin optimal de \(e\) à l'état but:
	\begin{equation*}
		\forall e, h(e) \meq g*(e)
	\end{equation*}
	où \(g*(e)\) dénote le cout d'un des chemins optimaux de
	\(e\) à l'état but le plus proche.

	On considère la stratégie \(A^\ast\) avec l'algorithme \(\text{Explorer}\).
	\begin{enumerate}
		\item Exhiber un exemple montrant que si l'heuristique utilisée n'est pas
		admissible alors la solution trouvée peut ne pas être optimale.

		\item Montrer que si l'heuristique est admissible alors la solution
		trouvée est optimale.

		\item Est-ce vrai pour l'algorithme \(\text{ExplorerOptimise}\)?
	\end{enumerate}
\end{td-exo}

% ----- Solutions exo 5 ----- %
\iftoggle{showsolutions}{
	\begin{td-sol}[]\ %
		A faire %TODO sol exo 5
	\end{td-sol}
}{}


% ----- Consignes exo 6 ----- %
\begin{td-exo}[] % 6
	A faire %TODO consignes exo 6
\end{td-exo}

% ----- Solutions exo 6 ----- %
\iftoggle{showsolutions}{
	\begin{td-sol}[]\ %
		A faire %TODO sol exo 6
	\end{td-sol}
}{}


% ----- Consignes exo 7 ----- %
\begin{td-exo}[] % 7
	A faire %TODO consignes exo 7
\end{td-exo}

% ----- Solutions exo 7 ----- %
\iftoggle{showsolutions}{
	\begin{td-sol}[]\ %
		A faire %TODO sol exo 7
	\end{td-sol}
}{}


% ----- Consignes exo 8 ----- %
\begin{td-exo}[] % 8
	A faire %TODO consignes exo 8
\end{td-exo}

% ----- Solutions exo 8 ----- %
\iftoggle{showsolutions}{
	\begin{td-sol}[]\ %
		A faire %TODO sol exo 8
	\end{td-sol}
}{}
