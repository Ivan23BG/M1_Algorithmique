% ----- Consignes exo 1 ----- %
\begin{td-exo}[Le puzzle du zèbre]
    Le puzzle du zèbre est un jeu logique bien connu, atribué à Albert Einstein
    ou à Lewis Caroll, sans certitude que l'inventeur soit l'un des deux. Il
    existe plusieurs variantes de ce jeu, voici l'énoncé d'origine.
    
    % TODO inserer enonce ici

    Il faut aussi ajouter que les maisons sont supposées être sur une ligne. La
    question \og{}qui boit de l'eau\fg{} doit être comprise comme \og{}sachant que
    quelqu'un boit de leau, qui est-ce?\fg{} (sinon on peut trouver une solution
    où personne ne boit de l'eau). De même la question \og{}qui possède le zèbre\fg{} 
    doit être comprise comme \og{}sachant que quelqu'un possède le zèbre, qui est-ce?\fg{}.
    Si on sait que quelqu'un boit de l'eau et que quelqu'un possède un zèbre, on peut 
    en fait déterminer qui vit où, la couleur de sa maison, sa nationalité, ce qu'il 
    boit et fume et son animal de compagnie.

    Modéliser ce problème comme un problème de satisfaction de contraintes.

    Quelle est la taille de l'espace de recherche?

    Peut-on reformuler certaines contraintes pour diminuer l'espace de recherche?
\end{td-exo}


% ----- Solutions exo 1 ----- %
\iftoggle{showsolutions}{
	\begin{td-sol}[]\ %
		On a 25 variables représentant les caractéristiques de chaque maison.
        Plus précisément, on aura 5 variables représentant chacune la
        \begin{enumerate}
            \item nationalité,
            \item la couleur,
            \item la boisson,
            \item ce qui est fumé,
            \item l'animal de compagnie.
        \end{enumerate}
        des 5 maisons. Pour différencier ces maisons, on leur fixe une position
        prédéfinie de 1 à 5 (ie.\ on attribue un ordre aux maisons dont on pourra se servir
        plus tard).

        Pour ce qui est des domaines des variables, ce sont celles énoncées dans les contraintes.
        5 valeurs par domaine, 5 domaines, tous différent.

        Une contrainte est représentée comme suit:
        \begin{enumerate}
            \item \(C_2\): l'anglais habite dans la maison rouge:
            \begin{equation*}
                C = \left\{ c_1 \langle N_1, C_1\rangle = \left\{(E, R), (J, \not R), (N, \not R),\cdots\right\}\right\}
            \end{equation*}
            donc 17 tuples de contraintes :D. Pour la maison 1 et encore une fois pour chaque autre maison, donc 85 en tout, par contrainte.
        \end{enumerate}
	\end{td-sol}
}{}
