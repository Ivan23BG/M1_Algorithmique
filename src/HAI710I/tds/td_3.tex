\begin{td-exo}[Le puzzle du zèbre]
    Le puzzle du zèbre est un jeu logique bien connu, atribué à Albert Einstein
    ou à Lewis Caroll, sans certitude que l'inventeur soit l'un des deux. Il
    existe plusieurs variantes de ce jeu, voici l'énoncé d'origine.
    
    % TODO inserer enonce ici

    Il faut aussi ajouter que les maisons sont supposées être sur une ligne. La
    question \og{}qui boit de l'eau\fg{} doit être comprise comme \og{}sachant que
    quelqu'un boit de leau, qui est-ce?\fg{} (sinon on peut trouver une solution
    où personne ne boit de l'eau). De même la question \og{}qui possède le zèbre\fg{} 
    doit être comprise comme \og{}sachant que quelqu'un possède le zèbre, qui est-ce?\fg{}.
    Si on sait que quelqu'un boit de l'eau et que quelqu'un possède un zèbre, on peut 
    en fait déterminer qui vit où, la couleur de sa maison, sa nationalité, ce qu'il 
    boit et fume et son animal de compagnie.

    Modéliser ce problème comme un problème de satisfaction de contraintes.

    Quelle est la taille de l'espace de recherche?

    Peut-on reformuler certaines contraintes pour diminuer l'espace de recherche?
\end{td-exo}