% ----- Consignes exo 1 ----- %
\begin{td-exo}[] % 1
	Répondez aux questions suivantes:
	\begin{itemize}
		\item Supposons qu’un adversaire sache qu’un message, chiffré avec le chiffrement par décalage, 
		est soit \defemph{aaa}, \defemph{aab}, \defemph{aac}, \defemph{abc}, \defemph{acd} soit \defemph{abd}. 
		En examinant le texte chiffré obtenu, est-il (toujours) en mesure de déterminer le message d’origine?
		Expliquez votre réponse: fournissez une stratégie si la réponse est correcte ou 
		un contre-exemple (une clé) si elle ne l’est pas.

		\item Considérez le point précédent, mais cette fois-ci, 
		le message est chiffré avec le chiffrement de Vigenère. 
		Supposons d’abord que la période soit 2, puis que la période soit 3.
		Est-il (toujours) en mesure de déterminer le message d’origine?

		\item Cette fois-ci, les messages possibles, chiffrés avec le chiffrement de Vigenère de période 2,
		sont soit \defemph{abcd}, \defemph{aabc} soit \defemph{bccd}. 
		L’adversaire est-il capable de déterminer le texte en clair à partir du texte chiffré? 
		Pourquoi?
	\end{itemize}
\end{td-exo}

% ----- Solutions exo 1 ----- %
\iftoggle{showsolutions}{
	\begin{td-sol}[]\ %
		\begin{itemize}
			\item Oui, il est toujours en mesure de déterminer le message d’origine.
			En effet, il suffit de décaler toutes les lettres du texte chiffré jusqu'à obtenir un des messages possibles.
			Comme les messages possibles ne peuvent pas être obtenus par décalage les uns des autres,
			il y a une et une seule possibilité.

			\item Non, il n’est pas toujours en mesure de déterminer le message d’origine. 
			Si 2 messages partagent les mêmes lettres aux positions 1 et 3 (pour la période 2)
			alors ce n'est pas possible.

			\item Oui, facile, regarder les écarts entre les paires.
		\end{itemize}
	\end{td-sol}
}{}


% ----- Consignes exo 2 ----- %
\begin{td-exo}[] % 2
	Nous chiffrons le message abacbc avec le chiffrement de Vigenère et une clé de période 2. 
	Trouvez une clé telle que le texte chiffré contienne exactement 4 fois le caractère c. 
	Existe-t-il une clé pour laquelle le caractère c apparaît plus de 4 fois dans le texte chiffré?
	Et pour les autres caractères?
\end{td-exo}

% ----- Solutions exo 2 ----- %
\iftoggle{showsolutions}{
	\begin{td-sol}[]\ %
		Oui, pour obtenir exactement 4 fois un caractère donné, on peut utiliser le chiffrement suivant:
		\begin{itemize}
			\item Pour \(c\): la clé est \((2,0)\)
			\item Pour toute autre lettre, la clé est celle du \(c\) décalée de la différence entre \(c\) et la lettre voulue.
		\end{itemize}
		
		Ce n'est pas possible d'avoir plus de 4 fois un caractère dans le texte chiffré.
	\end{td-sol}
}{}


% ----- Consignes exo 3 ----- %
\begin{td-exo}[] % 3
	Montrer que les chiffrements par décalage et par substitution mono-alphabétique
	ne sont pas inconditionnellement sûrs.
\end{td-exo}

% ----- Solutions exo 3 ----- %
\iftoggle{showsolutions}{
	\tdsolspec{
		Le mot \(aaa\) a une probabilité non nulle d'être encodée en \(bbb\) avec le chiffrement par décalage
		ou par substitution mono-alphabétique. Ce n'est en revanche pas le cas du mot \(aab\). Donc
		les deux chiffrements ne sont pas inconditionnellement sûrs.
	}
}{}


% ----- Consignes exo 4 ----- %
\begin{td-exo}[] % 4
	A partir de la définition de la sécurité inconditionnelle que nous avons donnée, 
	prouver la proposition vue pendant le cours:
	\begin{proposition}
		Un schéma de chiffrement (\(\gen\), \(\enc\), \(\dec\)) avec un espace de messages \(M\) est
		inconditionnellement sûr si et seulement si, pour chaque
		\(m, m' \in M\) et chaque \(c \in C\), on a
		\begin{equation*}
			\bb P(\enc_K (m) = c) = \bb P(\enc_K (m') = c).
		\end{equation*}
	\end{proposition}
\end{td-exo}


% ----- Solutions exo 4 ----- %
\iftoggle{showsolutions}{
	\begin{td-sol}[]\ %
		Exercice solution
	\end{td-sol}
}{}


% ----- Consignes exo 5 ----- %
\begin{td-exo}[] % 5

\end{td-exo}

% ----- Solutions exo 2 ----- %
\iftoggle{showsolutions}{
	\begin{td-sol}[]\ %
		Exercice solution
	\end{td-sol}
}{}


% ----- Consignes exo 2 ----- %
\begin{td-exo}[Optional title 2] % 2
Exercise 2 content
\end{td-exo}

% ----- Solutions exo 2 ----- %
\iftoggle{showsolutions}{
	\begin{td-sol}[]\ %
		Exercice solution
	\end{td-sol}
}{}


% ----- Consignes exo 2 ----- %
\begin{td-exo}[Optional title 2] % 2
Exercise 2 content
\end{td-exo}

% ----- Solutions exo 2 ----- %
\iftoggle{showsolutions}{
	\begin{td-sol}[]\ %
		Exercice solution
	\end{td-sol}
}{}
