% ----- Consignes exo 1 ----- %
\begin{td-exo}[] % 1
	\, % TODO add content
\end{td-exo}

% ----- Solutions exo 1 ----- %
\iftoggle{showsolutions}{
	\begin{td-sol}[]\ %
		\begin{itemize}
			\item On a la chaine suivante:
			\begin{equation*}
				\begin{aligned}
					&1000\\
					&1100\\
					&0110\\
					&1011\\
					&1101\\
					&0110\\
					&1011\\
					&1101\\
					&0110
				\end{aligned}
			\end{equation*}
			Où on peut remarquer que la période est de 3. On ne parcourt pas les 15 combinaisons
			donc elle n'est pas de longueur maximale.
			\item La longueur est de 15 et elle est maximale.
			% TODO finir le 3
		\end{itemize}
	\end{td-sol}
}{}


% ----- Consignes exo 2 ----- %
\begin{td-exo}[] % 2
	% TODO add content
\end{td-exo}

% ----- Solutions exo 2 ----- %
\iftoggle{showsolutions}{
	\begin{td-sol}[]\ %
		\begin{itemize}
			\item Avec \(g(S_{n-1},\ldots,S_0) = S_0\wedge S_1\), on a
			\begin{equation*}
				\bb P(g(\cdots) = 0) = \frac34,\quad \bb P(g(\cdots) = 1) = \frac14
			\end{equation*}
			ce qui n'est pas uniforme donc ce \(g\) n'est pas un 
			bon générateur pseudo-aléatoire.

			\item Avec \(g(S_{n-1},\ldots,S_0) = (S_0\wedge S_1) \oplus S_2\), on a
			\begin{equation*}
				\begin{aligned}
					\bb P(g = 0) = \bb P(S_2 = 0)\cdot \frac34 + \bb P(S_2 = 1)\cdot \frac14 =\frac12\\
					\bb P(g = 1) = \frac12
				\end{aligned}
			\end{equation*}
			ce qui n'est pas uniforme donc ce \(g\) n'est pas un 
			bon générateur pseudo-aléatoire.
		\end{itemize}
	\end{td-sol}
}{}


% ----- Consignes exo 3 ----- %
\begin{td-exo}[] % 3
	% TODO add content
\end{td-exo}

% ----- Solutions exo 3 ----- %
\iftoggle{showsolutions}{
	\begin{td-sol}[]\ %
		\begin{itemize}
			\item A l'etape 1, on a:
			\begin{equation*}
				\begin{aligned}
					\bb P(S_0\ff2=0 \wedge S_0\ff1\neq2)
					&= \frac1{256} \cdot \left(1-\frac{1}{255}\right)\\
					&= \frac{254}{256\cdot255}
				\end{aligned}
			\end{equation*}

			\item A l'étape 2, on a:
			\begin{equation*}
				\begin{aligned}
					&S_0\ff2=0, S_0\ff1\neq2\\
					&i=1, J=S_0\ff1 = X\\
					&S_1\ff1=S_0\ff J = S_0\ff X\\
					&S_1\ff X = S_0\ff1 = X
				\end{aligned}
			\end{equation*}

			\item A l'étape 3, on a:
			\begin{equation*}
				\begin{aligned}
					i=2, J 
					&= X + S_1\ff2\\
					&= X + S_0\ff2\\
					&= X + 0\\
					&= X
				\end{aligned}
			\end{equation*}
			et
			\begin{equation*}
				\begin{aligned}
					&S_2\ff2 = S_1\ff X = X\\
					&S_2\ff X = S_1\ff2 = 0\\
					&t = S_2\ff1 + S_2\ff J = X + 0 = X\\
					&Y = S_2\ff t = S_2\ff X = 0
				\end{aligned}
			\end{equation*}

			\item On en déduit que le 2e octet de la clé est 0. 
			De plus, on a une probabilité uniforme sur cet octet.
			
			\item On en conclut que le 2e octet est biaisé
			avec probabilité \(\frac{1}{128} > \frac{1}{256}\).
		\end{itemize}
	\end{td-sol}
}{}
