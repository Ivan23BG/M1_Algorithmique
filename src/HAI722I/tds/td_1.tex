\subsection{Convexité: ensembles et fonctions}\label{subsec:ss_1}

% ----- Consignes exo 1 ----- %
\begin{td-exo}[Convexité] % 1
Exercise 1 content
\end{td-exo}

% ----- Solutions exo 1 ----- %
\iftoggle{showsolutions}{
	\begin{td-sol}[]\ %
		Exercice solution
	\end{td-sol}
}{}


% ----- Consignes exo 2 ----- %
\begin{td-exo}[Combinaison convexe] % 2
Exercise 2 content
\end{td-exo}

% ----- Solutions exo 2 ----- %
\iftoggle{showsolutions}{
	\begin{td-sol}[]\ %
		Exercice solution
	\end{td-sol}
}{}


% ----- Consignes exo 3 ----- %
\begin{td-exo}[Ensembles convexe] % 3
	Montrer qu'étant donné un sous-ensemble convexe \(C\) et deux réels positifs \(\alpha\) et \(\beta\)
	alors on a
	\begin{equation*}
		\alpha C + \beta C = (\alpha + \beta) C.
	\end{equation*}
\end{td-exo}

% ----- Solutions exo 2 ----- %
\iftoggle{showsolutions}{
	\begin{td-sol}[]\ %
		Commencons par montrer l'inclusion \(\left(\alpha + \beta\right) C \subset \alpha C + \beta C\).

		Soit \(x \in \left(\alpha + \beta\right) C\). Alors, il existe \(x_0\in C\) tel que
		\begin{equation*}
			x = \left(\alpha + \beta\right) x_0 = \alpha x_0 + \beta x_0.
		\end{equation*}
		Donc \(x \in \alpha C + \beta C\).

		Montrons maintenant l'inclusion \(\alpha C + \beta C \subset \left(\alpha + \beta\right) C\).

		Soit \(x \in \alpha C + \beta C\). Alors, il existe \(x_1, x_2 \in C\) tels que
		\begin{equation*}
			x = \alpha x_1 + \beta x_2 = \left(\alpha + \beta\right) \left(\frac{\alpha}{\alpha + \beta} x_1 + \frac{\beta}{\alpha + \beta} x_2\right).
		\end{equation*}
	\end{td-sol}
}{}


% ----- Consignes exo 4 ----- %
\begin{td-exo}[Ensembles convexes] % 4
	Soit \(S\sub\bb R^n\) vérifiant la propriété de \defemph{demi-somme} suivante:
	\begin{equation*}
		\forall x,y \in S,\quad \frac{x+y}{2} \in S.
	\end{equation*}

	\begin{enumerate}
		\item \(S\) est-il convexe?

		\item Même question si on suppose que \(S\) est fermé.
	\end{enumerate}
\end{td-exo}

% ----- Solutions exo 4 ----- %
\iftoggle{showsolutions}{
	\begin{td-sol}[]\ %
		\begin{enumerate}
			\item Non. Par exemple, le sous-ensemble \(S\) suivant:
			\begin{equation*}
				S = \left\{ x \in \ff{0,1} \mid x = \sum_{i=I} \frac{1}{2^i} \right\} = \left\{ 0, \frac{1}{2}, \frac{1}{4}, \frac{3}{4}, \ldots \right\}
			\end{equation*}
			vérifie la propriété de demi-somme mais n'est pas convexe, car par exemple \(\sqrt{2}/2 \in \ff{0,1} \notin S\).
		\end{enumerate}
	\end{td-sol}
}{}


% ----- Consignes exo 5 ----- %
\begin{td-exo}[Ensembles convexes] % 5
Exercise 2 content
\end{td-exo}

% ----- Solutions exo 5 ----- %
\iftoggle{showsolutions}{
	\begin{td-sol}[]\ %
		Exercice solution
	\end{td-sol}
}{}


% ----- Consignes exo 6 ----- %
\begin{td-exo}[Ensembles convexes] % 6
Exercise 2 content
\end{td-exo}

% ----- Solutions exo 6 ----- %
\iftoggle{showsolutions}{
	\begin{td-sol}[]\ %
		Exercice solution
	\end{td-sol}
}{}


% ----- Consignes exo 7 ----- %
\begin{td-exo}[Fonction convexe]\ % 7
	\begin{enumerate}
		\item Est-ce qu'une combinaison linéaire à coefficients positifs de fonctions convexes est convexe?

		\item Est-ce que le produit de deux fonctions convexes est convexe?

		\item Si \(f_1\) et \(f_2\) sont deux fonctions convexes, est-ce que \(\max\left(f_1,f_2\right)\) est convexe?

		\item Montrer que la fonction \(f\ \colon\ x\mapsto x^2\) est une fonction convexe sur \(\bb R\).
	\end{enumerate}
\end{td-exo}

% ----- Solutions exo 7 ----- %
\iftoggle{showsolutions}{
	\begin{td-sol}[]\ %
		\begin{enumerate}
			\item Oui. On pose \(g(x) = \sum_{i\in I} \alpha_i f_i(x)\). Alors
			\begin{equation*}
				\begin{aligned}
					g(\lambda x + (1-\lambda) y) 
					& = \sum_{i\in I} \alpha_i f_i(\lambda x + (1-\lambda) y) \\
					& \leq \sum_{i\in I} \alpha_i \left(\lambda f_i(x) + (1-\lambda) f_i(y)\right) \\
					& = \lambda g(x) + (1-\lambda) g(y).
				\end{aligned}
			\end{equation*}

			\item Non. Par exemple, \(f_1(x) = x\) et \(f_2(x) = x^2\) sont convexes mais \(f_1(x) f_2(x) = x^3\) n'est pas convexe.

			\item Oui. On pose \(g(x) = \max\left(f_1(x), f_2(x)\right)\). Alors
			\begin{equation*}
				\begin{aligned}
					g(\lambda x + (1-\lambda) y) 
					& = \max\left(f_1(\lambda x + (1-\lambda) y), f_2(\lambda x + (1-\lambda) y)\right) \\
					& \leq\sim \max\left(\lambda f_1(x) + (1-\lambda) f_1(y), \lambda f_2(x) + (1-\lambda) f_2(y)\right) \\
					& \leq \lambda \max\left(f_1(x), f_2(x)\right) + (1-\lambda) \max\left(f_1(y), f_2(y)\right) \\
					& = \lambda g(x) + (1-\lambda) g(y).
				\end{aligned}
			\end{equation*}

			\item Soit \(x,y \in \bb R\) et \(\lambda \in [0,1]\). Alors
			\begin{equation*}
				\begin{aligned}
					f(\lambda x + (1-\lambda) y) 
					& = (\lambda x + (1-\lambda) y)^2 \\
					& = \lambda^2 x^2 + (1-\lambda)^2 y^2 + 2\lambda(1-\lambda) xy \\
					& \leq \lambda^2 x^2 + (1-\lambda)^2 y^2 + \lambda(1-\lambda)(x^2 + y^2) \\
					& = \lambda x^2 + (1-\lambda) y^2 \\
					& = \lambda f(x) + (1-\lambda) f(y).
				\end{aligned}
			\end{equation*}
		\end{enumerate}
	\end{td-sol}
}{}


% ----- Consignes exo 2 ----- %
\begin{td-exo}[Optional title 2] % 2
Exercise 2 content
\end{td-exo}

% ----- Solutions exo 2 ----- %
\iftoggle{showsolutions}{
	\begin{td-sol}[]\ %
		Exercice solution
	\end{td-sol}
}{}