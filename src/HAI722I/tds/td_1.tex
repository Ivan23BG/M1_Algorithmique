\subsection{Convexité: ensembles et fonctions}\label{subsec:ss_1}

% ----- Consignes exo 1 ----- %
\begin{td-exo}[Convexité]\, % 1
	\begin{enumerate}
		\item Soit une famille (éventuellement infinie) d'inégalités linéaires
		\(a_i^T x \leq b_i,i\in I\). Soit \(C\) son ensemble de solutions.
		Montrer que \(C\) est convexe.

		\item Montrer que la boule fermée \(\mathsf{B}(a,r)\) est convexe
		pour tout \(a\in\bb R^n\) et \(r\in \bb R^+\).

		\item Soit \(S\subseteq\bb R^n\) et soit \(W\) l'ensemble de toutes les
		combinaisons convexes de points de \(S\). Montrer que \(W\) est convexe.

		\item Soit \(C\) un convexe. Montrer que 
		\begin{equation*}
			\bigcup_{0\leq\lambda\leq 1}\lambda C
		\end{equation*}
		est convexe.

		\item Une matrice \(A=(a_{ij})\) de dimension \(n\times n\) est bistochastique si 
		elle satisfait
		\begin{equation*}
			\begin{aligned}
				\forall i\in\{1,\ldots,n\}, & \sum_{j=1}^n a_{ij} = 1, \\
				\forall j\in\{1,\ldots,n\}, & \sum_{i=1}^n a_{ij} = 1, \\
				\forall (i,j)\in{\{1,\ldots,n\}}^2, & a_{ij}\geq 0.
			\end{aligned}
		\end{equation*}
		Une matrice de permutation \(P\) est une matrice bistochastique à valeurs 
		entières, c'est-à-dire que dans chaque ligne de \(P\) il y a un et un seul élément égal à 1,
		et les autres sont nuls. De même pour chaque colonne.
		\begin{enumerate}
			\item Montrer que pour toute matrice bistochastique \(A\), il existe
			une matrice de permutation \(P\) de même dimension telle que \(p_{ij}=0\)
			si \(a_{ij}=0\).

			\item Est-ce qu'une combinaison convexe de matrices de permutation est 
			une matrice bistochastique?

			\item Montrer que toute matrice bistochastique \(A\) est une combinaison convexe
			de matrices de permutation.

			\item Trouver la combinaison convexe pour la matrice \(A\) suivante:
			\begin{equation*}
				A = \begin{pmatrix}
					0.15 & 0.37 & 0 & 0.48 \\
					0.02 & 0.15 & 0.67 & 0.16 \\
					0.46 & 0.02 & 0.16 & 0.36 \\
					0.37 & 0.46 & 0.17 & 0
				\end{pmatrix}.
			\end{equation*}
		\end{enumerate}

		\item Soient maintenant \(C_1\) et \(C_2\) deux convexes disjoints et
		\begin{equation*}
			D_1 = \bigcup_{0\leq\lambda\leq 1}\lambda C_1, \quad i=1,2.
		\end{equation*}
		Montrer que l'un des deux convexes \(C_1\cap D_2\) ou \(C_2\cap D_1\) est vide.
	\end{enumerate}
\end{td-exo}

% ----- Solutions exo 1 ----- %
\iftoggle{showsolutions}{
	\begin{td-sol}[]\ %
		A remplir %TODO solve exercise 1
	\end{td-sol}
}{}


% ----- Consignes exo 2 ----- %
\begin{td-exo}[Combinaison convexe]\, % 2
	\begin{enumerate}
		\item Rappeler la définition d'une combinaison convexe.
		\item Est-ce que le point \(A\) de coordonnées \((1,1,1)\) est une combinaison convexe
		des points \((2,2,0), (0,0,3), (0,0,0)\)?
		\item Déterminer si le point de coordonnées \((0,7)\) est une combinaison convexe
		des points \((3,6),(-6,9),(2,1),(-1,1)\).
		\item Déterminer gra
	\end{enumerate}
\end{td-exo}

% ----- Solutions exo 2 ----- %
\iftoggle{showsolutions}{
	\begin{td-sol}[]\ %
		Exercice solution
	\end{td-sol}
}{}


% ----- Consignes exo 3 ----- %
\begin{td-exo}[Ensembles convexe] % 3
	Montrer qu'étant donné un sous-ensemble convexe \(C\) et deux réels positifs \(\alpha\) et \(\beta\)
	alors on a
	\begin{equation*}
		\alpha C + \beta C = (\alpha + \beta) C.
	\end{equation*}
\end{td-exo}

% ----- Solutions exo 2 ----- %
\iftoggle{showsolutions}{
	\begin{td-sol}[]\ %
		Commencons par montrer l'inclusion \(\left(\alpha + \beta\right) C \subset \alpha C + \beta C\).

		Soit \(x \in \left(\alpha + \beta\right) C\). Alors, il existe \(x_0\in C\) tel que
		\begin{equation*}
			x = \left(\alpha + \beta\right) x_0 = \alpha x_0 + \beta x_0.
		\end{equation*}
		Donc \(x \in \alpha C + \beta C\).

		Montrons maintenant l'inclusion \(\alpha C + \beta C \subset \left(\alpha + \beta\right) C\).

		Soit \(x \in \alpha C + \beta C\). Alors, il existe \(x_1, x_2 \in C\) tels que
		\begin{equation*}
			x = \alpha x_1 + \beta x_2 = \left(\alpha + \beta\right) \left(\frac{\alpha}{\alpha + \beta} x_1 + \frac{\beta}{\alpha + \beta} x_2\right).
		\end{equation*}
	\end{td-sol}
}{}


% ----- Consignes exo 4 ----- %
\begin{td-exo}[Ensembles convexes] % 4
	Soit \(S\sub\bb R^n\) vérifiant la propriété de \defemph{demi-somme} suivante:
	\begin{equation*}
		\forall x,y \in S,\quad \frac{x+y}{2} \in S.
	\end{equation*}

	\begin{enumerate}
		\item \(S\) est-il convexe?

		\item Même question si on suppose que \(S\) est fermé.
	\end{enumerate}
\end{td-exo}

% ----- Solutions exo 4 ----- %
\iftoggle{showsolutions}{
	\begin{td-sol}[]\ %
		\begin{enumerate}
			\item Non. Par exemple, le sous-ensemble \(S\) suivant:
			\begin{equation*}
				S = \left\{ x \in \ff{0,1} \mid x = \sum_{i=I} \frac{1}{2^i} \right\} = \left\{ 0, \frac{1}{2}, \frac{1}{4}, \frac{3}{4}, \ldots \right\}
			\end{equation*}
			vérifie la propriété de demi-somme mais n'est pas convexe, car par exemple \(\sqrt{2}/2 \in \ff{0,1} \notin S\).
		\end{enumerate}
	\end{td-sol}
}{}


% ----- Consignes exo 5 ----- %
\begin{td-exo}[Ensembles convexes] % 5
Exercise 2 content
\end{td-exo}

% ----- Solutions exo 5 ----- %
\iftoggle{showsolutions}{
	\begin{td-sol}[]\ %
		Exercice solution
	\end{td-sol}
}{}


% ----- Consignes exo 6 ----- %
\begin{td-exo}[Ensembles convexes] % 6
Exercise 2 content
\end{td-exo}

% ----- Solutions exo 6 ----- %
\iftoggle{showsolutions}{
	\begin{td-sol}[]\ %
		Exercice solution
	\end{td-sol}
}{}


% ----- Consignes exo 7 ----- %
\begin{td-exo}[Fonction convexe]\ % 7
	\begin{enumerate}
		\item Est-ce qu'une combinaison linéaire à coefficients positifs de fonctions convexes est convexe?

		\item Est-ce que le produit de deux fonctions convexes est convexe?

		\item Si \(f_1\) et \(f_2\) sont deux fonctions convexes, est-ce que \(\max\left(f_1,f_2\right)\) est convexe?

		\item Montrer que la fonction \(f\ \colon\ x\mapsto x^2\) est une fonction convexe sur \(\bb R\).
	\end{enumerate}
\end{td-exo}

% ----- Solutions exo 7 ----- %
\iftoggle{showsolutions}{
	\begin{td-sol}[]\ %
		\begin{enumerate}
			\item Oui. On pose \(g(x) = \sum_{i\in I} \alpha_i f_i(x)\). Alors
			\begin{equation*}
				\begin{aligned}
					g(\lambda x + (1-\lambda) y) 
					& = \sum_{i\in I} \alpha_i f_i(\lambda x + (1-\lambda) y) \\
					& \leq \sum_{i\in I} \alpha_i \left(\lambda f_i(x) + (1-\lambda) f_i(y)\right) \\
					& = \lambda g(x) + (1-\lambda) g(y).
				\end{aligned}
			\end{equation*}

			\item Non. Par exemple, \(f_1(x) = x\) et \(f_2(x) = x^2\) sont convexes mais \(f_1(x) f_2(x) = x^3\) n'est pas convexe.

			\item Oui. On pose \(g(x) = \max\left(f_1(x), f_2(x)\right)\). Alors
			\begin{equation*}
				\begin{aligned}
					g(\lambda x + (1-\lambda) y) 
					& = \max\left(f_1(\lambda x + (1-\lambda) y), f_2(\lambda x + (1-\lambda) y)\right) \\
					& \leq \max\left(\lambda f_1(x) + (1-\lambda) f_1(y), \lambda f_2(x) + (1-\lambda) f_2(y)\right) \\
					& \leq \lambda \max\left(f_1(x), f_2(x)\right) + (1-\lambda) \max\left(f_1(y), f_2(y)\right) \\
					& = \lambda g(x) + (1-\lambda) g(y).
				\end{aligned}
			\end{equation*}

			\item Soit \(x,y \in \bb R\) et \(\lambda \in [0,1]\). Alors
			\begin{equation*}
				\begin{aligned}
					f(\lambda x + (1-\lambda) y) 
					& = {(\lambda x + (1-\lambda) y)}^2 \\
					& = \lambda^2 x^2 + {(1-\lambda)}^2 y^2 + 2\lambda(1-\lambda) xy \\
					& \iff \lambda^2 x^2 + {(1-\lambda)}^2 y^2 + 2\lambda(1-\lambda) xy - \lambda x^2 - (1-\lambda) y^2 \leq 0 \\
					& \iff \lambda (1-\lambda)\left(\frac{\lambda}{1-\lambda} x^2 + \frac{1-\lambda}{\lambda} y^2 + 2xy - \frac{x^2}{1-\lambda} - \frac{y^2}{\lambda}\right) \leq 0 \\
					& \iff \lambda (1-\lambda)\left(-{\left(x - y\right)}^2\right) \leq 0.
				\end{aligned}
			\end{equation*}
			Or tous les termes sont positifs sauf le dernier. Donc l'inégalité est vérifiée.
		\end{enumerate}
	\end{td-sol}
}{}


% ----- Consignes exo 8 ----- %
\begin{td-exo}[Fonction convexe] % 8
	Soit \(f\ \colon\ \bb R^n \to \bb R\) une fonction continue telle que
	\begin{equation*}
		\forall (x,y) \in \bb R^2, \quad f\left(\frac{x+y}{2}\right) \leq \frac{f(x) + f(y)}{2}.
	\end{equation*}
	Prouver que \(f\) est convexe.\\
	Indication: Montrer par récurrence que sur \(\n\geq 2\), on a
	\begin{equation*}
		\forall (x,y) \in \bb R^2, \forall p\in\{0,1,\ldots,2^n\}, \quad f \left( \frac p{2^n}x + \left(1 - \frac p{2^n}\right)y\right) \leq \frac p{2^n} f(x) + \left(1 - \frac p{2^n}\right) f(y).
	\end{equation*}
\end{td-exo}

% ----- Solutions exo 2 ----- %
\iftoggle{showsolutions}{
	\begin{td-sol}[]\ %
		
	\end{td-sol}
}{}


% ----- Consignes exo 2 ----- %
\begin{td-exo}[] % 2

\end{td-exo}

% ----- Solutions exo 2 ----- %
\iftoggle{showsolutions}{
	\begin{td-sol}[]\ %
		
	\end{td-sol}
}{}


% ----- Consignes exo 2 ----- %
\begin{td-exo}[] % 2

\end{td-exo}

% ----- Solutions exo 2 ----- %
\iftoggle{showsolutions}{
	\begin{td-sol}[]\ %
		
	\end{td-sol}
}{}


% ----- Consignes exo 12 ----- %
\begin{td-exo}[Forme standard et forme canonique] % 12
	Dans cet exercice vous devez mettre les programmes suivants sous forme standard et donner
	également la forme matricielle.
	\begin{enumerate}
		\item
		\begin{equation*}
			\begin{cases}
				\max z = x_1 + x_2 \\
				x_1 + 5 x_2 \leq 5 \\
				2x_1 + x_2 \leq 4 \\
				x_1 \geq 0,\quad i=1,2.
			\end{cases}
		\end{equation*}

		\item
		\begin{equation*}
			\begin{cases}
				\max z = 80 x_1 + 60 x_2 \\
				0.2x_1 + 0.32 x_2 \leq 0.25\\
				x_1 + x_2 = 1 \\
				x_1 \geq 0,\quad i=1,2.
			\end{cases}
		\end{equation*}

		\item Réécrire le programme précédent dans le cas où la fonction objectif
		est la minimisation.

		\item
		\begin{equation*}
			\begin{cases}
				\max z = 5x_1 + 2x_2 \\
				6x_1 + x_2 \geq 6 \\
				4x_1 + 3x_2  \geq 12 \\
				x_1 + 2x_2 \geq 4 \\
				x_1 \geq 0,\quad i=1,2.
			\end{cases}
		\end{equation*}
	\end{enumerate}
\end{td-exo}

% ----- Solutions exo 12 ----- %
\iftoggle{showsolutions}{
	\begin{td-sol}[]\ %
		\begin{enumerate}
			\item La forme standard est
			\begin{equation*}
				\begin{cases}
					\max z = x_1 + x_2 + 0\cdot x_3 + 0\cdot x_4 \\
					x_1 + 5 x_2 + x_3 = 5 \\
					2x_1 + x_2 + x_4 = 4 \\
					x_1, x_2, x_3, x_4 \geq 0.
				\end{cases}
			\end{equation*}
			La forme matricielle est
			\begin{equation*}
				\begin{cases}
					\max z = \begin{pmatrix} 1 & 1 & 0 & 0 \end{pmatrix} \begin{pmatrix} x_1 \\ x_2 \\ x_3 \\ x_4 \end{pmatrix} \\
					\begin{pmatrix}
						1 & 5 & 1 & 0 \\
						2 & 1 & 0 & 1
					\end{pmatrix}
					\begin{pmatrix} x_1 \\ x_2 \\ x_3 \\ x_4 \end{pmatrix}
					=
					\begin{pmatrix} 5 \\ 4 \end{pmatrix} \\
					\begin{pmatrix} x_1 \\ x_2 \\ x_3 \\ x_4 \end{pmatrix} \geq 0.
				\end{cases}
			\end{equation*}

			\item La forme standard est
			\begin{equation*}
				\begin{cases}
					\max z = 80 x_1 + 60 x_2 + 0\cdot x_3 - M \cdot x_4\\
					0.2x_1 + 0.32 x_2 + x_3 = 0.25\\
					x_1 + x_2 + x_4 = 1 \\
					x_1, x_2, x_3, x_4 \geq 0.
				\end{cases}
			\end{equation*}
			La forme matricielle est
			\begin{equation*}
				\begin{cases}
					\max z = \begin{pmatrix} 80 & 60 & 0 & -M \end{pmatrix} \begin{pmatrix} x_1 \\ x_2 \\ x_3 \\ x_4 \end{pmatrix} \\
					\begin{pmatrix}
						0.2 & 0.32 & 1 & 0 \\
						1 & 1 & 0 & 1
					\end{pmatrix}
					\begin{pmatrix} x_1 \\ x_2 \\ x_3 \\ x_4 \end{pmatrix}
					=
					\begin{pmatrix} 0.25 \\ 1 \end{pmatrix} \\
					\begin{pmatrix} x_1 \\ x_2 \\ x_3 \\ x_4 \end{pmatrix} \geq 0.
				\end{cases}
			\end{equation*}

			\item La forme standard est
			\begin{equation*}
				\begin{cases}
					\min z = -80 x_1 - 60 x_2 + 0\cdot x_3 + M \cdot x_4\\
					0.2x_1 + 0.32 x_2 + x_3 = 0.25\\
					x_1 + x_2 + x_4 = 1 \\
					x_1, x_2, x_3, x_4 \geq 0.
				\end{cases}
			\end{equation*}
			La forme matricielle est
			\begin{equation*}
				\begin{cases}
					\min z = \begin{pmatrix} -80 & -60 & 0 & M \end{pmatrix} \begin{pmatrix} x_1 \\ x_2 \\ x_3 \\ x_4 \end{pmatrix} \\
					\begin{pmatrix}
						0.2 & 0.32 & 1 & 0 \\
						1 & 1 & 0 & 1
					\end{pmatrix}
					\begin{pmatrix} x_1 \\ x_2 \\ x_3 \\ x_4 \end{pmatrix}
					=
					\begin{pmatrix} 0.25 \\ 1 \end{pmatrix} \\
					\begin{pmatrix} x_1 \\ x_2 \\ x_3 \\ x_4 \end{pmatrix} \geq 0.
				\end{cases}
			\end{equation*}

			\item La forme standard est
			\begin{equation*}
				\begin{cases}
					\max z = 5x_1 + 2x_2 + 0\cdot x_3 + 0\cdot x_4 + 0\cdot x_5 + M\cdot (x_6 + x_7 + x_8)\\
					6x_1 + x_2 - x_3 + x_6= 6 \\
					4x_1 + 3x_2 - x_4 + x_7 = 12 \\
					x_1 + 2x_2 - x_5 + x_8 = 4 \\
					x_1, x_2, x_3, x_4, x_5, x_6, x_7, x_8 \geq 0.
				\end{cases}
			\end{equation*}
			La forme matricielle est
			\begin{equation*}
				\begin{cases}
					\max z = \begin{pmatrix} 5 & 2 & 0 & 0 & 0 & M & M & M \end{pmatrix} \begin{pmatrix} x_1 \\ x_2 \\ x_3 \\ x_4 \\ x_5 \\ x_6 \\ x_7 \\ x_8 \end{pmatrix} \\
					\begin{pmatrix}
						6 & 1 & -1 & 0 & 0 & 1 & 0 & 0 \\
						4 & 3 & 0 & -1 & 0 & 0 & 1 & 0 \\
						1 & 2 & 0 & 0 & -1 & 0 & 0 & 1
					\end{pmatrix}
					\begin{pmatrix} x_1 \\ x_2 \\ x_3 \\ x_4 \\ x_5 \\ x_6 \\ x_7 \\ x_8 \end{pmatrix}
					=
					\begin{pmatrix} 6 \\ 12 \\ 4 \end{pmatrix} \\
					\begin{pmatrix} x_1 \\ x_2 \\ x_3 \\ x_4 \\ x_5 \\ x_6 \\ x_7 \\ x_8 \end{pmatrix} \geq 0.
				\end{cases}
			\end{equation*}
		\end{enumerate}
	\end{td-sol}
}{}


% ----- Consignes exo 2 ----- %
\begin{td-exo}[] % 2

\end{td-exo}

% ----- Solutions exo 2 ----- %
\iftoggle{showsolutions}{
	\begin{td-sol}[]\ %
		
	\end{td-sol}
}{}


% ----- Consignes exo 2 ----- %
\begin{td-exo}[] % 2

\end{td-exo}

% ----- Solutions exo 2 ----- %
\iftoggle{showsolutions}{
	\begin{td-sol}[]\ %
		
	\end{td-sol}
}{}


% ----- Consignes exo 2 ----- %
\begin{td-exo}[] % 2

\end{td-exo}

% ----- Solutions exo 2 ----- %
\iftoggle{showsolutions}{
	\begin{td-sol}[]\ %
		
	\end{td-sol}
}{}


% ----- Consignes exo 2 ----- %
\begin{td-exo}[] % 2

\end{td-exo}

% ----- Solutions exo 2 ----- %
\iftoggle{showsolutions}{
	\begin{td-sol}[]\ %
		
	\end{td-sol}
}{}


% ----- Consignes exo 2 ----- %
\begin{td-exo}[] % 2

\end{td-exo}

% ----- Solutions exo 2 ----- %
\iftoggle{showsolutions}{
	\begin{td-sol}[]\ %
		
	\end{td-sol}
}{}


% ----- Consignes exo 2 ----- %
\begin{td-exo}[] % 2

\end{td-exo}

% ----- Solutions exo 2 ----- %
\iftoggle{showsolutions}{
	\begin{td-sol}[]\ %
		
	\end{td-sol}
}{}