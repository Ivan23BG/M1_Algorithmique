% ----- Consignes exo 1 ----- %
\begin{td-exo}[Théorie de l'incrément (TD logique 3)]\,\\ % 1
    Dans
\end{td-exo}

% ----- Solutions exo 1 ----- %
\iftoggle{showsolutions}{
	\begin{td-sol}[]\,\\ % 1
		Exercice solution
	\end{td-sol}
}{}

% ----- Consignes exo 2 ----- %
\begin{td-exo}[Théories complètes]\, % 2
    \begin{enumerate}
        \item Montrez qu'une théorie \(T\) est complète si et seulement si tous ses modèles sont élémentairement équivalents.
        
        \item Soit \(T\) un ensemble de formules quelconque.
        On note \(\theta (T)\) les conséquences de \(T\), donc 
        \begin{equation*}
            \theta(T) = \{\phi, T \vdash \phi\}.
        \end{equation*}
        Montrez que \(T\) et \(\theta(T)\) ont les mêmes théorèmes
    \end{enumerate}
\end{td-exo}

% ----- Solutions exo 2 ----- %
\iftoggle{showsolutions}{
	\begin{td-sol}[]\, % 2
		\begin{enumerate}
            \item Exercice solution
        \end{enumerate}
	\end{td-sol}
}{}

% ----- Consignes exo 3 ----- %
\begin{td-exo}[Optional title xx]\,\\ % 3
Exercise xx content
\end{td-exo}

% ----- Solutions exo 3 ----- %
\iftoggle{showsolutions}{
	\begin{td-sol}[]\,\\ % 3
		Exercice solution
	\end{td-sol}
}{}