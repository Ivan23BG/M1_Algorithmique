% ----- Consignes exo 1 ----- %
\begin{td-exo}[Rice]\,\\ % 1
    Soit \(A = \left\{x,\exists n>0,\forall y, [x\ |\ y] = y^n\right\}\)
    \begin{enumerate}
        \item Exprimez cet ensemble avec des mots en français.
        \item Peut-on écrire \(A=P_{\mathcal{C}}\) pour une certaine propriété \(\mathcal{C}\) où
        \(P_{\mathcal{C}}=\{\langle x, [x\ |\ \cdot] \in \mathcal{C}\}\)? Si oui, proposez une telle propriété,
        sinon faites une preuve.
        \item En appliquant proprement le théorème de Rice, montrez que \(A\) est indécidable.
    \end{enumerate}
\end{td-exo}

% ----- Solutions exo 1 ----- %
\iftoggle{showsolutions}{
	\begin{td-sol}[]\, % 2
		\begin{enumerate}
            \item L'ensemble des programmes qui renvoient un monome de degré \(n\).
            \item Une telle propriété existe:
            \begin{equation*}
                f\in\mathcal{C} \iff f(y)=y^n, n>0
            \end{equation*}
            \item On voit que \(A\) est non trivial car:
            \begin{equation*}
                x\mapsto x \in A
            \end{equation*}
            et
            \begin{equation*}
                x\mapsto \perp
            \end{equation*}
            donc en appliquant le théorème de Rice, on en déduit que \(A\) est indécidable.
        \end{enumerate}
	\end{td-sol}
}{}


% ----- Consignes exo 2 ----- %
\begin{td-exo}[Progressif]\,\\ % 2
    Soit \(g\) une fonction calculable totale qui ne s'annule jamais. 
    
    Soit \(A = \{x, [x\ |\ 0]\uparrow \text{ et } [x\ |\ g(x)] = 1\}\).
    On définit le programme \(p\) suivant:
    
    % insert program code here
    On définit aussi la fonction \(f\) par \(f(x) = S_1^1\langle p,x\rangle\).
    \begin{enumerate}
        \item La fonction \(f\) est-elle calculable? Est-elle totale? Justifier.
        \item Soit \(x\) tel que \([x\ |\ x]\downarrow\). Quelle fonction est calculée par \(y\mapsto [f(x)\ |\ y]\)?
        \item Soit \(x\) tel que \([x\ |\ x]\uparrow\). Quelle fonction est calculée par \(y\mapsto [f(x)\ |\ y]\)?
        \item Montrez qur \(\bb K \prec A\).
        \item Montrez que l'ensemble \(A\) n'est pas récursif.
        \item Aurait-on pu montrer que \(A\) n'est pas récursif en appliquant le Théorème de Rice?
        \item Ecrivez un programme \(q\) tel que les deux conditions suivantes soient vérifiées:
        \begin{itemize}
            \item \(\forall x,[q\ |\ \langle x,0\rangle]\downarrow \iff x\in\bb K\),
            \item \(\forall x,y,y\neq 0 \implies [q\ |\ \langle x,y\rangle] = 1\).
        \end{itemize}
        \item Montrez que \(\ol{\bb K}\prec A\).
    \end{enumerate}
\end{td-exo}

% ----- Solutions exo 2 ----- %
\iftoggle{showsolutions}{
	\begin{td-sol}[]\, % 2
        \begin{enumerate}
            \item\label{exam:s2:1} D'après le théorème SNM qu'on utilise pour construire la fonction \(f\), \(f\) est bien calculable.

            \item\label{exam:s2:2} La fonction calculée est celle qui renvoie 1 partout sauf en 0 où elle n'est pas définie.

            \item\label{exam:s2:3} La fonction calculée est \(\upmodels\).

            \item On veut mmontrer qu'il existe une fonction \(F\) calculable totale telle que
            pour tout \(x\):
            \begin{equation*}
                x\in \bb K \iff f(x)\in A
            \end{equation*}
            On a montré que \(f\) est calculable totale au~\ref{exam:s2:1} que
            \(f\) est calculable totale. On a montré au~\ref{exam:s2:2} que si \(x\in \bb K\) alors
            \(f(x)\in A\) et au~\ref{exam:s2:3} que si \(x\notin \bb K\) alors \(f(x)\notin A\).
            Donc \(f\) est bien la fonction cherchée.

            \item Si \(A\) était récursif, alors \(\bb K\) serait récursif, ce qui n'est pas le cas 
            donc \(A\) n'est pas récursif.

            \item Non car l'ensemble est défini par des propriétés sur le programme et non sur la fonction calculée.

            \item On pose \(q\) comme suit:
            % insert code
            % if y = 0 then return [x|x]
            % else return 1

            \item On pose \(f'(x) = S_1^1\langle q,x\rangle\).
            \begin{equation*}
                \forall x,x\in \ol{\bb K} \iff f'(x)\in A
            \end{equation*}
            et \(f'\) est calculable totale. Donc \(\ol{\bb K}\prec A\).

            \item On a \(\ol{\bb K}\prec A\) donc \(\bb K \prec \ol{A}\) et donc \(\ol{A}\) n'est pas énumérable.
            De même, \(\bb K\prec A\) donc \(A\) n'est pas énumérable.
        \end{enumerate}
	\end{td-sol}
}{}


% ----- Consignes exo 3 ----- %
\begin{td-exo}[Enumérabilité] % 3
    
\end{td-exo}

% ----- Solutions exo 3 ----- %
\iftoggle{showsolutions}{
	\begin{td-sol}[]\ % 3
        
	\end{td-sol}
}{}


% ----- Consignes exo 4 ----- %
\begin{td-exo}[Points fixes]\,\\ % 4
    Soit \(F:x\mapsto a\) où \(a\) est un entier fixé. Notons \(n_0\) un point fixe de \(F\).
    \begin{enumerate}
        \item Que calcule le programme \(n_0\)?
        \item Est-ce que \(a\) est un point fixe de \(F\)?
        % insert escaped text here
        \item Peut-on avoir \(n_1 = a\)? \(n_1 = n_0\)? Que calcule \(n_1\)?
        \item Montrez qu'il existe un programme partour convergent qui prend en entrée \(a\) et \(b\)
        et donne en sortie le point fixe \(n_1\). En d'autres termes, il faut montrer que \(n_1\)
        est récursif en \(\langle a,b\rangle\).
        \item En vous inspirant de ce qui précède, montrez qu'il existe un programme partout
        convergent \(toto\) qui, sur l'entrée \(\langle a,b,i\rangle\) donne un programme noté \(n_i\) 
        qui calcule la même fonction que \(a\) \(\left( [a\ |\ \cdot = [n_i\ |\ \cdot]]\right)\),
        tous les \(n_i\) étant différents quand \(i\) varie, \(a\) et \(b\) étant fixés.
        % insert second escaped text here
        \item Si le programme \(a\) calcule la fonction constante de valeur 2, 
        que donnent \(toto\langle a,b_0,i\rangle\) et \(toto\langle a,b_1,i\rangle\)?
        \item Si le programme \(a\) calcule la fonction constante de valeur 1, 
        que donnent \(toto\langle a,b_0,i\rangle\) et \(toto\langle a,b_1,i\rangle\)?
        \item A l'aide de \(b_0,b_1\) et \(toto\), définissez un programme partout convergent qui,
        sur l'entrée \(\langle a,i\rangle\) donne un programme \(n_i\) qui calcule la même fonction que \(a\),
        tous les \(n_i\) étant différents quand \(i\) varie, \(a\) étant fixé.
    \end{enumerate}
\end{td-exo}

% ----- Solutions exo 4 ----- %
\iftoggle{showsolutions}{
	\begin{td-sol}[]\, % 4
        \begin{enumerate}
            \item On applique le théorème de point fixe qui nous dit que
            \begin{equation*}
                \exists n,[n\ |\ \cdot] = [F(n)\ |\ \cdot] = [a\ |\ \cdot]
            \end{equation*}
            Donc le programme \(n_0\) calcule la même chose que \(a\).

            \item Oui car \(F(a) = a\).

            \item Non car si \(n_1 = a\) ou \(n_1 = n_0\), on aurait à la fois
            \begin{equation*}
                F(n_1) = n_1 = a \text{ ou } n_0
            \end{equation*}
            et 
            \begin{equation*}
                F(n_1) = b
            \end{equation*}
            On arrive à une contradiction donc \(n_1\neq a\) et \(n_1\neq n_0\).

            \item \(n_1\) est récursif en \(a\) et \(b\) car il est récursif dans \(F_1\) (par le théorème 
            du point fixe effectif) et ce code dépend récursivement de \(a\) et \(b\).

            \item On va définir \(F_i\) comme suit:
            % if x = a or x = n0 or ... or x = n_(i)
            % then return b
            % else return a

            où \(n_i\) est un point fixe de \(F_i\). On applique le théorème du point fixe effectif pour obtenir
            \(n_i\) récursif en \(a,b,i\).

            \item Il renvoie un programme qui calcule la fonction constante de valeur 2 dans
            les deux cas.

            \item Il renvoie un programme qui calcule la fonction constante de valeur 1 dans
            le premier cas et dans le deuxième cas, il peut renvoyer \(a\).

            \item 
        \end{enumerate}
	\end{td-sol}
}{}


% % ----- Consignes exo xx ----- %
% \begin{td-exo}[yy]\,\\ % xx
    
% \end{td-exo}

% % ----- Solutions exo xx ----- %
% \iftoggle{showsolutions}{
% 	\begin{td-sol}[]\, % xx
        
% 	\end{td-sol}
% }{}
