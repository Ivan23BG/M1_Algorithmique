% ----- Consignes exo 1 ----- %
\begin{td-exo}[Non standard on refait]\,\\ % 1
    On considère \(\mathcal{M} = (\bb Z, <)\) les entiers relatifs avec un seul symbole de prédicat qui est l'ordre habituel inférieur strict (total) noté \(<\).
    Le langage du modèle est noté \(\mathcal L\).
    On note \(T = \mathsf{th}(\mathcal{M})\), ensemble de formules sur le langage \(\mathcal L\).
    On ajoute maintenant à \(\mathcal L\) deux symboles de constante \(a\) et \(b\) et on obtient le langage \(\mathcal{L}'\).
    \begin{enumerate}
        \item On considère le \(\mathcal L'\)-modèle
    \end{enumerate}
\end{td-exo}

% ----- Solutions exo 1 ----- %
\iftoggle{showsolutions}{
	\begin{td-sol}[]\,\\ % 1
		Exercice solution
	\end{td-sol}
}{}


% ----- Consignes exo 2 ----- %
\begin{td-exo}[Inexprimabilité de l'infini]\,\\ % 2
    On commence par étudier une théorie sur le langage avec uniquement un prédicat binaire \(<\).
    \begin{enumerate}
        \item Rappelez sous forme de formules closes les propriétés du prédicat \(<\) que l'on souhaite de façon à ce qu'il représente un ordre (strict) total.
    \end{enumerate}
\end{td-exo}

% ----- Solutions exo 2 ----- %
\iftoggle{showsolutions}{
	\begin{td-sol}[]\,\\ % 2
		\begin{enumerate}
            \item % TODO
            \item \(M\) est fini alors l'univers de \(M\) est fini, donc a un plus grand élément \(a\) et \(M \models \lnot \exists y\quad a<y\).
            
            \item Non, par exemple dans \((\bb Z \setminus \bb N, <_{\bb{Z}})\), le modèle est infini mais il existe un élément maximum.

            \item On se donne \(\mathcal L\) le langage quelconque et \(T\) une théorie cohérente qui admet un modèle de cardinal \(n\) pour tout \(n \in \bb N\).

            \item \(\varphi\) correspond à \og{}le modèle a \(n\) éléments au moins\fg{}. C'est à dire
            \begin{equation*}
                M \models \varphi_n \iff |M| \geq n.
            \end{equation*}
            Donc
            \begin{equation*}
                \varphi_n = \exists x_1, \dots, x_n \bigwedge_{i \neq j} x_i \neq x_j.
            \end{equation*}
            Donc
            \begin{equation*}
                M \models \varphi_n \iff |M| \geq n.
            \end{equation*}
            On pose \(G = \{ \varphi_n, n \in \bb N\}\) et on considère \(T \cup G\).

            \item On prend \(M\) un modèle fini de \(T\). Alors 
            \begin{itemize}
                \item \(M \models T\) car \(M\) est un modèle de \(T\).
                \item et \(M \not\models G\).
            \end{itemize}
            car \(m = |M|\) et alors

            \item On suppose \(T\cup G \vdash \perp\). Montrons qu'il existe \(F \subseteq T\cup G\) fini et tel que \(F \vdash \perp\).
            
            Il suffit de prendre \(M\) assez grand pour que \(|M| \geq n\) pour tout \(\varphi_n \in F \cap G\).
        \end{enumerate}
	\end{td-sol}
}{}


% ----- Consignes exo 3 ----- %
\begin{td-exo}[Théorie de l'incrément]\,\\ % 3
Exercise 3 content
\end{td-exo}

% ----- Solutions exo 3 ----- %
\iftoggle{showsolutions}{
	\begin{td-sol}[]\,\\ % 3
		Exercice solution
	\end{td-sol}
}{}