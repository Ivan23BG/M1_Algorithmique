% ----- Consignes exo 1 ----- %
\begin{td-exo}[Reductions]\ % 1
	Soit \(A = \{x, \forall y, [x\mid y] \downarrow\}\).
	\begin{enumerate}
		\item En utilisant avec soin le théorème de Rice, montrez que \(A\) n'est pas récursif.
		\item Montrez que \(\bb K\prec A\).
		\item Montrez que \(\bb K\prec \ol A\) (où \(A\) est le complémentaire de \(A\)).
		\item Montrez que ni \(A\) ni \(\ol A\) ne sont énumérables.
	\end{enumerate}
\end{td-exo}

% ----- Solutions exo 1 ----- %
\iftoggle{showsolutions}{
	\begin{td-sol}[]\ % 1
		\begin{enumerate}
			\item On pose \(A = P_{\scr C}\) pour \(\mathscr{C} = \{f,\mathsf{Dom}(f) = \bb N\}\).
			En d'autres termes c'est l'ensemble des fonctions totales où qui sont définies partout.
			Alors
			\begin{equation*}
				\begin{aligned}
					P_{\mathscr{C}}
					&=\{
						x,[x\mid\cdot]\in\mathscr{C}
					\}\\
					&=\{
						x,\forall y,[x\mid y]\downarrow
					\}
				\end{aligned}
			\end{equation*}
			De plus, \(A\) n'est pas trivial car:
			\begin{itemize}
				\item le programme suivant est dans \(A\):
				\begin{equation*}
					a: x\mapsto \text{return }42 
				\end{equation*}
				\item et celui ci ne l'est pas:
				\begin{equation*}
					b: x\mapsto \text{while true } x \leftarrow x+1.
				\end{equation*}
			\end{itemize}
			Donc, d'après le théorème de Rice, \(A\) est indécidable et donc non récursif.

			\item On pose
			\begin{equation*}
				c: \langle x,y\rangle \mapsto \text{if } [x\mid x]\downarrow \text{ then return } [a\mid y]
			\end{equation*}
			et ensuite
			\begin{equation*}
				S_1^1\langle c, x\rangle: y \mapsto \cdots\cdots
			\end{equation*}
			(où \(\cdots\cdots\) correspond au meme code que \(c\)).
			Enfin, on pose \(f(x) = S_1^1\langle c, x\rangle\). Alors
			\begin{equation*}
				\forall x, x\in \bb K \iff f(x)\in A.
			\end{equation*}
			Donc \(\bb K\prec A\).

			\item On pose
			\begin{equation*}
				d\colon \langle x,y\rangle \mapsto \text{if }\mathsf{Step}\langle x, x, y\rangle = 0 \text{ then return } 3
			\end{equation*}
			et on pose un \(S_1^1\) similaire avec \(d\).
			Enfin, on pose \(g(x) = S_1^1\langle d, x\rangle\). Alors
			\begin{equation*}
				\forall x, x\in \bb K \iff g(x)\in \ol A.
			\end{equation*}
			Donc \(\bb K\prec \ol A\).

			\item On sait que \(\ol{\bb K}\) n'est pas énumérable donc \(A\) est énumérable.
			De même pour \(\ol A\). Donc ni \(A\) ni \(\ol A\) ne sont énumérables.
		\end{enumerate}
	\end{td-sol}
}{}


% ----- Consignes exo 2 ----- %
\begin{td-exo}[Reductions]\ % 2
	Soit \(B = \{x,[x\mid0]\downarrow \text{ et } [x\mid1]\uparrow\}\).
	\begin{enumerate}
		\item En utilisant avec soin le théorème de Rice, montrez que \(B\) n'est pas récursif.
		\item \(B\) et son complémentaire \(\ol B\) sont-ils énumérables?
	\end{enumerate}
\end{td-exo}

% ----- Solutions exo 2 ----- %
\iftoggle{showsolutions}{
	\begin{td-sol}[]\ %
		\begin{enumerate}
			\item On pose \(B = P_{\scr C}\) pour \(\mathscr{C} = \{f, 0\in \mathsf{Dom}(f) \text{ et } 1\notin \mathsf{Dom}(f)\}\).
			Alors si \(a\) calcule \(\upmodels\) alors \(a\notin B\) et le programme suivant:
			\begin{equation*}
				b: x\mapsto \text{if } x=0 \text{ then return } 42 \text{ else }\perp
			\end{equation*}
			est dans \(B\). Donc \(B\) n'est pas trivial et d'après le théorème de Rice, \(B\) n'est pas récursif.
		\end{enumerate}
	\end{td-sol}
}{}

