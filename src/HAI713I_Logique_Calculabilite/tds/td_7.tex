% ----- Consignes exo 1 ----- %
\begin{td-exo}[Modèles, théories et propriétés]
	\begin{enumerate}
		\item Pour le modèle \(\bb R, |x-y| = 1\), c'est-à-dire \(\bb R\) avec un prédicat binaire \(d\) tel que 
		\(d(x,y)\) si et seulement si \(|x-y| = 1\) (ce qui exprime que \(x\) et \(y\) sont à distance \(1\) dans \(\bb R\)),
		donner une formule \emph{une formule du langage ayant uniquement le prédicat binaire} \(d\) qui exprime \(|x-y| = 2\).

		\item Rappelez ce qu'est un automorphisme. Pour le modèle \((\bb R, |x-y|=1)\) montrez que la fonction \(x\to x+1\) 
		est un automorphisme. En déduire que dans ce modèle on ne peut pas exprimer la propriété \(x=0\).

		\item Montrez que dans \((\bb R, |x-y|=1)\) on ne peut pas exprimer la propriété \(x\) est un entier relatif.

		\item Montrez que dans \((\bb Z, x+y=z)\) on peut exprimer les propriétés \(x=0\) et \(x\) est pair.

		\item Montrez que dans \((\bb Z, x+y=z)\) on ne peut pas exprimer la propriété \(x>0\). 
	\end{enumerate}
\end{td-exo}

	
% ----- Solutions exo 1 ----- %
\iftoggle{showsolutions}{
	\begin{td-sol}[]\ % 1
        \begin{enumerate}
			\item On peut utiliser la formule:
			\begin{equation*}
				P(x,y) \iff \exists z,\quad x\neq z \land d(x,z) \land d(z,y)
			\end{equation*}

			\item La fonction \(f : x \mapsto x + 1\) est bien bijective donc on peut faire comme suit:
			\begin{equation*}
				\begin{aligned}
					M_a M &\models d(x,y) \\
					&\iff d(f(x), f(y)) \\
					&\iff d(x+1, y+1) \\
					&\iff |(x+1) - (y+1)| = 1 \\
					&\iff |x-y| = 1
				\end{aligned}
			\end{equation*}

			\item Si on pouvait exprimer la propriété \(P(x) \sim x = 0\) alors on aurait \(M \models P(x_0)\) et en particulier \(M \models P(f(x_0)) \equiv P(x+1)\). On aurait alors \(1=0\) ce qui est impossible, donc \(P\) ne peut pas exprimer \og{}\(x = 0\)\fg{}.

			\item On veut montrer que
			\begin{equation*}
				M \models P(\underbrace{x}_{\in \bb Z}) \iff M \models P(\underbrace{f(x)}_{\notin \bb Z})
			\end{equation*}
			Supposons qu'il existe \(P : P(x) \iff x\in \bb Z\). Utilisons l'isomorphisme suivant:
			\begin{equation*}
				g : x\mapsto x + \frac12
			\end{equation*}
			Alors pour \(x=1\), on a \(M\models P(1)\) et \(M\models P(1.5)\) qui n'est pas entier. Alors \(P\) n'exprime pas \og{}\(x\in \bb Z\)\fg{}

			\item On peut utiliser la formule suivante:
			\begin{equation*}
				F(x) \iff \exists y (x + y = y)
			\end{equation*}
			qui exprime bien \og{}\(x = 0\)\fg{}. Pour l'autre formule on peut utiliser:
			\begin{equation*}
				G(x) \iff \exists y\quad y+y=x
			\end{equation*}

			\item On peut utiliser l'isomorphisme suivant:
			\begin{equation*}
				f : x\mapsto -x
			\end{equation*}
			Il en suit
			\begin{equation*}
				\begin{aligned}
					-(x+y)=-z
					& \iff x+y = z \\
					& \iff f(x) + f(y) = f(z)
				\end{aligned}
			\end{equation*}
		\end{enumerate}
	\end{td-sol}
}{} 


% ----- Consignes exo 2 ----- %
\begin{td-exo}[Compacité --- ordres complétés]
	Soit \(\mathcal{A} = (A, <, \{a_0, a_1,\ldots\})\) un modèle dénombrable où chaque élément de \(\mathcal A\) apparaît comme une constante \(a_i\). Dans la suite, \(i\) et \(j\) sont des symboles de l'exercice et non pas de la théorie; ils représentent des entiers. La relation \(<\) est un ordre (partiel et strict) quelconque. On note \(T\) la théorie de l'ordre partiel (voir précédents exercices) et on a donc \(A\models T\). Nous allons démontrer qu'on peut étendre l'ordre de \(A\) en un (nouvel) ordre qui, lui, est total. On appelle \(f_{i,j}\) la formule \(a_i < a_j \lor a_j < a_i\).
	\begin{enumerate}
		\item Pour \(i\neq j\), trouvez un modèle \(\mathcal{A}_{i,j}\) de même langage que \(\mathcal A\) tel que \(\mathcal{A}_{i,j} \models T\cup \{f_{i,j}\}\) et où l'ordre sur \(\mathcal{A}_{i,j}\) étend celui sur \(\mathcal A\).

		\item Trouvez un modèle \(\mathcal{A}_n\) de même langage que \(\mathcal A\) tel que \(A_n \models T \cup \left\{ \bigwedge\limits_{\substack{i\leq n \\ j\leq n \\ i\neq j}} f_{i,j}\right\}\) et où l'ordre sur \(\mathcal{A}_n\) étend celui sur \(\mathcal A\).

		\item En déduire qu'on peut étendre l'ordre en un ordre total sur \(\mathcal A\).
		
	\end{enumerate}
\end{td-exo}
	

	
% ----- Solutions exo 2 ----- %
\iftoggle{showsolutions}{
	\begin{td-sol}[]\ % 2
        \begin{enumerate}
			\item On prend \(T\) la théorie de l'ordre partiel. 

			\begin{itemize}
				\item Soit \(a_i < a_j\) ou \(a_j < a_i\).
				Alors, \(\mathcal{A} \models f_{i,j}\) donc \(\mathcal{A} \models T \cup \{f_{i,j}\}\).

				\item Sinon, on pose \(<^{\prime} = < \cup \{a_i < a_j\}\). Alors, pour tout \(b\in A\) tel que 
				\begin{equation*}
					\begin{cases}
						a_i < a_j \\
						a_j < b
					\end{cases}
				\end{equation*}
				En rajoutant \(a_i < b\) on a
				\begin{equation*}
					\mathcal{A}' = \left( A, <^{\prime}, \ldots \right)\models T \cup \{f_{i,j}\}
				\end{equation*}
			\end{itemize}

			\item On peut juste redémontrer la même chose pour chaque \(\{f_{i,j}\}\).

			\item On veut construire un modèle de \(T\cup \left\{ \bigwedge\limits_{i\neq j} f_{i,j}\right\}\).
			\begin{equation*}
				\begin{aligned}
					&T_f \subseteq T \cup \left\{ \bigwedge\limits_{i\neq j} f_{i,j}\right\}\\
					& T_f = T \cup \left\{ f_{i_1, j_1}, f_{i_2, j_2}, \ldots \right\} \\
					& \exists n, \forall i, \forall j, \quad f_{i,j} \in T_f \implies i\leq n \land j\leq n\\
					& \mathcal{A}_n \models T\cup \left\{ \bigwedge\limits_{\substack{i\leq n \\ j\leq n \\ i\neq j}} f_{i,j}\right\}\\
					& \mathcal{A}_n \models T_f
				\end{aligned}
			\end{equation*}
			Donc par \(H_1\) compacité, \(T\) a un modèle \(B\):
			\begin{equation*}
				B \models T \cup \left\{ \bigwedge\limits_{i\neq j} f_{i,j}\right\}
			\end{equation*}
			\end{itemize}
		\end{enumerate}
	\end{td-sol}
}{} 


% ----- Consignes exo 3 ----- %
\begin{td-exo}[Compacité et \(\bb Z\)]
	On dispose ici d'une théorie \(T\) de l'ordre total strict et d'une fonction \(S\) vérifiant \(\forall x\quad x < S(x)\) et \(\lnot \left( \exists x, \exists y, x < y < S(x) \right)\). 
	\begin{enumerate}
		\item Ré-écrivez la formule \(\lnot \left( \exists x, \exists y, x < y < S(x) \right)\) avec uniquement des quantificateurs, le symbole de fonction \(S\) et celui de prédicat \(<\), et des symboles \(=, \land, \lor, \lnot\) où on n'appliquera ce dernier qu'à des formules atomiques.
		Ensuite, transformez la formule en une formule équivalente sans \(\lnot\) en tenant compte des autres formules de la théorie \(T\).

		\item On ajoute maintenant les symboles de constante \(a\) et \(b\) au langage. Pour tout entier \(n\) trouver une formule \(F_n(a,b)\) qui exprime \(a+n < b\).

		\item Pour tout entier \(n\) trouvez un modèle d'univers \(\bb N\) où \(F_n(a,b)\) est vrai, \(a,b\) étant des constantes du modèle.

		\item Montrez que tout ensemble fini de formules \(F_n\) ajouté à la théorie \(T\) est cohérent.

		\item Montrez que l'ensemble infini des formules \(F_n\) ajouté à la théorie \(T\) a un modèle.
		Décrivez un tel modèle.

		\item Que se passe-t-il si on ajoute à la théorie la formule \(\forall x, \exists y, S(y) = x\)?
	\end{enumerate}
\end{td-exo}

% ----- Solutions exo 3 ----- %
\iftoggle{showsolutions}{
	\begin{td-sol}[]\ % 3
		On rappelle que dans un ordre total, on a les propriétés suivantes:
			\begin{itemize}
				\item Antiréflexivité: \(\forall x, \lnot (x < x)\),
				\item Asymétrie: \(\forall x, \forall y, (x < y) \implies \lnot (y < x)\),
				\item Transitivité: \(\forall x, \forall y, \forall z, (x < y \land y < z) \implies (x < z)\),
				\item Totalité: \(\forall x, \forall y, x = y \lor x < y \lor y < x\).
				\item Une fonction \(S\) telle que
				\begin{itemize}
					\item \(\forall x, S(x) > x\),
					\item \(\lnot \left( \exists x, \exists y, x < y < S(x) \right)\).
				\end{itemize}
			\end{itemize}
		\begin{enumerate}
			\item On a
			\begin{equation*}
				\forall x, \forall y\quad \lnot (x < y) \lor \lnot (y < S(x))
			\end{equation*}
			et
			\begin{equation*}
				\forall x, \forall y \quad y < x \lor x = y \lor S(x) < y \lor S(x) = y
			\end{equation*}

			\item On a
			\begin{equation*}
				F_n(a,b) = \underbrace{S(S(\cdots (S(x))))}_{n \text{ fois}} < b
			\end{equation*}

			\item On pose
			\begin{equation*}
				\begin{aligned}
					M &= \left(
						\bb N, <_{\bb N}, S_{\bb N}, a, b
					\right)\\
					{[a]}^M &= 0\\
					{[b]}^M &= n+1\\
					M &\models F_n(a,b)
				\end{aligned}
			\end{equation*}
			Où \(T \cup \{F_n\}\) est cohérente.

			Alors, 
			\begin{equation*}
				F_3(a, b) = S(S(S(a))) < b
			\end{equation*}

			\item\label{log:td2:ex3} On a \(T \cup \{ F_{K_1}, \ldots, F_{K_m} \}\) avec \(m\) un entier quelconque et \(K\) une constante.
			Alors 
			\begin{equation*}
				\exists m, \forall n \geq m\quad F_n \notin T \cup \{ F_{K_1}, \ldots, F_{K_m} \}
			\end{equation*}
			Donc 
			\begin{equation*}
				{[a]}^M = 10,\quad {[b]}^M = 10 + m + 1
			\end{equation*}
			Donc
			\begin{equation*}
				M \models F_{K_i}
			\end{equation*}
			pour tout \(K_i\). Donc 
			\begin{equation*}
				M \models T \cup \{ F_{K_1}, \ldots, F_{K_m} \}
			\end{equation*}
			Donc l'extension de \(T\) par un nombre fini de formules \(F_n\) est cohérente.

			\item \(T \cup \{ F_n,n\in \bb N\}\). Par\~\ref{log:td2:ex3}, toute partie finie de \(T \cup \{F_n\}\) a un modèle, donc \(T \cup \{F_n\}\) a un modèle \(M\). Alors 
			\begin{equation*}
				M = \left(
					\bb N, <_{\bb N}, S_{\bb N}, a, b
				\right)
			\end{equation*}
			et \(F_n(a,b)\) est équivalent à \og{}\(a, b\) à distance au moins \(n\)\fg{}. Si 
			\begin{equation*}
				\exists n_0\quad a + n_0 = b,\quad M\models F_{n_0+1}
			\end{equation*}
			ce qui est absurde donc \(n\) est cohérente.

			%% inserer schema ici de deux \bb Ns concatenes a dans l'un, b dans l'autre
			\begin{equation*}
				U = \left\{
					(1, m) \text{ avec } i \in \{0, 1\} \text{ et } m \in \bb N
				\right\}
			\end{equation*}

			\item \(\forall x, \exissts y, S(y) = x\). 
		\end{enumerate}
	\end{td-sol}
}{}